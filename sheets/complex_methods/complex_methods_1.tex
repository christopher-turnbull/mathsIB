\documentclass[a4paper]{article}
\usepackage{amsmath}
\def\npart {IB}
\def\nterm {Lent}
\def\nyear {2018}
\def\nlecturer {Prof. Haynes (P.H.Haynes@damtp.cam.ac.uk)}
\def\ncourse {Complex Methods Example Sheet 1}

% Imports
\ifx \nauthor\undefined
  \def\nauthor{Christopher Turnbull}
\else
\fi

\author{Supervised by \nlecturer \\\small Solutions presented by \nauthor}
\date{\nterm\ \nyear}

\usepackage{alltt}
\usepackage{amsfonts}
\usepackage{amsmath}
\usepackage{amssymb}
\usepackage{amsthm}
\usepackage{booktabs}
\usepackage{caption}
\usepackage{enumitem}
\usepackage{fancyhdr}
\usepackage{graphicx}
\usepackage{mathdots}
\usepackage{mathtools}
\usepackage{microtype}
\usepackage{multirow}
\usepackage{pdflscape}
\usepackage{pgfplots}
\usepackage{siunitx}
\usepackage{slashed}
\usepackage{tabularx}
\usepackage{tikz}
\usepackage{tkz-euclide}
\usepackage[normalem]{ulem}
\usepackage[all]{xy}
\usepackage{imakeidx}

\makeindex[intoc, title=Index]
\indexsetup{othercode={\lhead{\emph{Index}}}}

\ifx \nextra \undefined
  \usepackage[pdftex,
    hidelinks,
    pdfauthor={Christopher Turnbull},
    pdfsubject={Cambridge Maths Notes: Part \npart\ - \ncourse},
    pdftitle={Part \npart\ - \ncourse},
  pdfkeywords={Cambridge Mathematics Maths Math \npart\ \nterm\ \nyear\ \ncourse}]{hyperref}
  \title{Part \npart\ --- \ncourse}
\else
  \usepackage[pdftex,
    hidelinks,
    pdfauthor={Christopher Turnbull},
    pdfsubject={Cambridge Maths Notes: Part \npart\ - \ncourse\ (\nextra)},
    pdftitle={Part \npart\ - \ncourse\ (\nextra)},
  pdfkeywords={Cambridge Mathematics Maths Math \npart\ \nterm\ \nyear\ \ncourse\ \nextra}]{hyperref}

  \title{Part \npart\ --- \ncourse \\ {\Large \nextra}}
  \renewcommand\printindex{}
\fi

\pgfplotsset{compat=1.12}

\pagestyle{fancyplain}
\lhead{\emph{\nouppercase{\leftmark}}}
\ifx \nextra \undefined
  \rhead{
    \ifnum\thepage=1
    \else
      \npart\ \ncourse
    \fi}
\else
  \rhead{
    \ifnum\thepage=1
    \else
      \npart\ \ncourse\ (\nextra)
    \fi}
\fi
\usetikzlibrary{arrows.meta}
\usetikzlibrary{decorations.markings}
\usetikzlibrary{decorations.pathmorphing}
\usetikzlibrary{positioning}
\usetikzlibrary{fadings}
\usetikzlibrary{intersections}
\usetikzlibrary{cd}

\newcommand*{\Cdot}{{\raisebox{-0.25ex}{\scalebox{1.5}{$\cdot$}}}}
\newcommand {\pd}[2][ ]{
  \ifx #1 { }
    \frac{\partial}{\partial #2}
  \else
    \frac{\partial^{#1}}{\partial #2^{#1}}
  \fi
}
\ifx \nhtml \undefined
\else
  \renewcommand\printindex{}
  \makeatletter
  \DisableLigatures[f]{family = *}
  \let\Contentsline\contentsline
  \renewcommand\contentsline[3]{\Contentsline{#1}{#2}{}}
  \renewcommand{\@dotsep}{10000}
  \newlength\currentparindent
  \setlength\currentparindent\parindent

  \newcommand\@minipagerestore{\setlength{\parindent}{\currentparindent}}
  \usepackage[active,tightpage,pdftex]{preview}
  \renewcommand{\PreviewBorder}{0.1cm}

  \newenvironment{stretchpage}%
  {\begin{preview}\begin{minipage}{\hsize}}%
    {\end{minipage}\end{preview}}
  \AtBeginDocument{\begin{stretchpage}}
  \AtEndDocument{\end{stretchpage}}

  \newcommand{\@@newpage}{\end{stretchpage}\begin{stretchpage}}

  \let\@real@section\section
  \renewcommand{\section}{\@@newpage\@real@section}
  \let\@real@subsection\subsection
  \renewcommand{\subsection}{\@@newpage\@real@subsection}
  \makeatother
\fi

% Theorems
\theoremstyle{definition}
\newtheorem*{aim}{Aim}
\newtheorem*{axiom}{Axiom}
\newtheorem*{claim}{Claim}
\newtheorem*{cor}{Corollary}
\newtheorem*{conjecture}{Conjecture}
\newtheorem*{defi}{Definition}
\newtheorem*{eg}{Example}
\newtheorem*{ex}{Exercise}
\newtheorem*{fact}{Fact}
\newtheorem*{law}{Law}
\newtheorem*{lemma}{Lemma}
\newtheorem*{notation}{Notation}
\newtheorem*{prop}{Proposition}
\newtheorem*{soln}{Solution}
\newtheorem*{thm}{Theorem}

\newtheorem*{remark}{Remark}
\newtheorem*{warning}{Warning}
\newtheorem*{exercise}{Exercise}

\newtheorem{nthm}{Theorem}[section]
\newtheorem{nlemma}[nthm]{Lemma}
\newtheorem{nprop}[nthm]{Proposition}
\newtheorem{ncor}[nthm]{Corollary}


\renewcommand{\labelitemi}{--}
\renewcommand{\labelitemii}{$\circ$}
\renewcommand{\labelenumi}{(\roman{*})}

\let\stdsection\section
\renewcommand\section{\newpage\stdsection}

% Strike through
\def\st{\bgroup \ULdepth=-.55ex \ULset}

% Maths symbols
\newcommand{\abs}[1]{\left\lvert #1\right\rvert}
\newcommand\ad{\mathrm{ad}}
\newcommand\AND{\mathsf{AND}}
\newcommand\Art{\mathrm{Art}}
\newcommand{\Bilin}{\mathrm{Bilin}}
\newcommand{\bket}[1]{\left\lvert #1\right\rangle}
\newcommand{\B}{\mathcal{B}}
\newcommand{\bolds}[1]{{\bfseries #1}}
\newcommand{\brak}[1]{\left\langle #1 \right\rvert}
\newcommand{\braket}[2]{\left\langle #1\middle\vert #2 \right\rangle}
\newcommand{\bra}{\langle}
\newcommand{\cat}[1]{\mathsf{#1}}
\newcommand{\C}{\mathbb{C}}
\newcommand{\CP}{\mathbb{CP}}
\newcommand{\cU}{\mathcal{U}}
\newcommand{\Der}{\mathrm{Der}}
\newcommand{\D}{\mathrm{D}}
\newcommand{\dR}{\mathrm{dR}}
\newcommand{\E}{\mathbb{E}}
\newcommand{\F}{\mathbb{F}}
\newcommand{\Frob}{\mathrm{Frob}}
\newcommand{\GG}{\mathbb{G}}
\newcommand{\gl}{\mathfrak{gl}}
\newcommand{\GL}{\mathrm{GL}}
\newcommand{\G}{\mathcal{G}}
\newcommand{\Gr}{\mathrm{Gr}}
\newcommand{\haut}{\mathrm{ht}}
\newcommand{\Id}{\mathrm{Id}}
\newcommand{\ket}{\rangle}
\newcommand{\lie}[1]{\mathfrak{#1}}
\newcommand{\Mat}{\mathrm{Mat}}
\newcommand{\N}{\mathbb{N}}
\newcommand{\norm}[1]{\left\lVert #1\right\rVert}
\newcommand{\normalorder}[1]{\mathop{:}\nolimits\!#1\!\mathop{:}\nolimits}
\newcommand\NOT{\mathsf{NOT}}
\newcommand{\Oc}{\mathcal{O}}
\newcommand{\Or}{\mathrm{O}}
\newcommand\OR{\mathsf{OR}}
\newcommand{\ort}{\mathfrak{o}}
\newcommand{\PGL}{\mathrm{PGL}}
\newcommand{\ph}{\,\cdot\,}
\newcommand{\pr}{\mathrm{pr}}
\newcommand{\Prob}{\mathbb{P}}
\newcommand{\PSL}{\mathrm{PSL}}
\newcommand{\Ps}{\mathcal{P}}
\newcommand{\PSU}{\mathrm{PSU}}
\newcommand{\pt}{\mathrm{pt}}
\newcommand{\qeq}{\mathrel{``{=}"}}
\newcommand{\Q}{\mathbb{Q}}
\newcommand{\R}{\mathbb{R}}
\newcommand{\RP}{\mathbb{RP}}
\newcommand{\Rs}{\mathcal{R}}
\newcommand{\SL}{\mathrm{SL}}
\newcommand{\so}{\mathfrak{so}}
\newcommand{\SO}{\mathrm{SO}}
\newcommand{\Spin}{\mathrm{Spin}}
\newcommand{\Sp}{\mathrm{Sp}}
\newcommand{\su}{\mathfrak{su}}
\newcommand{\SU}{\mathrm{SU}}
\newcommand{\term}[1]{\emph{#1}\index{#1}}
\newcommand{\T}{\mathbb{T}}
\newcommand{\tv}[1]{|#1|}
\newcommand{\U}{\mathrm{U}}
\newcommand{\uu}{\mathfrak{u}}
\newcommand{\Vect}{\mathrm{Vect}}
\newcommand{\wsto}{\stackrel{\mathrm{w}^*}{\to}}
\newcommand{\wt}{\mathrm{wt}}
\newcommand{\wto}{\stackrel{\mathrm{w}}{\to}}
\newcommand{\Z}{\mathbb{Z}}
\renewcommand{\d}{\mathrm{d}}
\renewcommand{\H}{\mathbb{H}}
\renewcommand{\P}{\mathbb{P}}
\renewcommand{\sl}{\mathfrak{sl}}
\renewcommand{\vec}[1]{\boldsymbol{\mathbf{#1}}}
%\renewcommand{\F}{\mathcal{F}}

\let\Im\relax
\let\Re\relax

\DeclareMathOperator{\adj}{adj}
\DeclareMathOperator{\Ann}{Ann}
\DeclareMathOperator{\area}{area}
\DeclareMathOperator{\Aut}{Aut}
\DeclareMathOperator{\Bernoulli}{Bernoulli}
\DeclareMathOperator{\betaD}{beta}
\DeclareMathOperator{\bias}{bias}
\DeclareMathOperator{\binomial}{binomial}
\DeclareMathOperator{\card}{card}
\DeclareMathOperator{\ccl}{ccl}
\DeclareMathOperator{\Char}{char}
\DeclareMathOperator{\ch}{ch}
\DeclareMathOperator{\cl}{cl}
\DeclareMathOperator{\cls}{\overline{\mathrm{span}}}
\DeclareMathOperator{\conv}{conv}
\DeclareMathOperator{\corr}{corr}
\DeclareMathOperator{\cosec}{cosec}
\DeclareMathOperator{\cosech}{cosech}
\DeclareMathOperator{\cov}{cov}
\DeclareMathOperator{\covol}{covol}
\DeclareMathOperator{\diag}{diag}
\DeclareMathOperator{\diam}{diam}
\DeclareMathOperator{\Diff}{Diff}
\DeclareMathOperator{\disc}{disc}
\DeclareMathOperator{\dom}{dom}
\DeclareMathOperator{\End}{End}
\DeclareMathOperator{\energy}{energy}
\DeclareMathOperator{\erfc}{erfc}
\DeclareMathOperator{\erf}{erf}
\DeclareMathOperator*{\esssup}{ess\,sup}
\DeclareMathOperator{\ev}{ev}
\DeclareMathOperator{\Ext}{Ext}
\DeclareMathOperator{\Fit}{Fit}
\DeclareMathOperator{\fix}{fix}
\DeclareMathOperator{\Frac}{Frac}
\DeclareMathOperator{\Gal}{Gal}
\DeclareMathOperator{\gammaD}{gamma}
\DeclareMathOperator{\gr}{gr}
\DeclareMathOperator{\hcf}{hcf}
\DeclareMathOperator{\Hom}{Hom}
\DeclareMathOperator{\id}{id}
\DeclareMathOperator{\image}{image}
\DeclareMathOperator{\im}{im}
\DeclareMathOperator{\Im}{Im}
\DeclareMathOperator{\Ind}{Ind}
\DeclareMathOperator{\Int}{Int}
\DeclareMathOperator{\Isom}{Isom}
\DeclareMathOperator{\lcm}{lcm}
\DeclareMathOperator{\length}{length}
\DeclareMathOperator{\Lie}{Lie}
\DeclareMathOperator{\like}{like}
\DeclareMathOperator{\Lk}{Lk}
\DeclareMathOperator{\mse}{mse}
\DeclareMathOperator{\multinomial}{multinomial}
\DeclareMathOperator{\orb}{orb}
\DeclareMathOperator{\ord}{ord}
\DeclareMathOperator{\otp}{otp}
\DeclareMathOperator{\Poisson}{Poisson}
\DeclareMathOperator{\poly}{poly}
\DeclareMathOperator{\rank}{rank}
\DeclareMathOperator{\rel}{rel}
\DeclareMathOperator{\Re}{Re}
\DeclareMathOperator*{\res}{res}
\DeclareMathOperator{\Res}{Res}
\DeclareMathOperator{\rk}{rk}
\DeclareMathOperator{\Root}{Root}
\DeclareMathOperator{\sech}{sech}
\DeclareMathOperator{\sgn}{sgn}
\DeclareMathOperator{\spn}{span}
\DeclareMathOperator{\stab}{stab}
\DeclareMathOperator{\St}{St}
\DeclareMathOperator{\supp}{supp}
\DeclareMathOperator{\Syl}{Syl}
\DeclareMathOperator{\Sym}{Sym}
\DeclareMathOperator{\tr}{tr}
\DeclareMathOperator{\Tr}{Tr}
\DeclareMathOperator{\var}{var}
\DeclareMathOperator{\vol}{vol}

\pgfarrowsdeclarecombine{twolatex'}{twolatex'}{latex'}{latex'}{latex'}{latex'}
\tikzset{->/.style = {decoration={markings,
                                  mark=at position 1 with {\arrow[scale=2]{latex'}}},
                      postaction={decorate}}}
\tikzset{<-/.style = {decoration={markings,
                                  mark=at position 0 with {\arrowreversed[scale=2]{latex'}}},
                      postaction={decorate}}}
\tikzset{<->/.style = {decoration={markings,
                                   mark=at position 0 with {\arrowreversed[scale=2]{latex'}},
                                   mark=at position 1 with {\arrow[scale=2]{latex'}}},
                       postaction={decorate}}}
\tikzset{->-/.style = {decoration={markings,
                                   mark=at position #1 with {\arrow[scale=2]{latex'}}},
                       postaction={decorate}}}
\tikzset{-<-/.style = {decoration={markings,
                                   mark=at position #1 with {\arrowreversed[scale=2]{latex'}}},
                       postaction={decorate}}}
\tikzset{->>/.style = {decoration={markings,
                                  mark=at position 1 with {\arrow[scale=2]{latex'}}},
                      postaction={decorate}}}
\tikzset{<<-/.style = {decoration={markings,
                                  mark=at position 0 with {\arrowreversed[scale=2]{twolatex'}}},
                      postaction={decorate}}}
\tikzset{<<->>/.style = {decoration={markings,
                                   mark=at position 0 with {\arrowreversed[scale=2]{twolatex'}},
                                   mark=at position 1 with {\arrow[scale=2]{twolatex'}}},
                       postaction={decorate}}}
\tikzset{->>-/.style = {decoration={markings,
                                   mark=at position #1 with {\arrow[scale=2]{twolatex'}}},
                       postaction={decorate}}}
\tikzset{-<<-/.style = {decoration={markings,
                                   mark=at position #1 with {\arrowreversed[scale=2]{twolatex'}}},
                       postaction={decorate}}}

\tikzset{circ/.style = {fill, circle, inner sep = 0, minimum size = 3}}
\tikzset{mstate/.style={circle, draw, blue, text=black, minimum width=0.7cm}}

\tikzset{commutative diagrams/.cd,cdmap/.style={/tikz/column 1/.append style={anchor=base east},/tikz/column 2/.append style={anchor=base west},row sep=tiny}}

\definecolor{mblue}{rgb}{0.2, 0.3, 0.8}
\definecolor{morange}{rgb}{1, 0.5, 0}
\definecolor{mgreen}{rgb}{0.1, 0.4, 0.2}
\definecolor{mred}{rgb}{0.5, 0, 0}

\def\drawcirculararc(#1,#2)(#3,#4)(#5,#6){%
    \pgfmathsetmacro\cA{(#1*#1+#2*#2-#3*#3-#4*#4)/2}%
    \pgfmathsetmacro\cB{(#1*#1+#2*#2-#5*#5-#6*#6)/2}%
    \pgfmathsetmacro\cy{(\cB*(#1-#3)-\cA*(#1-#5))/%
                        ((#2-#6)*(#1-#3)-(#2-#4)*(#1-#5))}%
    \pgfmathsetmacro\cx{(\cA-\cy*(#2-#4))/(#1-#3)}%
    \pgfmathsetmacro\cr{sqrt((#1-\cx)*(#1-\cx)+(#2-\cy)*(#2-\cy))}%
    \pgfmathsetmacro\cA{atan2(#2-\cy,#1-\cx)}%
    \pgfmathsetmacro\cB{atan2(#6-\cy,#5-\cx)}%
    \pgfmathparse{\cB<\cA}%
    \ifnum\pgfmathresult=1
        \pgfmathsetmacro\cB{\cB+360}%
    \fi
    \draw (#1,#2) arc (\cA:\cB:\cr);%
}
\newcommand\getCoord[3]{\newdimen{#1}\newdimen{#2}\pgfextractx{#1}{\pgfpointanchor{#3}{center}}\pgfextracty{#2}{\pgfpointanchor{#3}{center}}}

\def\Xint#1{\mathchoice
   {\XXint\displaystyle\textstyle{#1}}%
   {\XXint\textstyle\scriptstyle{#1}}%
   {\XXint\scriptstyle\scriptscriptstyle{#1}}%
   {\XXint\scriptscriptstyle\scriptscriptstyle{#1}}%
   \!\int}
\def\XXint#1#2#3{{\setbox0=\hbox{$#1{#2#3}{\int}$}
     \vcenter{\hbox{$#2#3$}}\kern-.5\wd0}}
\def\ddashint{\Xint=}
\def\dashint{\Xint-}

\newcommand\separator{{\centering\rule{2cm}{0.2pt}\vspace{2pt}\par}}

\newenvironment{own}{\color{gray!70!black}}{}

\newcommand\makecenter[1]{\raisebox{-0.5\height}{#1}}

\newtheorem*{soln}{Solution}

\renewcommand{\thesection}{}
\renewcommand{\thesubsection}{\arabic{section}.\arabic{subsection}}
\makeatletter
\def\@seccntformat#1{\csname #1ignore\expandafter\endcsname\csname the#1\endcsname\quad}
\let\sectionignore\@gobbletwo
\let\latex@numberline\numberline
\def\numberline#1{\if\relax#1\relax\else\latex@numberline{#1}\fi}
\makeatother


\begin{document}
	
\maketitle

\section{QUESTION 1}

\begin{enumerate}
	\item \emph{[For each of the following we let $ f(z) = u(x,y) + iv(x,y) $ and check the Cauchy-Riemann equations.] }
	
	\begin{itemize}
		\item $ f(z) = \Im z$. This has $ u = y $, $ v = 0 $. But
		
		\[ \frac{\partial u }{\partial y} = 1 \neq 0 = - \frac{\partial u }{\partial x} \]
		
		So $ \Im z $ is nowhere differentiable, and hence nowhere analytic.
		
		\item $ f(z) = | z |^{2} = x^{2} + y^{2} $. This has $ u = x^{2} + y^{2} $, $ y = 0 $. Have
		
		\[ \frac{\partial u }{\partial x} = 2x, \frac{\partial u }{\partial y} = 2y \quad \frac{\partial v }{\partial y} = \frac{\partial v }{\partial x} = 0  \]
		
		Hence the Cauchy-Riemann equations are only satisfied at the origin. So $ f $ in only differentiable at $ z = 0 $, however it is not analytic since there is no neighbourhood of 0 throughout which $ f $ is differentiable.
		
		\item $ f(z) = \sech z $. First note that if $ f(z) = u + iv \neq 0 $, then \[ \frac{1}{f(z)} = \frac{u}{u^{2} + v^{2}} - \frac{iv}{u^{2} + v^{2}} \] 
		
		So if $ f(z) $ is analytic, then $ \frac{1}{f(z)} $ is analytic provided $ f(z) \neq 0 $. 
		
		
		$ g(z) : = \cosh(z) = \frac{1}{2} (e^{z} + e^{-z}) $ is entire since $ e^{z} $ is entire (from lectures). Checking when $ g $ is zero gives us $ z = \frac{1}{2} \log(-1) = \frac{1}{2} \left[ \log(1) + (2 n + 1) i \pi \right] $ for integer $ n $.
		
		Hence $ \sech (z) $ is differentiable at all points expect those at $ (0,(n + \frac{1}{2}) \pi) $ for integer $ n $, and hence also analytic everywhere but these points. 
	\end{itemize}
	
	
	\item Writing $ z= r(\cos \theta + i \sin \theta) $, we obtain 
	
	\[ u = r\cos 5 \theta \quad v = r\sin 5\theta \]
	
	Using the chain rule with $ r = \sqrt{x^{2} + y^{2}}, \tan \theta = \frac{y}{x} $,
	
	\st{This is going to be messy} First note that
	
	\[ \frac{\partial \theta }{\partial x} = \frac{-y}{x^{2} + y^{2}}, \qquad \frac{\partial \theta }{\partial y} = \frac{x}{x^{2} + y^{2}} \]
	
	and
	
	\[ \frac{\partial r }{\partial x} = x (x^{2} + y^{2})^{-1/2}, \qquad \frac{\partial r }{\partial y} y (x^{2} + y^{2})^{-1/2} \]
	
	The first Cauchy Riemann equation is
	
	\begin{align*}
	\frac{\partial u }{\partial x} & = \frac{\partial u }{\partial r} \frac{\partial r}{\partial x} + \frac{\partial u }{\partial \theta} \frac{\partial \theta}{\partial x} \\
	& = \cos (5 \theta) x (x^{2} + y^{2})^{-1/2} - r \sin 5 \theta \frac{-y}{x^{2} + y^{2}} \\
	& = \cos (5 \theta) r \cos \theta r^{-1}  - r \sin 5 \theta \frac{-r \sin \theta }{r^{2}} \\
	& = \cos 4 \theta
	\end{align*}
	
	Similarly, 
	
	\begin{align*}
	\frac{\partial v }{\partial y} & = \frac{\partial v }{\partial r} \frac{\partial r}{\partial y} + \frac{\partial v }{\partial \theta} \frac{\partial \theta}{\partial y} \\
	& = \sin (5 \theta) y (x^{2} + y^{2})^{-1/2} + r \cos 5 \theta \frac{x}{x^{2} + y^{2}} \\
	& = \sin (5 \theta) r \sin \theta r^{-1} + r \cos 5 \theta \frac{r \cos \theta}{r^{2}} \\
	& = \cos 4 \theta
	\end{align*}
	
	For second CR equation,
	
	\begin{align*}
	\frac{\partial v }{\partial x} & = \frac{\partial v }{\partial r} \frac{\partial r}{\partial x} + \frac{\partial v }{\partial \theta} \frac{\partial \theta}{\partial x} \\
	& = \sin (5 \theta) x (x^{2} + y^{2})^{-1/2} + r \cos 5 \theta \frac{-y}{x^{2} + y^{2}} \\
	& = \sin (5 \theta) r \cos \theta r^{-1}  + r \cos 5 \theta \frac{-r \sin \theta }{r^{2}} \\
	& = \sin 4 \theta
	\end{align*}
	
	and 
	
	\begin{align*}
	\frac{\partial u }{\partial y} & = \frac{\partial u }{\partial r} \frac{\partial r}{\partial y} + \frac{\partial u }{\partial \theta} \frac{\partial \theta}{\partial y} \\
	& = \cos (5 \theta) y (x^{2} + y^{2})^{-1/2} + - \sin 5 \theta \frac{x}{x^{2} + y^{2}} \\
	& = \cos (5 \theta) r \sin \theta r^{-1} - r \sin 5 \theta \frac{r \cos \theta}{r^{2}} \\
	& = -  \sin 4 \theta
	\end{align*}
	
	We conclude in fact that the Cauchy-Riemann equations are satisfied everywhere. 
	
	Now, looking closer at $ \frac{\partial u }{\partial x} $, we have 
	
	\begin{align*}
	\frac{\partial u }{\partial x} & = \cos 4 \theta \\
	& = \cos^{2} 2\theta - \sin^{2} 2 \theta \\
	& = ( \cos^{2} \theta - \sin^{2} \theta )^{2} - 4 \sin^{2} \theta \cos^{2} \theta \\
	& = ( \cos^{2} \theta + \sin^{2} \theta )^{2} - 8 \sin^{2} \theta \cos^{2} \theta \\
	& = 1 - \frac{8x^{2}y^{2}}{r^{4}} \\
	& = 1 - \frac{8x^{2}y^{2}}{(x^{2} + y^{2})^{2}}
 	\end{align*}
 	
 	Now we see that when $ x = y = 0 $, $ \frac{\partial u }{\partial x} $ is not defined, which is enough to show that $ f $ is not differentiable at the origin.
	
	
	
	
	

\end{enumerate}

\section{QUESTION 2}
\emph{[Each of the following analytical functions will be of the form $ f(z) = u(x,y) + iv(x,y) $. Given $ u $, we find $ v $ using the Cauchy-Riemann equations, and thus $ f $.] }
\begin{enumerate}
	
	\item The first Cauchy Riemann equation determines
	
	\[ \frac{\partial u }{\partial x} = \frac{\partial v }{\partial y} = 1 \implies v = y + g(x) \]
	
	The other Cauchy Riemann equation gives
	
	\[ 0 = - \frac{\partial u }{\partial y} = \frac{\partial v }{\partial x} = g'(x) \]
	
	So $ g $ must be a constant, say $ \alpha $. Wlog set it to zero. The corresponding analytic function is therefore
	
	\[ f(z) = x + iy   = z  \]
	
	\item $ u = xy, $ so the first Cauchy Riemann equation determines
	
	\[ \frac{\partial u }{\partial x} = \frac{\partial v }{\partial y} = y \implies  v = \frac{1}{2} y^{2} + g(x) \]
	
	The other Cauchy Riemann equation gives
	
	\[ -x = - \frac{\partial u }{\partial y} = \frac{\partial v }{\partial x} = g'(x) \]
	
	So $ g'(x) = - x $, giving us $ g(x) = - \frac{1}{2} x^{2} + \alpha $ for some constant $ \alpha $, wlog 0. The corresponding analytic function is therefore
	
	\begin{align*}
	f(z) & = xy +  \frac{1}{2} i (y^{2} - x^{2}) \\
	& = \frac{1}{2} i \left( y^{2} - 2ixy - x^{2}    \right) \\
	& =  - \frac{1}{2} i \left( x^{2} + 2 i x y - y^{2}    \right) \\
	& = - \frac{1}{2} i \left( x + i y   \right)^{2} \\
	& = - \frac{1}{2} i z^{2}
	\end{align*}
	
	
		
	\item $ u = \sin x \cosh y, $ so the first Cauchy Riemann equation determines
	
	\[ \frac{\partial u }{\partial x} = \frac{\partial v }{\partial y} =   \cos x \cosh y\implies v = \cos x \sinh y + g(x) \]
	
	The other Cauchy Riemann equation gives
	
	\[ \sin x \sinh y = \frac{\partial u }{\partial y} = - \frac{\partial v }{\partial x} = \sin x \sinh y + g'(x) \]
	
	So $ g'(x) = 0 $, giving us $ v(x) = \cos x \sinh y + \alpha $ for some constant $ \alpha $ (wlog set it to zero). The corresponding analytic function is therefore
	
	
	\begin{align*}
	f(z) & = \sin x \cosh y  +  i \cos x \sinh y \\
	& = \frac{1}{2} e^{y} (\sin x + i \cos x) + \frac{1}{2} e^{-y} ( \sin x - i \cos x) \\
	& = \frac{1}{2} i [   e^{y - i x} - e^{ix - y} ] \\
	& = i \sinh (z^{*})
	\end{align*}
	
	
		
	\item $ u = \log(x^{2}+y^{2}), $ so Cauchy Riemann determine that
	
	Recall $ \int \frac{1}{a^{2} + x^{2}} \; \d x = \frac{1}{a} \arctan(x/a) $
	
	\[ \frac{\partial v }{\partial y} = \frac{\partial u }{\partial x } = \frac{2x}{x^{2}+y^{2}} \implies y = 2 \arctan (y/x) + g(x)   \]
	Next,
	
	\[ \frac{2y}{x^{2}+y^{2}}  = \frac{\partial u }{\partial y} = - \frac{\partial v}{\partial x} =  \frac{2y}{x^{2}+y^{2}} + g'(x) \]
		
	Hence $ g'(x) = 0 $, set $ g(x) = 0 $ wlog, have that 
	
	
	\begin{align*}
	f(z) & = \log(x^{2}+y^{2})  +  i 2 \arctan (y/x) \\
	& = \log(| z |^{2}) + 2 i \text{sgn}(x) arg(z) \\
	\end{align*}
	
	
	
	
		
	\item $ u = \frac{y}{(x+1)^{2} + y^{2}} $, so Cauchy Riemann determine that
	
	
		
	\[ \frac{\partial u }{\partial x} = \frac{\partial v }{\partial y} =   \frac{-2y(x+1)}{[(x+1)^{2} + y^{2}]^{2}}  \]
	
	\begin{align*}
	\implies v & = -(x+1)\int 2y [(x+1)^{2} + y^{2}]^{-2} \; \d y   \\
	& = \frac{-(x+1)}{(x+1)^{2} + y^{2}} + g(x)
	\end{align*}
	
	CBA checking the next one (is this necessary?)
	
	The corresponding analytic function is therefore
	
	\begin{align*}
	f(z) & = \frac{y}{(x+1)^{2} + y^{2}}  + - i \frac{(x+1)}{(x+1)^{2} + y^{2}} \\
	& = y + i(x+1) \\
	& = i(x - iy) + i \\
	& = i z^{*} + i
	\end{align*}
	
	\item $ u = \arctan\left(   \frac{2xy}{x^{2} - y^{2}} \right)   $, so Cauchy Riemann determine that
	
	
	
	\begin{align*}
	\frac{\partial u }{\partial x} = \frac{\partial v }{\partial y} & = \frac{(x^{2} - y^{2})2y - 2xy(2x)}{(x^{2} - y^{2})^{2} + (2xy)^{2}  }   \\
	& = \frac{2y[(x^{2} - y^{2}) - 2x^{2}]}{(x^{2} + y^{2})^{2} } \\
	& = \frac{-2y(x^{2} + y^{2})}{(x^{2} + y^{2})^{2} } \\
	& = \frac{-2y}{x^{2} + y^{2} }
	\end{align*}
	
	\begin{align*}
	\implies v & = \int \frac{-2y}{x^{2} + y^{2} }\; \d y \\
	& = - \log ( x^{2} + y^{2} ) + g(x) \\	
	\end{align*}
	
	Hmm, deduce that $ g'(x) = 0 $, set the constant to zero, so we have 
	
	\begin{align*}
	f(z) & = \arctan\left(   \frac{2xy}{x^{2} - y^{2}} \right)  - i \log ( x^{2} + y^{2} ) \\
	& = 
	\end{align*}
		
	
\end{enumerate}


Now, if these $ f = u + iv $ are analytic, (and therefore satisfy the Cauchy-Riemann equations) we can compute

\begin{align*}
\frac{\partial^{2} u}{\partial x^{2}} & = \frac{\partial }{\partial x} \left(  \frac{\partial u }{\partial x} \right)  \\
& = \frac{\partial }{\partial x} \left(  \frac{\partial v }{\partial y} \right)\\
& = \frac{\partial }{\partial y} \left(  \frac{\partial v }{\partial x} \right)\\
& = \frac{\partial }{\partial y} \left(  - \frac{\partial u }{\partial y} \right)\\
& =  - \frac{\partial^{2} u }{\partial y^{2}}  \\
\end{align*}

Therefore, when the CR equations are satisfied, the function $ u $ is harmonic.

Hence, for the above questions, we have $ u $ harmonic on

\begin{enumerate}
	\item $ \R^{2} $
	\item $ \R^{2} $
	\item $ \R^{2} $
	\item $ \R^{2} $
	\item $ \R^{2} $
	\item $ \R^{2} $
	
\end{enumerate}

\section{QUESTION 3}
\section{QUESTION 4}

\[ \phi(x,y) = e^{x}(x \cos y - y \sin y) \]

Calculating the partial derivatives,

\begin{align*}
\partial_{x} \phi & = \phi + e^{x}\cos y \\
\partial_{xx} \phi & = \partial_{x} \phi +  e^{x}\cos y  \\
& = \phi + 2 e^{x}\cos y
\end{align*}



\begin{align*}
\partial_{y} \phi & = e^{x}(-x \sin y - \sin y - y \cos y) \\
\partial_{yy} \phi & = e^{x}(-x \cos y - 2\cos y + y \sin y ) \\
& = - \phi - 2 e^{x}\cos y
\end{align*}

Hence $ \partial_{xx} \phi + \partial_{yy} \phi = 0  $ and the function is indeed harmonic.

The harmonic conjugate $ \psi(x,y) $ satisfies the Cauchy Riemann equations

\[ \frac{\partial \phi }{\partial x} = \frac{\partial \psi }{\partial y}, \qquad  \frac{\partial \phi }{\partial y} = - \frac{\partial \psi }{\partial x} \]

The first of these gives 

\[ \frac{\partial \phi }{\partial x} = \frac{\partial \psi }{\partial y} = e^{x}(x \cos y - y \sin y) + e^{x}\cos y \]

Noting $ \int y \sin y \d y = - y \cos y + \sin y $, we must have $ \psi = e^{x}(x \sin y + y \cos y) + g(x) $. The other Cauchy Riemann equation gives

\[ e^{x}(x \sin y + \sin y + y \cos y) = - \frac{\partial \phi }{\partial y} = \frac{\partial v }{\partial x} = e^{x}(x \sin y + \sin y + y \cos y) + g'(x) \]

So $ g $ must be a constant, say $ 0 $, so the harmonic conjugate of $ \phi $ is

\[ \psi(x,y) = e^{x}(x \sin y + y \cos y) \]

Can now show that $ \nabla \phi \cdot \nabla  \psi = 0 $ (by the CR equations), ie. contours of harmonic conjugate function are perpendicular (in 2D). 

Recall that a gradient of a function is perpendicular to its contours. 





\section{QUESTION 5}
\section{QUESTION 6}
\section{QUESTION 7}
\section{QUESTION 8}
\section{QUESTION 9}
\section{QUESTION 10}
\section{QUESTION 11}
\section{QUESTION 12}



\end{document}