\documentclass[a4paper]{article}
\usepackage{amsmath}
\def\npart {IB}
\def\nterm {Lent}
\def\nyear {2018}
\def\nlecturer {Prof. Haynes (P.H.Haynes@damtp.cam.ac.uk)}
\def\ncourse {Complex Methods Example Sheet 3}

\input{header}

\newtheorem*{soln}{Solution}

\renewcommand{\thesection}{}
\renewcommand{\thesubsection}{\arabic{section}.\arabic{subsection}}
\makeatletter
\def\@seccntformat#1{\csname #1ignore\expandafter\endcsname\csname the#1\endcsname\quad}
\let\sectionignore\@gobbletwo
\let\latex@numberline\numberline
\def\numberline#1{\if\relax#1\relax\else\latex@numberline{#1}\fi}
\makeatother


\begin{document}
	
\maketitle

\section{QUESTION 1}

Know that 

\[ \tilde{f}(\omega) = \int_{-\infty}^{\infty} f(t) e^{-i \omega t} \; \d t, \quad f(t) = \frac{1}{2 \pi} \int_{-\infty}^{\infty} \tilde{f}(\omega) e^{i \omega t} \; \d \omega  \]

First, let $ f(t) = e^{-  a | t |} $. Then

\begin{align*}
\tilde{f}(\omega) & = \int_{-\infty}^{\infty} e^{-a | t |} e^{-i \omega t} \; \d t \\
& =  \int_{-\infty}^{0} e^{ a t } e^{-i \omega t} \; \d t + \int_{0}^{\infty} e^{- a t } e^{-i \omega t} \; \d t \\
& = \frac{1}{a - i \omega } \left[   e^{ a t } e^{-i \omega t} \right]_{-\infty}^{0} + \frac{1}{-a - i \omega } \left[ e^{ - a t } e^{-i \omega t} \right]_{0}^{\infty} \\
& = \frac{1}{a - i \omega} + \frac{1}{a + i \omega} \\
& = \frac{2a}{a^{2} + \omega^{2}}
\end{align*}

So using the inverse Fourier transform relation we have

\[ e^{-  a | t |} = \frac{1}{2 \pi} \int_{-\infty}^{\infty} \frac{2a}{a^{2} + \omega^{2}} e^{i \omega t} \; \d \omega \]

\[ \iff \frac{\pi}{a} e^{-  a | t |} = \int_{-\infty}^{\infty} \frac{1}{a^{2} + \omega^{2}} e^{i \omega t} \; \d \omega  \]

as required.

Next, let $ f(t) = e^{-at} \sin bt H(t) $. Then 


\begin{align*}
\tilde{f}(\omega) & = \int_{-\infty}^{\infty} e^{-at} \sin bt e^{-i \omega t} H(t) \; \d t \\
& = \int_{0}^{\infty} e^{-at} e^{-i \omega t} \sin bt \; \d t \\
& = \int_{0}^{\infty} e^{-at} e^{-i \omega t} \frac{i}{2} \left(  e^{-i b t} - e^{i b t} \right)  \; \d t \\
& = \frac{i}{2}  \int_{0}^{\infty}  e^{-at}e^{-i(\omega + b)t}   -  e^{-at}e^{-i(\omega - b)t}   \; \d t \\
& = \frac{i}{2} \left(  -  \frac{1}{-a -i(\omega + b)} - \frac{-1}{-a - i(\omega - b)} \right) \\
& =  \frac{i}{2} \left(  \frac{1}{a + i\omega + ib} - \frac{1}{a + i \omega - ib} \right) \\
& = \frac{i( - i b )}{(a + i \omega)^{2} + b^{2} } \\
& = \frac{b}{(a + i \omega)^{2} + b^{2} }
\end{align*}


So using the inverse Fourier transform relation we have

\[ e^{-at} \sin bt H(t) = \frac{1}{2 \pi} \int_{-\infty}^{\infty} \frac{b}{(a + i \omega)^{2} + b^{2} } e^{i \omega t} \; \d \omega \]

\[ \iff 2 \pi  e^{-at} \sin bt H(t) = \int_{-\infty}^{\infty} \frac{b}{(a + i \omega)^{2} + b^{2} } e^{i \omega t} \; \d \omega  \]

as required.

Not sure about the $ a < 0 $, $ a = 0 $ parts. 







\section{QUESTION 2}

Given

\[ f(x) = \begin{cases}  1  & \text{ if } | x | < \frac{1}{2} a \\ 0 & \text{ otherwise } \end{cases} \]

We compute the Fourier transform as 

\begin{align*}
\tilde{f}(k) & = \int_{-\infty}^{\infty} f(x) e^{- i k x} \; \d x \\
& = \int_{-a/2}^{a/2} e^{-ikx} \; \d x \\
& = \left[ \frac{1}{-ik} e^{-ikx}  \right]_{-a/2}^{a/2} \\
& = \frac{i}{k} \left(  e^{-ika/2} - e^{ika/2} \right) \\
& =  \frac{2}{k} \sin \left( \frac{ak}{2} \right) 
\end{align*}

as required. Similarly, given

\[ g(x) = \begin{cases} a - | x |  & \text{ if } | x | < a \\ 0 & \text{ otherwise } \end{cases} \]

We compute the Fourier transform as 

\begin{align*}
\tilde{g}(k) & = \int_{-\infty}^{\infty} g(x) e^{- i k x} \; \d x \\
& = \int_{-a}^{a} ( a - | x | )e^{-ikx} \; \d x \\
& =  \underbrace{\int_{-a}^{a} a e^{-ikx} \; \d x}_{(1)} - \left(  \underbrace{\int_{-a}^{0} - x e^{-ikx} \; \d x}_{(2)}   + \underbrace{\int_{0}^{a} x e^{-ikx} \; \d x }_{(3)} \right) \\
& =  \frac{2a}{k} \sin a k + \frac{a}{ik} (  e^{ika} + e^{-ika} ) - \frac{1}{(ik)^{2}}\left[ 1 - e^{ika} \right] + \frac{1}{(ik)^{2}} \left[  e^{-ika} - 1 \right] \\
& =   
\end{align*}

Now \begin{align*}
(1) & = a \left[ \frac{1}{-ik} e^{-ikx}  \right]_{-a}^{a} \\
& = a \frac{i}{k} \left(  e^{-ika} - e^{ika} \right) \\
& = \frac{2a}{k} \sin a k,
\end{align*}


\begin{align*}
(2) & = \left[  -x \left( - \frac{1}{ik} e^{-ikx} \right) - \int - \frac{1}{- ik} e^{-ikx}  \right]_{-a}^{0} \\
& = \left[  - \frac{a}{ik} e^{ika} + \frac{1}{(ik)^{2}} \left[ e^{-ikx} \right]_{-a}^{0} \right] \\
& = - \frac{a}{ik} e^{ika} - \frac{1}{k^{2}} \left(  1 - e^{ika} \right) 
\end{align*}

\begin{align*}
(3) & = \left[ x \left( - \frac{1}{ik} e^{-ikx} \right) - \int - \frac{1}{ik} e^{-ikx}  \right]_{0}^{a} \\
& = \left[  - \frac{a}{ik} e^{-ika} - \frac{1}{(ik)^{2}} \left[ e^{-ikx} \right]_{0}^{a} \right] \\
& = - \frac{a}{ik} e^{-ika} + \frac{1}{k^{2}} \left( e^{-ika} - 1 \right) 
\end{align*}






\section{QUESTION 3}


The convolution of $ g(x) = e^{-| x |} $ with itself is given by

\begin{align*}
g * g(x) & = \int_{-\infty}^{\infty} e^{-| x - y|} e^{-| y |} \; \d y  \\
\end{align*}

If $ x > 0 $ then we can split up the integral as 

\begin{align*}
g * g(x) & = \int_{-\infty}^{0} e^{x - y} e^{y} \; \d y + \int_{0}^{x} e^{x - y} e^{-y} \; \d y + \int_{x}^{\infty} e^{-(x - y)} e^{-y} \; \d y \\
& = 
\end{align*}


Similarly if $ x < 0 $ then

If $ x > 0 $ then we can split up the integral as 

\begin{align*}
g * g(x) & = \int_{-\infty}^{x} e^{x - y} e^{y} \; \d y + \int_{x}^{0} e^{-(x - y)} e^{y} \; \d y + \int_{0}^{\infty} e^{-(x - y)} e^{-y} \; \d y \\
& = 
\end{align*}


Next, the convolution theorem for Fourier transforms states that $ \mathcal{F}[g * g(x)] = \mathcal{F}[g] \mathcal{F}[g] $. Applied here, we have

\[ \int_{-\infty}^{\infty} (1 + | x |)e^{-| x |} e^{-ikx} \; \d x = \left[  \int_{-\infty}^{\infty} e^{-| x |} e^{ikx} \; \d x \right]^{2}   \]

\section{QUESTION 4}
\section{QUESTION 5}

Starting with 

\[ \mathcal{L}(1) = \frac{1}{s} \]

\begin{enumerate}
	\item By shifting, 
	
	\[ \mathcal{L}(e^{-2t}) = \frac{1}{s + 2} \]
	
	\item Starting with shifting,
	\[ \mathcal{L}(e^{-3t}) = \frac{1}{s + 3}  \]
	
	Then
	
	\begin{align*}
	\mathcal{L}(t^{3}e^{-3t}) & = - \frac{\d^{3} }{\d s^{3}} \mathcal{L}(e^{-3t})   \\
	& = - \frac{\d^{3} }{\d s^{3}} \frac{1}{s+3}   \\
	& = \frac{6}{(s+3)^{4}}
	\end{align*}
	
	\item Write $ \sin 4t = \frac{1}{2i} (e^{4it} - e^{-4it}) $. Then
	
	\begin{align*}
	\mathcal{L}(e^{3t} \sin 4t)& = \frac{1}{2i} \mathcal{L}(e^{(3 + 4i)t})   - \frac{1}{2i} \mathcal{L}(e^{(3 - 4i)t})   \\
	& = \frac{1}{2i} \left[   \frac{1}{s - (3 + 4i)} - \frac{1}{s - (3 - 4i)} \right] \\
	& = \frac{4}{(s-3)^{2} + 16}
	\end{align*}
	
	\item Writing $ \cosh 4t = \frac{1}{2}(e^{4t} + e^{-4t}) $, we have
	
	\begin{align*}
	\mathcal{L}(e^{-4t} \cosh 4t )& = \frac{1}{2} \mathcal{L}(1) + \frac{1}{2} \mathcal{L}(e^{-8t})   \\
	& = \frac{1}{2} \left[  \frac{1}{s} + \frac{1}{s + 8} \right] \\
	& = \frac{s+4}{s(s+8)}
	\end{align*}
	
	\item First by shifting,
	
	\[ \mathcal{L}(e^{-(t+1)}) = e^{-1} \mathcal{L}(e^{-t}) = \frac{e^{-1}}{s+1} \]
	
	Then by translation,
	
	\[ \mathcal{L}(e^{-t}H(t - 1)) = \frac{e^{-(s+1)}}{s+1} \]
	
	
\end{enumerate}

\section{QUESTION 6}

By partial fractions we have

\[ \hat{f}(s) = \frac{s+3}{(s-2)(s^{2} + 1)} = \frac{1}{s-2} - \frac{s+1}{s^{2} + 1} \]

Have 

\[ \mathcal{L}(e^{2t}) = \frac{1}{s-2}, \quad \mathcal{L}(\cos t) = \frac{s}{s^{2} + 1}, \quad \mathcal{L}(sin t) = \frac{1}{s^{2} + 1}  \]

Hence by linearity the inverse Laplace transform of $ \hat{f}(s) $ is given by

\[ f(t) = e^{2t} - \cos t - \sin t \]


Alternatively we can use the Bromwich inversion formula,
\[
f(t) = \sum_{k = 1}^n \res_{p = p_k} (\hat{f}(p) e^{pt}),
\]
as $ \hat{f}(s) $ has only a finite number of singularities, namely $ s = 2, i, -i $, and $ \hat{f}(s) \to 0 $ as $ | s | \to \infty $. First,

\begin{align*}
\res_{s = 2} \left( \frac{s+3}{(s-2)(s^{2} + 1)} e^{st} \right)  & = \lim\limits_{s \to 2} \left( \frac{s+3}{(s^{2} + 1)} e^{st} \right) \\
& = e^{2t}
\end{align*}

Next,

\begin{align*}
\res_{s = i} \left( \frac{s+3}{(s^{2} + 1)} e^{st} \right)  & = \lim\limits_{s \to i}  \left( \frac{s+3}{(s-2)(s+i)} e^{st} \right)\\
& = \frac{i+3}{2i(i-2)} e^{it} \\
& = - \frac{i+3}{4i + 2)} e^{it} \\
& = \frac{-10 + 10i}{20} e^{it} = \frac{-1+i}{2} ( \cos t + i \sin t)
\end{align*}

Similarly,

\begin{align*}
\res_{s = -i} \left( \frac{s+3}{(s^{2} + 1)} e^{st} \right)  & = \lim\limits_{s \to -i}  \left( \frac{s+3}{(s-2)(s-i)} e^{st} \right)\\
& = \frac{-i+3}{2i(i+2)} e^{-it} \\
& = \frac{-10 - 10i}{20} e^{-it} = \frac{-1-i}{2} ( \cos t - i \sin t)
\end{align*}

Hence adding the residues we achieve 

\[ f(t) = e^{2t} - \cos t - \sin t \]

in agreement with what we had above.


\section{QUESTION 7}

Consider the differential equation


\[ \frac{\d^{3} y }{\d t^{3}} - 3 \frac{\d^{2} y }{\d t^{2}} + 3 \frac{\d y}{\d t} - y = t^{2} e^{t} \]

\[ y(0) = 1, \dot{y}(0) = 0, \ddot{y}(0) = -2 \]

Taking the Laplace transform of this equation, where $ \mathcal{L}(y) = \hat{y} $, we have

\begin{align*}
\mathcal{L}(\dot{y}) & = p \hat{y} + y(0) \\
& =  p \hat{y} + 1
\end{align*}

Similarly

\begin{align*}
\mathcal{L}(\ddot{y}) & = p \mathcal{L}(\dot{y}) + \dot{y}(0) \\
& = p^{2} \hat{y} + p y(0) + \dot{y}(0) \\
& = p^{2} \hat{y} + p
\end{align*}

and,

\begin{align*}
\mathcal{L} \left( \frac{\d^{3} y  }{\d t^{3}} \right) & = p \mathcal{L}(\ddot{y}) + \ddot{y}(0) \\
& = p^{3} \hat{y} + p^{2} y(0) + p \dot{y}(0) + \ddot{y}(0) \\
& = p^{3} \hat{y} + p^{2} + -2
\end{align*}

The term on the RHS gives

\begin{align*}
\mathcal{L}(t^{2}e^{t}) & = \frac{\d^{2} }{\d p^{2}} \mathcal{L}(e^{t}) \\
& =  \frac{\d^{2} }{\d p^{2}} \left(  \frac{1}{p - 1} \right) \\
& = \frac{2}{(p-1)^{3}} 
\end{align*}

Thus substituting in gives:

\[ (p^{3} - 3p^{2} + 3 p - 1 )\hat{y} + p^{2} - 3p + 1 = \frac{2}{(p-1)^{3}} \]

Hence

\[ \hat{y} = \frac{2}{(p-1)^{6}} - \frac{1}{(p-1)} + \frac{p}{(p-1)^{3}} \]

We know $ \mathcal{L}(e^{t}) = 1/(p-1) $. For the other two terms, we will use the Bromwich inversion formula from the previous question, noting both tend to zero as $ | p | \to \infty $. 

\[
f(t) = \sum_{k = 1}^n \res_{p = p_k} (\hat{f}(p) e^{pt}),
\]

For the first term, the singularity $ p = 1 $ has a pole of order 6. Hence the residue here is given by

\begin{align*}
\lim_{p \to 1}\frac{1}{5!}  \frac{\d^{5}}{\d p^{5}} (p-1)^{5} \left( \frac{2}{(p-1)^{6}} e^{pt} \right) & = \lim_{p \to 1} \frac{1}{5!}  \frac{\d^{5}}{\d p^{5}}\left( -2e^{pt}(1-p)^{-1}\right) \\
& = \lim_{p \to 1} \frac{1}{5!} \frac{\d^{5}}{\d p^{5}} \left[ -2 e^{pt} \left( 1 + p + p^{2} + p^{3} + \cdots   \right) \right]  \\
& = \cdots
\end{align*}

The last term has a singularity at $ p = 3 $ with a pole of order $ 3 $ there; the residue is given by 

\begin{align*}
\lim_{p \to 1}\frac{1}{2!}  \frac{\d^{2}}{\d p^{2}} (p-1)^{2} \left( \frac{p}{(p-1)^{3}} e^{pt} \right) & = \lim_{p \to 1}\frac{1}{2!}  \frac{\d^{2}}{\d p^{2}}  \left( \frac{p}{1-p} e^{pt} \right) \\
& = \lim_{p \to 1}\frac{1}{2!}  \frac{\d}{\d p}  \left( \frac{1}{(1-p)^{2}} e^{pt} + \frac{p^{2}}{(1-p)} e^{pt} \right) \\
& = \lim_{p \to 1}\frac{1}{2!} \left( \frac{2}{(1-p)^{3}} e^{pt} + \frac{p}{(1-p)^{2}} e^{pt} + \frac{1-p^{2}}{(1-p)^{2}} e^{pt} + \frac{p^{3}}{(1-p)} e^{pt} \right)
\end{align*}

Not sure how to take limit.



\section{QUESTION 8}

The differential equation we are investigating is

\[ \ddot{y} - 2 \dot{y} - 2 y = \delta(t) - \delta(t - t_{0}) \]

Note that

\[ \mathcal{L}(\delta(t)) = \int_{0}^{\infty} \delta(t) e^{\pt} \; \d t = 1, \quad \mathcal{L}(\delta(t-t_{0})) = e^{-pt_{0}} \]

Taking the Laplace transform of the equation gives

\[ p^{2} \hat{y} + 2 (p \hat{y}) + 2 \hat{y} = 1 + e^{-pt_{0}} \]

which gives

\[ \hat{y} = \frac{1 + e^{-pt_{0}}}{(p^{2} + 2p + 2)} \]





\section{QUESTION 9}

Taking the Laplace transform of the equation gives

\[ \hat{f} + 4 \frac{1}{p} \hat{f} = \frac{1}{p^{2}} \]

Thus

\[ \hat{f} = \frac{1}{p^{2} + 4p} = \frac{1}{(p+2)^{2} -4} \]


\section{QUESTION 10}
\section{QUESTION 11}

(whenever I can be bothered) http://www.robots.ox.ac.uk/~jmb/lectures/pdelecture4.pdf


\section{QUESTION 12}



\end{document}