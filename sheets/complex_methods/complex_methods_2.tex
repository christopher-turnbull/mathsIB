\documentclass[a4paper]{article}
\usepackage{amsmath}
\def\npart {IB}
\def\nterm {Lent}
\def\nyear {2018}
\def\nlecturer {Prof. Haynes (P.H.Haynes@damtp.cam.ac.uk)}
\def\ncourse {Complex Methods Example Sheet 2}

% Imports
\ifx \nauthor\undefined
  \def\nauthor{Christopher Turnbull}
\else
\fi

\author{Supervised by \nlecturer \\\small Solutions presented by \nauthor}
\date{\nterm\ \nyear}

\usepackage{alltt}
\usepackage{amsfonts}
\usepackage{amsmath}
\usepackage{amssymb}
\usepackage{amsthm}
\usepackage{booktabs}
\usepackage{caption}
\usepackage{enumitem}
\usepackage{fancyhdr}
\usepackage{graphicx}
\usepackage{mathdots}
\usepackage{mathtools}
\usepackage{microtype}
\usepackage{multirow}
\usepackage{pdflscape}
\usepackage{pgfplots}
\usepackage{siunitx}
\usepackage{slashed}
\usepackage{tabularx}
\usepackage{tikz}
\usepackage{tkz-euclide}
\usepackage[normalem]{ulem}
\usepackage[all]{xy}
\usepackage{imakeidx}

\makeindex[intoc, title=Index]
\indexsetup{othercode={\lhead{\emph{Index}}}}

\ifx \nextra \undefined
  \usepackage[pdftex,
    hidelinks,
    pdfauthor={Christopher Turnbull},
    pdfsubject={Cambridge Maths Notes: Part \npart\ - \ncourse},
    pdftitle={Part \npart\ - \ncourse},
  pdfkeywords={Cambridge Mathematics Maths Math \npart\ \nterm\ \nyear\ \ncourse}]{hyperref}
  \title{Part \npart\ --- \ncourse}
\else
  \usepackage[pdftex,
    hidelinks,
    pdfauthor={Christopher Turnbull},
    pdfsubject={Cambridge Maths Notes: Part \npart\ - \ncourse\ (\nextra)},
    pdftitle={Part \npart\ - \ncourse\ (\nextra)},
  pdfkeywords={Cambridge Mathematics Maths Math \npart\ \nterm\ \nyear\ \ncourse\ \nextra}]{hyperref}

  \title{Part \npart\ --- \ncourse \\ {\Large \nextra}}
  \renewcommand\printindex{}
\fi

\pgfplotsset{compat=1.12}

\pagestyle{fancyplain}
\lhead{\emph{\nouppercase{\leftmark}}}
\ifx \nextra \undefined
  \rhead{
    \ifnum\thepage=1
    \else
      \npart\ \ncourse
    \fi}
\else
  \rhead{
    \ifnum\thepage=1
    \else
      \npart\ \ncourse\ (\nextra)
    \fi}
\fi
\usetikzlibrary{arrows.meta}
\usetikzlibrary{decorations.markings}
\usetikzlibrary{decorations.pathmorphing}
\usetikzlibrary{positioning}
\usetikzlibrary{fadings}
\usetikzlibrary{intersections}
\usetikzlibrary{cd}

\newcommand*{\Cdot}{{\raisebox{-0.25ex}{\scalebox{1.5}{$\cdot$}}}}
\newcommand {\pd}[2][ ]{
  \ifx #1 { }
    \frac{\partial}{\partial #2}
  \else
    \frac{\partial^{#1}}{\partial #2^{#1}}
  \fi
}
\ifx \nhtml \undefined
\else
  \renewcommand\printindex{}
  \makeatletter
  \DisableLigatures[f]{family = *}
  \let\Contentsline\contentsline
  \renewcommand\contentsline[3]{\Contentsline{#1}{#2}{}}
  \renewcommand{\@dotsep}{10000}
  \newlength\currentparindent
  \setlength\currentparindent\parindent

  \newcommand\@minipagerestore{\setlength{\parindent}{\currentparindent}}
  \usepackage[active,tightpage,pdftex]{preview}
  \renewcommand{\PreviewBorder}{0.1cm}

  \newenvironment{stretchpage}%
  {\begin{preview}\begin{minipage}{\hsize}}%
    {\end{minipage}\end{preview}}
  \AtBeginDocument{\begin{stretchpage}}
  \AtEndDocument{\end{stretchpage}}

  \newcommand{\@@newpage}{\end{stretchpage}\begin{stretchpage}}

  \let\@real@section\section
  \renewcommand{\section}{\@@newpage\@real@section}
  \let\@real@subsection\subsection
  \renewcommand{\subsection}{\@@newpage\@real@subsection}
  \makeatother
\fi

% Theorems
\theoremstyle{definition}
\newtheorem*{aim}{Aim}
\newtheorem*{axiom}{Axiom}
\newtheorem*{claim}{Claim}
\newtheorem*{cor}{Corollary}
\newtheorem*{conjecture}{Conjecture}
\newtheorem*{defi}{Definition}
\newtheorem*{eg}{Example}
\newtheorem*{ex}{Exercise}
\newtheorem*{fact}{Fact}
\newtheorem*{law}{Law}
\newtheorem*{lemma}{Lemma}
\newtheorem*{notation}{Notation}
\newtheorem*{prop}{Proposition}
\newtheorem*{soln}{Solution}
\newtheorem*{thm}{Theorem}

\newtheorem*{remark}{Remark}
\newtheorem*{warning}{Warning}
\newtheorem*{exercise}{Exercise}

\newtheorem{nthm}{Theorem}[section]
\newtheorem{nlemma}[nthm]{Lemma}
\newtheorem{nprop}[nthm]{Proposition}
\newtheorem{ncor}[nthm]{Corollary}


\renewcommand{\labelitemi}{--}
\renewcommand{\labelitemii}{$\circ$}
\renewcommand{\labelenumi}{(\roman{*})}

\let\stdsection\section
\renewcommand\section{\newpage\stdsection}

% Strike through
\def\st{\bgroup \ULdepth=-.55ex \ULset}

% Maths symbols
\newcommand{\abs}[1]{\left\lvert #1\right\rvert}
\newcommand\ad{\mathrm{ad}}
\newcommand\AND{\mathsf{AND}}
\newcommand\Art{\mathrm{Art}}
\newcommand{\Bilin}{\mathrm{Bilin}}
\newcommand{\bket}[1]{\left\lvert #1\right\rangle}
\newcommand{\B}{\mathcal{B}}
\newcommand{\bolds}[1]{{\bfseries #1}}
\newcommand{\brak}[1]{\left\langle #1 \right\rvert}
\newcommand{\braket}[2]{\left\langle #1\middle\vert #2 \right\rangle}
\newcommand{\bra}{\langle}
\newcommand{\cat}[1]{\mathsf{#1}}
\newcommand{\C}{\mathbb{C}}
\newcommand{\CP}{\mathbb{CP}}
\newcommand{\cU}{\mathcal{U}}
\newcommand{\Der}{\mathrm{Der}}
\newcommand{\D}{\mathrm{D}}
\newcommand{\dR}{\mathrm{dR}}
\newcommand{\E}{\mathbb{E}}
\newcommand{\F}{\mathbb{F}}
\newcommand{\Frob}{\mathrm{Frob}}
\newcommand{\GG}{\mathbb{G}}
\newcommand{\gl}{\mathfrak{gl}}
\newcommand{\GL}{\mathrm{GL}}
\newcommand{\G}{\mathcal{G}}
\newcommand{\Gr}{\mathrm{Gr}}
\newcommand{\haut}{\mathrm{ht}}
\newcommand{\Id}{\mathrm{Id}}
\newcommand{\ket}{\rangle}
\newcommand{\lie}[1]{\mathfrak{#1}}
\newcommand{\Mat}{\mathrm{Mat}}
\newcommand{\N}{\mathbb{N}}
\newcommand{\norm}[1]{\left\lVert #1\right\rVert}
\newcommand{\normalorder}[1]{\mathop{:}\nolimits\!#1\!\mathop{:}\nolimits}
\newcommand\NOT{\mathsf{NOT}}
\newcommand{\Oc}{\mathcal{O}}
\newcommand{\Or}{\mathrm{O}}
\newcommand\OR{\mathsf{OR}}
\newcommand{\ort}{\mathfrak{o}}
\newcommand{\PGL}{\mathrm{PGL}}
\newcommand{\ph}{\,\cdot\,}
\newcommand{\pr}{\mathrm{pr}}
\newcommand{\Prob}{\mathbb{P}}
\newcommand{\PSL}{\mathrm{PSL}}
\newcommand{\Ps}{\mathcal{P}}
\newcommand{\PSU}{\mathrm{PSU}}
\newcommand{\pt}{\mathrm{pt}}
\newcommand{\qeq}{\mathrel{``{=}"}}
\newcommand{\Q}{\mathbb{Q}}
\newcommand{\R}{\mathbb{R}}
\newcommand{\RP}{\mathbb{RP}}
\newcommand{\Rs}{\mathcal{R}}
\newcommand{\SL}{\mathrm{SL}}
\newcommand{\so}{\mathfrak{so}}
\newcommand{\SO}{\mathrm{SO}}
\newcommand{\Spin}{\mathrm{Spin}}
\newcommand{\Sp}{\mathrm{Sp}}
\newcommand{\su}{\mathfrak{su}}
\newcommand{\SU}{\mathrm{SU}}
\newcommand{\term}[1]{\emph{#1}\index{#1}}
\newcommand{\T}{\mathbb{T}}
\newcommand{\tv}[1]{|#1|}
\newcommand{\U}{\mathrm{U}}
\newcommand{\uu}{\mathfrak{u}}
\newcommand{\Vect}{\mathrm{Vect}}
\newcommand{\wsto}{\stackrel{\mathrm{w}^*}{\to}}
\newcommand{\wt}{\mathrm{wt}}
\newcommand{\wto}{\stackrel{\mathrm{w}}{\to}}
\newcommand{\Z}{\mathbb{Z}}
\renewcommand{\d}{\mathrm{d}}
\renewcommand{\H}{\mathbb{H}}
\renewcommand{\P}{\mathbb{P}}
\renewcommand{\sl}{\mathfrak{sl}}
\renewcommand{\vec}[1]{\boldsymbol{\mathbf{#1}}}
%\renewcommand{\F}{\mathcal{F}}

\let\Im\relax
\let\Re\relax

\DeclareMathOperator{\adj}{adj}
\DeclareMathOperator{\Ann}{Ann}
\DeclareMathOperator{\area}{area}
\DeclareMathOperator{\Aut}{Aut}
\DeclareMathOperator{\Bernoulli}{Bernoulli}
\DeclareMathOperator{\betaD}{beta}
\DeclareMathOperator{\bias}{bias}
\DeclareMathOperator{\binomial}{binomial}
\DeclareMathOperator{\card}{card}
\DeclareMathOperator{\ccl}{ccl}
\DeclareMathOperator{\Char}{char}
\DeclareMathOperator{\ch}{ch}
\DeclareMathOperator{\cl}{cl}
\DeclareMathOperator{\cls}{\overline{\mathrm{span}}}
\DeclareMathOperator{\conv}{conv}
\DeclareMathOperator{\corr}{corr}
\DeclareMathOperator{\cosec}{cosec}
\DeclareMathOperator{\cosech}{cosech}
\DeclareMathOperator{\cov}{cov}
\DeclareMathOperator{\covol}{covol}
\DeclareMathOperator{\diag}{diag}
\DeclareMathOperator{\diam}{diam}
\DeclareMathOperator{\Diff}{Diff}
\DeclareMathOperator{\disc}{disc}
\DeclareMathOperator{\dom}{dom}
\DeclareMathOperator{\End}{End}
\DeclareMathOperator{\energy}{energy}
\DeclareMathOperator{\erfc}{erfc}
\DeclareMathOperator{\erf}{erf}
\DeclareMathOperator*{\esssup}{ess\,sup}
\DeclareMathOperator{\ev}{ev}
\DeclareMathOperator{\Ext}{Ext}
\DeclareMathOperator{\Fit}{Fit}
\DeclareMathOperator{\fix}{fix}
\DeclareMathOperator{\Frac}{Frac}
\DeclareMathOperator{\Gal}{Gal}
\DeclareMathOperator{\gammaD}{gamma}
\DeclareMathOperator{\gr}{gr}
\DeclareMathOperator{\hcf}{hcf}
\DeclareMathOperator{\Hom}{Hom}
\DeclareMathOperator{\id}{id}
\DeclareMathOperator{\image}{image}
\DeclareMathOperator{\im}{im}
\DeclareMathOperator{\Im}{Im}
\DeclareMathOperator{\Ind}{Ind}
\DeclareMathOperator{\Int}{Int}
\DeclareMathOperator{\Isom}{Isom}
\DeclareMathOperator{\lcm}{lcm}
\DeclareMathOperator{\length}{length}
\DeclareMathOperator{\Lie}{Lie}
\DeclareMathOperator{\like}{like}
\DeclareMathOperator{\Lk}{Lk}
\DeclareMathOperator{\mse}{mse}
\DeclareMathOperator{\multinomial}{multinomial}
\DeclareMathOperator{\orb}{orb}
\DeclareMathOperator{\ord}{ord}
\DeclareMathOperator{\otp}{otp}
\DeclareMathOperator{\Poisson}{Poisson}
\DeclareMathOperator{\poly}{poly}
\DeclareMathOperator{\rank}{rank}
\DeclareMathOperator{\rel}{rel}
\DeclareMathOperator{\Re}{Re}
\DeclareMathOperator*{\res}{res}
\DeclareMathOperator{\Res}{Res}
\DeclareMathOperator{\rk}{rk}
\DeclareMathOperator{\Root}{Root}
\DeclareMathOperator{\sech}{sech}
\DeclareMathOperator{\sgn}{sgn}
\DeclareMathOperator{\spn}{span}
\DeclareMathOperator{\stab}{stab}
\DeclareMathOperator{\St}{St}
\DeclareMathOperator{\supp}{supp}
\DeclareMathOperator{\Syl}{Syl}
\DeclareMathOperator{\Sym}{Sym}
\DeclareMathOperator{\tr}{tr}
\DeclareMathOperator{\Tr}{Tr}
\DeclareMathOperator{\var}{var}
\DeclareMathOperator{\vol}{vol}

\pgfarrowsdeclarecombine{twolatex'}{twolatex'}{latex'}{latex'}{latex'}{latex'}
\tikzset{->/.style = {decoration={markings,
                                  mark=at position 1 with {\arrow[scale=2]{latex'}}},
                      postaction={decorate}}}
\tikzset{<-/.style = {decoration={markings,
                                  mark=at position 0 with {\arrowreversed[scale=2]{latex'}}},
                      postaction={decorate}}}
\tikzset{<->/.style = {decoration={markings,
                                   mark=at position 0 with {\arrowreversed[scale=2]{latex'}},
                                   mark=at position 1 with {\arrow[scale=2]{latex'}}},
                       postaction={decorate}}}
\tikzset{->-/.style = {decoration={markings,
                                   mark=at position #1 with {\arrow[scale=2]{latex'}}},
                       postaction={decorate}}}
\tikzset{-<-/.style = {decoration={markings,
                                   mark=at position #1 with {\arrowreversed[scale=2]{latex'}}},
                       postaction={decorate}}}
\tikzset{->>/.style = {decoration={markings,
                                  mark=at position 1 with {\arrow[scale=2]{latex'}}},
                      postaction={decorate}}}
\tikzset{<<-/.style = {decoration={markings,
                                  mark=at position 0 with {\arrowreversed[scale=2]{twolatex'}}},
                      postaction={decorate}}}
\tikzset{<<->>/.style = {decoration={markings,
                                   mark=at position 0 with {\arrowreversed[scale=2]{twolatex'}},
                                   mark=at position 1 with {\arrow[scale=2]{twolatex'}}},
                       postaction={decorate}}}
\tikzset{->>-/.style = {decoration={markings,
                                   mark=at position #1 with {\arrow[scale=2]{twolatex'}}},
                       postaction={decorate}}}
\tikzset{-<<-/.style = {decoration={markings,
                                   mark=at position #1 with {\arrowreversed[scale=2]{twolatex'}}},
                       postaction={decorate}}}

\tikzset{circ/.style = {fill, circle, inner sep = 0, minimum size = 3}}
\tikzset{mstate/.style={circle, draw, blue, text=black, minimum width=0.7cm}}

\tikzset{commutative diagrams/.cd,cdmap/.style={/tikz/column 1/.append style={anchor=base east},/tikz/column 2/.append style={anchor=base west},row sep=tiny}}

\definecolor{mblue}{rgb}{0.2, 0.3, 0.8}
\definecolor{morange}{rgb}{1, 0.5, 0}
\definecolor{mgreen}{rgb}{0.1, 0.4, 0.2}
\definecolor{mred}{rgb}{0.5, 0, 0}

\def\drawcirculararc(#1,#2)(#3,#4)(#5,#6){%
    \pgfmathsetmacro\cA{(#1*#1+#2*#2-#3*#3-#4*#4)/2}%
    \pgfmathsetmacro\cB{(#1*#1+#2*#2-#5*#5-#6*#6)/2}%
    \pgfmathsetmacro\cy{(\cB*(#1-#3)-\cA*(#1-#5))/%
                        ((#2-#6)*(#1-#3)-(#2-#4)*(#1-#5))}%
    \pgfmathsetmacro\cx{(\cA-\cy*(#2-#4))/(#1-#3)}%
    \pgfmathsetmacro\cr{sqrt((#1-\cx)*(#1-\cx)+(#2-\cy)*(#2-\cy))}%
    \pgfmathsetmacro\cA{atan2(#2-\cy,#1-\cx)}%
    \pgfmathsetmacro\cB{atan2(#6-\cy,#5-\cx)}%
    \pgfmathparse{\cB<\cA}%
    \ifnum\pgfmathresult=1
        \pgfmathsetmacro\cB{\cB+360}%
    \fi
    \draw (#1,#2) arc (\cA:\cB:\cr);%
}
\newcommand\getCoord[3]{\newdimen{#1}\newdimen{#2}\pgfextractx{#1}{\pgfpointanchor{#3}{center}}\pgfextracty{#2}{\pgfpointanchor{#3}{center}}}

\def\Xint#1{\mathchoice
   {\XXint\displaystyle\textstyle{#1}}%
   {\XXint\textstyle\scriptstyle{#1}}%
   {\XXint\scriptstyle\scriptscriptstyle{#1}}%
   {\XXint\scriptscriptstyle\scriptscriptstyle{#1}}%
   \!\int}
\def\XXint#1#2#3{{\setbox0=\hbox{$#1{#2#3}{\int}$}
     \vcenter{\hbox{$#2#3$}}\kern-.5\wd0}}
\def\ddashint{\Xint=}
\def\dashint{\Xint-}

\newcommand\separator{{\centering\rule{2cm}{0.2pt}\vspace{2pt}\par}}

\newenvironment{own}{\color{gray!70!black}}{}

\newcommand\makecenter[1]{\raisebox{-0.5\height}{#1}}

\newtheorem*{soln}{Solution}

\renewcommand{\thesection}{}
\renewcommand{\thesubsection}{\arabic{section}.\arabic{subsection}}
\makeatletter
\def\@seccntformat#1{\csname #1ignore\expandafter\endcsname\csname the#1\endcsname\quad}
\let\sectionignore\@gobbletwo 
\let\latex@numberline\numberline
\def\numberline#1{\if\relax#1\relax\else\latex@numberline{#1}\fi}
\makeatother


\begin{document}
	
\maketitle

\section{QUESTION 1}


\begin{enumerate}
	\item \begin{align*}
	z/\log(1+z)& = z \left[  z - \frac{z^{2}}{2} + \frac{z^{3}}{3} - \cdots     \right]^{-1}  \\
	& = \left[  1 - \frac{z}{2} + \frac{z^{2}}{3} - \cdots     \right]^{-1} \\
	& = \left(  1 - \frac{z}{2} \right)^{-1} + O(z^{2}) \\
	& = 1 + \frac{z}{2} + O(z^{2}) 
	\end{align*}
	
	
	Owing to the $ \log(1+z) $ term, this series expansion converges if $ | z | < 1 $. 
	
	
	\item \begin{align*}
	(\cos z)^{1/2} -1 & = \left[   1 - \frac{z^{2}}{2} +  \frac{z^{4}}{4!} - \cdots  \right]^{1/2} - 1  \\
	& = \left(  1 - \left(  \frac{z^{2}}{2!} - \frac{z^{4}}{4!}  \right)  \right)^{1/2} - 1 + O(z^{6}) \\
	& = - \frac{1}{4} z + \frac{5}{96} z^{4} + O(z^{6}) 
	\end{align*}
	
	Converges for all $ z \in \C $.
	
	\item For $ | e^{z}  | < 1 \iff | e^{x} e^{iy} | < 1 \iff x < 0 $, we have, comparing coefficients of powers of $ z $, 
	
	\begin{align*} \log(1+e^{z})  & = e^{z} - \frac{e^{2z}}{2} + \frac{e^{3z}}{3} - \frac{e^{4z}}{4} + \cdots   \\
	& = \underbrace{\left(  1 - \frac{1}{2} + \frac{1}{3} - \cdots \right)}_{= \log 2} + \left( 1 - 1 + 1 - 1 + \cdots \right) z  \\
	& \quad + \frac{1}{2}\left( 1 - 2 + 3 - 4 + \cdots   \right) z^{2} + \frac{1}{3!} \left( 1^{2} - 2^{2} + 3^{2} - 4^{2} +  \cdots   \right) z^{3} +  O(z^{4}) 
	\end{align*}
	
	And if $ x > 0 $, have
	
		
	\begin{align*} \log(1+e^{z})  & = \log(e^{z}(1 + e^{-z})) = z + e^{-z} - \frac{e^{-2z}}{2} + \frac{e^{-3z}}{3} - \frac{e^{-4z}}{4} + \cdots   \\
	& = \underbrace{\left(  1 - \frac{1}{2} + \frac{1}{3} - \cdots \right)}_{= \log 2} + \left( 1 - 1 + 1 - 1 + \cdots \right) z  \\
	& \quad + \frac{1}{2}\left( 1 - 2 + 3 - 4 + \cdots   \right) z^{2} + \frac{1}{3!} \left( 1^{2} - 2^{2} + 3^{2} - 4^{2} +  \cdots   \right) z^{3} +  O(z^{4}) 
	\end{align*}
	
	
	
	
	\item \begin{align*}
	e^{e^{z}} & = 1 + e^{z} +  \frac{1}{2!}e^{2z}+ \frac{1}{3!}  e^{3z} + \cdots \\
	& = \left(  1 + 1 + \frac{1}{2!} + \frac{1}{3!} + \cdots \right)  + \left(  1 + 1 + \frac{1}{2!} + \frac{1}{3!} + \cdots \right) z \\
	& \quad + \frac{1}{2!} \left(  1 + 1 + \frac{1}{2!} + \frac{1}{3!} + \cdots \right)z^{2} + \cdots
	\end{align*}
	
	which seems to be valid for all $ z $.  
	
\end{enumerate}









\section{QUESTION 2}


Using partial fractions,

\[ \frac{1}{(z-a)(z-b)} = \frac{1}{a-b} \left(   \frac{1}{z-a} - \frac{1}{z - b} \right)  \]


Have $ 0 < | a | < | b | $. In the region $ | z | < | a | $, we have no singularities,  ie our function is analytic here, and we can calculate the  Taylor series about $ z_{0} = 0 $. Note that (for $ | z | < | a | $)

\[ \frac{1}{z - a} = - \frac{1}{a} \left(  1 - \frac{z}{a} \right)^{-1} = - \sum_{n=0}^{\infty} \frac{1}{a^{n+1}} z^{n}  \]


Hence 

\[ \frac{1}{(z-a)(z-b)} = -\frac{1}{a-b} \sum_{n=0}^{\infty} \left(  \frac{1}{a^{n+1}} - \frac{1}{b^{n+1}} \right)  z^{n}   \]




In the region $ | a | < | z | < | b | $ we can determine a Laurent series for $ \frac{1}{z-a} $ in this annulus, (but $ \frac{1}{z-b} $ still has a Taylor series). Note that 

\[
\frac{1}{z - a} = \frac{1}{z} \left(1 - \frac{a}{z}\right)^{-1} = \sum_{m = 0}^\infty \frac{a^m}{z^{m + 1}} = \sum_{n = -\infty}^{-1} a^{-n - 1} z^n.
\]


Hence

\[ \frac{1}{(z-a)(z-b)} = \frac{1}{a-b} \left(  \sum_{n = -\infty}^{-1} a^{-n - 1} z^n + \sum_{n=0}^{\infty} \frac{1}{b^{n+1}} z^{n}   \right)   \]



Finally, in the region $ | z | > | b | $, this is an annulus, that goes from $|b|$ to infinity. So it has a Laurent series, given by



\[ \frac{1}{(z-a)(z-b)} = \frac{1}{a-b} \left(  \sum_{n = -\infty}^{-1} (a^{-n - 1}  + b^{-n-1})z^n  \right)   \]




\section{QUESTION 3}


We note that $ \sin z = \frac{e^{iz} - e^{-iz}}{2i} $ has zeros at $ e^{2iz} = 1 $ ie. $ z = n \pi $ for integer $ n $. 

The annulus $ 0 < | z | < \pi $ contains no singularities, thus there exists a Laurent series for $ \sin z $ in this annulus. 

\begin{align*}
\cosec^{2} z& = \left[     \left( z - \frac{z^{3}}{3!} + \frac{z^{5}}{5!} - \cdots \right)\left( z - \frac{z^{3}}{3!} + \frac{z^{5}}{5!} - \cdots \right)  \right]^{-1}  \\
& = \left[   z^{2} - \frac{z^{4}}{3} + \frac{z^{6}}{60} - \cdots \right]^{-1} \\
& = z^{-2} \left[   1 - \frac{1}{3} z^{2} + \frac{1}{60} z^{4} - \cdots \right]^{-1}
\end{align*}

Not sure how to do the binomial expansion in a valid way.

Next part, if $ 0 < | z | < \pi $. 

\begin{align*}
g(z) & = f(z) - z^{2} - \frac{1}{\pi^{2}}  \left(  1 + \frac{z}{\pi} \right)^{-2} -  \frac{1}{\pi^{2}} \left(  1 - \frac{z}{\pi} \right)^{-2}  \\
& = 
\end{align*}

Can see $ z^{-2} $ term is removed by $ f(z) - z^{-2} $, hence we $ a_{n} = 0 $ for all $ n < 0 $, and can remove the singularity at $ z = 0 $ by setting 


\[ G(z) =  \begin{cases} g(z)  & \text{ if } z \neq 0 \\ \text{constant term in } g(z) & \text{ if } z = 0 \end{cases}  \]

Not sure why $ z = | \pi |  $ is fine?



\section{QUESTION 4}


$ f(z) $ has a zero of order $ N $ at $ z = z_{0} $ if $ 0 = f(z_{0}) = f'(z_{0}) = \cdots = f^{N-1}(z_{0}) $, but $ f^{(N)}(z_{0}) \neq 0 $. 

If there is a $ N > 0 $ such that $ a_n = 0 $ for all $ n < -N $ but $ a_{-N} \neq 0$,  then $ f $ has a pole of order $ N $ at $ z_{0} $.

Not sure.


Write 

\[ f(z) = (z - z_{0})^{N} G(z) \]

for some $ G $ with $ G(z_{0}) \neq 0 $. Then $ \frac{1}{G(z)} $ has a Taylor series about $ z_{0} $, and then the result follows. 


\section{QUESTION 5}

\begin{enumerate}
	\item $ \frac{1}{z^{3}(z-1)^{2}} $ has isolated singularities at $ z = 0 $ and $ z = 1 $.
	
	\item $ \tan z $ has isolated singularities at $ z = (n + \frac{1}{2}) \pi $.
	
	\item  $\sinh z$ has zeros where $\frac{1}{2}(e^z - e^{-z}) = 0$, i.e.\ $e^{2z} = 1$, i.e.\ $z = n \pi i$, where $n \in \Z$. Hence $ z \coth z $ has isolated singularities here. 
	
	\item $ \frac{e^{z} - e}{(1-z)^{3}} $ has a singularity at $ z = 1 $, 
	
	\item $ \exp(\tan z) $ has singularities $ z = \frac{\pi}{2} + n \pi $.
	
	\item $ \sinh \frac{z}{z^{2} - 1} $ has singularities at $ z = \pm 1 $
	
	\item $ \log(1+e^{z}) $ has singularities $ z = ( 1 + 2n )i \pi  $. 
	
	\item $ \tan(z^{-1}) $ has singularities $ 1/(  \frac{\pi}{2} + n \pi  ) = \frac{2}{\pi(2n+1)} $.
	
\end{enumerate}





\section{QUESTION 6}


Firstly $ \int_{-1}^{1} z \; \d z $ evaluated along $ \gamma_{1} $, the straight line from $ -1 $ to $ +1 $ is simply $ \left[  \frac{z^{2}}{2} \right]_{-1}^{1} = 0 $. 

We integrate along the semicircular contour by making the substitution $ z = e^{i \theta} $, $ \d z = i e^{i \theta} \; \ d \theta $. Then

\begin{align*}
\int_{\gamma_{2}} z \; \d z & = \int_{\pi}^{0} e^{i \theta} \cdot i e^{i \theta} \; \d \theta  \\
& = \int_{\pi}^{0} i e^{2i \theta} \; \d \theta \\
& = \left[ \frac{1}{2} e^{2 i \theta}  \right]_{\pi}^{0} \\
& = 0
\end{align*}

Next, consider 

\[ I_{3} = \oint_{\gamma_{3}} \bar{z} \; \d z, \quad I_{4} = \oint_{\gamma_{4}} \bar{z} \; \d z\]

where $ \gamma_{3} $ is the unit circle $ | z | = 1 $, and $ \gamma_{4} $ is the translated unit circle $ | z - 1 | = 1 $. For $ I_{3} $ we again make the substitution $ z = e^{i \theta} $, $ \d z = i e^{i \theta} \; \ d \theta $, so 

\begin{align*}
I_{3} & = \int_{0}^{2\pi} e^{-i \theta} i e^{i \theta} \; \d \theta \\
& = 2 \pi
\end{align*}

For $ I_{4} $ we make the substitution $ z = 1 + e^{i \theta} $, $ \d z = i e^{i \theta} \; \ d \theta $, so 

\begin{align*}
I_{4} & = \int_{0}^{2\pi} (1 + e^{-i \theta}) i e^{i \theta} \; \d \theta \\
& = \int_{0}^{2\pi} i( 1 + e^{i \theta}) \; \d \theta \\
& = 2 \pi i
\end{align*}



\section{QUESTION 7}

 At a \emph{simple} pole, the residue is given by
\[
\res_{z = z_0}f(z) = \lim_{z \to z_0} (z - z_0) f(z).
\]

Hence 

\begin{enumerate}
	\item 
	
	\begin{align*}
	\res_{z = z_0}f(z)/ (z- z_{0}) & = \lim_{z \to z_0} f(z) \\
	& = f(z_{0})
	\end{align*}
	
	\item 
	
	\begin{align*}
	\res_{z = z_0}f(z)/ g(z) & = \lim_{z \to z_0} (z - z_0) f(z)/ g(z) \\
	& = f(z_{0})
	\end{align*}
	
	
	\item 
	
	\begin{prop}
		At a pole of order $N$, the residue is given by
		\[
		\lim_{z \to z_0} \frac{1}{(N - 1)!} \frac{\d^{N - 1}}{\d z^{N - 1}} (z - z_0)^N f(z).
		\]
	\end{prop}

	\begin{proof}
		We can simply expand the right hand side to obtain
		\begin{align*}
		& \lim_{z \to z_0} \frac{1}{(N - 1)!} \frac{\d^{N - 1}}{\d z^{N - 1}} (z - z_0)^{N}   \left(  \frac{a_{-N}}{(z - z_0)^{N}} + \cdots +  \frac{a_{-2}}{z - z_0}^{2} +    \frac{a_{-1}}{z - z_0} + a_0 + a_1(z - z_0) + \cdots\right) \\
		& \lim_{z \to z_0} \frac{1}{(N - 1)!} \frac{\d^{N - 1}}{\d z^{N - 1}} \left(  a_{-N} + a_{-N+1} (z - z_{0}) + \cdots +  a_{-1} (z-z_{0})^{N-1}  + a_0(z - z_{0}  )^N + \cdots   \right) \\
		& \lim_{z \to z_0} \frac{1}{(N - 1)!} \left((N-1)! a_{-1} + N! a_0(z - z_{0}  ) + \cdots   \right) \\
		&= \lim_{z \to z_0} (a_{-1} + N a_0(z - z_0) +\cdots) \\
		&= a_{-1}
		\end{align*}
		as required.
	\end{proof}
	
	
\end{enumerate}

We now compute the residues of the poles given in question 5.

\begin{enumerate}
	\item We can use the fact that $f$ has a pole of order $3$ at $z = 0$. So we can use the formula to obtain
	\[
	\res_{z = 0} f(z) = \lim_{z \to 0} \frac{1}{2!} \frac{\d^2}{\d z^2} (z^3 f(z)) = \lim_{z \to 0} \frac{1}{2} \frac{\d^2}{\d z^2} \frac{1}{(z-1)^{2}} = \frac{1}{2}.
	\]
	
\end{enumerate}


\section{QUESTION 8}
\section{QUESTION 9}

 We shall evaluate
\[
I = \int_{\gamma} \frac{z^{n} \; \d z}{(z-a)(z-a^{-1})},
\]

where $ \gamma $ is the unit circle. Making the substitution $ z = e^{i \theta} $ and traversing anticlockwise from $ z = 1 $ gives

\begin{align*}
I & = \int_{0}^{2\pi}  \frac{e^{i n \theta}}{(e^{i \theta} - a)(e^{i \theta} - a^{-1})} i e^{i \theta} \; \d \theta  \\
& = \int_{0}^{2\pi} \frac{ i e^{i n \theta}}{e^{ i \theta} - a - a^{-1}  + e^{-i \theta} }  \; \d \theta \\
& = - ia \int_{0}^{2\pi} \frac{ \cos n \theta + i \sin n \theta}{1   - 2 a \cos \theta +  a^{2}    }  \; \d \theta \\
\end{align*}

Now evaluating $ I $ by the residue theorem, the poles of the integrand are at $z_0 = a $ and $z_1 = a^{-1} $, with $ z_{1} $ lying inside the contour with residue
\[
\frac{a^{-n+1}}{1-a^{2}}
\]

Hence we get

\[
I = 2\pi i \frac{a^{-n+1}}{1-a^{2}}
\]





\section{QUESTION 10}

 We shall evaluate
\[
I = \int_{-\infty}^{\infty} \frac{\d x}{1 + x + x^2},
\]

Consider
\[
\oint_\gamma \frac{\d z}{1 + z + z^2},
\]
where $\gamma$ is the contour ``closing in the upper-half plane'', shown: from $-R$ to $R$ along the real axis ($\gamma_0$), then returning to $-R$ via a semicircle of radius $R$ in the upper half plane ($\gamma_R$).
\begin{center}
	\begin{tikzpicture}
	\draw [->] (-3, 0) -- (3, 0);
	\draw [->] (0, -1) -- (0, 3);
	
	\draw [mred, thick, ->-=0.3, ->-=0.8] (-2, 0) -- (2, 0) node [pos=0.3, below] {$\gamma_0$} arc(0:180:2) node [pos=0.3, right] {$\gamma_R$};
	
	\node [below] at (-2, 0) {$-R$};
	\node [below] at (2, 0) {$R$};
	
	\node at (-1, 1) {$\times$};
	\node [right] at (-1, 1) {$z_{0}$};
	\end{tikzpicture}
\end{center}
Now we have
\[
\frac{1}{1 + z + z^2} = \frac{1}{(z - z_{0})(z - \bar{z_{0}})}.
\]

where $ z_{0} = - \frac{1}{2} + \frac{\sqrt{3}}{2} i  $. So the only singularity enclosed by $\gamma$ is a simple pole at $z = z_{0}$, where the residue is
\[
\lim_{z \to z_{0}} \frac{1}{z - \bar{z_{0}}} = -\frac{1}{\sqrt{3}i} .
\]
Hence
\[
\int_{\gamma_0} \frac{\d z}{1 + z + z^2} + \int_{\gamma_R} \frac{\d z}{1 +z +  z^2} = \int_\gamma \frac{\d z}{1 + z + z^2}= 2\pi i \cdot -\frac{1}{\sqrt{3}i} = - \frac{2 \sqrt{3}}{3} \pi.
\]
Let's now look at the terms individually. We know
\[
\int_{\gamma_0} \frac{\d z}{1 + z^2} = \int_{-R}^R \frac{\d x}{1 + x^2} \to I
\]
as $R \to \infty$. Also,
\[
\int_{\gamma_R} \frac{\d z}{1 + z^2} \to 0
\]
as $R \to \infty$ (see below). So we obtain in the limit
\[
I + 0 = - \frac{2 \sqrt{3}}{3} \pi.
\]
So
\[
I = - \frac{2 \sqrt{3}}{3} \pi.
\]
Finally, we need to show that the integral about $\gamma_R$ vanishes as $R \to \infty$. We can also do this informally, by writing
\[
\left|\int_{\gamma_R} \frac{\d z}{1 + z + z^2}\right| \leq \pi R \sup_{z \in \gamma_R} \left|\frac{1}{1 + z + z^2}\right| = \pi R \cdot O(R^{-2}) = O(R^{-1}) \to 0.
\]


\section{QUESTION 11}

we want to integrate
\[
I = \int_0^\infty \frac{x^{a-1}}{1 +x}\;\d x,
\]

with $0 < a < 1$ so that the integral converges. 
 We need a branch cut for $z^{a-1}$. We take our branch cut to be along the positive real axis, and define
\[
z^\alpha = r^\alpha e^{i\alpha \theta},
\]
where $z = re^{i\theta}$ and $0 \leq \theta < 2\pi$. We use the following keyhole contour:
\begin{center}
	\begin{tikzpicture}
	\draw [->] (-3, 0) -- (3, 0);
	\draw [->] (0, -3) -- (0, 3);
	\draw [thick,decorate, decoration=zigzag] (0, 0) -- (3, 0);
	
	\draw [mred, thick, ->-=0.1, ->-=0.45, ->-=0.75, ->-=0.84, ->-=0.95] (1.977, 0.3) arc(8.63:351.37:2) node [pos=0.15, anchor = south west] {$C_R$} -- (0.4, -0.3) arc(323.13:36.87:0.5) node [pos=0.7, left] {$C_\varepsilon$} -- cycle;
	
	\node [circ] at (0, 0) {};
	
	\node at (-1, -1) {$\times$};
	\node [right] at (-1, -1) {$e^{5\pi i/4}$};
	\node at (-1, 1) {$\times$};
	\node [right] at (-1, 1) {$e^{3\pi i/4}$};
	\end{tikzpicture}
\end{center}
This consists of a large circle $C_R$ of radius $R$, a small circle $C_\varepsilon$ of radius $\varepsilon$, and the two lines just above and below the branch cut. We will simultaneously take the limit $\varepsilon \to 0$ and $R \to \infty$.

We have four integrals to work out. The first is
\[
\int_{\gamma_R} \frac{z^{a-1}}{1 +z} \;\d z= O(R^{a - 2}) \cdot 2\pi R = O(R^{a - 1}) \to 0
\]
as $R \to \infty$. To obtain the contribution from $\gamma_\varepsilon$, we substitute $z = \varepsilon e^{i\theta}$, and obtain
\[
\int_{2\pi}^0 \frac{\varepsilon^{a-1} e^{i(a-1)\theta}}{1 + \varepsilon e^{i\theta}} i \varepsilon e^{i\theta}\;\d \theta = O(\varepsilon^{a - 1}) \to 0.
\]
Finally, we look at the integrals above and below the branch cut. The contribution from just above the branch cut is
\[
\int_\varepsilon^R \frac{x^{a-1}}{1 +x}\;\d x \to I.
\]
Similarly, the integral below is
\[
\int_R^\varepsilon \frac{x^{a-1}e^{2 a \pi i}}{1 +x}\;\d x \to -e^{2a  \pi i}I.
\]
So we get
\[
\oint_\gamma \frac{z^(a-1)}{1 + z}\;\d z \to (1 - e^{2a \pi i})I.
\]
All that remains is to compute the residues. The only pole is the simple pole at $z = -1$, with residue 


\[ (-1)^{a-1} \]

Hence we know
\[
(1 - e^{2a \pi i})I = 2\pi i\left( (-1)^{a-1} \right).
\]


In other words, we get
\[
e^{a \pi i} \frac{1}{2i} (e^{-a \pi i} - e^{a \pi i})I =  \pi (-1)^{a-1} e^{- \pi a i}
\]
Thus we have
\[
I = \frac{\pi}{\sin \pi a}
\]



\section{QUESTION 12}


\section{QUESTION 13}

Not too sure about these trigonometric integrals, but will attempt before supervision.


\section{QUESTION 14}


\section{QUESTION 15}

Consider the integral
\[
\int_\gamma \frac{\cot z}{z^2 + \pi^{2} a^{2}}\;\d z,
\]
where $\gamma$ is the square contour shown with corners at $(N + \frac{1}{2})(\pm 1 \pm i)$, where $N$ is a large integer, avoiding the singularities
\begin{center}
	\begin{tikzpicture}
	\draw [->] (-3, 0) -- (3, 0);
	\draw [->] (0, -3) -- (0, 3);
	
	\foreach \x in {-2.5,-2,...,2.5} {
		\node at (\x, 0) {$\times$};
	}
	
	\draw [mred, thick] (1.75, 1.75) rectangle (-1.75, -1.75);
	
	\node [anchor = south west] at (0, 1.75) {$(N + \frac{1}{2})i$};
	\node [anchor = north west] at (0, -1.75) {$-(N + \frac{1}{2})i$};
	\node [anchor = south west] at (1.75, 0) {$N + \frac{1}{2}$};
	\node [anchor = south east] at (-1.75, 0) {$-(N + \frac{1}{2})$};
	
	\node [circ] at (0, 1.75) {};
	\node [circ] at (0, -1.75) {};
	\node [circ] at (1.75, 0) {};
	\node [circ] at (-1.75, 0) {};
	\end{tikzpicture}
\end{center}
There are simple poles at $z = n \pi, \; n \in \Z\setminus \{0\}$, with residues $\frac{1}{(n^2 + a^{2}) \pi}$, and a two poles at $z = \pm i \pi a$ with residue $-\frac{1}{\pi a} \coth \pi a$ each. It turns out the integrals along the sides all vanish as $N \to \infty$ (see later). So we know
\[
2\pi i\left(2 \sum_{n = 1}^N \frac{1}{(n^2 + a^{2}) \pi} - \frac{2}{\pi a}  \coth \pi a \right) \to 0
\]
as $N \to \infty$. In other words,
\[
\sum_{n = 1}^N \frac{1}{n^2 + a^{2}} = \frac{\pi}{a} \coth \pi a.
\]

[Not sure of the details of calculating these residues...]
Hence all that remains is to show that the integrals along the sides vanish. On the right-hand side, we can write $z = N + \frac{1}{2} + iy$. Then
\[
|\cot z| = \left|\cot\left( \left(N + \frac{1}{2}\right) + i y\right)\right| = |-\tan i y| = |\tanh y| \leq 1.
\]
So $\cot \pi z$ is bounded on the vertical side. Since we are integrating $\frac{\cot \pi z}{z^2 + \pi^{2} a^{2}}$, the integral vanishes as $N \to \infty$.

Along the top, we get $z = x + \left(N + \frac{1}{2}\right)i$. This gives
\[
|\cot z| = \frac{\sqrt{\cosh^2 \left(N + \frac{1}{2}\right) - \sin^2 x}}{\sqrt{\sinh^2 \left(N + \frac{1}{2}\right) + \sin^2 x}} \leq \coth\left(N + \frac{1}{2}\right) \leq \coth \frac{1}{2}.
\]
So again $\cot \pi $ is bounded on the top side. So again, the integral vanishes as $N \to \infty$.

Similarly the left and bottom boundary both vanish too, hence the required result.



\end{document}