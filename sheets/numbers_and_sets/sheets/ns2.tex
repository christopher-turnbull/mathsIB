\documentclass[a4paper]{article}
\usepackage{amsmath}
\def\npart {IA}
\def\nterm {Michaelmas}
\def\nyear {2017}
\def\nlecturer {Dr Forster}
\def\ncourse {Numbers and Sets Example Sheet 3}

\input{header}

\newtheorem*{soln}{Solution}

\renewcommand{\thesection}{}
\renewcommand{\thesubsection}{\arabic{section}.\arabic{subsection}}
\makeatletter
\def\@seccntformat#1{\csname #1ignore\expandafter\endcsname\csname the#1\endcsname\quad}
\let\sectionignore\@gobbletwo
\let\latex@numberline\numberline
\def\numberline#1{\if\relax#1\relax\else\latex@numberline{#1}\fi}
\makeatother


\begin{document}
	
\maketitle

\section{QUESTION 1}

As $ (n+1) \equiv (n+4) $ (mod 3), and exactly one of $ n,n+1,n+2 $ is divisible by 3, it follows that exactly of $ n, n+2,n+4 $ is divisible by 3 also.

So $ 3,5,7 $ are three primes of this form, but this only occurs once.

\section{QUESTION 2}

Must have last digits 3,5,7,9.
Consider the block of numbers $ 10k  $ to $ 10k + 10 $.
We see that

\[ 10k \equiv 10 \quad (\text{mod } 30) \]

otherwise if $   10k \equiv 0 \quad (\text{mod} 30)$, then the number with last digit 3 in our block would be divisible by 3. We follow a similar strategy with primes greater than 3, and obtain

\[ k \equiv 1 \quad  (\text{mod } 3) \]
\[ k \equiv 1, 3, 5 \quad  (\text{mod } 7) \]
\[ k \equiv 2, 4,5,6,8,0 \quad  (\text{mod } 11) \]

Now solve simultaneously for solutions.

\section{QUESTION 3}

Each number in the sequence takes the form

\[ T_{n} = 41 + 2(n-1)  \]

and clearly $ T_{42} $ is divisible by 41.

\section{QUESTION 4}




\section{QUESTION 5}
\section{QUESTION 6}
\section{QUESTION 7}
\section{QUESTION 8}
\section{QUESTION 9}
\section{QUESTION 10}
\section{QUESTION 11}
\section{QUESTION 12}
\section{QUESTION 13}


\end{document}