\documentclass[a4paper]{article}
\usepackage{amsmath}
\def\npart {IA}
\def\nterm {Michaelmas}
\def\nyear {2017}
\def\nlecturer {Dr. Forster}
\def\ncourse {Numbers and Sets }

\usepackage{amsmath}
% Imports
\ifx \nauthor\undefined
  \def\nauthor{Christopher Turnbull}
\else
\fi

\author{Supervised by \nlecturer \\\small Solutions presented by \nauthor}
\date{\nterm\ \nyear}

\usepackage{alltt}
\usepackage{amsfonts}
\usepackage{amsmath}
\usepackage{amssymb}
\usepackage{amsthm}
\usepackage{booktabs}
\usepackage{caption}
\usepackage{enumitem}
\usepackage{fancyhdr}
\usepackage{graphicx}
\usepackage{mathdots}
\usepackage{mathtools}
\usepackage{microtype}
\usepackage{multirow}
\usepackage{pdflscape}
\usepackage{pgfplots}
\usepackage{siunitx}
\usepackage{slashed}
\usepackage{tabularx}
\usepackage{tikz}
\usepackage{tkz-euclide}
\usepackage[normalem]{ulem}
\usepackage[all]{xy}
\usepackage{imakeidx}

\makeindex[intoc, title=Index]
\indexsetup{othercode={\lhead{\emph{Index}}}}

\ifx \nextra \undefined
  \usepackage[pdftex,
    hidelinks,
    pdfauthor={Christopher Turnbull},
    pdfsubject={Cambridge Maths Notes: Part \npart\ - \ncourse},
    pdftitle={Part \npart\ - \ncourse},
  pdfkeywords={Cambridge Mathematics Maths Math \npart\ \nterm\ \nyear\ \ncourse}]{hyperref}
  \title{Part \npart\ --- \ncourse}
\else
  \usepackage[pdftex,
    hidelinks,
    pdfauthor={Christopher Turnbull},
    pdfsubject={Cambridge Maths Notes: Part \npart\ - \ncourse\ (\nextra)},
    pdftitle={Part \npart\ - \ncourse\ (\nextra)},
  pdfkeywords={Cambridge Mathematics Maths Math \npart\ \nterm\ \nyear\ \ncourse\ \nextra}]{hyperref}

  \title{Part \npart\ --- \ncourse \\ {\Large \nextra}}
  \renewcommand\printindex{}
\fi

\pgfplotsset{compat=1.12}

\pagestyle{fancyplain}
\lhead{\emph{\nouppercase{\leftmark}}}
\ifx \nextra \undefined
  \rhead{
    \ifnum\thepage=1
    \else
      \npart\ \ncourse
    \fi}
\else
  \rhead{
    \ifnum\thepage=1
    \else
      \npart\ \ncourse\ (\nextra)
    \fi}
\fi
\usetikzlibrary{arrows.meta}
\usetikzlibrary{decorations.markings}
\usetikzlibrary{decorations.pathmorphing}
\usetikzlibrary{positioning}
\usetikzlibrary{fadings}
\usetikzlibrary{intersections}
\usetikzlibrary{cd}

\newcommand*{\Cdot}{{\raisebox{-0.25ex}{\scalebox{1.5}{$\cdot$}}}}
\newcommand {\pd}[2][ ]{
  \ifx #1 { }
    \frac{\partial}{\partial #2}
  \else
    \frac{\partial^{#1}}{\partial #2^{#1}}
  \fi
}
\ifx \nhtml \undefined
\else
  \renewcommand\printindex{}
  \makeatletter
  \DisableLigatures[f]{family = *}
  \let\Contentsline\contentsline
  \renewcommand\contentsline[3]{\Contentsline{#1}{#2}{}}
  \renewcommand{\@dotsep}{10000}
  \newlength\currentparindent
  \setlength\currentparindent\parindent

  \newcommand\@minipagerestore{\setlength{\parindent}{\currentparindent}}
  \usepackage[active,tightpage,pdftex]{preview}
  \renewcommand{\PreviewBorder}{0.1cm}

  \newenvironment{stretchpage}%
  {\begin{preview}\begin{minipage}{\hsize}}%
    {\end{minipage}\end{preview}}
  \AtBeginDocument{\begin{stretchpage}}
  \AtEndDocument{\end{stretchpage}}

  \newcommand{\@@newpage}{\end{stretchpage}\begin{stretchpage}}

  \let\@real@section\section
  \renewcommand{\section}{\@@newpage\@real@section}
  \let\@real@subsection\subsection
  \renewcommand{\subsection}{\@@newpage\@real@subsection}
  \makeatother
\fi

% Theorems
\theoremstyle{definition}
\newtheorem*{aim}{Aim}
\newtheorem*{axiom}{Axiom}
\newtheorem*{claim}{Claim}
\newtheorem*{cor}{Corollary}
\newtheorem*{conjecture}{Conjecture}
\newtheorem*{defi}{Definition}
\newtheorem*{eg}{Example}
\newtheorem*{ex}{Exercise}
\newtheorem*{fact}{Fact}
\newtheorem*{law}{Law}
\newtheorem*{lemma}{Lemma}
\newtheorem*{notation}{Notation}
\newtheorem*{prop}{Proposition}
\newtheorem*{soln}{Solution}
\newtheorem*{thm}{Theorem}

\newtheorem*{remark}{Remark}
\newtheorem*{warning}{Warning}
\newtheorem*{exercise}{Exercise}

\newtheorem{nthm}{Theorem}[section]
\newtheorem{nlemma}[nthm]{Lemma}
\newtheorem{nprop}[nthm]{Proposition}
\newtheorem{ncor}[nthm]{Corollary}


\renewcommand{\labelitemi}{--}
\renewcommand{\labelitemii}{$\circ$}
\renewcommand{\labelenumi}{(\roman{*})}

\let\stdsection\section
\renewcommand\section{\newpage\stdsection}

% Strike through
\def\st{\bgroup \ULdepth=-.55ex \ULset}

% Maths symbols
\newcommand{\abs}[1]{\left\lvert #1\right\rvert}
\newcommand\ad{\mathrm{ad}}
\newcommand\AND{\mathsf{AND}}
\newcommand\Art{\mathrm{Art}}
\newcommand{\Bilin}{\mathrm{Bilin}}
\newcommand{\bket}[1]{\left\lvert #1\right\rangle}
\newcommand{\B}{\mathcal{B}}
\newcommand{\bolds}[1]{{\bfseries #1}}
\newcommand{\brak}[1]{\left\langle #1 \right\rvert}
\newcommand{\braket}[2]{\left\langle #1\middle\vert #2 \right\rangle}
\newcommand{\bra}{\langle}
\newcommand{\cat}[1]{\mathsf{#1}}
\newcommand{\C}{\mathbb{C}}
\newcommand{\CP}{\mathbb{CP}}
\newcommand{\cU}{\mathcal{U}}
\newcommand{\Der}{\mathrm{Der}}
\newcommand{\D}{\mathrm{D}}
\newcommand{\dR}{\mathrm{dR}}
\newcommand{\E}{\mathbb{E}}
\newcommand{\F}{\mathbb{F}}
\newcommand{\Frob}{\mathrm{Frob}}
\newcommand{\GG}{\mathbb{G}}
\newcommand{\gl}{\mathfrak{gl}}
\newcommand{\GL}{\mathrm{GL}}
\newcommand{\G}{\mathcal{G}}
\newcommand{\Gr}{\mathrm{Gr}}
\newcommand{\haut}{\mathrm{ht}}
\newcommand{\Id}{\mathrm{Id}}
\newcommand{\ket}{\rangle}
\newcommand{\lie}[1]{\mathfrak{#1}}
\newcommand{\Mat}{\mathrm{Mat}}
\newcommand{\N}{\mathbb{N}}
\newcommand{\norm}[1]{\left\lVert #1\right\rVert}
\newcommand{\normalorder}[1]{\mathop{:}\nolimits\!#1\!\mathop{:}\nolimits}
\newcommand\NOT{\mathsf{NOT}}
\newcommand{\Oc}{\mathcal{O}}
\newcommand{\Or}{\mathrm{O}}
\newcommand\OR{\mathsf{OR}}
\newcommand{\ort}{\mathfrak{o}}
\newcommand{\PGL}{\mathrm{PGL}}
\newcommand{\ph}{\,\cdot\,}
\newcommand{\pr}{\mathrm{pr}}
\newcommand{\Prob}{\mathbb{P}}
\newcommand{\PSL}{\mathrm{PSL}}
\newcommand{\Ps}{\mathcal{P}}
\newcommand{\PSU}{\mathrm{PSU}}
\newcommand{\pt}{\mathrm{pt}}
\newcommand{\qeq}{\mathrel{``{=}"}}
\newcommand{\Q}{\mathbb{Q}}
\newcommand{\R}{\mathbb{R}}
\newcommand{\RP}{\mathbb{RP}}
\newcommand{\Rs}{\mathcal{R}}
\newcommand{\SL}{\mathrm{SL}}
\newcommand{\so}{\mathfrak{so}}
\newcommand{\SO}{\mathrm{SO}}
\newcommand{\Spin}{\mathrm{Spin}}
\newcommand{\Sp}{\mathrm{Sp}}
\newcommand{\su}{\mathfrak{su}}
\newcommand{\SU}{\mathrm{SU}}
\newcommand{\term}[1]{\emph{#1}\index{#1}}
\newcommand{\T}{\mathbb{T}}
\newcommand{\tv}[1]{|#1|}
\newcommand{\U}{\mathrm{U}}
\newcommand{\uu}{\mathfrak{u}}
\newcommand{\Vect}{\mathrm{Vect}}
\newcommand{\wsto}{\stackrel{\mathrm{w}^*}{\to}}
\newcommand{\wt}{\mathrm{wt}}
\newcommand{\wto}{\stackrel{\mathrm{w}}{\to}}
\newcommand{\Z}{\mathbb{Z}}
\renewcommand{\d}{\mathrm{d}}
\renewcommand{\H}{\mathbb{H}}
\renewcommand{\P}{\mathbb{P}}
\renewcommand{\sl}{\mathfrak{sl}}
\renewcommand{\vec}[1]{\boldsymbol{\mathbf{#1}}}
%\renewcommand{\F}{\mathcal{F}}

\let\Im\relax
\let\Re\relax

\DeclareMathOperator{\adj}{adj}
\DeclareMathOperator{\Ann}{Ann}
\DeclareMathOperator{\area}{area}
\DeclareMathOperator{\Aut}{Aut}
\DeclareMathOperator{\Bernoulli}{Bernoulli}
\DeclareMathOperator{\betaD}{beta}
\DeclareMathOperator{\bias}{bias}
\DeclareMathOperator{\binomial}{binomial}
\DeclareMathOperator{\card}{card}
\DeclareMathOperator{\ccl}{ccl}
\DeclareMathOperator{\Char}{char}
\DeclareMathOperator{\ch}{ch}
\DeclareMathOperator{\cl}{cl}
\DeclareMathOperator{\cls}{\overline{\mathrm{span}}}
\DeclareMathOperator{\conv}{conv}
\DeclareMathOperator{\corr}{corr}
\DeclareMathOperator{\cosec}{cosec}
\DeclareMathOperator{\cosech}{cosech}
\DeclareMathOperator{\cov}{cov}
\DeclareMathOperator{\covol}{covol}
\DeclareMathOperator{\diag}{diag}
\DeclareMathOperator{\diam}{diam}
\DeclareMathOperator{\Diff}{Diff}
\DeclareMathOperator{\disc}{disc}
\DeclareMathOperator{\dom}{dom}
\DeclareMathOperator{\End}{End}
\DeclareMathOperator{\energy}{energy}
\DeclareMathOperator{\erfc}{erfc}
\DeclareMathOperator{\erf}{erf}
\DeclareMathOperator*{\esssup}{ess\,sup}
\DeclareMathOperator{\ev}{ev}
\DeclareMathOperator{\Ext}{Ext}
\DeclareMathOperator{\Fit}{Fit}
\DeclareMathOperator{\fix}{fix}
\DeclareMathOperator{\Frac}{Frac}
\DeclareMathOperator{\Gal}{Gal}
\DeclareMathOperator{\gammaD}{gamma}
\DeclareMathOperator{\gr}{gr}
\DeclareMathOperator{\hcf}{hcf}
\DeclareMathOperator{\Hom}{Hom}
\DeclareMathOperator{\id}{id}
\DeclareMathOperator{\image}{image}
\DeclareMathOperator{\im}{im}
\DeclareMathOperator{\Im}{Im}
\DeclareMathOperator{\Ind}{Ind}
\DeclareMathOperator{\Int}{Int}
\DeclareMathOperator{\Isom}{Isom}
\DeclareMathOperator{\lcm}{lcm}
\DeclareMathOperator{\length}{length}
\DeclareMathOperator{\Lie}{Lie}
\DeclareMathOperator{\like}{like}
\DeclareMathOperator{\Lk}{Lk}
\DeclareMathOperator{\mse}{mse}
\DeclareMathOperator{\multinomial}{multinomial}
\DeclareMathOperator{\orb}{orb}
\DeclareMathOperator{\ord}{ord}
\DeclareMathOperator{\otp}{otp}
\DeclareMathOperator{\Poisson}{Poisson}
\DeclareMathOperator{\poly}{poly}
\DeclareMathOperator{\rank}{rank}
\DeclareMathOperator{\rel}{rel}
\DeclareMathOperator{\Re}{Re}
\DeclareMathOperator*{\res}{res}
\DeclareMathOperator{\Res}{Res}
\DeclareMathOperator{\rk}{rk}
\DeclareMathOperator{\Root}{Root}
\DeclareMathOperator{\sech}{sech}
\DeclareMathOperator{\sgn}{sgn}
\DeclareMathOperator{\spn}{span}
\DeclareMathOperator{\stab}{stab}
\DeclareMathOperator{\St}{St}
\DeclareMathOperator{\supp}{supp}
\DeclareMathOperator{\Syl}{Syl}
\DeclareMathOperator{\Sym}{Sym}
\DeclareMathOperator{\tr}{tr}
\DeclareMathOperator{\Tr}{Tr}
\DeclareMathOperator{\var}{var}
\DeclareMathOperator{\vol}{vol}

\pgfarrowsdeclarecombine{twolatex'}{twolatex'}{latex'}{latex'}{latex'}{latex'}
\tikzset{->/.style = {decoration={markings,
                                  mark=at position 1 with {\arrow[scale=2]{latex'}}},
                      postaction={decorate}}}
\tikzset{<-/.style = {decoration={markings,
                                  mark=at position 0 with {\arrowreversed[scale=2]{latex'}}},
                      postaction={decorate}}}
\tikzset{<->/.style = {decoration={markings,
                                   mark=at position 0 with {\arrowreversed[scale=2]{latex'}},
                                   mark=at position 1 with {\arrow[scale=2]{latex'}}},
                       postaction={decorate}}}
\tikzset{->-/.style = {decoration={markings,
                                   mark=at position #1 with {\arrow[scale=2]{latex'}}},
                       postaction={decorate}}}
\tikzset{-<-/.style = {decoration={markings,
                                   mark=at position #1 with {\arrowreversed[scale=2]{latex'}}},
                       postaction={decorate}}}
\tikzset{->>/.style = {decoration={markings,
                                  mark=at position 1 with {\arrow[scale=2]{latex'}}},
                      postaction={decorate}}}
\tikzset{<<-/.style = {decoration={markings,
                                  mark=at position 0 with {\arrowreversed[scale=2]{twolatex'}}},
                      postaction={decorate}}}
\tikzset{<<->>/.style = {decoration={markings,
                                   mark=at position 0 with {\arrowreversed[scale=2]{twolatex'}},
                                   mark=at position 1 with {\arrow[scale=2]{twolatex'}}},
                       postaction={decorate}}}
\tikzset{->>-/.style = {decoration={markings,
                                   mark=at position #1 with {\arrow[scale=2]{twolatex'}}},
                       postaction={decorate}}}
\tikzset{-<<-/.style = {decoration={markings,
                                   mark=at position #1 with {\arrowreversed[scale=2]{twolatex'}}},
                       postaction={decorate}}}

\tikzset{circ/.style = {fill, circle, inner sep = 0, minimum size = 3}}
\tikzset{mstate/.style={circle, draw, blue, text=black, minimum width=0.7cm}}

\tikzset{commutative diagrams/.cd,cdmap/.style={/tikz/column 1/.append style={anchor=base east},/tikz/column 2/.append style={anchor=base west},row sep=tiny}}

\definecolor{mblue}{rgb}{0.2, 0.3, 0.8}
\definecolor{morange}{rgb}{1, 0.5, 0}
\definecolor{mgreen}{rgb}{0.1, 0.4, 0.2}
\definecolor{mred}{rgb}{0.5, 0, 0}

\def\drawcirculararc(#1,#2)(#3,#4)(#5,#6){%
    \pgfmathsetmacro\cA{(#1*#1+#2*#2-#3*#3-#4*#4)/2}%
    \pgfmathsetmacro\cB{(#1*#1+#2*#2-#5*#5-#6*#6)/2}%
    \pgfmathsetmacro\cy{(\cB*(#1-#3)-\cA*(#1-#5))/%
                        ((#2-#6)*(#1-#3)-(#2-#4)*(#1-#5))}%
    \pgfmathsetmacro\cx{(\cA-\cy*(#2-#4))/(#1-#3)}%
    \pgfmathsetmacro\cr{sqrt((#1-\cx)*(#1-\cx)+(#2-\cy)*(#2-\cy))}%
    \pgfmathsetmacro\cA{atan2(#2-\cy,#1-\cx)}%
    \pgfmathsetmacro\cB{atan2(#6-\cy,#5-\cx)}%
    \pgfmathparse{\cB<\cA}%
    \ifnum\pgfmathresult=1
        \pgfmathsetmacro\cB{\cB+360}%
    \fi
    \draw (#1,#2) arc (\cA:\cB:\cr);%
}
\newcommand\getCoord[3]{\newdimen{#1}\newdimen{#2}\pgfextractx{#1}{\pgfpointanchor{#3}{center}}\pgfextracty{#2}{\pgfpointanchor{#3}{center}}}

\def\Xint#1{\mathchoice
   {\XXint\displaystyle\textstyle{#1}}%
   {\XXint\textstyle\scriptstyle{#1}}%
   {\XXint\scriptstyle\scriptscriptstyle{#1}}%
   {\XXint\scriptscriptstyle\scriptscriptstyle{#1}}%
   \!\int}
\def\XXint#1#2#3{{\setbox0=\hbox{$#1{#2#3}{\int}$}
     \vcenter{\hbox{$#2#3$}}\kern-.5\wd0}}
\def\ddashint{\Xint=}
\def\dashint{\Xint-}

\newcommand\separator{{\centering\rule{2cm}{0.2pt}\vspace{2pt}\par}}

\newenvironment{own}{\color{gray!70!black}}{}

\newcommand\makecenter[1]{\raisebox{-0.5\height}{#1}}
\def\checkmark{\tikz\fill[scale=0.4](0,.35) -- (.25,0) -- (1,.7) -- (.25,.15) -- cycle;}

\newcommand{\dd}{\mathcal{D}}
\begin{document}
	
\maketitle


\subsection*{1D 2012}

\begin{enumerate}
	\item By Euclid,
	
	\begin{align*}
	23 & = 18 + 5 \\
	18 & = 3 \times 5 + 5\\
	5 & = 3 + 2\\
	3 & = 2 + 1 
	\end{align*}
	
	So gcd(23,18) $ = 1 $.
	
	Now expressing 1 as a linear combination of 23 and 18,
	
	\begin{align*}
	1 & = 3 - 2 \\
	& = 3 - (5 - 3) \\
	& = 2 \times 3 - 5 \\
	& = 2(18 - 5 \times 3) - 5\\
	& = 2 \times 18 - 7 \times 5 \\
	& = 2 \times 18 - 7 \times (23 - 18) \\
	& = 9 \times 18 - 7 \times 23
	\end{align*}
	
Hence multiplying by 101,

\[ 909 \times 18 - 707 \times 23 = 101 \]

we see that

\[ x = 909, y = - 707 \]

\item We know 

\begin{align*}
9 \times 18 & = 1 \quad (\text{mod } 23) \\
-7 \times 23 & = 1 \quad (\text{mod } 18)
\end{align*}

So put 

\[ x = (9 \times 18 \times 2) - (7 \times 23 \times 3) \]

ie. \[ x = 106 \]
\end{enumerate}


\subsection*{2D 2012}

A relation $ a R b $ on elements of a set $ a,b \in X $ is an \emph{equivalence relation} if it is

\begin{itemize}
	\item Reflexive: $ a R a $ $ \forall \; a \in X $
	\item Symmetric: $ a R b \; \iff \; b R a \; \forall \; a,b \in X $
	\item Transitive: $ a R b $ and $ b R c \; \Rightarrow \; a R c \; \forall \; a,b,c \in X$
\end{itemize}

If $ \sim  $ is an equivalence relation on $ X $, then the equivalence classes of $ \sim $ form a partition of $ X $

\begin{proof}
	By reflexivity, $ x \in [x] \; \forall \; x \in X$.
	
	Now suppose $ [x] \cap [y] \neq \emptyset $. Let $ z \in [x] \cap [y] $.
	Then $ x \sim z $ and $ y \sim z $. By symmetry, $ z \sim y $. By transitivity, $ x \sim y $.
	
	For all $ x' \in X$, we have $ x' \sim x $, thus by transitivity, $ x' \sim y $, and $ [x] \subseteq [y] $. Similarly, $ [y] \subseteq [x] $, and $ [x] = [y] $.
\end{proof}

\begin{enumerate}
	\item $ V $ is an equivalence relation:
	$ x R x, x S x \; \forall \; x \in X  $ hence $ x V x \; \forall \; x \in X $. Similarly, symmetry and transitivity follow exactly.
	
	\item $ W $ is not necessarily an equivalence relation: take $ X = \{  1,2,3 \} $, let $ R $ act such that $ 1 R 2 $, with 3 in its own equivalence class, and let $ S $ act such that $ 2 S 3 $ with 1 in it's own class.
	
	By the definition of $ W $, $ 1 W 2 $ and $ 2 W 3 $, but 1 is not related to 3, so transitivity fails. 
\end{enumerate}

\subsection*{5D 2012}

\begin{enumerate}
	\item Not true if $ X $ is infinite. Let $ X = \N$, $ g(x) = x + 1 $, 
	
	\[ f(x) = \begin{cases} 1  & \text{ if } x = 1 \\ x - 1 & \text{ otherwise }  \end{cases} \]
	
	Then $ f \circ g $ is the identity map, but $ g(f(1)) = 2 $.
	
	If $ X $ is finite and $ f \circ g $ is the identity, 
	
	\begin{itemize}
		\item $ f \circ g  $ is injective
		\item $ f \circ g  $ injective $ \Rightarrow \; g $ injective 
		\item $ X $ finite, $ g $ injective $ \Rightarrow \; g $ bijective
		\item $ \Rightarrow \; $ f bijective, $ f = g^{-1} $,
		\item Hence $ f \circ g $ identity $ \Rightarrow \; g \circ f $ identity
	\end{itemize}

\item Can be false: Let $ X \subseteq \N $, take $ g(x) $ to be the constant function $ g(x) = 1 $, and take $ f(x) = x^{2} $.

(``$ g $ destroys a lot of information, $ f $ has a lot of leeway'')

\item Take $ f(x) = 1 $ for all $ x \in X $. It doesn't matter what $ g $ does now, but certainly need not be the identity, for any set $ X $. 

	 
\end{enumerate}


\begin{itemize}
	\item If $ X $ is a finite set, for each $ x_{i} \in X $ there exists a positive integer $ n_{i} $ such that $ f^{n_{i}}(x_{i}) = x_{i} $. Now simply take lcm($ x_{1},\cdots,x_{N} $), thus $ f^{N}(x) = x $ for all $ x \in X $
	\item If $ X $ is a countably infinite set, biject it with $ \N $ and take the function that maps
	
	\[ f(1) = 2, f(2) = 1 \]
	\[ f(3) = 4, f(4) = 5, f(5) = 3 \]
	
	and so on. Respectively we have $ n = 2,3,\cdots $, and there is no positive integer $ N $ such that $ f^{N}(x)  = x $ for all $ x \in X $
	
	\item If $ X $ is an uncountably infinite set, eg. $ \R $, simply set the function to be equal to the identity map on the points in $ \R \setminus \N $, and equal to our previous function for points in $ \N $.
	
	(we can always biject eg. $ \R^{2} to \R $)
\end{itemize}

\subsection*{6D 2012}

\begin{thm} Fermat's (Little) Theorem.
	Let $ p $ be a prime. Then $ a^{p} \equiv a \text{ (mod p)} $, for all $ a \in \Z $.
\end{thm}

\begin{thm} Wilsons Theorem.
	Let $ p $ be a prime. Then $ (p-1)! \equiv -1 \text{ (mod p)} $
\end{thm}

\begin{prop} $ x^{2} \equiv -1 \text{ (mod } p) $ has a solution iff $ p \equiv 1 $ ( mod 4)
\end{prop}

\begin{proof}
	By Wilson's,
	
	\begin{align*}
	-1 \equiv (p-1)! & \equiv (1)(2) \cdots  \left( \frac{p-1}{2} \right) \left( - \frac{p-1}{2} \right) \cdots (-2)(-1) \\
	& = (-1)^{\frac{p-1}{2}} \left( \frac{p-1}{2} \right)!^{2} 
	\end{align*}


If $ p \equiv 1 \text{ mod 4} $, write $ p = 4k + 1 $, and the RHS becomes $ (2k!)^{2} $. But for $ p \equiv -1 \text{ (mod 4)}$, ie. $ p = 4k + 3 $, suppose we have some $ x $ st. $ x^{2} \equiv -1 \text{ (mod p)} $.

Then, by Fermat's, $ x^{p} \equiv x \; \Rightarrow x^{p-1} \equiv 1 $, and

\begin{align*}
1 & \equiv x^{4k + 2}  \\
& = (x^{2})^{2k + 1} \\
& \equiv (-1)^{2k + 1}\\
& = -1 
\end{align*} 
Contradiction. 
\end{proof}

$ x $ has order $ d $ (mod $ p $), so $ d $ is the least positive integer st. $ x^{d} \equiv 1 $ (mod $ p $).
Suppose $ x^{k} \equiv 1 $ (mod $ p $). Then $ k > d $, so write $ k = qd + r $ for $ q > 0 $, with remainder $ r \in \{ 0,\cdots,d-1 \} $.

Then

\begin{align*}
1 = x^{k} & = x^{qd + r} \\
& = (x^{d})^{q} x^{r} \\
& \equiv x^{r} \text{ (mod p)}
\end{align*}

ie $ x^{r} \equiv 1 $Since $ r \in \{  0,\cdots,d-1\} $, and $ d $ is the least positive integer st. $ x^{d} \equiv 1 $, we must have $ r = 0 $, and hence $ d $ divides $ k $.

Now, suppose $ p  $ is a prime factor of $ F_{n} = 2^{2^{n}} + 1 $. We want to determine the order of $ 2 $ (mod $ p $), ie. the least positive integer $ d $ st. $ 2^{d} \equiv 1 $ ( mod $ p $). Now, as $ p $ is a factor of $ F_{n} $ we have

\[ 2^{2^{n}} + 1 \equiv 0 \; (\text{mod }p) \]
\[ 2^{2^{n}} \equiv -1 \; (\text{mod } p) \]

and by squaring both sides we have
\[ 2^{2^{n+1}} \equiv 1 \; (\text{mod } p) \]

thus the order of 2 (mod $ p $) divides $ 2^{n+1} $.

I think the order is $ 2^{n+1} $ but don't know how to justify this.
$ F_{n} $ and $ F_{m} $ are pairwise relatively coprime iff their gcd is 1. 

Next, if $ p $ is of the form $ 4k + 3 $ and is a factor of some $ F_{n} $, we have $ \left( 2^{2^{n-1}} \right)^{2} \equiv -1 $, but in the first part of the question we showed that $ x^{2} $ only has a solution when $ p $ is of the form $ 4k + 1 $.


\subsection*{7D 2012}
	\begin{enumerate}
		\item 
			CLAIM: $ \sqrt{6} $ is irrational
		
			Assume otherwise,
			
			\[ \sqrt{6} = \frac{p}{q}, \quad (p,q) = 1 \]
			
			Then $ 6q^{2} = p^{2}  \Rightarrow 2 | p^{2} $. 
			
			CLAIM: $ p^{2} $ even $ \Rightarrow \; p $ even.
			
		\begin{proof}
				Proof by contrapositive, if $ p $ not even, write $ p = 2k + 1 $ for some integer $ k $. Then
			
			\begin{align*}
			p^2 & = 4k^2 + 4k + 1 \\
			& = 2 (2k^2 + 2) + 1
			\end{align*}
			Thus $ p^{2} $ is not even. Hence $ p^{2} $ even $ \Rightarrow \; p $ even.
		\end{proof}
			Thus $ p $ even, and we can write $ p = 2p' $ and $ \sqrt{6} = \frac{2p'}{q} \Rightarrow 2 | q $ also, which contradicts $ (p,q) = 1 $.

	
		Now to show $ \sqrt{2} + \sqrt{3} $ is irrational, all we need to do is assume it is rational and we get the following contradiction: then so is $ (\sqrt{2} + \sqrt{3})^{2} = 5 + 2 \sqrt{6} $. But this is absurd since we have just showed $ \sqrt{6} $ is irrational
	
		
	
	

	\item 
	
	\[ e : = 1 + \frac{1}{1!} + \frac{1}{2!} + \frac{1}{3!} + \cdots \]
	
		Suppose is rational, $e = \frac{p}{q}$ for some integers $ p,q $ s.t. $ (p,q) = 1 $  Then $q!e \in \N$. But
	\[
	q!e = \underbrace{q! + q! + \frac{q!}{2!} + \frac{q!}{3!} + \cdots + \frac{q!}{q!}}_{n} + \underbrace{\frac{q!}{(q + 1)!} + \frac{q!}{(q + 2)!} + \cdots}_{x},
	\]
	where $n \in \N$ and 
	\[
	x = \frac{1}{q + 1} + \frac{1}{(q + 1)(q + 2)} + \cdots.
	\]
	We can bound it by
	\[
	0 < x < \frac{1}{q+1} +\frac{1}{(q + 1)^2} + \frac{1}{(q + 1)^3} + \cdots = \frac{1}{q + 1}\cdot \frac{1}{1 - 1/(q + 1)} = \frac{1}{q}.
	\]
	
	Now clearly $ e > 2 $, and
	
	\begin{align*}
	\frac{1}{2!}  + \frac{1}{3!} + \cdots & < \frac{1}{2} + \frac{1}{2^{2}} + \frac{1}{2^{3}} + \cdots \\
	& < \frac{1/2}{1 - 1/2} = 1 
	\end{align*}
	
	Thus $ 2 < e < 3 $, so $ e  = \frac{p}{q} $ is not an integer $ \Rightarrow q > 1 $. Hence
	
	\[ x < 5\frac{1}{q} < 1 \]
	
	Thus $q!e$ is the sum of an integer part $n$ plus a non-integer part $x$.
	Contradiction.

	
	\item Suppose the real root $ x = \frac{p}{q} $, $ (p,q) = 1 $ Then
	
	\begin{align*}
	\frac{p^{3}}{q^{3}} + 4 \frac{p}{q} - 7 = 0 & \Rightarrow \frac{p^{3}}{q} + 4pq - 7q^{2}  \\
	& \Rightarrow \frac{p^{3}}{q} = 7 q^{2} - 4pq 
	\end{align*}
	
	Then RHS is an integer $ \Rightarrow $ LHS is an integer also, which contradicts the fact that $ (p,q) = 1 $.
	
	
	\item 
		Let $ \log_{2} 3 = \frac{p}{q} $, $ (p,q) = 1 $ Now it must hold that 
	\[ 2^{q} = 3^{p} \]
	
	which is nonsense as LHS is even, while the RHS is odd.
	Note this could be true if either $ p $ or $ q $ are equal to zero, but it is clear that this is not the case here.	
	\end{enumerate}


\subsection*{8D 2012}

\begin{prop} 
	There is no injection from the power-set of $ \R  $ to $ \R $
\end{prop}
	\begin{proof}
	
	
		\begin{itemize}
			\item Suppose for the sake of contradiction that $ f: \mathcal{P} (\R) \to \R $ is an injection.
			\item For each $ t \in \Im(f) $, there exists a unique $ s \in \mathcal{P} ( \R ) $ st. $ f(s) = t $. Define $ g $ as 
			
			\[ g(t) = \begin{cases}  s & \text{ if } (t \in \Im f) \text{ and } (f(s) = t)\\ s_{0} & \text{ if } t \notin \Im f \end{cases} \]
			
			where $ s_{0} $ is any element of $ S $.
			
			By construction, given any $ s \in S \; \exists f(s) \in \R $ that maps to $ s $ under $ g $, so $ g : \R \to \mathcal{P}(R) $ is a surjection.
			
			\item Let $ S = \{  r \in \R : r \notin f(r) \}$. Since $ g $ is surjective, there must exist $ r \in \R $ st. $ g(r) = S $. If $ r \in \R $, then $ r \notin \R $ by the definition of $ S $. Conversely if $ r \notin S $, then $ r \in S $.
			
			\item This is absurd, and we arrive at the conclusion that $ f $ cannot be an injection.
		
	
		\end{itemize}
	\end{proof}


\begin{proof}
	Suppose such an injection exists, $ f: R \to \mathcal{P}(\R)  $. Take 
	
	\[ S = \{ r \in \R \; : \; r \notin f^{-1} (r) \} \]
	
	where $ f^{-1} $ denotes the preimage of $ f $. Nah, I don't think this works unless we use the fact that the preimage is surjective...?
\end{proof}


\begin{prop} There is an injection from $ \R^{2} $ to $ \R $	
	
\end{prop}

\begin{proof}
	Let's construct an injective function $ f : (0,1) \times (0,1) \to (0,1) $. Since there exist bijections between $ \R $ and $ (0,1) $ (eg. take $ g(t) = (\tan t + \frac{\pi}{2})/2 $), the proposed function $ f $ is sufficient to show such an injection exists.
	
	
	Let the decimal representation of $ x $ be $ 0.x_{1}x_{2}x_{3}\cdots $, and that of $ y $ be $ 0.y_{1}y_{2}y_{3}\cdots $. Let $ f(x,y) $ be $ 0.x_{1}y_{1}x_{2}y_{2}x_{3}y_{3}\cdots $
	
	To make this function well-defined, avoid decimal representations that end with infinite successions of $ 9 $s. Then, $ f $ is injective. 
\end{proof}

To specify some $ f \in X : = \{  f : f(x) = x \text{ for all but finitely many } x \in \R   \}$, I need

\[ (r_{1},f(r_{1}), r_{2},f(r_{2} ),\cdots,(r_{n},f(r_{n})) \] ie. a finite set of ordered pairs or reals, where the $ r_{i} $ represents the points at which the function is not the identity.

Given the number of ordered pairs $ n \in \N $ we then encode these orders pairs as a member of the set $ \N \times \R $, and inject this into $ \R $:

\[ N \times \R \to \R \times \R \to \R \]

Hence an injection $ X \to \R $ 


\end{document}
