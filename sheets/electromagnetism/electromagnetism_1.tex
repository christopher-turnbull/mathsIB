\documentclass[a4paper]{article}
\usepackage{amsmath}
\def\npart {IB}
\def\nterm {Lent}
\def\nyear {2018}
\def\nlecturer {Dr. Warnick (cmw50@cam.ac.uk)}
\def\ncourse {Electromagnetism Example Sheet 1}

% Imports
\ifx \nauthor\undefined
  \def\nauthor{Christopher Turnbull}
\else
\fi

\author{Supervised by \nlecturer \\\small Solutions presented by \nauthor}
\date{\nterm\ \nyear}

\usepackage{alltt}
\usepackage{amsfonts}
\usepackage{amsmath}
\usepackage{amssymb}
\usepackage{amsthm}
\usepackage{booktabs}
\usepackage{caption}
\usepackage{enumitem}
\usepackage{fancyhdr}
\usepackage{graphicx}
\usepackage{mathdots}
\usepackage{mathtools}
\usepackage{microtype}
\usepackage{multirow}
\usepackage{pdflscape}
\usepackage{pgfplots}
\usepackage{siunitx}
\usepackage{slashed}
\usepackage{tabularx}
\usepackage{tikz}
\usepackage{tkz-euclide}
\usepackage[normalem]{ulem}
\usepackage[all]{xy}
\usepackage{imakeidx}

\makeindex[intoc, title=Index]
\indexsetup{othercode={\lhead{\emph{Index}}}}

\ifx \nextra \undefined
  \usepackage[pdftex,
    hidelinks,
    pdfauthor={Christopher Turnbull},
    pdfsubject={Cambridge Maths Notes: Part \npart\ - \ncourse},
    pdftitle={Part \npart\ - \ncourse},
  pdfkeywords={Cambridge Mathematics Maths Math \npart\ \nterm\ \nyear\ \ncourse}]{hyperref}
  \title{Part \npart\ --- \ncourse}
\else
  \usepackage[pdftex,
    hidelinks,
    pdfauthor={Christopher Turnbull},
    pdfsubject={Cambridge Maths Notes: Part \npart\ - \ncourse\ (\nextra)},
    pdftitle={Part \npart\ - \ncourse\ (\nextra)},
  pdfkeywords={Cambridge Mathematics Maths Math \npart\ \nterm\ \nyear\ \ncourse\ \nextra}]{hyperref}

  \title{Part \npart\ --- \ncourse \\ {\Large \nextra}}
  \renewcommand\printindex{}
\fi

\pgfplotsset{compat=1.12}

\pagestyle{fancyplain}
\lhead{\emph{\nouppercase{\leftmark}}}
\ifx \nextra \undefined
  \rhead{
    \ifnum\thepage=1
    \else
      \npart\ \ncourse
    \fi}
\else
  \rhead{
    \ifnum\thepage=1
    \else
      \npart\ \ncourse\ (\nextra)
    \fi}
\fi
\usetikzlibrary{arrows.meta}
\usetikzlibrary{decorations.markings}
\usetikzlibrary{decorations.pathmorphing}
\usetikzlibrary{positioning}
\usetikzlibrary{fadings}
\usetikzlibrary{intersections}
\usetikzlibrary{cd}

\newcommand*{\Cdot}{{\raisebox{-0.25ex}{\scalebox{1.5}{$\cdot$}}}}
\newcommand {\pd}[2][ ]{
  \ifx #1 { }
    \frac{\partial}{\partial #2}
  \else
    \frac{\partial^{#1}}{\partial #2^{#1}}
  \fi
}
\ifx \nhtml \undefined
\else
  \renewcommand\printindex{}
  \makeatletter
  \DisableLigatures[f]{family = *}
  \let\Contentsline\contentsline
  \renewcommand\contentsline[3]{\Contentsline{#1}{#2}{}}
  \renewcommand{\@dotsep}{10000}
  \newlength\currentparindent
  \setlength\currentparindent\parindent

  \newcommand\@minipagerestore{\setlength{\parindent}{\currentparindent}}
  \usepackage[active,tightpage,pdftex]{preview}
  \renewcommand{\PreviewBorder}{0.1cm}

  \newenvironment{stretchpage}%
  {\begin{preview}\begin{minipage}{\hsize}}%
    {\end{minipage}\end{preview}}
  \AtBeginDocument{\begin{stretchpage}}
  \AtEndDocument{\end{stretchpage}}

  \newcommand{\@@newpage}{\end{stretchpage}\begin{stretchpage}}

  \let\@real@section\section
  \renewcommand{\section}{\@@newpage\@real@section}
  \let\@real@subsection\subsection
  \renewcommand{\subsection}{\@@newpage\@real@subsection}
  \makeatother
\fi

% Theorems
\theoremstyle{definition}
\newtheorem*{aim}{Aim}
\newtheorem*{axiom}{Axiom}
\newtheorem*{claim}{Claim}
\newtheorem*{cor}{Corollary}
\newtheorem*{conjecture}{Conjecture}
\newtheorem*{defi}{Definition}
\newtheorem*{eg}{Example}
\newtheorem*{ex}{Exercise}
\newtheorem*{fact}{Fact}
\newtheorem*{law}{Law}
\newtheorem*{lemma}{Lemma}
\newtheorem*{notation}{Notation}
\newtheorem*{prop}{Proposition}
\newtheorem*{soln}{Solution}
\newtheorem*{thm}{Theorem}

\newtheorem*{remark}{Remark}
\newtheorem*{warning}{Warning}
\newtheorem*{exercise}{Exercise}

\newtheorem{nthm}{Theorem}[section]
\newtheorem{nlemma}[nthm]{Lemma}
\newtheorem{nprop}[nthm]{Proposition}
\newtheorem{ncor}[nthm]{Corollary}


\renewcommand{\labelitemi}{--}
\renewcommand{\labelitemii}{$\circ$}
\renewcommand{\labelenumi}{(\roman{*})}

\let\stdsection\section
\renewcommand\section{\newpage\stdsection}

% Strike through
\def\st{\bgroup \ULdepth=-.55ex \ULset}

% Maths symbols
\newcommand{\abs}[1]{\left\lvert #1\right\rvert}
\newcommand\ad{\mathrm{ad}}
\newcommand\AND{\mathsf{AND}}
\newcommand\Art{\mathrm{Art}}
\newcommand{\Bilin}{\mathrm{Bilin}}
\newcommand{\bket}[1]{\left\lvert #1\right\rangle}
\newcommand{\B}{\mathcal{B}}
\newcommand{\bolds}[1]{{\bfseries #1}}
\newcommand{\brak}[1]{\left\langle #1 \right\rvert}
\newcommand{\braket}[2]{\left\langle #1\middle\vert #2 \right\rangle}
\newcommand{\bra}{\langle}
\newcommand{\cat}[1]{\mathsf{#1}}
\newcommand{\C}{\mathbb{C}}
\newcommand{\CP}{\mathbb{CP}}
\newcommand{\cU}{\mathcal{U}}
\newcommand{\Der}{\mathrm{Der}}
\newcommand{\D}{\mathrm{D}}
\newcommand{\dR}{\mathrm{dR}}
\newcommand{\E}{\mathbb{E}}
\newcommand{\F}{\mathbb{F}}
\newcommand{\Frob}{\mathrm{Frob}}
\newcommand{\GG}{\mathbb{G}}
\newcommand{\gl}{\mathfrak{gl}}
\newcommand{\GL}{\mathrm{GL}}
\newcommand{\G}{\mathcal{G}}
\newcommand{\Gr}{\mathrm{Gr}}
\newcommand{\haut}{\mathrm{ht}}
\newcommand{\Id}{\mathrm{Id}}
\newcommand{\ket}{\rangle}
\newcommand{\lie}[1]{\mathfrak{#1}}
\newcommand{\Mat}{\mathrm{Mat}}
\newcommand{\N}{\mathbb{N}}
\newcommand{\norm}[1]{\left\lVert #1\right\rVert}
\newcommand{\normalorder}[1]{\mathop{:}\nolimits\!#1\!\mathop{:}\nolimits}
\newcommand\NOT{\mathsf{NOT}}
\newcommand{\Oc}{\mathcal{O}}
\newcommand{\Or}{\mathrm{O}}
\newcommand\OR{\mathsf{OR}}
\newcommand{\ort}{\mathfrak{o}}
\newcommand{\PGL}{\mathrm{PGL}}
\newcommand{\ph}{\,\cdot\,}
\newcommand{\pr}{\mathrm{pr}}
\newcommand{\Prob}{\mathbb{P}}
\newcommand{\PSL}{\mathrm{PSL}}
\newcommand{\Ps}{\mathcal{P}}
\newcommand{\PSU}{\mathrm{PSU}}
\newcommand{\pt}{\mathrm{pt}}
\newcommand{\qeq}{\mathrel{``{=}"}}
\newcommand{\Q}{\mathbb{Q}}
\newcommand{\R}{\mathbb{R}}
\newcommand{\RP}{\mathbb{RP}}
\newcommand{\Rs}{\mathcal{R}}
\newcommand{\SL}{\mathrm{SL}}
\newcommand{\so}{\mathfrak{so}}
\newcommand{\SO}{\mathrm{SO}}
\newcommand{\Spin}{\mathrm{Spin}}
\newcommand{\Sp}{\mathrm{Sp}}
\newcommand{\su}{\mathfrak{su}}
\newcommand{\SU}{\mathrm{SU}}
\newcommand{\term}[1]{\emph{#1}\index{#1}}
\newcommand{\T}{\mathbb{T}}
\newcommand{\tv}[1]{|#1|}
\newcommand{\U}{\mathrm{U}}
\newcommand{\uu}{\mathfrak{u}}
\newcommand{\Vect}{\mathrm{Vect}}
\newcommand{\wsto}{\stackrel{\mathrm{w}^*}{\to}}
\newcommand{\wt}{\mathrm{wt}}
\newcommand{\wto}{\stackrel{\mathrm{w}}{\to}}
\newcommand{\Z}{\mathbb{Z}}
\renewcommand{\d}{\mathrm{d}}
\renewcommand{\H}{\mathbb{H}}
\renewcommand{\P}{\mathbb{P}}
\renewcommand{\sl}{\mathfrak{sl}}
\renewcommand{\vec}[1]{\boldsymbol{\mathbf{#1}}}
%\renewcommand{\F}{\mathcal{F}}

\let\Im\relax
\let\Re\relax

\DeclareMathOperator{\adj}{adj}
\DeclareMathOperator{\Ann}{Ann}
\DeclareMathOperator{\area}{area}
\DeclareMathOperator{\Aut}{Aut}
\DeclareMathOperator{\Bernoulli}{Bernoulli}
\DeclareMathOperator{\betaD}{beta}
\DeclareMathOperator{\bias}{bias}
\DeclareMathOperator{\binomial}{binomial}
\DeclareMathOperator{\card}{card}
\DeclareMathOperator{\ccl}{ccl}
\DeclareMathOperator{\Char}{char}
\DeclareMathOperator{\ch}{ch}
\DeclareMathOperator{\cl}{cl}
\DeclareMathOperator{\cls}{\overline{\mathrm{span}}}
\DeclareMathOperator{\conv}{conv}
\DeclareMathOperator{\corr}{corr}
\DeclareMathOperator{\cosec}{cosec}
\DeclareMathOperator{\cosech}{cosech}
\DeclareMathOperator{\cov}{cov}
\DeclareMathOperator{\covol}{covol}
\DeclareMathOperator{\diag}{diag}
\DeclareMathOperator{\diam}{diam}
\DeclareMathOperator{\Diff}{Diff}
\DeclareMathOperator{\disc}{disc}
\DeclareMathOperator{\dom}{dom}
\DeclareMathOperator{\End}{End}
\DeclareMathOperator{\energy}{energy}
\DeclareMathOperator{\erfc}{erfc}
\DeclareMathOperator{\erf}{erf}
\DeclareMathOperator*{\esssup}{ess\,sup}
\DeclareMathOperator{\ev}{ev}
\DeclareMathOperator{\Ext}{Ext}
\DeclareMathOperator{\Fit}{Fit}
\DeclareMathOperator{\fix}{fix}
\DeclareMathOperator{\Frac}{Frac}
\DeclareMathOperator{\Gal}{Gal}
\DeclareMathOperator{\gammaD}{gamma}
\DeclareMathOperator{\gr}{gr}
\DeclareMathOperator{\hcf}{hcf}
\DeclareMathOperator{\Hom}{Hom}
\DeclareMathOperator{\id}{id}
\DeclareMathOperator{\image}{image}
\DeclareMathOperator{\im}{im}
\DeclareMathOperator{\Im}{Im}
\DeclareMathOperator{\Ind}{Ind}
\DeclareMathOperator{\Int}{Int}
\DeclareMathOperator{\Isom}{Isom}
\DeclareMathOperator{\lcm}{lcm}
\DeclareMathOperator{\length}{length}
\DeclareMathOperator{\Lie}{Lie}
\DeclareMathOperator{\like}{like}
\DeclareMathOperator{\Lk}{Lk}
\DeclareMathOperator{\mse}{mse}
\DeclareMathOperator{\multinomial}{multinomial}
\DeclareMathOperator{\orb}{orb}
\DeclareMathOperator{\ord}{ord}
\DeclareMathOperator{\otp}{otp}
\DeclareMathOperator{\Poisson}{Poisson}
\DeclareMathOperator{\poly}{poly}
\DeclareMathOperator{\rank}{rank}
\DeclareMathOperator{\rel}{rel}
\DeclareMathOperator{\Re}{Re}
\DeclareMathOperator*{\res}{res}
\DeclareMathOperator{\Res}{Res}
\DeclareMathOperator{\rk}{rk}
\DeclareMathOperator{\Root}{Root}
\DeclareMathOperator{\sech}{sech}
\DeclareMathOperator{\sgn}{sgn}
\DeclareMathOperator{\spn}{span}
\DeclareMathOperator{\stab}{stab}
\DeclareMathOperator{\St}{St}
\DeclareMathOperator{\supp}{supp}
\DeclareMathOperator{\Syl}{Syl}
\DeclareMathOperator{\Sym}{Sym}
\DeclareMathOperator{\tr}{tr}
\DeclareMathOperator{\Tr}{Tr}
\DeclareMathOperator{\var}{var}
\DeclareMathOperator{\vol}{vol}

\pgfarrowsdeclarecombine{twolatex'}{twolatex'}{latex'}{latex'}{latex'}{latex'}
\tikzset{->/.style = {decoration={markings,
                                  mark=at position 1 with {\arrow[scale=2]{latex'}}},
                      postaction={decorate}}}
\tikzset{<-/.style = {decoration={markings,
                                  mark=at position 0 with {\arrowreversed[scale=2]{latex'}}},
                      postaction={decorate}}}
\tikzset{<->/.style = {decoration={markings,
                                   mark=at position 0 with {\arrowreversed[scale=2]{latex'}},
                                   mark=at position 1 with {\arrow[scale=2]{latex'}}},
                       postaction={decorate}}}
\tikzset{->-/.style = {decoration={markings,
                                   mark=at position #1 with {\arrow[scale=2]{latex'}}},
                       postaction={decorate}}}
\tikzset{-<-/.style = {decoration={markings,
                                   mark=at position #1 with {\arrowreversed[scale=2]{latex'}}},
                       postaction={decorate}}}
\tikzset{->>/.style = {decoration={markings,
                                  mark=at position 1 with {\arrow[scale=2]{latex'}}},
                      postaction={decorate}}}
\tikzset{<<-/.style = {decoration={markings,
                                  mark=at position 0 with {\arrowreversed[scale=2]{twolatex'}}},
                      postaction={decorate}}}
\tikzset{<<->>/.style = {decoration={markings,
                                   mark=at position 0 with {\arrowreversed[scale=2]{twolatex'}},
                                   mark=at position 1 with {\arrow[scale=2]{twolatex'}}},
                       postaction={decorate}}}
\tikzset{->>-/.style = {decoration={markings,
                                   mark=at position #1 with {\arrow[scale=2]{twolatex'}}},
                       postaction={decorate}}}
\tikzset{-<<-/.style = {decoration={markings,
                                   mark=at position #1 with {\arrowreversed[scale=2]{twolatex'}}},
                       postaction={decorate}}}

\tikzset{circ/.style = {fill, circle, inner sep = 0, minimum size = 3}}
\tikzset{mstate/.style={circle, draw, blue, text=black, minimum width=0.7cm}}

\tikzset{commutative diagrams/.cd,cdmap/.style={/tikz/column 1/.append style={anchor=base east},/tikz/column 2/.append style={anchor=base west},row sep=tiny}}

\definecolor{mblue}{rgb}{0.2, 0.3, 0.8}
\definecolor{morange}{rgb}{1, 0.5, 0}
\definecolor{mgreen}{rgb}{0.1, 0.4, 0.2}
\definecolor{mred}{rgb}{0.5, 0, 0}

\def\drawcirculararc(#1,#2)(#3,#4)(#5,#6){%
    \pgfmathsetmacro\cA{(#1*#1+#2*#2-#3*#3-#4*#4)/2}%
    \pgfmathsetmacro\cB{(#1*#1+#2*#2-#5*#5-#6*#6)/2}%
    \pgfmathsetmacro\cy{(\cB*(#1-#3)-\cA*(#1-#5))/%
                        ((#2-#6)*(#1-#3)-(#2-#4)*(#1-#5))}%
    \pgfmathsetmacro\cx{(\cA-\cy*(#2-#4))/(#1-#3)}%
    \pgfmathsetmacro\cr{sqrt((#1-\cx)*(#1-\cx)+(#2-\cy)*(#2-\cy))}%
    \pgfmathsetmacro\cA{atan2(#2-\cy,#1-\cx)}%
    \pgfmathsetmacro\cB{atan2(#6-\cy,#5-\cx)}%
    \pgfmathparse{\cB<\cA}%
    \ifnum\pgfmathresult=1
        \pgfmathsetmacro\cB{\cB+360}%
    \fi
    \draw (#1,#2) arc (\cA:\cB:\cr);%
}
\newcommand\getCoord[3]{\newdimen{#1}\newdimen{#2}\pgfextractx{#1}{\pgfpointanchor{#3}{center}}\pgfextracty{#2}{\pgfpointanchor{#3}{center}}}

\def\Xint#1{\mathchoice
   {\XXint\displaystyle\textstyle{#1}}%
   {\XXint\textstyle\scriptstyle{#1}}%
   {\XXint\scriptstyle\scriptscriptstyle{#1}}%
   {\XXint\scriptscriptstyle\scriptscriptstyle{#1}}%
   \!\int}
\def\XXint#1#2#3{{\setbox0=\hbox{$#1{#2#3}{\int}$}
     \vcenter{\hbox{$#2#3$}}\kern-.5\wd0}}
\def\ddashint{\Xint=}
\def\dashint{\Xint-}

\newcommand\separator{{\centering\rule{2cm}{0.2pt}\vspace{2pt}\par}}

\newenvironment{own}{\color{gray!70!black}}{}

\newcommand\makecenter[1]{\raisebox{-0.5\height}{#1}}

\newtheorem*{soln}{Solution}

\renewcommand{\thesection}{}
\renewcommand{\thesubsection}{\arabic{section}.\arabic{subsection}}
\makeatletter
\def\@seccntformat#1{\csname #1ignore\expandafter\endcsname\csname the#1\endcsname\quad}
\let\sectionignore\@gobbletwo
\let\latex@numberline\numberline
\def\numberline#1{\if\relax#1\relax\else\latex@numberline{#1}\fi}
\makeatother


\begin{document}
	
\maketitle

\section{QUESTION 1}

Equation for conservation of charge is

\[ \frac{\partial \rho }{\partial t}  + \nabla \cdot \mathbf{J} = 0 \]

Have $  \mathbf{J} = C \mathbf{r} e^{-atr^{2}} $, so 

\begin{align*}
\nabla \cdot \mathbf{J} & = C \nabla \cdot (e^{-atr^{2}} \mathbf{r} )   \\
& = C e^{-atr^{2}} \nabla \cdot \mathbf{r} + C \mathbf{r} \cdot \nabla (e^{-atr^{2}})
\end{align*}

Now $ \mathbf{r}_{i} = x_{i} $ so $ \nabla \cdot \mathbf{r} = \frac{\partial x_{j}}{\partial x_{j}} = 3 $, and 


\begin{align*}                                         
 \nabla e^{-atr^{2}} & = \frac{\partial e^{-atr^{2}} }{\partial r} \hat{\mathbf{r}} \\
& = - 2ate^{-atr^{2}} \mathbf{r}
\end{align*}
                                                       
%\begin{align*}                                         
%[\nabla (e^{-atr^{2}})]_{i} & = \frac{\partial }{\partial x_{i}} (e^{-atx_{j}x_{j}}) \\
%& = -at (e^{-atx_{j}x_{j}})\frac{\partial }{\partial x_{i}}(x_{j}x_{j}) \\
%& = - ate^{-atr^{2}} 2 x_{j} \delta_{ij} \\
%& = - 2ate^{-atr^{2}} x_{i}
%\end{align*}

Hence

\begin{align*}
\nabla \cdot \mathbf{J} & = 3 C e^{-atr^{2}} - 2C  r^{2} ate^{-atr^{2}}
\end{align*}

Suppose that $ \rho = (f + tg)e^{-atr^{2}} $. Then we have

\begin{align*}
\frac{\partial \rho }{\partial t} & = (-ar^{2} f + g - ar^{2}tg)e^{-atr^{2}} \\
& = (g - ar^{2} f) e^{-atr^{2}} - gt ar^{2} e^{-atr^{2}} \\
\end{align*}

Hence we conclude that

\[ g - a r^{2} f = - 3 C , \qquad g = -2C \]

\[ \Rightarrow f = \frac{C}{ar^{2}} \]

\section{QUESTION 2}

Using the continuity equation, 

\begin{align*}
\frac{\partial \rho }{\partial t} & = - \nabla \cdot \mathbf{J}  \\
& = - \nabla \cdot (-D \nabla  \rho) \\
& = D \nabla^{2} \rho 
\end{align*}

showing $ \rho(\mathbf{x},t) $ obeys the heat equation with diffusion constant $ D $.

Let $ \rho(\mathbf{r},t) $ be defined as 

\[ \rho(\mathbf{r},t) = \frac{\rho_{0}a^{3}}{	(4D(t - t_{0}) + a^{2})^{3/2}} \exp \left(  - \frac{r^{2}}{4D(t - t_{0}) + a^{2}} \right)  \]

Taking time derivatives,

\begin{align*}
\frac{\partial \rho }{\partial t} & = \left(  \frac{-6D \rho_{0}a^{3}}{	(4D(t - t_{0}) + a^{2})^{5/2}} + \frac{4D r^{2} \rho_{0}a^{3} }{(4D(t-t_{0}) + a^{2})^{7/2}} \right) \exp \left( - \frac{r^{2}}{4D(t - t_{0}) + a^{2}} \right) \\
\end{align*}

Now

%\begin{align*}
%\nabla^{2} e^{\lambda r^{2}} & = \frac{\partial^{2} }{\partial x_{j} \partial x_{j}} \left[ e^{\lambda x_{p}x_{p}} \right]  \\
%& = \frac{\partial }{\partial x_{j}} \left[ 2\lambda x_{j} e^{\lambda x_{p}x_{p}} \right] \\
%& =  6 \lambda e^{\lambda x_{j}x_{j}}  + 4 \lambda^{2} x_{j} x_{j} e^{\lambda x_{p}x_{p}} \qquad \left(  \text{ as } \frac{\partial x_{j} }{\partial x_{j}} = 3 \right)    \\
%& =  \lambda ( 6  + 4 \lambda r^{2} )e^{\lambda r^{2}}
%\end{align*}

\begin{align*}
\nabla^{2} e^{\lambda r^{2}} & = \frac{1}{r^{2} }\frac{\d }{\d r} \left(  r^{2} \frac{\d }{\d r}( e^{\lambda r^{2}} ) \right) \\
& = \frac{2 \lambda}{r^{2}} \frac{\d }{\d r} \left(  r^{3} e^{\lambda r^{2}} \right) \\
& =  \frac{2 \lambda}{r^{2}} \left[   3r^{2} + 2 \lambda r^{4} \right] e^{\lambda r^{2}} \\
& = \lambda(6 + 4r^{2}) e^{\lambda r^{2}}
\end{align*}

Thus with $ \lambda = - \frac{1}{4D(t - t_{0}) + a^{2}}  $, we have 

\begin{align*}
\nabla^{2} \rho  & = \frac{- \rho_{0}a^{3}}{	(4D(t - t_{0}) + a^{2})^{5/2}} \left(  6 - \frac{4 r^{2}}{4D(t - t_{0}) + a^{2}}  \right) \exp \left( - \frac{r^{2}}{4D(t - t_{0}) + a^{2}} \right)  \\
& = \left( -  \frac{6 \rho_{0}a^{3}}{	(4D(t - t_{0}) + a^{2})^{5/2}} + \frac{4 r^{2} \rho_{0}a^{3} }{(4D(t - t_{0}) + a^{2})^{-7/2}} \right) \exp \left( - \frac{r^{2}}{4D(t - t_{0}) + a^{2}} \right)
\end{align*}

Hence we can see that $ \frac{\partial \rho }{\partial t} = D \nabla^{2} \rho  $, as required.



\section{QUESTION 3}

Considering the infinite plane $ z = 0 $, we see this has uniform charge density $ \rho_{0} $. 

By symmetry, the field points vertically, and the field on the bottom is opposite of that on top, we must have

\[ \mathbf{E} = E(z) \hat{\mathbf{z}} \]

with

\[ E(z) = -E(-z) \]

Consider a vertical cylinder of height $ 2h $ and cross-sectional area $ A $. Now only the end caps contribute.

First, 

\begin{align*}
Q & = \int_{V} \rho_{0} e^{-k| z |}  \; \d V \\
& = \int_{-h}^{h} \int_{0}^{2\pi}  \int_{0}^{R} \rho_{0} e^{-k| z |}  \rho \; \d \rho \d \phi \d z \\
& = 2 \pi \frac{R^{2}}{2} \rho_{0} \int_{-h}^{h}  e^{-k| z |}  \;  \d z \\
& =  A \rho_{0} \int_{0}^{h} 2 e^{-kz}   \;  \d z \\
& =  2 A \rho_{0} \left[ - \frac{1}{k} e^{-kz} \right]_{0}^{h} \\
& =  2 A \frac{\rho_{0}}{k} (1 - e^{-kh}) 
\end{align*}

And

\begin{align*}
\int_{S} \mathbf{E} \cdot \d \mathbf{S} & = E(h) A - E(-h) A = 2AE(h)  =  2 A \frac{\rho_{0}}{k \varepsilon_{0}} (1 - e^{-kh}) \\
\end{align*}

Hence 

\[ E(z)  =  \frac{\rho_{0}}{k \varepsilon_{0}} (1 - e^{-kz}) \]

as required.




\section{QUESTION 4}

We have that

\[ \rho(r) = \begin{cases} 0  & \text{ if } r < a  \\ \rho & \text{ if } a < r < b \\ 0 & \text{ if } r > b \end{cases} \]


Note that if $ r < a $ there is no field; Gauss' law tells us that the flux only depends on the total charge contained inside the surface. Now consider $ r > b $. By symmetry, the force is the same in all directions and points outwards radially. So

\[  \mathbf{E} = E(r) \hat{\mathbf{r}} \]

Put $ S $ to be a sphere of radius $ r > b $. Then the total flux is 

\begin{align*}
\int_{S} \mathbf{E} \cdot \; \d S  & = \int_{S} E(r) \hat{\mathbf{r}} \cdot \d \mathbf{S} \\
& = E(r) \int_{S}  \hat{\mathbf{r}} \cdot \d \mathbf{S} \\
& = E(r) \cdot 4 \pi r^{2}
\end{align*}

By Gauss's law, we know this is equal to $ Q/\varepsilon_{0} $, and $ Q = \frac{4}{3} \pi (b^{3} - a^{3}) \rho $. Therefore,

\[ E(r) = \frac{ (b^{3} - a^{3}) \rho}{3\varepsilon_{0}r^{2}} \]

and

\[  \mathbf{E}(r) = \frac{ (b^{3} - a^{3})\rho}{3\varepsilon_{0}r^{2}} \hat{\mathbf{r}}  \]


Now suppose we are inside the region, $ a < r < b $. Then


\[ \int_{S} \mathbf{E} \cdot \; \d S = E(r) 4 \pi r^{2} = \frac{Q}{\varepsilon_{0}} \left(  \frac{r^{3}-a^{3}}{b^{3}-a^{3}} \right)  \]

So 

\begin{align*}
\mathbf{E}(r) & = \frac{Q(r^{3}-a^{3})}{4 \pi \varepsilon_{0}(b^{3}-a^{3}) r^{2}} \\
& = \frac{Q(r^{3}-a^{3})\rho}{3 \varepsilon_{0} r^{2}}
\end{align*}


Finally if $ r < a $, Gauss' law tells us that the flux depends only on the total charge contained inside the surface, which in this case is none. So $ \mathbf{E}(r) = 0 $.

Note that the electric field is discontinuous across the surface. We have

\begin{align*}
E(r \to b +  ) - E(r \to b-) & =  \frac{ (b-a)(b^{2} + 2ab + a^{2}) \rho  }{3\varepsilon_{0}b^{2}} \\
& = \frac{\sigma}{\varepsilon_{0}}
\end{align*}

as expected.

\section{QUESTION 5}


The field lines for a positive charge are:
\begin{center}
	\begin{tikzpicture}
	\node [draw, circle, inner sep = 0, minimum size=13] {+};
	\draw [dashed] circle [radius=1.4];
	\draw [dashed] circle [radius=.7];
	\draw [->] (0, 0.3) -- (0, 1.5);
	\draw [->] (0.3, 0) -- (1.5, 0);
	\draw [->] (0, -0.3) -- (0, -1.5);
	\draw [->] (-0.3, 0) -- (-1.5, 0);
	\end{tikzpicture}
\end{center}

For two positive charges,



\begin{center}
	\begin{tikzpicture}
	\node [draw, circle, inner sep = 0, minimum size=13] (+) {+};
	\node [draw, circle, inner sep = 0, minimum size=13] (-) at (2, 0) {+};
	\draw [->-=0.6] (+) to [bend right=30] (0.5,1);
	\draw [->-=0.6] (+) to [bend left=30] (0.5,-1);
	\draw [->-=0.6] (-) to [bend left=30] (1.5,1);
	\draw [->-=0.6] (-) to [bend right=30] (1.5,-1);
	\draw [->-=0.7] (+) to [bend right=30] +(-1, 0.5);
	\draw [->-=0.7] (+) to [bend left=30] +(-1, -0.5);
	\draw [->-=0.7] (+) to +(-1, 0);
	\draw [-<-=0.4] (-) to [bend left=30] +(1, 0.5);
	\draw [-<-=0.4] (-) to [bend right=30] +(1, -0.5);
	\draw [-<-=0.4] (-) to +(1, 0);
	\end{tikzpicture}
\end{center}


We can also draw field lines for dipoles:
\begin{center}
	\begin{tikzpicture}
	\node [draw, circle, inner sep = 0, minimum size=13] (+) {+};
	\node [draw, circle, inner sep = 0, minimum size=13] (-) at (2, 0) {-};
	\draw [->-=0.6] (+) -- (-);
	\draw [->-=0.6] (+) to [bend left=30] (-);
	\draw [->-=0.6] (+) to [bend right=30] (-);
	\draw [->-=0.6] (+) to [bend left=60] (-);
	\draw [->-=0.6] (+) to [bend right=60] (-);
	\draw [->-=0.7] (+) to [bend right=30] +(-1, 0.5);
	\draw [->-=0.7] (+) to [bend left=30] +(-1, -0.5);
	\draw [->-=0.7] (+) to +(-1, 0);
	\draw [-<-=0.4] (-) to [bend left=30] +(1, 0.5);
	\draw [-<-=0.4] (-) to [bend right=30] +(1, -0.5);
	\draw [-<-=0.4] (-) to +(1, 0);
	\end{tikzpicture}
\end{center}


\section{QUESTION 6}

The inverse square law, or Coulomb's Law, states that the electric field generated by a particle with total charge $ Q $ (at the origin) is given by

\[ \mathbf{E}(r) = \frac{Q}{4 \pi \varepsilon_{0} r^{2}} \hat{\mathbf{r}} \]

Now consider an infinite line with uniform charge density per unit length $ \eta $.
We use cylindrical polar coordinates. By symmetry, the field is radial, ie.

\[ \mathbf{E}(r) = E(r) \hat{\mathbf{r}} \]

Consider an arbitrary point at $ (r,z_{0}) $. We will integrate along the $ z $-axis to find the field at this point; summing the contributions from the changes at $ (0,z) $ as $ z $ goes from $ -\infty $ to $ \infty $. By Coloumb's law; 


%\begin{center}
%	\begin{tikzpicture}
%	
%	\draw (0,0) to (0,1);
%	\draw (0,0) -- (2,0);
%	\draw (2,0) -- (0,1);
%	\node[draw] at (0.5, 0) ($r_{0}$)  ;
%	\end{tikzpicture}
%\end{center}

Here,

\begin{align*}
E(r) & = \int_{-\infty}^{\infty} \frac{\eta}{4 \pi \varepsilon_{0}} \frac{1}{r^{2} + (z-z_{0})^{2}} \; \d z \\
& = \frac{\eta}{4 \pi \varepsilon_{0}} \int_{-\infty}^{\infty}  \frac{1}{r^{2} + z^{2}} \; \d z \\
& = \frac{\eta}{4 \pi \varepsilon_{0}} \left[   \frac{1}{r} \arctan \left(  \frac{z}{r} \right)   \right]_{-\infty}^{\infty}  \\
& = \frac{\eta}{4 \varepsilon_{0} r} 
\end{align*}

But this is a different result than what we want... Can't spot my error. 




\section{QUESTION 7}

The Green's function for the Laplacian is definied to be the solution to:

\[ \nabla^{2} G(\mathbf{r};\mathbf{r}') = \delta^{3} (\mathbf{r} - \mathbf{r}')  \]

The Green's function in three dimensions is:

\[ G(\mathbf{r};\mathbf{r}') = - \frac{1}{4 \pi} \frac{1}{| \mathbf{r} - \mathbf{r}' |}  \]

We assume all the charge is contained within some compact region $ V $, then

\begin{align*}
\phi(\mathbf{r}) & = - \frac{1}{\varepsilon_{0}} \int_{V} \rho(\mathbf{r}') G(\mathbf{r},\mathbf{r}') \; \d^{3} \mathbf{r} \\
& = \frac{1}{4 \pi \varepsilon_{0}} \int_{V} \frac{\rho(\mathbf{r}')}{| \mathbf{r} - \mathbf{r}' |} \; \d^{3} \mathbf{r}
\end{align*}

Here, the charge is contained in a circular disk of radius $ a $, uniform charge density $ \sigma $. Using cylindrical polars, the charge at $ \mathbf{r} = (0,0,z) $, due to $ \mathbf{r}' = (r \cos \phi, r \sin \phi,0) $, we have $ | \mathbf{r} - \mathbf{r}' | = \sqrt{r^{2} + z^{2}} $ and hence

\begin{align*}
\phi(\mathbf{r}) & = \frac{1}{4 \pi \varepsilon_{0}} \int_{0}^{2\pi} \int_{0}^{a} \frac{\sigma}{\sqrt{r^{2} + z^{2}}} r \; \d r \;  \d \phi  \ \\
& = \frac{\sigma}{2 \varepsilon_{0}} \left[  \sqrt{r^{2} + z^{2}} \right]_{r=0}^{a} \\
& = \frac{\sigma}{2 \varepsilon_{0}} \left( \sqrt{a^{2} + z^{2}} - | z |  \right) 
\end{align*}

Then 

\begin{align*}
\mathbf{E}(\mathbf{r}) & = - \nabla  \phi (\mathbf{r}) \\
& = - \frac{\sigma}{2 \varepsilon_{0}} \left( \frac{z}{\sqrt{a^{2} + z^{2}}} - \sgn (z)  \right) 
\end{align*}

Thus again, with $ \mathbf{E} = E(z) \hat{\mathbf{z}} $, we see we have the expected discontinuity

\[ E (z \to 0+) - E(z \to 0-) = \frac{\sigma}{\varepsilon_{0}} \]

As $ z \to \infty $, 

\begin{align*}
\frac{z}{\sqrt{a^{2} + z^{2}}} &  =  \left(  1 + \frac{z^{2}}{a^{2}} \right)^{-1/2}  \\
& = 1 + - \frac{z^{2}}{2a^{2}} + \cdots
\end{align*}

\[ \mathbf{E} \approx \frac{\sigma a^{2}}{4 \pi \varepsilon_{0} z^{2}} \mathbf{\hat{z}} \]

which is Coloumb's Law for a particle due to charge $ Q = \sigma a^{2} $


\section{QUESTION 8}

From Q7 we have the result that

\begin{align*}
\phi(\mathbf{r}) & = \frac{1}{4 \pi \varepsilon_{0}} \int_{V} \frac{\rho(\mathbf{r}')}{| \mathbf{r} - \mathbf{r}' |} \; \d^{3} \mathbf{r}
\end{align*}

Very far from $ V $, ie. $ | \mathbf{r} | \gg | \mathbf{r}' | $, we can use the Taylor expansion

\begin{align*}
\frac{1}{| \mathbf{r} - \mathbf{r}' |} & =  \frac{1}{r} + \mathbf{r}' \cdot \nabla \left(  \frac{1}{r} \right) - \frac{1}{2} (\mathbf{r}' \cdot \nabla )^{2} \left( \frac{1}{r} \right)  + \cdots  \\
& = \frac{1}{r} + \frac{\mathbf{r} \cdot \mathbf{r}'}{r^{3}} - \frac{1}{2} \left(  \frac{\mathbf{r}' \cdot \mathbf{r}'}{r^{3}} - \frac{3 (\mathbf{r}' \cdot \mathbf{r})^{2} }{r^{5}} \right) + \cdots 
\end{align*}

Then we get 


\begin{align*}
\phi(\mathbf{r}) & = \frac{1}{4 \pi \varepsilon_{0}} \int_{V} \rho(\mathbf{r}')  \left\{  \frac{1}{r} + \frac{\mathbf{r} \cdot \mathbf{r}'}{r^{3}} - \frac{1}{2} \left(  \frac{\mathbf{r}' \cdot \mathbf{r}'}{r^{3}} - \frac{3 (\mathbf{r}' \cdot \mathbf{r})^{2} }{r^{5}} \right) + \cdots  \right\}  \; \d^{3} \mathbf{r}' \\
& = \frac{1}{4 \pi \varepsilon_{0}} \left(  \frac{Q}{r} + \frac{\mathbf{p} \cdot \hat{\mathbf{r}} }{r^{2}} + \frac{1}{2} \frac{\Q_{ij} r_{i}r_{j}}{r^{5}} + \cdots \right) 
\end{align*}

where 

\begin{align*}
Q &= \int_V \rho(\mathbf{r}')\;\d V' \\
\mathbf{p} &= \int_V \mathbf{r}'\rho(\mathbf{r}')\; \d V'\\
\hat{\mathbf{r}} &= \frac{\mathbf{r}}{\|\mathbf{r}\|}.\\
\Q_{ij} & = \int_{V} \d^{3} r' ( 3 r_{i}' r_{j}' - \delta_{ij} r'^{2} ) \rho(\mathbf{r}') 
\end{align*}

\begin{itemize}
	\item For the first two charges we have
	
	\[ \rho(\mathbf{r}') = q \delta(\mathbf{r'}) - q \delta(\mathbf{r'} - \mathbf{d})  \]
	
	Then
	
	\begin{align*}
	Q &= \int_V q \delta(\mathbf{r'}) - q \delta(\mathbf{r'} - \mathbf{d}) \;\d V' \\
	& = 1 - 1 = 0,
	\end{align*}
	
	\begin{align*}
	\mathbf{p} & = \int_{V} q \delta(\mathbf{r'}) - q \delta(\mathbf{r'} - \mathbf{d}) \mathbf{r}'  \; \d V' \\
	& = q(0 - \mathbf{d}) \\
	& = -q\mathbf{d}
	\end{align*}
	
	and 
	
	\begin{align*}
	\Q_{ij} & = \int_{V} ( 3 r'_{i} r'_{j} - \delta_{ij} r'^{2} ) \left[  q \delta(\mathbf{r'}) - q \delta(\mathbf{r'} - \mathbf{d}) \right] \;  \d^{3} r' \\
	& = - q \int_{V} ( 3 r'_{i} r'_{j} - \delta_{ij} r'^{2} ) \delta(\mathbf{r'} - \mathbf{d}) \;  \d^{3} r' \\
	& = - q \left(  3 d_{i} d_{j} - \delta_{ij} | d |^{2}  \right)  \\
	\end{align*}
	
	Now $ \mathbf{d} = (d,0,0) $ so $ \Q_{11} = -q(3d^{2} - d^{2}) = - 2d^{2}  $, and $ \Q_{ij} = 0 \text{ for } i,j \neq 1 $
	
	\item Next, similarly we have 
	
	
	\[ \rho(\mathbf{r}') = q \delta(\mathbf{r'}) - q \delta(\mathbf{r'} - \mathbf{d}_{1}) - q \delta(\mathbf{r'} - \mathbf{d}_{2}) + - q \delta(\mathbf{r'} - \mathbf{d}_{3})   \]
	
	where $ \mathbf{d}_{1} = (d,0,0), \mathbf{d}_{2} = (0,d,0), $ and $ \mathbf{d}_{3} = (d,d,0) $. 
	
	Again it can be easily verified that $ Q = 0 $. The dipole this times gives
	
	\begin{align*}
	p & = q(0 - \mathbf{d}_{1} - \mathbf{d}_{2} + \mathbf{d}_{3}) \\
	& = 0
	\end{align*}
	
	And the Quadrupole:
	
	\begin{align*}
	\Q_{ij} & = - q \int_{V} ( 3 r'_{i} r'_{j} - \delta_{ij} r'^{2} ) \left[   \delta(\mathbf{r'} - \mathbf{d}_{1}) +  \delta(\mathbf{r'} -  \mathbf{d}_{2}) -  \delta(\mathbf{r'} -  \mathbf{d}_{3}) \right]  \;  \d^{3} r' \\
	\end{align*}
	
	\begin{align*}
	\Q_{11} & = -q ( (3d^{2} - d^{2}) +  (- d^{2})-  (3d^{2} - 2d^{2}) ) \\
	& = 0
	\end{align*}
	
	Similarly $ \Q_{22} = 0 $. Also have
	
	\begin{align*}
	\Q_{12} & =   - q \left(  (0 - 0) + (0 - 0) - (3d^{2} - 0   \right)  \\
	& = 3qd^{2} = \Q_{21}
	\end{align*}
	
	\item Now, 	
	
	
	
	
	
	
	
	
\end{itemize}




\section{QUESTION 9}

We define the electric dipole moment to be $ \mathbf{p} = Q \mathbf{d} $. Now

\begin{align*}
\frac{\d \mathbf{p}}{\d t} & = \mathbf{d} \frac{\d Q}{\d t} \\
& = \mathbf{d} \int_{V} \frac{\partial }{\partial t} \rho \; \d V \quad V \text{ fixed} \\
& = \mathbf{d} \int_{V} - \nabla \cdot \mathbf{J} \; \d V \quad \text{ by continuity equation} \\
& = - \mathbf{d} \int_{S} \mathbf{J}  \d \mathbf{S}
\end{align*}


\section{QUESTION 10}

\[
U = \frac{1}{2} \int \rho (\mathbf{r}) \phi(\mathbf{r}) \;\d^3 \mathbf{r}.
\]
Hence we obtain
\begin{align*}
U &= \frac{\varepsilon_0}{2}\int (\nabla\cdot \mathbf{E}) \phi \;\d^3 \mathbf{r}\\
&= \frac{\varepsilon_0}{2}\int [\nabla\cdot (\mathbf{E}\phi) - \mathbf{E}\cdot \nabla \phi]\;\d^3 \mathbf{r}.
\end{align*}
The first term is a total derivative and vanishes. In the second term, we use the definition $\mathbf{E} = -\nabla \phi$ and obtain


	\[
	U = \frac{\varepsilon_0}{2}\int \mathbf{E}\cdot \mathbf{E} \;\d^3 \mathbf{r}.
	\]


This result shows that the potential energy depends only on the field itself, and not the charges. 


Next, considering a charge $ Q $ contained within some compact region $ V $; we have

\begin{align*}
\phi(\mathbf{r}) & = \frac{1}{4 \pi \varepsilon_{0}} \int_{V} \frac{\rho(\mathbf{r}')}{| \mathbf{r} - \mathbf{r}' |} \; \d^{3} \mathbf{r}
\end{align*}

Here, $ \rho(\mathbf{r}') = Q $, and using spherical polars, we model $ \mathbf{r}(r,\theta,\phi) $ in the usual way; take our point $ \mathbf{r} = (a,0,0) $, $ \mathbf{r}' = (r \sin \theta \cos \phi, r \sin \theta \sin \phi, r \cos \theta )  $, so $ | \mathbf{r} - \mathbf{r}' | = \sqrt{r^{2} \sin^{2} \theta + (a - r \cos \theta)^{2}} = \sqrt{a^{2} - 2 a r \cos \theta + r^{2}}  $


\begin{align*}
\phi(\mathbf{r}) 
\end{align*}

Did not finish due to time constraints.



\section{QUESTION 11}



\end{document}