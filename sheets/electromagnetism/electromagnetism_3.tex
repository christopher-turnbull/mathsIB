\documentclass[a4paper]{article}
\usepackage{amsmath}
\def\npart {IB}
\def\nterm {Lent}
\def\nyear {2018}
\def\nlecturer {Dr. Warnick (cmw50@cam.ac.uk)}
\def\ncourse {Electromagnetism Example Sheet 3}

\input{header}

\newtheorem*{soln}{Solution}

\renewcommand{\thesection}{}
\renewcommand{\thesubsection}{\arabic{section}.\arabic{subsection}}
\makeatletter
\def\@seccntformat#1{\csname #1ignore\expandafter\endcsname\csname the#1\endcsname\quad}
\let\sectionignore\@gobbletwo
\let\latex@numberline\numberline
\def\numberline#1{\if\relax#1\relax\else\latex@numberline{#1}\fi}
\makeatother


\begin{document}
	
\maketitle

\section{QUESTION 1}


 We use cylindrical polar coordinates $(r, \phi, z)$, where $z$ is along the direction of the current, and $r$ points in the radial direction.
\begin{center}
	\begin{tikzpicture}
	\draw (0, 1.5) circle [x radius=0.5, y radius = 0.15];
	\draw [dashed] (0.5, -1.5) arc (0: 180:0.5 and 0.15);
	\draw (-0.5, -1.5) arc (180: 360:0.5 and 0.15);
	\draw (0.5, 1.5) -- (0.5, -1.5);
	\draw (-0.5, 1.5) -- (-0.5, -1.5);
	
	\draw [->] (0, 1.5) -- (0, 2) node [above] {$I$};
	
	\draw [fill = gray!50!white] circle [x radius = 1, y radius = 0.3];
	\draw [fill = white] circle [x radius = 0.5, y radius = 0.15];
	\draw [fill = white, draw = none] (-0.5, 0) rectangle (0.5, 1);
	\draw (-0.5, 0) -- (-0.5, 1);
	\draw (0.5, 0) -- (0.5, 1);
	\node at (1, 0) [right] {$S$};
	
	\draw [->] (3, 0) -- (3, 1) node [above] {$z$};
	\draw [->] (3, 0) -- (4, 0) node [right] {$r$};
	\end{tikzpicture}
\end{center}


We know that inside a conductor, $ \mathbf{E} = 0 $, and at the surface, $ \mathbf{E}_{\parallel} = 0 $. 


%Let the $ a $ be the radius of the cylinder, so 

%\[ \mathbf{E} = \begin{cases} 0  & \text{ if } r < a \\ E(r) \hat{\mathbf{r}} & \text{ if } r = a \end{cases} \]

Since

\[ \mathbf{J} = \sigma \mathbf{E} \]

we know that inside the conductor, $ \mathbf{J} = 0 $. Hence the current, given by $ I = \int_{S} \mathbf{J} \cdot \d s $, must be constant across the cylinder. 


Using the result
  \[
\hat{\mathbf{n}}\cdot \mathbf{E}_{\text{outside}} - \hat{\mathbf{n}}\cdot \mathbf{E}_{\text{inside}} = \frac{k}{\varepsilon_0}.
\]

where $ k $ is the surface current density. Since $ \mathbf{E}_{\text{inside}} = 0 $, dotting with $ \hat{\mathbf{r}} = \hat{\mathbf{n}} $ gives

\[ \mathbf{E}_{\text{outside}} = \frac{k}{\varepsilon_0} \hat{\mathbf{r}}
\]

Maxwell's equations say that $ \nabla \times \mathbf{E} = - \frac{\partial \mathbf{B} }{\partial t} $, and since $ \mathbf{E} $ is constant, we also have that $ \mathbf{B} $ is constant. 

I think inside the conductor $ \mathbf{B} = 0 $ but not sure.



Pontying vector: $ \mathbf{S} = \frac{1}{\mu_{0}} \mathbf{E} \times \mathbf{B}  $.

\begin{defi}[Joule heating]
	\emph{Joule heating} is the energy lost in a circuit due to friction. It is given by
	\[
	\frac{\d W}{\d t} = I^2 R.
	\]
\end{defi}

From the definition of resistivity and conductivity, we have $ R = \frac{L}{A \sigma} $, which is simply $ 1/\sigma $ if we're talking about unit lengths.




Not sure how to finish this off. 


 

\section{QUESTION 2}

\begin{center}
	\begin{tikzpicture}
	\draw [fill=gray] (0, 2) rectangle (1, -2);
	\draw [->] (-2, 0) -- (-1, 0) node [pos =0.5, above] {$\mathbf{E}_\mathrm{inc}$};
	\end{tikzpicture}
\end{center}

We have

\[ \mathbf{E} = E_{0} \hat{\mathbf{x}} e^{i(kz - \omega t)} -  E_{0} \hat{\mathbf{x}} e^{i(-kz - \omega t)}   \]

\[ \mathbf{B} = \frac{E_{0}}{c} \hat{\mathbf{y}} e^{i(kz - \omega t)} -  \frac{E_{0}}{c} \hat{\mathbf{y}} e^{i(-kz - \omega t)}   \]

The Maxwell equations in free space $ \nabla \cdot \mathbf{E} = 0 $,$ \nabla  \cdot \mathbf{B} = 0  $ are trivially satisfied as $ E_{x} $ and $ B_{y} $ have no dependence on $ x $ and $ y $ respectively. Next we have to satisfy

\begin{align*}
\nabla  \times \mathbf{E} & = - \frac{\partial \mathbf{B} }{\partial t} \\
\nabla  \times \mathbf{B} & = \mu_{0}\varepsilon_{0} \frac{\partial \mathbf{E} }{\partial t} 
\end{align*}

Noting that $ \nabla \times E(z) \hat{\mathbf{x}} = E'(z) \hat{\mathbf{y}} $ and $ \nabla \times B(z) \hat{\mathbf{y}} = - B'(z) \hat{\mathbf{x}} $, we have 

\begin{align*}
\nabla  \times \mathbf{E} & = i k E_{0} \hat{\mathbf{y}} e^{i(kz - \omega t)} - ( ik E_{0} \hat{\mathbf{y}} e^{i(-kz - \omega t)}  )
\end{align*}

and

\begin{align*}
\frac{\partial \mathbf{B} }{\partial t} & = - i \omega \frac{E_{0}}{c} \hat{\mathbf{y}} e^{i(kz - \omega t)} - ( - i \omega \frac{E_{0}}{c} \hat{\mathbf{y}} e^{i(-kz - \omega t)} )
\end{align*}

so the the first equation is satisfied if $ k = \omega / c  $. Next, we have 

\begin{align*}
\nabla \times \mathbf{B }& = - ik \frac{E_{0}}{c} \hat{\mathbf{x}} e^{i(kz - \omega t)} -  (- ik \frac{E_{0}}{c} \hat{\mathbf{x}} e^{i(-kz - \omega t)} )  \\
\end{align*}

so can see the next equation is satisfied if $ \frac{k}{c} = \omega \mu_{0} \varepsilon_{0} $, which is equivalent to $ k = \omega / c $ since $ c = \frac{1}{\sqrt{\mu_{0} \varepsilon_{0} }} $.

We know at the surface of a conductor, $ \mathbf{E}_{\parallel} =  0 $. Clearly $ \mathbf{E} \cdot \hat{y}|_{z = 0} = 0 $, and 


\[  \mathbf{E} \cdot \hat{x}|_{z = 0} = E_{0} e^{-\i \omega t} - E_{0} e^{-\i \omega t} = 0 \]


Not sure how to do rest/ ran out of time.








\section{QUESTION 3}

Plane wave solutions which propagate in the $y$ direction are independent of $x$ and $z$. So we can write our electric field as
\[
\mathbf{E}(\mathbf{x}) = (E_x(y, t), E_y(y, t), E_z(y, t)).
\]
Hence any derivatives wrt $x$ and $z$ are zero. Since we know that $\nabla \cdot \mathbf{E} = 0$, $E_y$ must be constant. We take $E_y = 0$. wlog, assume $E_z = 0$, ie the wave propagate in the $y$ direction and oscillates in the $x$ direction. Then we look for solutions of the form

\[
\mathbf{E} = (E(y, t), 0, 0),
\]
with
\[
\frac{1}{c^2}\frac{\partial^2 \mathbf{E}}{\partial t^2} - \frac{\partial^2 \mathbf{E}}{\partial y^2} = 0.
\]
The general solution is
\[
E(y, t) = f(y - ct) + g(y + ct).
\]
The most important solutions are the \emph{monochromatic} waves
\[
E(y,t) = E_0 \sin (ky - \omega t).
\]

Note that here, $ \omega = c k $. The given boundary conditions imposed are that $ E(0,t) = E(a,t) = 0 $. Thus

\[ \sin \left( \frac{a \omega}{c} - \omega t  \right) = 0, \quad \sin(- \omega t) = 0 \]

from which we deduce that both arguments are integer multiples of $ \pi $. It then follows that $ \frac{a \omega}{c} = n \pi $ for some $ n \in \Z $, and hence $ \omega = \frac{n \pi c}{a}  $.







\section{QUESTION 4}


We check that the given electric and magnetic fields satisfy the equations:

\[ \frac{1}{c^{2}} \frac{\partial^{2} \mathbf{E} }{\partial t^{2}} - \nabla^{2} \mathbf{E} = 0 , \quad \nabla \times \mathbf{E} = - \frac{\partial \mathbf{B} }{\partial t}  \]

First, 

\begin{align*}
 \nabla^{2} \mathbf{E} & = \frac{\partial^{2} \mathbf{E} }{\partial y^{2}} + \frac{\partial^{2} \mathbf{E} }{\partial z^{2}} \\
& = \left[  - \left( \frac{n \pi}{a} \right)^{2}  - k^{2} \right]  \omega A \sin \left( \frac{n \pi y}{a} \right) \sin(kz - \omega t) 
\end{align*}

Next,

\begin{align*}
\frac{1}{c^{2}} \frac{\partial^{2} \mathbf{E} }{\partial t^{2}} & = - \frac{\omega^{3}}{c^{2}} A \sin \left( \frac{n \pi y}{a} \right) \sin(kz - \omega t)  \\
\end{align*}

Thus the first equation reduces to

\[ \omega^{2} = \left(  \frac{n \pi c}{a} \right)^{2} + c^{2} k^{2}  \quad (*)\]

which is the criteria for the wave to propagate between the plates. 

Next, we have 

\begin{align*}
\quad \nabla \times \mathbf{E}  & =  \partial_{z} E_{x} \hat{\mathbf{y}}- \partial_{y} E_{x} \hat{\mathbf{z}}   \\
\end{align*}

which is trivial to show, matches the negative first time derivative of $ B_{y} $ and $ B_{z} $.

Next, we have $ \lambda = 2 \pi / k $ and

\[ k^{2} = \frac{\omega^{2}}{c^{2}} -  \left(  \frac{n \pi}{a} \right)^{2}   \]

from (*). Hence

\begin{align*}
 \frac{1}{\lambda^{2}} & = k^{2} / 4 \pi^{2}  \\
& = \left(  \frac{\omega}{2 \pi c} \right)^{2} - \frac{n^{2}}{4 a^{2}} 
\end{align*}

This suggests that $ \lambda_{\infty} = \frac{2 \pi c}{\omega} $ but I'm not sure why this would be true.

Also not sure if (*) is true; don't we have $ \omega^{2} = c^{2} k^2 $ for monochromatic waves?


\section{QUESTION 5}


Recall we define $\mathbf{E}$ and $\mathbf{B}$ in terms of $\phi$ and $\mathbf{A}$ as follows:

\begin{align*}
\mathbf{E} &= -\nabla\phi - \frac{\partial \mathbf{A}}{\partial t}\\
\mathbf{B} &= \nabla\times \mathbf{A}.
\end{align*}

Since this is an electromagnetic wave, it must satisfy $ \nabla  \cdot \mathbf{E} = 0 $. Now

\begin{align*}
\nabla  \cdot \frac{\partial \mathbf{A} }{\partial t} & = \partial_{j} [ - \omega \mathbf{A}_{0} e^{i(\mathbf{k} \cdot \mathbf{r} - \omega t )} ]_{j}  \\
& = - \omega [\mathbf{A}_{0}]_{j} k_{j} e^{i(\mathbf{k} \cdot \mathbf{r} - \omega t )} \\
& = - \omega \mathbf{A}_{0} \cdot \mathbf{k} e^{i(\mathbf{k} \cdot \mathbf{r} - \omega t )}
\end{align*}

and

\begin{align*}
\nabla  \cdot \nabla  \phi & = \nabla^{2} \phi \\
& = k^{2} \phi_{0} e^{i(\mathbf{k} \cdot \mathbf{r} - \omega t)}
\end{align*}

Therefore we must have

\[ \omega \mathbf{A}_{0} \cdot \mathbf{k} = k^{2} \phi_{0} \]







\section{QUESTION 6}

Under a boost by $v$ in the $x$-direction , we have
\begin{align*}
E_x' &= E_x\\
E_y' &= \gamma(E_y - vB_z)\\
E_z' &= \gamma(E_z + vB_y)\\
B_x' &= B_x\\
B_y' &= \gamma\left(B_y + \frac{v}{c^2}E_z\right)\\
B_z' &= \gamma\left(B_z - \frac{v}{c^2} E_y\right)
\end{align*}

The conditions are:

\[ E_{x} B_{x} + E_{y} B_{y} + E_{z} B_{z} = 0 \]

and

\[ E_{x}^{2} + E_{y}^{2}+ E_{z}^{2} - c^{2} \left( B_{x}^{2} + B_{y}^{2} + B_{z}^{2}  \right) \neq 0  \]

Not sure what I'm doing here.



\section{QUESTION 7}

We know that inside a conductor, $\mathbf{E} = 0$, and at the surface, $\mathbf{E}_{\parallel} = 0$. So $\mathbf{E}_0 \cdot \hat{\mathbf{y}}|_{x = 0} = 0$. This is clearly satisfied by our field, as it just becomes $ f(t) - f(t) = 0$. 


Maxwell's equations says $\nabla\times \mathbf{E} = -\frac{\partial \mathbf{B}}{\partial t}$. So

\begin{align*}
\nabla\times \mathbf{E} & = \hat{\mathbf{z}} \left[  \frac{\partial f(t_{-}) }{\partial x } - \frac{\partial f(t_{+}) }{\partial x } \right] \\
& =  
\end{align*}

Note sure how to link in the time derivatives of $ f $.

Under a boost by $v$ in the $x$-direction , we have
\begin{align*}
E_x' &= E_x\\
E_y' &= \gamma(E_y - vB_z)\\
E_z' &= \gamma(E_z + vB_y)\\
B_x' &= B_x\\
B_y' &= \gamma\left(B_y + \frac{v}{c^2}E_z\right)\\
B_z' &= \gamma\left(B_z - \frac{v}{c^2} E_y\right)
\end{align*}



\section{QUESTION 8}

\[
F_{\mu\nu} = \partial_\mu A_\nu -\partial_\nu A_\mu.
\]
Since this is antisymmetric, the diagonals are all 0, and $A_{\mu\nu} = -A_{\nu\mu}$. So this thing has $(d \times d - d)/2 = $ independent components (for $ d \geq 3 $).


\section{QUESTION 9}

The Lorentz force law in relativistic from is 

\[ \frac{\d P^{\mu}}{\d \tau} = q F^{\mu \nu} U_{\nu} \]

Here $ U_{\nu} = \gamma(c,0,-p_{0} /m,0 ) $, and 

\[   F^{\mu\nu} =
\begin{pmatrix}
0 & -E/c & 0 & 0\\
E/c & 0 & 0 & 0\\
0 & 0 & 0 & 0\\
0 & 0 & 0 & 0
\end{pmatrix} \]

So here, the RHS is $ (0,qE,0,0) $.

Then

\[ m \frac{\d \gamma u}{\d t} = q E \]

which gives 

\[ m \gamma u = q E t \]

where 

\[ \gamma(\mathbf{u}) = \frac{1}{\sqrt{ 1 - \frac{u^{2} + v^{2}}{c^{2}}}} \]








\section{QUESTION 10}




\end{document}