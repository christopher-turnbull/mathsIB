\documentclass[a4paper]{article}
\usepackage{amsmath}
\def\npart {IB}
\def\nterm {Michaelmas}
\def\nyear {2017}
\def\nlecturer {Dr. Saxton}
\def\ncourse {Methods Example Sheet 1}

\input{header}
\newtheorem*{soln}{Solution}

\renewcommand{\thesection}{}
\renewcommand{\thesubsection}{\arabic{section}.\arabic{subsection}}
\makeatletter
\def\@seccntformat#1{\csname #1ignore\expandafter\endcsname\csname the#1\endcsname\quad}
\let\sectionignore\@gobbletwo
\let\latex@numberline\numberline
\def\numberline#1{\if\relax#1\relax\else\latex@numberline{#1}\fi}
\makeatother


\begin{document}
	
\maketitle

\section{QUESTION 1}

\[ \frac{f(x_{+} + f(x_{-}))}{2} = \frac{1}{2} a_{0} + \sum_{n=1}^{\infty} \left[ a_{n} \cos \left( \frac{n \pi x}{L} \right) + b_{n} \sin \left( \frac{n \pi x}{L} \right)  \right]  \]

For $ f(x) = (x-1)^{2} $ on the interval $ -1 \leq x \leq 1 $, $ f(x) $ is an even function, thus $ b_{n} = 0 $. We have $ L = 1 $, and 

\begin{align*}
\frac{1}{2} a_{0} & = \frac{1}{2L} \int_{-L}^{L} f(x) \; \d x \\
& = \frac{1}{2} \int_{-1}^{1} x^{4} - 2x^{2} + 1 \; \d x \\
& = \int_{0}^{1} x^{4} - 2x^{2} + 1 \; \d x \\
& = \frac{8}{15}
\end{align*}

and

\begin{align*}
a_{n} & = \frac{1}{L} \int_{-1}^{1} f(x) \cos \left( \frac{n \pi x}{L} \right) \; \d x   \\
& = \int_{-1}^{1} x^{4} \cos  n \pi x  \; \d x - 2 \int_{-1}^{1} x^{2} \cos  n \pi x   \; \d x + \int_{-1}^{1} \cos  n \pi x   \; \d x
\end{align*}

Evaluating each integral separately, we have:

\begin{enumerate}
	\item  \[ \int_{-1}^{1} \cos  n \pi x   \; \d x = \left[ \frac{\sin n \pi x}{n \pi} \right]_{-1}^{-1} = 0  \]
	
	as $ \sin n \pi x = 0 \; \forall \; n $
	
	\item By parts, 
	
	\begin{align*}
	\int_{-1}^{1} x^{2} \cos  n \pi x   \; \d x & = \left[ \frac{x^{2} \sin n \pi x}{n \pi} - \frac{1}{n \pi }\int 2x \sin n \pi x \; \d x \right]_{-1}^{1}  \\
	& = - \frac{2}{n \pi} \int_{-1}^{1} x \sin n \pi x \; \d x
	\end{align*}
	
	and
	
	\begin{align*}
	\int_{-1}^{1} x \sin n \pi x \; \d x & = \left[  \frac{- x \cos n \pi x}{n \pi} + \frac{1}{n \pi} \int \cos n \pi x \; \d x   \right]_{-1}^{1}  \\
	& = \frac{-2 \cos n \pi x }{(n \pi )^{2}}
	\end{align*}
	
	Thus the second integral contributes to give
	
	\[ - \frac{8 cos n \pi x}{(n \pi)^{2}} \]
	
	
	\item 
	
	\begin{align*}
	\int_{-1}^{1} x^{4} \cos  n \pi x  \; \d x & = \left[ \frac{x^{4} \sin n \pi x}{n \pi} - \frac{1}{n \pi }\int 4x^{3} \sin n \pi x \; \d x  \right]_{-1}^{1} \\
	& = - \frac{4}{n \pi} \int_{-1}^{1} x^{3} \sin n \pi x \; \d x
	\end{align*}
	
	and 
	
	\begin{align*}
	\int_{-1}^{1} x^{3} \sin n \pi x \; \d x & = \left[ \frac{-x^{3} \cos n \pi x}{n \pi} + \frac{1}{n \pi} \int 3 x^{2} \cos n \pi x \right]_{-1}^{1}  \\
	& = \frac{-2 \cos n \pi}{n \pi} + \frac{3}{n \pi} \int_{-1}^{1} x^{2} \cos n \pi x \; \d x
	\end{align*}
	
	Whence
	
	\begin{align*}
	\int_{-1}^{1} x^{4} \cos  n \pi x  \; \d x & = \frac{8\cos n \pi}{n \pi} - \frac{12}{(n \pi)^{2} } \int_{-1}^{1} x^{2} \cos n \pi x \; \d x  \\
	& = \frac{8\cos n \pi}{n \pi} - \frac{48 \cos n \pi }{(n \pi)^{4}}
	\end{align*}
	
	using (ii).
	
	
	
\end{enumerate}

Finally,

\begin{align*}
a_{n} & = - \frac{48 \cos n \pi }{(n \pi)^{4}} \\
& = \frac{48(-1)^{n+1}}{(n \pi)^{4}}
\end{align*}

as $ \cos n \pi x = (-1)^{n}$ 

Hence the Fourier Series is given by

\begin{align*}
f(x) & = \frac{1}{2}a_{0} + \sum_{n=1}^{\infty} a_{n} \cos n \pi x \\
& = \frac{8}{15} + \frac{48}{\pi^{4}}\sum_{n=1}^{\infty} \frac{(-1)^{n+1}}{n^{4}} \cos n \pi x
\end{align*}

\begin{center}
	\begin{tikzpicture}
	\draw [->] (-5, 0) -- (5, 0) node [right] {$x$};
	\draw [->, use as bounding box] (0, -2.5) -- (0, 2.5) node [above] {$f(x)$};
	
	\end{tikzpicture}
\end{center}


$ f(x) $ satisfies the Dirichlet conditions. The $ 1^{\text{st}} $ derivative is the lowest derivative which is discontinuous (at the endpoints, as $ f(x) $ even fn $ \Rightarrow \; f'(x) $ odd), so Fourier coefficients are $ \Oc(\frac{1}{n^{2}}) $ as $ n \to \infty $

\section{QUESTION 2}

Extending on range $ (-\pi,\pi) $ so $ L = \pi $ and

\begin{enumerate}[label = (\alph*)]
	\item \[ \frac{f(x_{+} + f(x_{-}))}{2} =  \sum_{n=1}^{\infty} b_{n} \sin  n x   \]
	
	where 
	\begin{align*}
	b_{n} & = \frac{2}{L} \int_{0}^{L} f(x) \sin \left( \frac{n \pi x}{L} \right) \; \d x  \\
	& = \frac{2}{\pi} \int_{0}^{\pi} x^{2} \sin n x \; \d x
	\end{align*}
	
	Integrating by parts,
	
	\begin{align*}
	\int_{0}^{\pi} x^{2} \sin n x \; \d x & = \left[  \frac{- x^{2} \cos n x }{n} + \frac{1}{n} \int 2x \cos n x \; \d x \right]_{0}^{\pi}  \\
	& = \frac{- \pi^{2} \cos n \pi}{n} + \frac{2}{n} \int_{0}^{\pi} x \cos n x \; \d x
	\end{align*}
	
	and once again,
	
	\begin{align*}
	\int_{0}^{\pi} x \cos n x \; \d x & =\left[ \frac{x \sin n x}{n} - \frac{1}{n} \int \sin n x \; \d x \right]_{0}^{\pi}  \\
	& = - \frac{1}{n} \int_{0}^{\pi} \sin n x \; \d x \\
	& = - \frac{1}{n} \left[ - \frac{\cos n x}{n} \right]_{0}^{\pi}\\
	& = \frac{1}{n^{2}} (\cos n \pi - 1) 
	\end{align*}
	
	Back substituting in, 
	
	\begin{align*}
	b_{n} & = \frac{2}{\pi} \left( \frac{- \pi^{2} \cos n \pi}{n} + \frac{2}{n^{3}} (\cos n \pi - 1)   \right)  \\
	& = \frac{2}{\pi n^{3}} \left(   - 2 + (2 - (\pi n)^{2} )\cos n \pi       \right) 
	\end{align*}
	
	Hence Fourier sine series given by:
	
	\[ f(x) = \sum_{n=1}^{\infty} \frac{2}{\pi n^{3}} \left(   - 2 + (2 - (\pi n)^{2} ) (-1)^{n} \right) \sin n x  \]
	
	\item Similarly,
	
	\[ \frac{f(x_{+} + f(x_{-}))}{2} =  \frac{a_{0}}{2} + \sum_{n=1}^{\infty} a_{n} \cos n x  \]
	
	where 
	
	\begin{align*}
	\frac{a_{0}}{2} & = \frac{1}{L} \int_{0}^{L} f(x) \; \d x \\
	& = \frac{1}{\pi} \int_{0}^{\pi} x^{2} \; \d x\\
	& = \frac{\pi^{2}}{3}
	\end{align*}
	
	and 
	
	\begin{align*}
	a_{n} & = \frac{2}{L} \int_{0}^{L} f(x) \cos \left( \frac{n \pi x}{L} \right) \; \d x  \\
	& = \frac{2}{\pi} \int_{0}^{\pi} x^{2} \cos n x \; \d x
	\end{align*}
	
	Integrating by parts,
	
	
	\begin{align*}
	\int_{0}^{\pi} x^{2} \cos n x \; \d x & = \left[ \frac{x^{2}\sin n x}{n}  - \frac{1}{n}\int 2x \sin n x \; \d x \right]_{0}^{\pi} \\
	& = \frac{-2}{n} \int_{0}^{\pi} x \sin n x \; \d x
	\end{align*}
	
	and once again,
	
	\begin{align*}
	\int_{0}^{\pi} x \sin n x \; \d x & = \left[  \frac{- x \cos n x}{n} + \frac{1}{n} \int \cos n x \; \d x \right]_{0}^{\pi} \\
	& = \frac{- \pi \cos n \pi}{n} + \frac{1}{n} \left[ \frac{\sin n x}{n \pi} \right]_{0}^{\pi} \\
	& = \frac{- \pi \cos n \pi}{n}
	\end{align*}
	
	Thus
	
	\[ a_{n} = \frac{4}{n^{2}} \cos n \pi  \]
	
	and the Fourier cosine series is given by
	
	\[ f(x) = \frac{\pi^{2}}{3} + \sum_{n=0}^{\infty} \frac{4}{n^{2}} (-1)^{n} \cos n x \]
	
\end{enumerate}

\begin{enumerate}
	\item \begin{center}
		\begin{tikzpicture}
		\draw [->] (-5, 0) -- (5, 0) node [right] {$x$};
		\draw [->, use as bounding box] (0, -2.5) -- (0, 2.5) node [above]
		{$f(x)$};
		\draw node at (-4.5,0) [below] {$ -6 \pi$}; 
		\draw node at (-3,0) [below] {$ -4 \pi$}; 
		\draw node at (-1.5,0) [below] {$ -2 \pi$}; 
		\draw node at (1.5,0) [below] {$ 2 \pi$}; 
		\draw node at (3,0) [below] {$ 4 \pi$}; 
		\draw node at (4.5,0) [below] {$ 6 \pi$}; 
		
		\end{tikzpicture}
	\end{center}
	
	\item \begin{center}
		\begin{tikzpicture}
		\draw [->] (-5, 0) -- (5, 0) node [right] {$x$};
		\draw [->, use as bounding box] (0, -2.5) -- (0, 2.5) node [above] {$f(x)$};
		\draw node at (-4.5,0) [below] {$ -6 \pi$}; 
		\draw node at (-3,0) [below] {$ -4 \pi$}; 
		\draw node at (-1.5,0) [below] {$ -2 \pi$}; 
		\draw node at (1.5,0) [below] {$ 2 \pi$}; 
		\draw node at (3,0) [below] {$ 4 \pi$}; 
		\draw node at (4.5,0) [below] {$ 6 \pi$}; 
		
		\end{tikzpicture}
	\end{center}

\end{enumerate}

Fourier series for $ g(x) = 2x $ (odd function) in the range $ (-\pi,\pi) $ given by


\[ \frac{f(x_{+} + f(x_{-}))}{2} =  \sum_{n=1}^{\infty} b_{n} \sin  n x   \]

where 
\begin{align*}
b_{n} & = \frac{1}{L} \int_{-L}^{L} f(x) \sin \left( \frac{n \pi x}{L} \right) \; \d x  \\
& = \frac{2}{\pi} \int_{-\pi}^{\pi} x \sin n x \; \d x
\end{align*}

Integrating by parts,

\begin{align*}
\int_{-\pi}^{\pi} x \sin n x \; \d x & = \left[  \frac{- x \cos n x}{n} + \frac{1}{n} \int \cos n x \; \d x \right]_{-\pi}^{\pi} \\
& = \frac{- 2 \pi \cos n \pi}{n} + \frac{1}{n} \left[ \frac{\sin n x}{n \pi} \right]_{-\pi}^{\pi} \\
& = \frac{- 2 \pi (-1)^{n}}{n}
\end{align*}

Whence 

\[ g(x) = \sum_{n=1}^{\infty} \frac{ 4 \pi^{2} (-1)^{n+1}}{n^{2}} \sin  n x  \]


Fourier series for $ h(x) = 2| x | $ (even function) in the range $ (-\pi,\pi) $ given by


\[ \frac{f(x_{+} + f(x_{-}))}{2} =  \frac{1}{2} a_{0} +  \sum_{n=1}^{\infty} a_{n} \cos  n x   \]


where 

\begin{align*}
\frac{1}{2} a_{0} & = \frac{1}{2L} \int_{-L}^{L} f(x) \; \d x \\
& = \frac{1}{2\pi} \int_{-\pi}^{\pi} 2| x | \; \d x \\
& = \frac{2}{\pi} \int_{0}^{\pi} x \; \d x \\
& = \pi
\end{align*}
and 
\begin{align*}
a_{n} & = \frac{1}{L} \int_{-L}^{L} f(x) \cos \left( \frac{n \pi x}{L} \right) \; \d x  \\
& = \frac{2}{\pi} \int_{-\pi}^{\pi} | x | \cos n x \; \d x
\end{align*}





Integrating by parts,

\begin{align*}
\int_{-\pi}^{\pi} | x | \cos n x \; \d x & = 2 \int_{0}^{\pi} x  \cos n x \; \d x \\
& = 2 \left[ \frac{x \sin n x}{n} - \frac{1}{n} \int \sin n x \; \d x \right]_{0}^{\pi}  \\
& = - \frac{2}{n} \int_{0}^{\pi} \sin n x \; \d x \\
& = - \frac{2}{n} \left[ - \frac{\cos n x}{n} \right]_{0}^{\pi}\\
& = \frac{2}{n^{2}} (\cos n \pi - 1) 
\end{align*}

Whence

\[ h(x) = \pi +  \sum_{n=1}^{\infty} \frac{2}{n^{2}} ((-1)^{n} - 1)  \cos  n x \]







\section{QUESTION 3}


$ f(x) = e^{x} $ on $ (-\pi,\pi) $ has Fourier series given by 


\[ \frac{f(x_{+} + f(x_{-}))}{2} = \frac{1}{2} a_{0} + \sum_{n=1}^{\infty} \left[ a_{n} \cos n x + b_{n} \sin n x  \right]  \]


where 

\begin{align*}
\frac{1}{2} a_{0} & = \frac{1}{2\pi} \int_{-\pi}^{\pi} e^{x} \; \d x \\
& = \frac{1}{2\pi} \left(  e^{\pi} - e^{-\pi} \right) \\
& = \frac{1}{\pi} \sinh \pi
\end{align*}
and 
\begin{align*}
a_{n} & = \frac{1}{\pi} \underbrace{ \int_{-\pi}^{\pi} e^{x} \cos n x \; \d x}_{I_{a}}  \\
I_{a} & =  \left[   e^{x} \cos n x + \int e^{x} n \sin n x  \; \d x  \right]_{-\pi}^{\pi} \\
& = (e^{\pi} - e^{-\pi}  )\cos n \pi + n \int_{-\pi}^{\pi} e^{x} \sin n x \;\d x \\
& = 2 \sinh \pi \;  (-1)^{n} + n \left[  e^{x} \sin n x - \int e^{x} n \cos x \; \d x  \right]_{-\pi}^{\pi} \\
& =  2 \sinh \pi \;  (-1)^{n} + -n^{2} \int_{-\pi}^{\pi} e^{x} \cos n x \; \d x \\
& =  2 \sinh \pi \;  (-1)^{n} + -n^{2} I_{a}
\end{align*}

Hence 

\[ a_{n} = \frac{1}{\pi} I_{a} \qquad I_{a} = \frac{2}{1 + n^{2}} \sinh \pi (-1)^{n} \]

Also,

\[ b_{n} =  \frac{1}{\pi} \underbrace{\int_{-\pi}^{\pi}  e^{x} \sin n x \; \d x }_{I_{b}} \]

\begin{align*}
I_{b} & = \left[  e^{x} \sin n x - \int e^{x} n \cos n x \; \d x  \right]_{-\pi}^{\pi}  \\
& = -n I_{a}
\end{align*}

\[ b_{n} = - \frac{n}{\pi} I_{a} \]

Combining these results, the Fourier series for $ e^{x} $ is given by

\begin{align*}
f(x) & = \frac{1}{\pi} \sinh \pi  + \sum_{n=1}^{\infty} \left[  \left(  \frac{1}{\pi} \cos n x - \frac{n}{\pi} \sin n x \right) I_{a}  \right]  \\
& = \frac{1}{\pi} \sinh \pi + \frac{2}{\pi} \sinh \pi \sum_{n=1}^{\infty} \left[  (\cos n x - n \sin n x) \frac{(-1)^{n}}{1+n^{2}} \right] 
\end{align*}

Setting $ x = \pi $ yields

\[ e^{\pi} = \frac{1}{\pi} \sinh \pi + \frac{2}{\pi} \sinh \pi \sum_{n=1}^{\infty} \frac{1}{1+n^{2}} \]

Thus

\begin{align*}
\sum_{n=1}^{\infty} \frac{1}{1+n^{2}} & = \frac{\pi e^{\pi} - \sinh \pi }{2 \sinh \pi} \\
\end{align*}

Setting $ x = -\pi $ similarly yields

\begin{align*}
\sum_{n=1}^{\infty} \frac{1}{1+n^{2}} & = \frac{\pi e^{-\pi} - \sinh \pi }{2 \sinh \pi} \\
\end{align*}

Adding and dividing by two, 

\begin{align*}
\sum_{n=1}^{\infty} \frac{1}{1+n^{2}} & = \frac{\pi (e^{\pi} + e^{-\pi}) - 2 \sinh \pi }{4 \sinh \pi} \\
& = \frac{2 \pi \cosh \pi  - 2 \sinh \pi }{4 \sinh \pi} \\
& = \frac{1}{2} ( \pi \coth \pi - 1)
\end{align*}

\section{QUESTION 4}

\begin{enumerate}
	\item Reposing the Fourier Series of $ f(t) $ using complex variables,
	
	\begin{align*}
	f(t) & = \frac{a_{0}}{2} + \sum_{n=1}^{\infty}  \left[   \frac{a_{n}}{2} \left(   e^{\frac{i n \pi t}{L}} + e^{\frac{- i n \pi t}{L}} \right) + \frac{b_{n}}{2i} \left( e^{\frac{i n \pi t}{L}} - e^{\frac{-i n \pi t}{L}} \right)   \right]  \\
	& = \sum_{n=-\infty}^{\infty}  c_{n} e^{\frac{i n \pi t}{L}}, \\
	c_{n} & = \frac{a_{n} - i b_{n}}{2} \; n > 0 ; \\
	c_{-n} & = \frac{a_{n} + i b_{n}}{2} \; n > 0 ; \\
	c_{0} & = \frac{a_{0}}{2}
	\end{align*}
	
	Using the orthogonality of complex exponentials and the properties of complex Fourier coefficients, we deduce that
	
	\begin{align*}
	\int_{-L}^{L} \left[  f(t) \right]^{2} \; \d t  & = \sum_{n = - \infty}^{\infty} \sum_{m = -\infty}^{\infty} c_{n} c_{m} \int_{-T}^{T} \exp \left[   \frac{i \pi t (n + m)}{L} \right]  \; \d t \\
	& = \sum_{n = - \infty}^{\infty} \sum_{m = -\infty}^{\infty} c_{n} c_{m} 2 T \delta_{n[-m]}  \; \\ 
	& = 2 T \sum_{n = - \infty}^{\infty} c_{n} c_{-n} \\
	& = 2 T \sum_{n = - \infty}^{\infty} c_{n} c_{n}^{*} \\
	& = 2 T \sum_{n = - \infty}^{\infty} | c_{n} |^{2} \\
	\end{align*}
	
	This can be then re-expressed in terms of the $ a_{n} $ and $ b_{n} $ as 
	
	\[ \int_{-L}^{L} \left[  f(t) \right]^{2} \; \d t  = L  \left[   \frac{a_{0}^{2}}{2} +\sum_{n=1}^{\infty} (a_{n}^{2} + b_{n}^{2}) \right]  \]
	
	as required. 
	 
	\item 
	
	\begin{center}
		\begin{tikzpicture}
		\draw [->] (-5, 0) -- (5, 0) node [right] {$t$};
		\draw [->, use as bounding box] (0, -2.5) -- (0, 2.5) node [above] {$f(t)$};
		\draw [-, mblue] (0, 1) node [left] {$ 1 $} -- (5, 1);
		\draw [-, mblue] (-5, -1)  -- (0, -1)  node [right] {$ -1 $};
		\end{tikzpicture}
	\end{center}
	
	The unit amplitude square wave has Fourier series (odd function)
	
	\[ f(t) = \sum_{n=1}^{\infty}  b_{n} \sin \left(  \frac{n \pi t}{T} \right)  \]
	
	Frequencies less than $ \frac{9}{2} \pi T^{-1} $ correspond to terms in the Fourier series with $ \frac{n \pi}{T} <  \frac{9}{2} \pi T^{-1} $, ie. $ n = 1,2,3,4 $.
	
	Also,
	
	\begin{align*}
	b_{n} & = \frac{1}{T} \int_{-T}^{T} \\
	& = 
	\end{align*}
	
\end{enumerate}



\section{QUESTION 5}

\begin{enumerate}
	\item \begin{center}
	\begin{tikzpicture}
	\draw [->] (0, 0) -- (5, 0) node [right] {$x$};
	\draw [->, use as bounding box] (0, 0) -- (0, 4) node [above] {$f(x)$};
	
	\end{tikzpicture}
\end{center}


$ f(x) $ on $ (0,2\pi) $ has Fourier series given by 


\[ \frac{f(x_{+} + f(x_{-}))}{2} = \frac{1}{2} a_{0} + \sum_{n=1}^{\infty} \left[ a_{n} \cos n x + b_{n} \sin n x  \right]  \]


where 

\begin{align*}
\frac{1}{2} a_{0} & = \frac{1}{2\pi} \int_{0}^{2\pi} f(x) \; \d x \\
& = \frac{1}{2\pi} \int_{\pi}^{2\pi} 1 \; \d x \\
& = \frac{1}{2}
\end{align*}
and 
\begin{align*}
a_{n} & = \frac{1}{\pi} \int_{0}^{2\pi} f(x) \cos n x    \; \d x  \\
& = \frac{1}{\pi} \int_{\pi}^{2\pi} \cos n x \; \d x \\
& = \frac{1}{\pi} \left[  \frac{1}{n} \sin n x \right]_{\pi}^{2\pi} \\
& = 0
\end{align*}

and 

\begin{align*}
b_{n} & = \frac{1}{\pi} \int_{0}^{2\pi} f(x) \sin n x    \; \d x  \\
& = \frac{1}{\pi} \int_{\pi}^{2\pi} \sin n x \; \d x \\
& = \frac{1}{\pi} \left[  -\frac{1}{n} \cos n x \right]_{\pi}^{2\pi} \\
& = - \frac{1}{\pi n} \left[ \cos n x  \right]_{\pi}^{2\pi} \\
& = 0 \text{ if }  n \text{ even or } - \frac{2}{n \pi} \text{ if } n \text{ odd}  
\end{align*}

Hence 

\[ f(x) = \frac{1}{2}  - \frac{2}{\pi} \sum_{n=1}^{\infty} \frac{\sin(2n -1)x}{2n - 1}  \]

\item Taking the hint, differentiating term by term gives

\[ \frac{\d }{\d x} [S_{n}(x)] = \frac{2}{\pi} \sum_{n=1}^{N} \cos(2n-1)x \]

Now

\begin{align*}
\sum_{n=1}^{N} \cos(2n-1)x & = \Re \left[  \sum_{n=1}^{N} e^{(2n-1)i} \right]   \\
& = \Re \left[  \frac{e^{(2N + 1)i} - e^{i}}{e^{2i} - 1} \right] 
\end{align*}
	

	
	
	
\end{enumerate}

\section{QUESTION 6}

Assuming the solution takes the form $ y \propto e^{\sigma x},$, we have

\[ y(x) = A \cos(\mu x) + B \sin(\mu x) \]

where $ A $ and $ B $ are constants, and $ \mu^{2} = \lambda $. Applying the boundary conditions, $ y(0) = 0 $ implies that $ A = 0 $. The other boundary condition implies

\[ B \sin \mu + B \mu \cos \mu = 0 \]

\[  \mu = - \tan \mu  \] 

This eigenvalue equation has an infinite number of solutions, $ \mu_{n} $ (and hence there an infinite number of positive eigenvalues $ \lambda_{n} = \mu_{n}^{2} $).

As $ n \to \infty $, $ \mu_{n} \to \infty $, so $ \mu $ is close to an odd multiple of $\frac{\pi}{2} $, ie. $ \mu_{n} \approx (2n+1)\pi/2 $, and hence $ \lambda_{n} \approx (2n+1)^{2}\pi^{2} / 4 $


\section{QUESTION 7}

\begin{enumerate}
	\item $ p(x) = \exp \left(  \int^{x} \frac{-2u}{1-u^{2}} \; \d u \right) = (1-x^{2}) $, thus integrating factor is $ -\frac{1}{1-x^{2}} \left( (1-x^{2}) \right) = -1   $. We can then rewrite the equation as
	
	\[ -(1-x^{2})y'' + 2x y' - n(n+1) y = 0  \] 
	
	and
	
	\[ -\frac{\d }{\d x} \left( (1-x^{2}) \frac{\d y}{\d x} \right) - n(n+1)y = 0  \]
	
	\item
	
	
	\begin{align*}
	p(x) & = \exp \left(  \int^{x} \frac{(1+a+b)u - c}{u(u-1)} \; \d u \right) \\
	& = \exp \left(  \int^{x} \frac{c}{u} + \frac{1 + a + b - c}{u-1} \; \d u \right) \\
	& = \exp \left(  c \log x + (1 + a + b - c)\log (x-1)  \right) \\
	& = x^{c} + (x-1)^{1 + a + b - c}
	\end{align*}
	
	Thus the required integrating factor is 
	
	\[ - \frac{x^{c} + (x-1)^{1 + a + b - c}}{x(x-1)} \]
	
	The equation becomes
	
	\[ - (x^{c} + (x-1)^{1 + a + b - c}) y''  + -[ (1 + a + b)x - c ] \frac{x^{c} + (x-1)^{1 + a + b - c}}{x(x-1)} y' - \frac{x^{c} + (x-1)^{1 + a + b - c}}{x(x-1)} ab y = 0 \]
	
	which, in Sturm-Liouville form, is 
	
	\[ -\frac{\d }{\d x} \left[  (x^{c} + (x-1)^{1 + a + b - c}) \frac{\d y }{\d x} \right] - \frac{x^{c} + (x-1)^{1 + a + b - c}}{x(x-1)} ab y = 0  \]
	
	\item Self-adjoint form, integrating factor $ - e^{4x} $, 
	
	\[ - \frac{\d }{\d x} \left( e^{4x} \frac{\d y}{\d x} \right) -4e^{4x} y = \lambda e^{4x} y \]
	
	weight function is hence $ e^{4x} $.
	
	Easier to consider original equation; assuming the solution takes the form $ y \propto e^{\sigma x}, \; \sigma $ satisfies the auxillary equation
	
	\[ \sigma^{2} + 4 \sigma + 4 + \lambda = 0 \Rightarrow \sigma = -2 \pm i\sqrt{\lambda}, \]
	
	\[ y(x) = Ae^{-2x} \cos(\mu x) + B e^{-2x} \sin(\mu x) \]
	
	where $ A $ and $ B $ are constants, and $ \mu^{2} = \lambda $. Applying the boundary conditions, $ y(0) = 0 $ implies that $ A = 0 $. The other boundary condition implies

	\begin{align*}
	& B e^{-2} \sin \mu  = 0 \\
	& \Rightarrow \mu = n \pi
	\end{align*}
	
	Thus infinite positive eigenvalues $ \lambda_{n} = n^{2} \pi^{2} $
	
	The associated eigenvectors are thus proportional to $ e^{-2x} \sin (n \pi x) $.
	
	Eigenvectors associated with distinct eigenvalues are indeed orthogonal on the interval, if the weight function $ e^{4x} $ is correctly included in the inner product integral $ I_{mn} $, $ (m \neq n) $ defined as
	
	\[ I_{mn} = \int_{0}^{1} e^{4x} Y_{n}(x)Y_{m}(x) \; \d x \]
	
	where $ Y_{n} $ are $ Y_{m} $ are normalized eigenfunction with distinct eigenvalues $ \lambda_{n} = n^{2} \pi^{2} $ and $ \lambda_{m} = m^{2} \pi^{2} $
	
	
	
	
	
\end{enumerate}



\section{QUESTION 8}

The Sturm-Louville problem

\[ - \frac{\d }{\d x} \left(  x \frac{\d u}{\d x} \right) = \lambda x u , \quad 0 < x < 1  \]

hence with weight function $ x $, can be reposed in original form as

\[ xu'' + u' + \lambda x u = 0 \]

We try $ u = \sum_{n=0}^{\infty} a_{n}x^{n} $, first writing the equation in equidimensional form by multiplying by $ x $

\[ (x^{2}u'') + (xu') + \lambda x^{2} u = 0 \]

However we cannot find any series solutions about $ x = 0 $, as it is an irregular singular point.









\section{QUESTION 9}





\end{document}