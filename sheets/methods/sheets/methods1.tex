\documentclass[a4paper]{article}
\usepackage{amsmath}
\def\npart {IB}
\def\nterm {Michaelmas}
\def\nyear {2017}
\def\nlecturer {Dr. Saxton}
\def\ncourse {Methods Example Sheet 1}

% Imports
\ifx \nauthor\undefined
  \def\nauthor{Christopher Turnbull}
\else
\fi

\author{Supervised by \nlecturer \\\small Solutions presented by \nauthor}
\date{\nterm\ \nyear}

\usepackage{alltt}
\usepackage{amsfonts}
\usepackage{amsmath}
\usepackage{amssymb}
\usepackage{amsthm}
\usepackage{booktabs}
\usepackage{caption}
\usepackage{enumitem}
\usepackage{fancyhdr}
\usepackage{graphicx}
\usepackage{mathdots}
\usepackage{mathtools}
\usepackage{microtype}
\usepackage{multirow}
\usepackage{pdflscape}
\usepackage{pgfplots}
\usepackage{siunitx}
\usepackage{slashed}
\usepackage{tabularx}
\usepackage{tikz}
\usepackage{tkz-euclide}
\usepackage[normalem]{ulem}
\usepackage[all]{xy}
\usepackage{imakeidx}

\makeindex[intoc, title=Index]
\indexsetup{othercode={\lhead{\emph{Index}}}}

\ifx \nextra \undefined
  \usepackage[pdftex,
    hidelinks,
    pdfauthor={Christopher Turnbull},
    pdfsubject={Cambridge Maths Notes: Part \npart\ - \ncourse},
    pdftitle={Part \npart\ - \ncourse},
  pdfkeywords={Cambridge Mathematics Maths Math \npart\ \nterm\ \nyear\ \ncourse}]{hyperref}
  \title{Part \npart\ --- \ncourse}
\else
  \usepackage[pdftex,
    hidelinks,
    pdfauthor={Christopher Turnbull},
    pdfsubject={Cambridge Maths Notes: Part \npart\ - \ncourse\ (\nextra)},
    pdftitle={Part \npart\ - \ncourse\ (\nextra)},
  pdfkeywords={Cambridge Mathematics Maths Math \npart\ \nterm\ \nyear\ \ncourse\ \nextra}]{hyperref}

  \title{Part \npart\ --- \ncourse \\ {\Large \nextra}}
  \renewcommand\printindex{}
\fi

\pgfplotsset{compat=1.12}

\pagestyle{fancyplain}
\lhead{\emph{\nouppercase{\leftmark}}}
\ifx \nextra \undefined
  \rhead{
    \ifnum\thepage=1
    \else
      \npart\ \ncourse
    \fi}
\else
  \rhead{
    \ifnum\thepage=1
    \else
      \npart\ \ncourse\ (\nextra)
    \fi}
\fi
\usetikzlibrary{arrows.meta}
\usetikzlibrary{decorations.markings}
\usetikzlibrary{decorations.pathmorphing}
\usetikzlibrary{positioning}
\usetikzlibrary{fadings}
\usetikzlibrary{intersections}
\usetikzlibrary{cd}

\newcommand*{\Cdot}{{\raisebox{-0.25ex}{\scalebox{1.5}{$\cdot$}}}}
\newcommand {\pd}[2][ ]{
  \ifx #1 { }
    \frac{\partial}{\partial #2}
  \else
    \frac{\partial^{#1}}{\partial #2^{#1}}
  \fi
}
\ifx \nhtml \undefined
\else
  \renewcommand\printindex{}
  \makeatletter
  \DisableLigatures[f]{family = *}
  \let\Contentsline\contentsline
  \renewcommand\contentsline[3]{\Contentsline{#1}{#2}{}}
  \renewcommand{\@dotsep}{10000}
  \newlength\currentparindent
  \setlength\currentparindent\parindent

  \newcommand\@minipagerestore{\setlength{\parindent}{\currentparindent}}
  \usepackage[active,tightpage,pdftex]{preview}
  \renewcommand{\PreviewBorder}{0.1cm}

  \newenvironment{stretchpage}%
  {\begin{preview}\begin{minipage}{\hsize}}%
    {\end{minipage}\end{preview}}
  \AtBeginDocument{\begin{stretchpage}}
  \AtEndDocument{\end{stretchpage}}

  \newcommand{\@@newpage}{\end{stretchpage}\begin{stretchpage}}

  \let\@real@section\section
  \renewcommand{\section}{\@@newpage\@real@section}
  \let\@real@subsection\subsection
  \renewcommand{\subsection}{\@@newpage\@real@subsection}
  \makeatother
\fi

% Theorems
\theoremstyle{definition}
\newtheorem*{aim}{Aim}
\newtheorem*{axiom}{Axiom}
\newtheorem*{claim}{Claim}
\newtheorem*{cor}{Corollary}
\newtheorem*{conjecture}{Conjecture}
\newtheorem*{defi}{Definition}
\newtheorem*{eg}{Example}
\newtheorem*{ex}{Exercise}
\newtheorem*{fact}{Fact}
\newtheorem*{law}{Law}
\newtheorem*{lemma}{Lemma}
\newtheorem*{notation}{Notation}
\newtheorem*{prop}{Proposition}
\newtheorem*{soln}{Solution}
\newtheorem*{thm}{Theorem}

\newtheorem*{remark}{Remark}
\newtheorem*{warning}{Warning}
\newtheorem*{exercise}{Exercise}

\newtheorem{nthm}{Theorem}[section]
\newtheorem{nlemma}[nthm]{Lemma}
\newtheorem{nprop}[nthm]{Proposition}
\newtheorem{ncor}[nthm]{Corollary}


\renewcommand{\labelitemi}{--}
\renewcommand{\labelitemii}{$\circ$}
\renewcommand{\labelenumi}{(\roman{*})}

\let\stdsection\section
\renewcommand\section{\newpage\stdsection}

% Strike through
\def\st{\bgroup \ULdepth=-.55ex \ULset}

% Maths symbols
\newcommand{\abs}[1]{\left\lvert #1\right\rvert}
\newcommand\ad{\mathrm{ad}}
\newcommand\AND{\mathsf{AND}}
\newcommand\Art{\mathrm{Art}}
\newcommand{\Bilin}{\mathrm{Bilin}}
\newcommand{\bket}[1]{\left\lvert #1\right\rangle}
\newcommand{\B}{\mathcal{B}}
\newcommand{\bolds}[1]{{\bfseries #1}}
\newcommand{\brak}[1]{\left\langle #1 \right\rvert}
\newcommand{\braket}[2]{\left\langle #1\middle\vert #2 \right\rangle}
\newcommand{\bra}{\langle}
\newcommand{\cat}[1]{\mathsf{#1}}
\newcommand{\C}{\mathbb{C}}
\newcommand{\CP}{\mathbb{CP}}
\newcommand{\cU}{\mathcal{U}}
\newcommand{\Der}{\mathrm{Der}}
\newcommand{\D}{\mathrm{D}}
\newcommand{\dR}{\mathrm{dR}}
\newcommand{\E}{\mathbb{E}}
\newcommand{\F}{\mathbb{F}}
\newcommand{\Frob}{\mathrm{Frob}}
\newcommand{\GG}{\mathbb{G}}
\newcommand{\gl}{\mathfrak{gl}}
\newcommand{\GL}{\mathrm{GL}}
\newcommand{\G}{\mathcal{G}}
\newcommand{\Gr}{\mathrm{Gr}}
\newcommand{\haut}{\mathrm{ht}}
\newcommand{\Id}{\mathrm{Id}}
\newcommand{\ket}{\rangle}
\newcommand{\lie}[1]{\mathfrak{#1}}
\newcommand{\Mat}{\mathrm{Mat}}
\newcommand{\N}{\mathbb{N}}
\newcommand{\norm}[1]{\left\lVert #1\right\rVert}
\newcommand{\normalorder}[1]{\mathop{:}\nolimits\!#1\!\mathop{:}\nolimits}
\newcommand\NOT{\mathsf{NOT}}
\newcommand{\Oc}{\mathcal{O}}
\newcommand{\Or}{\mathrm{O}}
\newcommand\OR{\mathsf{OR}}
\newcommand{\ort}{\mathfrak{o}}
\newcommand{\PGL}{\mathrm{PGL}}
\newcommand{\ph}{\,\cdot\,}
\newcommand{\pr}{\mathrm{pr}}
\newcommand{\Prob}{\mathbb{P}}
\newcommand{\PSL}{\mathrm{PSL}}
\newcommand{\Ps}{\mathcal{P}}
\newcommand{\PSU}{\mathrm{PSU}}
\newcommand{\pt}{\mathrm{pt}}
\newcommand{\qeq}{\mathrel{``{=}"}}
\newcommand{\Q}{\mathbb{Q}}
\newcommand{\R}{\mathbb{R}}
\newcommand{\RP}{\mathbb{RP}}
\newcommand{\Rs}{\mathcal{R}}
\newcommand{\SL}{\mathrm{SL}}
\newcommand{\so}{\mathfrak{so}}
\newcommand{\SO}{\mathrm{SO}}
\newcommand{\Spin}{\mathrm{Spin}}
\newcommand{\Sp}{\mathrm{Sp}}
\newcommand{\su}{\mathfrak{su}}
\newcommand{\SU}{\mathrm{SU}}
\newcommand{\term}[1]{\emph{#1}\index{#1}}
\newcommand{\T}{\mathbb{T}}
\newcommand{\tv}[1]{|#1|}
\newcommand{\U}{\mathrm{U}}
\newcommand{\uu}{\mathfrak{u}}
\newcommand{\Vect}{\mathrm{Vect}}
\newcommand{\wsto}{\stackrel{\mathrm{w}^*}{\to}}
\newcommand{\wt}{\mathrm{wt}}
\newcommand{\wto}{\stackrel{\mathrm{w}}{\to}}
\newcommand{\Z}{\mathbb{Z}}
\renewcommand{\d}{\mathrm{d}}
\renewcommand{\H}{\mathbb{H}}
\renewcommand{\P}{\mathbb{P}}
\renewcommand{\sl}{\mathfrak{sl}}
\renewcommand{\vec}[1]{\boldsymbol{\mathbf{#1}}}
%\renewcommand{\F}{\mathcal{F}}

\let\Im\relax
\let\Re\relax

\DeclareMathOperator{\adj}{adj}
\DeclareMathOperator{\Ann}{Ann}
\DeclareMathOperator{\area}{area}
\DeclareMathOperator{\Aut}{Aut}
\DeclareMathOperator{\Bernoulli}{Bernoulli}
\DeclareMathOperator{\betaD}{beta}
\DeclareMathOperator{\bias}{bias}
\DeclareMathOperator{\binomial}{binomial}
\DeclareMathOperator{\card}{card}
\DeclareMathOperator{\ccl}{ccl}
\DeclareMathOperator{\Char}{char}
\DeclareMathOperator{\ch}{ch}
\DeclareMathOperator{\cl}{cl}
\DeclareMathOperator{\cls}{\overline{\mathrm{span}}}
\DeclareMathOperator{\conv}{conv}
\DeclareMathOperator{\corr}{corr}
\DeclareMathOperator{\cosec}{cosec}
\DeclareMathOperator{\cosech}{cosech}
\DeclareMathOperator{\cov}{cov}
\DeclareMathOperator{\covol}{covol}
\DeclareMathOperator{\diag}{diag}
\DeclareMathOperator{\diam}{diam}
\DeclareMathOperator{\Diff}{Diff}
\DeclareMathOperator{\disc}{disc}
\DeclareMathOperator{\dom}{dom}
\DeclareMathOperator{\End}{End}
\DeclareMathOperator{\energy}{energy}
\DeclareMathOperator{\erfc}{erfc}
\DeclareMathOperator{\erf}{erf}
\DeclareMathOperator*{\esssup}{ess\,sup}
\DeclareMathOperator{\ev}{ev}
\DeclareMathOperator{\Ext}{Ext}
\DeclareMathOperator{\Fit}{Fit}
\DeclareMathOperator{\fix}{fix}
\DeclareMathOperator{\Frac}{Frac}
\DeclareMathOperator{\Gal}{Gal}
\DeclareMathOperator{\gammaD}{gamma}
\DeclareMathOperator{\gr}{gr}
\DeclareMathOperator{\hcf}{hcf}
\DeclareMathOperator{\Hom}{Hom}
\DeclareMathOperator{\id}{id}
\DeclareMathOperator{\image}{image}
\DeclareMathOperator{\im}{im}
\DeclareMathOperator{\Im}{Im}
\DeclareMathOperator{\Ind}{Ind}
\DeclareMathOperator{\Int}{Int}
\DeclareMathOperator{\Isom}{Isom}
\DeclareMathOperator{\lcm}{lcm}
\DeclareMathOperator{\length}{length}
\DeclareMathOperator{\Lie}{Lie}
\DeclareMathOperator{\like}{like}
\DeclareMathOperator{\Lk}{Lk}
\DeclareMathOperator{\mse}{mse}
\DeclareMathOperator{\multinomial}{multinomial}
\DeclareMathOperator{\orb}{orb}
\DeclareMathOperator{\ord}{ord}
\DeclareMathOperator{\otp}{otp}
\DeclareMathOperator{\Poisson}{Poisson}
\DeclareMathOperator{\poly}{poly}
\DeclareMathOperator{\rank}{rank}
\DeclareMathOperator{\rel}{rel}
\DeclareMathOperator{\Re}{Re}
\DeclareMathOperator*{\res}{res}
\DeclareMathOperator{\Res}{Res}
\DeclareMathOperator{\rk}{rk}
\DeclareMathOperator{\Root}{Root}
\DeclareMathOperator{\sech}{sech}
\DeclareMathOperator{\sgn}{sgn}
\DeclareMathOperator{\spn}{span}
\DeclareMathOperator{\stab}{stab}
\DeclareMathOperator{\St}{St}
\DeclareMathOperator{\supp}{supp}
\DeclareMathOperator{\Syl}{Syl}
\DeclareMathOperator{\Sym}{Sym}
\DeclareMathOperator{\tr}{tr}
\DeclareMathOperator{\Tr}{Tr}
\DeclareMathOperator{\var}{var}
\DeclareMathOperator{\vol}{vol}

\pgfarrowsdeclarecombine{twolatex'}{twolatex'}{latex'}{latex'}{latex'}{latex'}
\tikzset{->/.style = {decoration={markings,
                                  mark=at position 1 with {\arrow[scale=2]{latex'}}},
                      postaction={decorate}}}
\tikzset{<-/.style = {decoration={markings,
                                  mark=at position 0 with {\arrowreversed[scale=2]{latex'}}},
                      postaction={decorate}}}
\tikzset{<->/.style = {decoration={markings,
                                   mark=at position 0 with {\arrowreversed[scale=2]{latex'}},
                                   mark=at position 1 with {\arrow[scale=2]{latex'}}},
                       postaction={decorate}}}
\tikzset{->-/.style = {decoration={markings,
                                   mark=at position #1 with {\arrow[scale=2]{latex'}}},
                       postaction={decorate}}}
\tikzset{-<-/.style = {decoration={markings,
                                   mark=at position #1 with {\arrowreversed[scale=2]{latex'}}},
                       postaction={decorate}}}
\tikzset{->>/.style = {decoration={markings,
                                  mark=at position 1 with {\arrow[scale=2]{latex'}}},
                      postaction={decorate}}}
\tikzset{<<-/.style = {decoration={markings,
                                  mark=at position 0 with {\arrowreversed[scale=2]{twolatex'}}},
                      postaction={decorate}}}
\tikzset{<<->>/.style = {decoration={markings,
                                   mark=at position 0 with {\arrowreversed[scale=2]{twolatex'}},
                                   mark=at position 1 with {\arrow[scale=2]{twolatex'}}},
                       postaction={decorate}}}
\tikzset{->>-/.style = {decoration={markings,
                                   mark=at position #1 with {\arrow[scale=2]{twolatex'}}},
                       postaction={decorate}}}
\tikzset{-<<-/.style = {decoration={markings,
                                   mark=at position #1 with {\arrowreversed[scale=2]{twolatex'}}},
                       postaction={decorate}}}

\tikzset{circ/.style = {fill, circle, inner sep = 0, minimum size = 3}}
\tikzset{mstate/.style={circle, draw, blue, text=black, minimum width=0.7cm}}

\tikzset{commutative diagrams/.cd,cdmap/.style={/tikz/column 1/.append style={anchor=base east},/tikz/column 2/.append style={anchor=base west},row sep=tiny}}

\definecolor{mblue}{rgb}{0.2, 0.3, 0.8}
\definecolor{morange}{rgb}{1, 0.5, 0}
\definecolor{mgreen}{rgb}{0.1, 0.4, 0.2}
\definecolor{mred}{rgb}{0.5, 0, 0}

\def\drawcirculararc(#1,#2)(#3,#4)(#5,#6){%
    \pgfmathsetmacro\cA{(#1*#1+#2*#2-#3*#3-#4*#4)/2}%
    \pgfmathsetmacro\cB{(#1*#1+#2*#2-#5*#5-#6*#6)/2}%
    \pgfmathsetmacro\cy{(\cB*(#1-#3)-\cA*(#1-#5))/%
                        ((#2-#6)*(#1-#3)-(#2-#4)*(#1-#5))}%
    \pgfmathsetmacro\cx{(\cA-\cy*(#2-#4))/(#1-#3)}%
    \pgfmathsetmacro\cr{sqrt((#1-\cx)*(#1-\cx)+(#2-\cy)*(#2-\cy))}%
    \pgfmathsetmacro\cA{atan2(#2-\cy,#1-\cx)}%
    \pgfmathsetmacro\cB{atan2(#6-\cy,#5-\cx)}%
    \pgfmathparse{\cB<\cA}%
    \ifnum\pgfmathresult=1
        \pgfmathsetmacro\cB{\cB+360}%
    \fi
    \draw (#1,#2) arc (\cA:\cB:\cr);%
}
\newcommand\getCoord[3]{\newdimen{#1}\newdimen{#2}\pgfextractx{#1}{\pgfpointanchor{#3}{center}}\pgfextracty{#2}{\pgfpointanchor{#3}{center}}}

\def\Xint#1{\mathchoice
   {\XXint\displaystyle\textstyle{#1}}%
   {\XXint\textstyle\scriptstyle{#1}}%
   {\XXint\scriptstyle\scriptscriptstyle{#1}}%
   {\XXint\scriptscriptstyle\scriptscriptstyle{#1}}%
   \!\int}
\def\XXint#1#2#3{{\setbox0=\hbox{$#1{#2#3}{\int}$}
     \vcenter{\hbox{$#2#3$}}\kern-.5\wd0}}
\def\ddashint{\Xint=}
\def\dashint{\Xint-}

\newcommand\separator{{\centering\rule{2cm}{0.2pt}\vspace{2pt}\par}}

\newenvironment{own}{\color{gray!70!black}}{}

\newcommand\makecenter[1]{\raisebox{-0.5\height}{#1}}
\newtheorem*{soln}{Solution}

\renewcommand{\thesection}{}
\renewcommand{\thesubsection}{\arabic{section}.\arabic{subsection}}
\makeatletter
\def\@seccntformat#1{\csname #1ignore\expandafter\endcsname\csname the#1\endcsname\quad}
\let\sectionignore\@gobbletwo
\let\latex@numberline\numberline
\def\numberline#1{\if\relax#1\relax\else\latex@numberline{#1}\fi}
\makeatother


\begin{document}
	
\maketitle

\section{QUESTION 1}

\[ \frac{f(x_{+} + f(x_{-}))}{2} = \frac{1}{2} a_{0} + \sum_{n=1}^{\infty} \left[ a_{n} \cos \left( \frac{n \pi x}{L} \right) + b_{n} \sin \left( \frac{n \pi x}{L} \right)  \right]  \]

For $ f(x) = (x-1)^{2} $ on the interval $ -1 \leq x \leq 1 $, $ f(x) $ is an even function, thus $ b_{n} = 0 $. We have $ L = 1 $, and 

\begin{align*}
\frac{1}{2} a_{0} & = \frac{1}{2L} \int_{-L}^{L} f(x) \; \d x \\
& = \frac{1}{2} \int_{-1}^{1} x^{4} - 2x^{2} + 1 \; \d x \\
& = \int_{0}^{1} x^{4} - 2x^{2} + 1 \; \d x \\
& = \frac{8}{15}
\end{align*}

and

\begin{align*}
a_{n} & = \frac{1}{L} \int_{-1}^{1} f(x) \cos \left( \frac{n \pi x}{L} \right) \; \d x   \\
& = \int_{-1}^{1} x^{4} \cos  n \pi x  \; \d x - 2 \int_{-1}^{1} x^{2} \cos  n \pi x   \; \d x + \int_{-1}^{1} \cos  n \pi x   \; \d x
\end{align*}

Evaluating each integral separately, we have:

\begin{enumerate}
	\item  \[ \int_{-1}^{1} \cos  n \pi x   \; \d x = \left[ \frac{\sin n \pi x}{n \pi} \right]_{-1}^{-1} = 0  \]
	
	as $ \sin n \pi x = 0 \; \forall \; n $
	
	\item By parts, 
	
	\begin{align*}
	\int_{-1}^{1} x^{2} \cos  n \pi x   \; \d x & = \left[ \frac{x^{2} \sin n \pi x}{n \pi} - \frac{1}{n \pi }\int 2x \sin n \pi x \; \d x \right]_{-1}^{1}  \\
	& = - \frac{2}{n \pi} \int_{-1}^{1} x \sin n \pi x \; \d x
	\end{align*}
	
	and
	
	\begin{align*}
	\int_{-1}^{1} x \sin n \pi x \; \d x & = \left[  \frac{- x \cos n \pi x}{n \pi} + \frac{1}{n \pi} \int \cos n \pi x \; \d x   \right]_{-1}^{1}  \\
	& = \frac{-2 \cos n \pi x }{(n \pi )^{2}}
	\end{align*}
	
	Thus the second integral contributes to give
	
	\[ - \frac{8 cos n \pi x}{(n \pi)^{2}} \]
	
	
	\item 
	
	\begin{align*}
	\int_{-1}^{1} x^{4} \cos  n \pi x  \; \d x & = \left[ \frac{x^{4} \sin n \pi x}{n \pi} - \frac{1}{n \pi }\int 4x^{3} \sin n \pi x \; \d x  \right]_{-1}^{1} \\
	& = - \frac{4}{n \pi} \int_{-1}^{1} x^{3} \sin n \pi x \; \d x
	\end{align*}
	
	and 
	
	\begin{align*}
	\int_{-1}^{1} x^{3} \sin n \pi x \; \d x & = \left[ \frac{-x^{3} \cos n \pi x}{n \pi} + \frac{1}{n \pi} \int 3 x^{2} \cos n \pi x \right]_{-1}^{1}  \\
	& = \frac{-2 \cos n \pi}{n \pi} + \frac{3}{n \pi} \int_{-1}^{1} x^{2} \cos n \pi x \; \d x
	\end{align*}
	
	Whence
	
	\begin{align*}
	\int_{-1}^{1} x^{4} \cos  n \pi x  \; \d x & = \frac{8\cos n \pi}{n \pi} - \frac{12}{(n \pi)^{2} } \int_{-1}^{1} x^{2} \cos n \pi x \; \d x  \\
	& = \frac{8\cos n \pi}{n \pi} - \frac{48 \cos n \pi }{(n \pi)^{4}}
	\end{align*}
	
	using (ii).
	
	
	
\end{enumerate}

Finally,

\begin{align*}
a_{n} & = - \frac{48 \cos n \pi }{(n \pi)^{4}} \\
& = \frac{48(-1)^{n+1}}{(n \pi)^{4}}
\end{align*}

as $ \cos n \pi x = (-1)^{n}$ 

Hence the Fourier Series is given by

\begin{align*}
f(x) & = \frac{1}{2}a_{0} + \sum_{n=1}^{\infty} a_{n} \cos n \pi x \\
& = \frac{8}{15} + \frac{48}{\pi^{4}}\sum_{n=1}^{\infty} \frac{(-1)^{n+1}}{n^{4}} \cos n \pi x
\end{align*}

\begin{center}
	\begin{tikzpicture}
	\draw [->] (-5, 0) -- (5, 0) node [right] {$x$};
	\draw [->, use as bounding box] (0, -2.5) -- (0, 2.5) node [above] {$f(x)$};
	
	\end{tikzpicture}
\end{center}


$ f(x) $ satisfies the Dirichlet conditions. The $ 1^{\text{st}} $ derivative is the lowest derivative which is discontinuous (at the endpoints, as $ f(x) $ even fn $ \Rightarrow \; f'(x) $ odd), so Fourier coefficients are $ \Oc(\frac{1}{n^{2}}) $ as $ n \to \infty $

\section{QUESTION 2}

Extending on range $ (-\pi,\pi) $ so $ L = \pi $ and

\begin{enumerate}[label = (\alph*)]
	\item \[ \frac{f(x_{+} + f(x_{-}))}{2} =  \sum_{n=1}^{\infty} b_{n} \sin  n x   \]
	
	where 
	\begin{align*}
	b_{n} & = \frac{2}{L} \int_{0}^{L} f(x) \sin \left( \frac{n \pi x}{L} \right) \; \d x  \\
	& = \frac{2}{\pi} \int_{0}^{\pi} x^{2} \sin n x \; \d x
	\end{align*}
	
	Integrating by parts,
	
	\begin{align*}
	\int_{0}^{\pi} x^{2} \sin n x \; \d x & = \left[  \frac{- x^{2} \cos n x }{n} + \frac{1}{n} \int 2x \cos n x \; \d x \right]_{0}^{\pi}  \\
	& = \frac{- \pi^{2} \cos n \pi}{n} + \frac{2}{n} \int_{0}^{\pi} x \cos n x \; \d x
	\end{align*}
	
	and once again,
	
	\begin{align*}
	\int_{0}^{\pi} x \cos n x \; \d x & =\left[ \frac{x \sin n x}{n} - \frac{1}{n} \int \sin n x \; \d x \right]_{0}^{\pi}  \\
	& = - \frac{1}{n} \int_{0}^{\pi} \sin n x \; \d x \\
	& = - \frac{1}{n} \left[ - \frac{\cos n x}{n} \right]_{0}^{\pi}\\
	& = \frac{1}{n^{2}} (\cos n \pi - 1) 
	\end{align*}
	
	Back substituting in, 
	
	\begin{align*}
	b_{n} & = \frac{2}{\pi} \left( \frac{- \pi^{2} \cos n \pi}{n} + \frac{2}{n^{3}} (\cos n \pi - 1)   \right)  \\
	& = \frac{2}{\pi n^{3}} \left(   - 2 + (2 - (\pi n)^{2} )\cos n \pi       \right) 
	\end{align*}
	
	Hence Fourier sine series given by:
	
	\begin{align*}
	f(x)_{s} & = \sum_{n=1}^{\infty} \frac{2}{\pi n^{3}} \left(   - 2 + (2 - (\pi n)^{2} ) (-1)^{n} \right) \sin n x \\
	& = \sum_{n=1}^{\infty}\left\{  \frac{2 \pi (-1)^{n+1}  }{n} + \frac{4[(-1)^{n} - 1   ]}{n^{3} \pi}   \right\} \sin n x
	\end{align*}
	
	\item Similarly,
	
	\[ \frac{f(x_{+} + f(x_{-}))}{2} =  \frac{a_{0}}{2} + \sum_{n=1}^{\infty} a_{n} \cos n x  \]
	
	where 
	
	\begin{align*}
	\frac{a_{0}}{2} & = \frac{1}{L} \int_{0}^{L} f(x) \; \d x \\
	& = \frac{1}{\pi} \int_{0}^{\pi} x^{2} \; \d x\\
	& = \frac{\pi^{2}}{3}
	\end{align*}
	
	and 
	
	\begin{align*}
	a_{n} & = \frac{2}{L} \int_{0}^{L} f(x) \cos \left( \frac{n \pi x}{L} \right) \; \d x  \\
	& = \frac{2}{\pi} \int_{0}^{\pi} x^{2} \cos n x \; \d x
	\end{align*}
	
	Integrating by parts,
	
	
	\begin{align*}
	\int_{0}^{\pi} x^{2} \cos n x \; \d x & = \left[ \frac{x^{2}\sin n x}{n}  - \frac{1}{n}\int 2x \sin n x \; \d x \right]_{0}^{\pi} \\
	& = \frac{-2}{n} \int_{0}^{\pi} x \sin n x \; \d x
	\end{align*}
	
	and once again,
	
	\begin{align*}
	\int_{0}^{\pi} x \sin n x \; \d x & = \left[  \frac{- x \cos n x}{n} + \frac{1}{n} \int \cos n x \; \d x \right]_{0}^{\pi} \\
	& = \frac{- \pi \cos n \pi}{n} + \frac{1}{n} \left[ \frac{\sin n x}{n \pi} \right]_{0}^{\pi} \\
	& = \frac{- \pi \cos n \pi}{n}
	\end{align*}
	
	Thus
	
	\[ a_{n} = \frac{4}{n^{2}} \cos n \pi  \]
	
	and the Fourier cosine series is given by
	
	\[ f(x) = \frac{\pi^{2}}{3} + \sum_{n=0}^{\infty} \frac{4}{n^{2}} (-1)^{n} \cos n x \]
	
\end{enumerate}

\begin{enumerate}
	\item \begin{center}
		\begin{tikzpicture}
		\draw [->] (-5, 0) -- (5, 0) node [right] {$x$};
		\draw [->, use as bounding box] (0, -2.5) -- (0, 2.5) node [above]
		{$f(x)$};
		\draw node at (-4.5,0) [below] {$ -6 \pi$}; 
		\draw node at (-3,0) [below] {$ -4 \pi$}; 
		\draw node at (-1.5,0) [below] {$ -2 \pi$}; 
		\draw node at (1.5,0) [below] {$ 2 \pi$}; 
		\draw node at (3,0) [below] {$ 4 \pi$}; 
		\draw node at (4.5,0) [below] {$ 6 \pi$}; 
		
		\end{tikzpicture}
	\end{center}
	
	\item \begin{center}
		\begin{tikzpicture}
		\draw [->] (-5, 0) -- (5, 0) node [right] {$x$};
		\draw [->, use as bounding box] (0, -2.5) -- (0, 2.5) node [above] {$f(x)$};
		\draw node at (-4.5,0) [below] {$ -6 \pi$}; 
		\draw node at (-3,0) [below] {$ -4 \pi$}; 
		\draw node at (-1.5,0) [below] {$ -2 \pi$}; 
		\draw node at (1.5,0) [below] {$ 2 \pi$}; 
		\draw node at (3,0) [below] {$ 4 \pi$}; 
		\draw node at (4.5,0) [below] {$ 6 \pi$}; 
		
		\end{tikzpicture}
	\end{center}

\end{enumerate}

Fourier series for $ g(x) = 2x $ (odd function) in the range $ (-\pi,\pi) $ given by


\[ \frac{f(x_{+} + f(x_{-}))}{2} =  \sum_{n=1}^{\infty} b_{n} \sin  n x   \]

where 
\begin{align*}
b_{n} & = \frac{1}{L} \int_{-L}^{L} f(x) \sin \left( \frac{n \pi x}{L} \right) \; \d x  \\
& = \frac{2}{\pi} \int_{-\pi}^{\pi} x \sin n x \; \d x
\end{align*}

Integrating by parts,

\begin{align*}
\int_{-\pi}^{\pi} x \sin n x \; \d x & = \left[  \frac{- x \cos n x}{n} + \frac{1}{n} \int \cos n x \; \d x \right]_{-\pi}^{\pi} \\
& = \frac{- 2 \pi \cos n \pi}{n} + \frac{1}{n} \left[ \frac{\sin n x}{n \pi} \right]_{-\pi}^{\pi} \\
& = \frac{2 \pi (-1)^{n+1}}{n}
\end{align*}

Whence 

\[ g(x) = \sum_{n=1}^{\infty} \frac{ 4 \pi^{2} (-1)^{n+1}}{n^{2}} \sin  n x  \]


Fourier series for $ h(x) = 2| x | $ (even function) in the range $ (-\pi,\pi) $ given by


\[ \frac{f(x_{+} + f(x_{-}))}{2} =  \frac{1}{2} a_{0} +  \sum_{n=1}^{\infty} a_{n} \cos  n x   \]


where 

\begin{align*}
\frac{1}{2} a_{0} & = \frac{1}{2L} \int_{-L}^{L} f(x) \; \d x \\
& = \frac{1}{2\pi} \int_{-\pi}^{\pi} 2| x | \; \d x \\
& = \frac{2}{\pi} \int_{0}^{\pi} x \; \d x \\
& = \pi
\end{align*}
and 
\begin{align*}
a_{n} & = \frac{1}{L} \int_{-L}^{L} f(x) \cos \left( \frac{n \pi x}{L} \right) \; \d x  \\
& = \frac{2}{\pi} \int_{-\pi}^{\pi} | x | \cos n x \; \d x
\end{align*}





Integrating by parts,

\begin{align*}
\int_{-\pi}^{\pi} | x | \cos n x \; \d x & = 2 \int_{0}^{\pi} x  \cos n x \; \d x \\
& = 2 \left[ \frac{x \sin n x}{n} - \frac{1}{n} \int \sin n x \; \d x \right]_{0}^{\pi}  \\
& = - \frac{2}{n} \int_{0}^{\pi} \sin n x \; \d x \\
& = - \frac{2}{n} \left[ - \frac{\cos n x}{n} \right]_{0}^{\pi}\\
& = \frac{2}{n^{2}} (\cos n \pi - 1) 
\end{align*}

Whence

\[ h(x) = \pi +  \sum_{n=1}^{\infty} \frac{4[(-1)^{n} - 1]}{n^{2} \pi }  \cos  n x \]

Note that differentiating the Fourier sine series for $ x^{2} $ gives

\[ \frac{\d }{\d x} [ f_{s}(x) ] = \sum_{n=1}^{\infty} \left\{ 2 
\pi (-1)^{n+1} + \frac{4[(-1)^{n} - 1]}{n^{2} \pi }   \right\} \cos  n x   \]

These don't quite match up: what is $ \sum_{n=1}^{\infty}  2 
\pi (-1)^{n+1} \cos n x  $ the Fourier series for?

Note that the cos coefficients $ a_{n} = O(1) $ as $ n \to \infty $, so this function is terrible. Using the direchlet conditions, $ a_{n} = O(\frac{1}{n})  $, so $ f $ is discontinuous. 

This motivates us to check the Fourier series for the dirac delta function $ \delta(x) $, with period $ (- \pi, \pi) $

\begin{align*}
\delta (x) & \sim \frac{a_{0}}{2} + \sum_{n=1}^{\infty}  \left\{  a_{n} \cos n x + b_{n} \sin n x \right\}   \\
\end{align*} 

We find that

\begin{align*}
a_{n} & = \frac{1}{\pi} \int_{-pi}^{\pi} \delta(x) \cos n x \; \d x  \\
& = \frac{\cos 0}{\pi}\\
&  = \frac{1}{\pi}
\end{align*}

and similarly

\[ b_{n} = \frac{\sin 0}{\pi} = 0  \]

ie.

\[ \delta(x) \sim \sum_{n=1}^{\infty}  \frac{1}{\pi} \cos n x + \frac{1}{2 \pi}   \]

This isn't quite what we wanted, but making a small adjustment:

\[ \delta(x - \pi): a_{n} = \frac{\cos n \pi}{\pi} = \frac{(-1)^{n}}{\pi}  \quad b_{n} = 0\]

Finally, we conclude that

\[ f_{s}'(x) = 2 | x |  - 2 \pi^{2}  \delta(x - \pi) \]

I guess the morale is, don't differentiate term by term if $ f $ is discontinuous...




\section{QUESTION 3}


$ f(x) = e^{x} $ on $ (-\pi,\pi) $ has Fourier series given by 


\[ \frac{f(x_{+} + f(x_{-}))}{2} = \frac{1}{2} a_{0} + \sum_{n=1}^{\infty} \left[ a_{n} \cos n x + b_{n} \sin n x  \right]  \]


where 

\begin{align*}
\frac{1}{2} a_{0} & = \frac{1}{2\pi} \int_{-\pi}^{\pi} e^{x} \; \d x \\
& = \frac{1}{2\pi} \left(  e^{\pi} - e^{-\pi} \right) \\
& = \frac{1}{\pi} \sinh \pi
\end{align*}
and 
\begin{align*}
a_{n} & = \frac{1}{\pi} \underbrace{ \int_{-\pi}^{\pi} e^{x} \cos n x \; \d x}_{I_{a}}  \\
I_{a} & =  \left[   e^{x} \cos n x + \int e^{x} n \sin n x  \; \d x  \right]_{-\pi}^{\pi} \\
& = (e^{\pi} - e^{-\pi}  )\cos n \pi + n \int_{-\pi}^{\pi} e^{x} \sin n x \;\d x \\
& = 2 \sinh \pi \;  (-1)^{n} + n \left[  e^{x} \sin n x - \int e^{x} n \cos x \; \d x  \right]_{-\pi}^{\pi} \\
& =  2 \sinh \pi \;  (-1)^{n} + -n^{2} \int_{-\pi}^{\pi} e^{x} \cos n x \; \d x \\
& =  2 \sinh \pi \;  (-1)^{n} + -n^{2} I_{a}
\end{align*}

Hence 

\[ a_{n} = \frac{1}{\pi} I_{a} \qquad I_{a} = \frac{2}{1 + n^{2}} \sinh \pi (-1)^{n} \]

Also,

\[ b_{n} =  \frac{1}{\pi} \underbrace{\int_{-\pi}^{\pi}  e^{x} \sin n x \; \d x }_{I_{b}} \]

\begin{align*}
I_{b} & = \left[  e^{x} \sin n x - \int e^{x} n \cos n x \; \d x  \right]_{-\pi}^{\pi}  \\
& = -n I_{a}
\end{align*}

\[ b_{n} = - \frac{n}{\pi} I_{a} \]

Combining these results, the Fourier series for $ e^{x} $ is given by

\begin{align*}
f(x) & = \frac{1}{\pi} \sinh \pi  + \sum_{n=1}^{\infty} \left[  \left(  \frac{1}{\pi} \cos n x - \frac{n}{\pi} \sin n x \right) I_{a}  \right]  \\
& = \frac{1}{\pi} \sinh \pi + \frac{2}{\pi} \sinh \pi \sum_{n=1}^{\infty} \left[  (\cos n x - n \sin n x) \frac{(-1)^{n}}{1+n^{2}} \right] 
\end{align*}

Setting $ x = \pi $ yields

\[ e^{\pi} = \frac{1}{\pi} \sinh \pi + \frac{2}{\pi} \sinh \pi \sum_{n=1}^{\infty} \frac{1}{1+n^{2}} \]

Thus

\begin{align*}
\sum_{n=1}^{\infty} \frac{1}{1+n^{2}} & = \frac{\pi e^{\pi} - \sinh \pi }{2 \sinh \pi} \\
\end{align*}

Setting $ x = -\pi $ similarly yields

\begin{align*}
\sum_{n=1}^{\infty} \frac{1}{1+n^{2}} & = \frac{\pi e^{-\pi} - \sinh \pi }{2 \sinh \pi} \\
\end{align*}

Adding and dividing by two, 

\begin{align*}
\sum_{n=1}^{\infty} \frac{1}{1+n^{2}} & = \frac{\pi (e^{\pi} + e^{-\pi}) - 2 \sinh \pi }{4 \sinh \pi} \\
& = \frac{2 \pi \cosh \pi  - 2 \sinh \pi }{4 \sinh \pi} \\
& = \frac{1}{2} ( \pi \coth \pi - 1)
\end{align*}

\section{QUESTION 4}

\begin{enumerate}
	\item Reposing the Fourier Series of $ f(t) $ using complex variables,
	
	\begin{align*}
	f(t) & = \frac{a_{0}}{2} + \sum_{n=1}^{\infty}  \left[   \frac{a_{n}}{2} \left(   e^{\frac{i n \pi t}{L}} + e^{\frac{- i n \pi t}{L}} \right) + \frac{b_{n}}{2i} \left( e^{\frac{i n \pi t}{L}} - e^{\frac{-i n \pi t}{L}} \right)   \right]  \\
	& = \sum_{n=-\infty}^{\infty}  c_{n} e^{\frac{i n \pi t}{L}}, \\
	c_{n} & = \frac{a_{n} - i b_{n}}{2} \; n > 0 ; \\
	c_{-n} & = \frac{a_{n} + i b_{n}}{2} \; n > 0 ; \\
	c_{0} & = \frac{a_{0}}{2}
	\end{align*}
	
	Using the orthogonality of complex exponentials and the properties of complex Fourier coefficients, we deduce that
	
	\begin{align*}
	\int_{-L}^{L} \left[  f(t) \right]^{2} \; \d t  & = \sum_{n = - \infty}^{\infty} \sum_{m = -\infty}^{\infty} c_{n} c_{m} \int_{-T}^{T} \exp \left[   \frac{i \pi t (n + m)}{L} \right]  \; \d t \\
	& = \sum_{n = - \infty}^{\infty} \sum_{m = -\infty}^{\infty} c_{n} c_{m} 2 T \delta_{n[-m]}  \; \\ 
	& = 2 T \sum_{n = - \infty}^{\infty} c_{n} c_{-n} \\
	& = 2 T \sum_{n = - \infty}^{\infty} c_{n} c_{n}^{*} \\
	& = 2 T \sum_{n = - \infty}^{\infty} | c_{n} |^{2} \\
	\end{align*}
	
	This can be then re-expressed in terms of the $ a_{n} $ and $ b_{n} $ as 
	
	\[ \int_{-L}^{L} \left[  f(t) \right]^{2} \; \d t  = L  \left[   \frac{a_{0}^{2}}{2} +\sum_{n=1}^{\infty} (a_{n}^{2} + b_{n}^{2}) \right]  \]
	
	as required. 
	 
	\item 
	
	\begin{center}
		\begin{tikzpicture}
		\draw [->] (-5, 0) -- (5, 0) node [right] {$t$};
		\draw [->, use as bounding box] (0, -2.5) -- (0, 2.5) node [above] {$f(t)$};
		\draw [-, mblue] (0, 1) node [left] {$ 1 $} -- (5, 1);
		\draw [-, mblue] (-5, -1)  -- (0, -1)  node [right] {$ -1 $};
		\end{tikzpicture}
	\end{center}
	
	The unit amplitude square wave has Fourier series (odd function)
	
	\[ f(t) = \sum_{n=1}^{\infty}  b_{n} \sin \left(  \frac{n \pi t}{T} \right)  \]
	
	Frequencies less than $ \frac{9}{2} \pi T^{-1} $ correspond to terms in the Fourier series with $ \frac{n \pi}{T} <  \frac{9}{2} \pi T^{-1} $, ie. $ n = 1,2,3,4 $.
	
	Also,
	
	\begin{align*}
	b_{n} & = \frac{1}{T} \int_{-T}^{T} \\
	& = 
	\end{align*}
	
\end{enumerate}



\section{QUESTION 5}

\begin{enumerate}
	\item \begin{center}
	\begin{tikzpicture}
	\draw [->] (0, 0) -- (5, 0) node [right] {$x$};
	\draw [->, use as bounding box] (0, 0) -- (0, 4) node [above] {$f(x)$};
	
	\end{tikzpicture}
\end{center}


$ f(x) $ on $ (0,2\pi) $ has Fourier series given by 


\[ \frac{f(x_{+} + f(x_{-}))}{2} = \frac{1}{2} a_{0} + \sum_{n=1}^{\infty} \left[ a_{n} \cos n x + b_{n} \sin n x  \right]  \]


where 

\begin{align*}
\frac{1}{2} a_{0} & = \frac{1}{2\pi} \int_{0}^{2\pi} f(x) \; \d x \\
& = \frac{1}{2\pi} \int_{\pi}^{2\pi} 1 \; \d x \\
& = \frac{1}{2}
\end{align*}
and 
\begin{align*}
a_{n} & = \frac{1}{\pi} \int_{0}^{2\pi} f(x) \cos n x    \; \d x  \\
& = \frac{1}{\pi} \int_{\pi}^{2\pi} \cos n x \; \d x \\
& = \frac{1}{\pi} \left[  \frac{1}{n} \sin n x \right]_{\pi}^{2\pi} \\
& = 0
\end{align*}

and 

\begin{align*}
b_{n} & = \frac{1}{\pi} \int_{0}^{2\pi} f(x) \sin n x    \; \d x  \\
& = \frac{1}{\pi} \int_{\pi}^{2\pi} \sin n x \; \d x \\
& = \frac{1}{\pi} \left[  -\frac{1}{n} \cos n x \right]_{\pi}^{2\pi} \\
& = - \frac{1}{\pi n} \left[ \cos n x  \right]_{\pi}^{2\pi} \\
& = 0 \text{ if }  n \text{ even or } - \frac{2}{n \pi} \text{ if } n \text{ odd}  
\end{align*}

Hence 

\[ f(x) = \frac{1}{2}  - \frac{2}{\pi} \sum_{n=1}^{\infty} \frac{\sin(2n -1)x}{2n - 1}  \]

\item Taking the hint, differentiating term by term gives

\[ \frac{\d }{\d x} [S_{n}(x)] = \frac{2}{\pi} \sum_{n=1}^{N} \cos(2n-1)x \]

Now

\begin{align*}
\sum_{n=1}^{N} \cos(2n-1)x & = \Re \left[  \sum_{n=1}^{N} e^{(2n-1)i} \right]   \\
& = \Re \left[  \frac{e^{(2N + 1)i} - e^{i}}{e^{2i} - 1} \right] 
\end{align*}
	

	
	
	
\end{enumerate}

\section{QUESTION 6}

Assuming the solution takes the form $ y \propto e^{\sigma x},$, we have

\[ y(x) = A \cos(\mu x) + B \sin(\mu x) \]

where $ A $ and $ B $ are constants, and $ \mu^{2} = \lambda $. Applying the boundary conditions, $ y(0) = 0 $ implies that $ A = 0 $. The other boundary condition implies

\[ B \sin \mu + B \mu \cos \mu = 0 \]

\[  \mu = - \tan \mu  \] 

This eigenvalue equation has an infinite number of solutions, $ \mu_{n} $ (and hence there an infinite number of positive eigenvalues $ \lambda_{n} = \mu_{n}^{2} $).

As $ n \to \infty $, $ \mu_{n} \to \infty $, so $ \mu $ is close to an odd multiple of $\frac{\pi}{2} $, ie. $ \mu_{n} \approx (2n+1)\pi/2 $, and hence $ \lambda_{n} \approx (2n+1)^{2}\pi^{2} / 4 $


\section{QUESTION 7}

\begin{enumerate}
	\item $ p(x) = \exp \left(  \int^{x} \frac{-2u}{1-u^{2}} \; \d u \right) = (1-x^{2}) $, thus integrating factor is $ -\frac{1}{1-x^{2}} \left( (1-x^{2}) \right) = -1   $. We can then rewrite the equation as
	
	\[ -(1-x^{2})y'' + 2x y' - n(n+1) y = 0  \] 
	
	and
	
	\[ -\frac{\d }{\d x} \left( (1-x^{2}) \frac{\d y}{\d x} \right) - n(n+1)y = 0  \]
	
	\item
	
	
	\begin{align*}
	p(x) & = \exp \left(  \int^{x} \frac{(1+a+b)u - c}{u(u-1)} \; \d u \right) \\
	& = \exp \left(  \int^{x} \frac{c}{u} + \frac{1 + a + b - c}{u-1} \; \d u \right) \\
	& = \exp \left(  c \log x + (1 + a + b - c)\log (x-1)  \right) \\
	& = x^{c} + (x-1)^{1 + a + b - c}
	\end{align*}
	
	Thus the required integrating factor is 
	
	\[ - \frac{x^{c} + (x-1)^{1 + a + b - c}}{x(x-1)} \]
	
	The equation becomes
	
	\[ - (x^{c} + (x-1)^{1 + a + b - c}) y''  + -[ (1 + a + b)x - c ] \frac{x^{c} + (x-1)^{1 + a + b - c}}{x(x-1)} y' - \frac{x^{c} + (x-1)^{1 + a + b - c}}{x(x-1)} ab y = 0 \]
	
	which, in Sturm-Liouville form, is 
	
	\[ -\frac{\d }{\d x} \left[  (x^{c} + (x-1)^{1 + a + b - c}) \frac{\d y }{\d x} \right] - \frac{x^{c} + (x-1)^{1 + a + b - c}}{x(x-1)} ab y = 0  \]
	
	\item Self-adjoint form, integrating factor $ - e^{4x} $, 
	
	\[ - \frac{\d }{\d x} \left( e^{4x} \frac{\d y}{\d x} \right) -4e^{4x} y = \lambda e^{4x} y \]
	
	weight function is hence $ e^{4x} $.
	
	Easier to consider original equation; assuming the solution takes the form $ y \propto e^{\sigma x}, \; \sigma $ satisfies the auxillary equation
	
	\[ \sigma^{2} + 4 \sigma + 4 + \lambda = 0 \Rightarrow \sigma = -2 \pm i\sqrt{\lambda}, \]
	
	\[ y(x) = Ae^{-2x} \cos(\mu x) + B e^{-2x} \sin(\mu x) \]
	
	where $ A $ and $ B $ are constants, and $ \mu^{2} = \lambda $. Applying the boundary conditions, $ y(0) = 0 $ implies that $ A = 0 $. The other boundary condition implies

	\begin{align*}
	& B e^{-2} \sin \mu  = 0 \\
	& \Rightarrow \mu = n \pi
	\end{align*}
	
	Thus infinite positive eigenvalues $ \lambda_{n} = n^{2} \pi^{2} $
	
	The associated eigenvectors are thus proportional to $ e^{-2x} \sin (n \pi x) $.
	
	Eigenvectors associated with distinct eigenvalues are indeed orthogonal on the interval, if the weight function $ e^{4x} $ is correctly included in the inner product integral $ I_{mn} $, $ (m \neq n) $ defined as
	
	\[ I_{mn} = \int_{0}^{1} e^{4x} Y_{n}(x)Y_{m}(x) \; \d x \]
	
	where $ Y_{n} $ are $ Y_{m} $ are normalized eigenfunction with distinct eigenvalues $ \lambda_{n} = n^{2} \pi^{2} $ and $ \lambda_{m} = m^{2} \pi^{2} $
	
	
	
	
	
\end{enumerate}



\section{QUESTION 8}

\begin{enumerate}
	\item Using $ L $ to denote the operator, 

\[ L : = \frac{\d }{\d x} \left(  x \frac{\d u}{\d x} \right)  \]

the Sturm-Liouville form is

\[ L u = - \lambda x u , \quad 0 < x < 1  \]

hence with weight function $ x $.

We seek a linear substitution to turn this into Bessel's equation of order zero. We cannot make a substitution for $ u $ as the linearity of the operator makes this redundant; after trying a few things, we see that $ x = \frac{z}{\sqrt{\lambda}} $ is the way forward. Turns it into

\[ \frac{\d }{\d z} \left(  z \frac{\d}{\d z} \right) y = - z y   \]

which is Bessel's equation, and we are told the general solution is of the form

\[ y(z) = A J_{0}(z) + B[ R(z) + J_{0}(z)\log(z) ] \]

where $ A,B $ are constants and $ R(z) $ is a `regular function'.

Now we have $ u(x) $ bounded as $ x \to 0 $, same must be true for $ y(z) $ as $ z \to 0 $. From the series definition of $ J_{0} $, we know that $ J_{0}(0) = 1 $. Hence as $ \log(z) \to - \infty $ as $ z \to 0 $, we have $ B = 0 $, concluding that $ y(z) = A J_{0}(z) $, ie. 

\[ u(x) = A J_{0} (\sqrt{\lambda}  x)  \]

and $ u(1) = 0 \Rightarrow A J_{0} (\sqrt{\lambda}) = 0 $, so $ \sqrt{\lambda} = j_{0} $ for $ n = 1,2,\cdots $.

Thus the operator $ L $ has eigenfunctions $ u_{n}(x) = J_{0}(j_{n}x) $ with eigenvalues $ \lambda_{n} = j_{n}^{2} $.


\item Acting the operator $ L $ on its eigenfunction $ J_{0}(\alpha x) $ given 

\[ \frac{\d }{\d x}\left(  x \frac{\d }{\d x} \right) J_{0}(\alpha x) = - \alpha^{2} x J_{0} (\alpha x)  \]

Multiplying by $ J_{0}(\beta x) $ and integrating gives

\[ \int_{0}^{1}  J_{0}(\beta x) \frac{\d }{\d x}\left(  x \frac{\d }{\d x} \right) J_{0}(\alpha x) \; \d x = - \alpha^{2} \int_{0}^{1}  x J_{0} (\alpha x)J_{0}(\beta x ) \; \d x \]

\[ \cdots \]

Next part: setting $ \alpha = j_{n} $, $ \beta = j_{m} $, we note that $ J_{0}(j_{n}) = J_{0}(j_{m}) = 0 $, thus the identity follows.

Next: note that our first result is only valid for when $ \beta \neq \alpha $. So we set $ \alpha = j_{n} $, $ \beta = j_{n} + \varepsilon $, and the result should pop out.

\item Summarising what we have so far.

\[ L : = \frac{\d }{\d x} \left(  x \frac{\d }{\d x}   \right)  \qquad \text{with weight } w = x \]

\[ \lambda_{n} = j_{n}^{2}, \qquad u_{n}(x) = J_{0} (j_{n} x )   \]

\[ \text{orth. relation: } \int_{0}^{1} x J_{0}(j_{n} x) J_{0} (j_{m} x) \; \d x= \frac{1}{2} \left[  J_{0}'(j_{n}) \right]^{2} \delta_{mn}   \]


To solve 

\[ L u + \tilde{\lambda} x u = x f(x) \qquad (*)  \]

Seek eigenfunction expansions

\[ u = \sum_{n=1}^{\infty} a_{n} J_{0} (j_{n} x) \qquad f(x) = \sum_{n=1}^{\infty} b_{n} J_{0} (j_{n} x )   \]

Substitute into $ (*) $

\[ \sum_{n=1}^{\infty} a_{n} \underbrace{L [J_{0} (j_{n} x) ] }_{-j_{n}^{2} x J_{0}(j_{n}x) } + \tilde{\lambda}  x \sum_{n=1}^{\infty} a_{n} J_{0} (j_{n} x)  = x \sum_{n=1}^{\infty} b_{n} J_{0} (j_{n} x )    \]

Comparing coefficients (note how the $ x $ makes this easy)

\begin{align*}
& - a_{n} j_{n}^{2} + \tilde{\lambda} a_{n} = b_{n} \\
& \Rightarrow a_{n} = \frac{b_{n}}{\tilde{\lambda} - j_{n}^{2} }
\end{align*}

Noting that $ \tilde{\lambda} $ is not an eigenvalue, ie. $ \tilde{\lambda} \neq \lambda_{n} = j_{n}^{2} $.

To find the eigenfunction expansion of $ u $ it remains to find $ b_{n} $ st. 

\[ f(x) = \sum_{n=1}^{\infty} b_{n} J_{0}(j_{n}x)  \]


Multiply by $ x J_{0}(j_{n} x) $ and integrate, thus

\begin{align*}
\int_{0}^{1}  x f(x) J_{0}(j_{m} x) \; \d x  & =  \sum_{n=1}^{\infty} b_{n} \int_{0}^{1}  x J_{0}(j_{n} x) J_{0}(j_{m} x) \; \d x   \\
& = \sum_{n=1}^{\infty} b_{n} \frac{1}{2}  [  J_{0}'(j_{n}) ]^{2} \delta_{mn} \\
& = b_{m} \frac{1}{2}  [  J_{0}'(j_{m}) ]^{2} \\
\end{align*}
ie.
\[ b_{n}  = \frac{ 2 \int_{0}^{1}  t f(t) J_{0}(j_{m} t) \; \d t}{  [ J_{0}1(j_{n}) ]^{2} }  \]

Hence

\[ u(x) = 2 \sum_{n=1}^{\infty} \frac{\int_{0}^{1}  t f(t) J_{0}(j_{m} t) \; \d t}{ [ J_{0}1(j_{n}) ]^{2}  (\tilde{\lambda}   -j_{n}^{2} )   }J_{0}(j_{n} x)  \]
	
\end{enumerate}









\section{QUESTION 9}





\end{document}