\documentclass[a4paper]{article}
\usepackage{amsmath}
\def\npart {IB}
\def\nterm {Michaelmas}
\def\nyear {2017}
\def\nlecturer {Dr. Saxton}
\def\ncourse {Methods Example Sheet 4}

\input{header}
\newtheorem*{soln}{Solution}

\renewcommand{\thesection}{}
\renewcommand{\thesubsection}{\arabic{section}.\arabic{subsection}}
\makeatletter
\def\@seccntformat#1{\csname #1ignore\expandafter\endcsname\csname the#1\endcsname\quad}
\let\sectionignore\@gobbletwo
\let\latex@numberline\numberline
\def\numberline#1{\if\relax#1\relax\else\latex@numberline{#1}\fi}
\makeatother


\begin{document}
	
\maketitle

\section{QUESTION 1}
\begin{enumerate}
	\item Along characteristic curves,
	
	\[ \frac{\d x}{\d s} = 1 \quad \frac{\d y}{\d s} = y  \]
	
	which has general solution $ x = s + c $ and $ y = Ae^{s} $ for some constants $ c,A $. 
	
	Cauchy data is $ B(t) = \{ (x = 0, y = t) \} $, intersect $ B $ at $ s = 0 $, thus characteristic curves are
	
	\[ x = s \quad y = t e^{s} \]
	
	$ \frac{\partial u}{\partial s} \big|_{t} = 0 $ so that $ u = \text{cst} $ along these characteristics.
	
	 and the Cauchy data fixes $ u(s,t) = t^{3} $ on the $ t^{\text{th}} $ curve. Inverting gives
	
	\[ s = x \quad t = y e^{-x} \]
	
	and therefore the solution to our problem is 
	
	\[ u(x,y) = y^{3} e^{-3x} \]
	
	throughout $ \R^{2} $
	
	\item Along characteristic curves
	
	\[ \frac{\d x}{\d s} = y \quad \frac{\d y}{\d s} = x  \]
	
	\[ \Rightarrow \frac{\d^{2} x }{\d s^{2}} = x \]
	
	\[ x = A \sinh s + B \cosh s \]
	
	Cauchy data is $ B(t) = \{ (x = 0, y = t) \} $, intersect at $ s = 0 $ gives $ B = 0 $, thus
	
	\[ x = t \sinh s \qquad y = t \cosh s \]
	
	Then $ \frac{\partial u }{\partial s} \Big|_{t} = 0  \Rightarrow u = \text{ cst. }$ along these characteristics, and Cauchy data fixes $ u(s,t) = e^{-t^{2}} $, and have that
	
	\[ t^{2} = (t \cosh s)^{2} - (t \sinh s)^{2} = y^{2} - x^{2} \]
	
	therefore solution is
	
	\[ u(x,y) = e^{x^{2} - y^{2}} \]
	
	which is uniquely defined only in the upper quadrant of the plane $ \{ (x,y) \subset R^{2} : x \geq 0, y \geq 0 \} $
	
	\item PDE is $ u_{x} + u_{y} = e^{x + 2y} - u $.
	
	Along characteristic curves
	
	\[ \frac{\d x}{\d s} = 1 \quad \frac{\d y}{\d s} = 1  \]
	
	
	\[ x = s + c \qquad y = s + d \]
	
	Cauchy data is $ B(t) = \{ (x = t, y = 0) \} $, intersect at $ s = 0 $, thus
	
	\[ x = s + t  \qquad y = s \]

	Then $ \frac{\partial u }{\partial s} \Big|_{t} = e^{2x + y} - u = e^{3s + t} - u $ along these characteristics, this is an ODE in $ s $ so multiplying by the integrating factor $ e^{s} $ gives
	
	\[ e^{s} \frac{\d u}{\d s} + e^{s} u = e^{4s + t}  \]
	
	\[ \Rightarrow e^{s} u = \frac{1}{4} e^{4s + t } + \text{cst.} \]
	
	Cauchy data  $ u(t,0) = 0 $ fixes $ \text{cst} = -  \frac{1}{4} e^{t} $, and have that
	
	\[ u(s,t) = \frac{1}{4} e^{t  + 3s} -  \frac{1}{4} e^{t-s}  \]


	inverting the relations
	
	\[ s = y \qquad t = x - y \]
	
	the solution is
	
	\begin{align*}
	u(x,y) & = \frac{1}{4}e^{x + 2y} - \frac{1}{4} e^{x - 2y}  \\
	& = \frac{1}{2}e^{x} \sinh 2y
	\end{align*}
	
\end{enumerate}

\section{QUESTION 2}

Separation of variables $ \Rightarrow u(x,t) = X(x) T(t) $, thus

\[ X''T + X \dot{T} = 0 \]
\[ \Rightarrow \frac{X''}{X} = - \frac{\dot{T}}{T}  \]

LHS independent of $ x $, RHS independent of $ t $, so both sides constant.
Setting $ \lambda = \dot{T} / T $ we have

\[ X'' + \lambda X = 0 \]
\[ X(x) = A \cos \sqrt{\lambda} x + B \sin \sqrt{\lambda} x \]

Boundary conditions $ X(0) = X(\pi) = 0 $ imply that $ A = 0 $, $ \lambda = n^{2} $. Then solving $ \dot{T} = \lambda T $ gives $ T(t) = C e^{n^{2} t} $ for some constant $ C $, so the solution is 

\[ U(x,t) = \sum_{n=1}^{\infty} D_{n} e^{n^{2}t} \sin n x  \]

for some constants $ D_{n} $, which we can determine by using orthogonality conditions. 

But now, looking at what happens for large n, the solution then has oscillations with higher and higher wavenumber and larger and larger (indeed arbitrarily large) amplitude $ U(x) e^{n^{2}t} $, and so this problem is ill-posed.

\section{QUESTION 3}

\begin{enumerate}
	\item The principal part of the symbol of the differential operator is $ \mathbf{k}^{T} \mathbf{A} \mathbf{k} $, where 

	\[ A = \begin{pmatrix}
	1 & 0 \\
	0 & x
	\end{pmatrix} \]
	
	$ \det A  $ is product of eigenvalues, therefore we have 
	
	\begin{center}
		\begin{tabular}{rll}
			elliptic  & $ x > 0  $ \\
			parabolic  &  $ x = 0 $  \\
			hyperbolic &  $ x < 0$ 
		\end{tabular}
	\end{center}

	In the hyperbolic region the negative eigenvector points in the $ y $ direction. Thus, if $ f(x,y) = \text{cst.} $ is to be a characteristic surface, we need $ \partial_{y} f = \pm \sqrt{ \partial_{x} f  A_{xx} \partial_{x} f / x } = \pm \sqrt{x} \partial_{x} f  $. Letting $ p = 2 x^{1/2} $, this is $ (\partial_{y} \pm \partial_{p})f = 0 $, so the characteristic surfaces are the two curves of constant $ y \pm p $, that is
	
	\[ u = y + 2x^{1/2} \]
	\[ v = y - 2x^{1/2} \]
	
	for $ u $ constant and $ v $ constant.

	\item Here
	
	\[ \mathbf{A} = \text{diag}(1,y) \]
	
	so the PDE is hyperbolic in the $ y < 0 $ region with the negative eigenvector $ \mathbf{m} $ points in the $ y $-direction, with $ \mathbf{A} \mathbf{m} = y \mathbf{m} $. Thus, if $ f(x,y) = \text{cst.} $ is to be a characteristic surface, we need $ \partial_{y} f = \pm \sqrt{ \partial_{x} f  A_{xx} \partial_{x} f / y } = \pm \sqrt{y} \partial_{x} f  $. Letting $ \xi = x, \eta  = 2 y^{1/2} $, this is $ (\partial_{\xi} \pm \partial_{\eta})f = 0 $, (thus $ \partial_{\eta \nu})f = 0 $)  so the characteristic surfaces are the two curves of constant $ \xi \pm \eta $, that is
	
		
	\[ \nu_{1} = x + 2y^{1/2} \]
	\[ \nu_{2} = x - 2y^{1/2} \]
	
	for $ \nu_{1} $ constant and $ \nu_{2} $ constant.
	
	(Do we? Not sure about the steps here) have $ u_{\xi\eta} = 0 \Rightarrow u(\xi,\eta) = f(\xi + \eta) + g(\xi - \eta) $, so the result follows.
 	
	
	 
	
\end{enumerate}



\section{QUESTION 4}

Green's second identity is 

\[ \int_{\Omega}  \phi \nabla^{2} \psi  - \psi \nabla^{2} \phi \;  \d V = \int_{\delta \Omega} \phi ( \mathbf{n} \cdot \nabla \psi ) - \psi( \mathbf{n} \cdot \nabla \phi ) \d S   \]

where $ \Omega \subset \R^{n} $ is a compact set, and $ \phi, \psi : \Omega \to \R $ are a pair of functions on $ \Omega $ regular throughout $ \Omega $.

Green's function for the Laplacian on domain $ \Omega $ satisfies $ \nabla^{2} G(\mathbf{r};\mathbf{r}_{0}) = \delta(\mathbf{r} - \mathbf{r}_{0}) $, and $ G |_{\partial \Omega} = 0 $

Consider using this identity with $ \phi, \psi $ being $ G(\mathbf{r}'; \mathbf{r}), G(\mathbf{r}'; \mathbf{r_{0}})  $.

Assume Green's second identity is valid for Green's functions (it isn't).

Then the RHS is zero, as $ G |_{\partial \Omega} = 0 $. Then we have

\begin{align*}
0 & = \int_{\Omega}  G(\mathbf{r}'; \mathbf{r}) \nabla^{2} G(\mathbf{r}'; \mathbf{r_{0}}) - G(\mathbf{r}'; \mathbf{r_{0}}) \nabla^{2} G(\mathbf{r}'; \mathbf{r}) \;  \d V \\
& = \int_{\Omega}  G(\mathbf{r}'; \mathbf{r}) \delta(\mathbf{r}' - \mathbf{r}_{0}) - G(\mathbf{r}'; \mathbf{r_{0}}) \delta(\mathbf{r}' - \mathbf{r}) \;  \d V \\
& = \int_{\Omega}  G(\mathbf{r}_{0}; \mathbf{r})  - G(\mathbf{r}; \mathbf{r_{0}}) \;  \d V
\end{align*}

Conclude that $ G(\mathbf{r}_{0}; \mathbf{r})  - G(\mathbf{r}; \mathbf{r_{0}}) $ anywhere in $ \Omega $.

\section{QUESTION 5}

$ u(\mathbf{x}) $ is harmonic and therefore satisfies $ \nabla^{2} \mathbf{u} = 0 $. Consider a Dirichlet Green's function for the Laplace operator on $ D $; we have

\[ \nabla^{2} G ( \mathbf{r} ; \mathbf{r}_{0}) \]

It can be shown that

\[ G (\mathbf{x} ; \mathbf{x}_{0} ) = \frac{1}{2\pi} \log | \mathbf{x} - \mathbf{x}_{0} | \]

Now, the result is a direct application of Green's third identity (which I'm guessing the question wants us to derive). Let $ B_{r} $ and $ B_{\varepsilon} $ be circles about the point $ \mathbf{x}_{0} $, so

\[ B_{r} = \{ \mathbf{x} \in \R^{2} \; | \; | \mathbf{x} - \mathbf{x}_{0} | \leq r \} \]
\[ B_{\varepsilon} = \{ \mathbf{x} \in \R^{2} \; | \; | \mathbf{x} - \mathbf{x}_{0} | \leq \varepsilon \} \]

Now consider applying Green's identity on the region $ \Omega = B_{r} - B_{\varepsilon} $, so $ G(\mathbf{x};\mathbf{x}_{0}) $ is perfectly regular everywhere within $ \Omega $


 Using Green's second identity with $ \mathbf{u} $ and $ G $ we have

\begin{align*}
& \int_{\Omega}  u \nabla^{2} G  - G \nabla^{2} u \; \d V = 0 \\
& = \int_{\delta \Omega} u ( \mathbf{n} \cdot \nabla G ) - G ( \mathbf{n} \cdot \nabla u ) \d S  \\
& = \int_{S_{r}} u ( \mathbf{n} \cdot \nabla G ) - G ( \mathbf{n} \cdot \nabla u ) \d S + \int_{S_{\varepsilon}} u ( \mathbf{n} \cdot \nabla G ) - G ( \mathbf{n} \cdot \nabla u ) \d S \quad(*)
\end{align*}

where the first equality follows since $ \nabla^{2} G = 0 $ in $ \Omega $ and $ u $ is a harmonic function. We've included contributions from both boundary circles in the final line. On the inner boundary we have

\[ G|_{\text{inner bdry}} = \frac{1}{2\pi} \log | \varepsilon | \]

\[ \mathbf{n} \cdot G|_{\text{inner bdry}} = - \frac{1}{2\pi} \frac{1}{\varepsilon}  \]


The final term on the last line of $ (*) $ becomes

\[ - \int_{S_{\varepsilon}} G ( \mathbf{n} \cdot \nabla u ) \d S =  \oint \frac{1}{2\pi} \log | \varepsilon | \frac{\partial u }{\partial n} \varepsilon \d \theta  \]

Since $ u $ is regular by assumption, the value of this remaining integral is bounded, so this term vanishes as $ \varepsilon \to 0 $. On the other hand, the penultimate term in the final line of $ (*) $ becomes

\[ \int_{S_{\varepsilon}} u ( \mathbf{n} \cdot \nabla G ) = - \frac{1}{2\pi }\oint u \d \theta = - \tilde{u}  \]

where $ \tilde{u} $ is the average value of $ u $ on the small circle surrounding  $ \mathbf{x_{0}} $. As the radius of this sphere shrinks to zero we have $ \tilde{u} \to u(\mathbf{x}_{0}) $, the value of $ u $ at the centre of the sphere. 

Putting all this together we find

\[ u(\mathbf{x}_{0}) = \oint u \frac{\partial }{\partial n} \left(  \log| \mathbf{x} - \mathbf{x}_{0} |   \right) -  \log| \mathbf{x} - \mathbf{x}_{0} |  \frac{\partial u }{\partial n} \d l  \]

which is Green's third identity, where we have now taken the boundary of $ \Omega $ to be just the large circle (the radius of the inner circle having been shrunk to zero).







\section{QUESTION 6}

Following the hint, we set

\[ x - x_{0} = z_{0} s \cos \theta \qquad y - y_{0} = z_{0} s \sin \theta \]

so that $ z_{0}^{2}s^{2} = (x-x_{0})^{2} + (y - y_{0})^{2} $. The derivatives change as 

\[ \d x \d y = \left| \frac{\partial (x,y)}{\partial(s,\theta)} \right|\; \d s \d \theta \]

We calculate the Jacobian as 

\[ \left| \frac{\partial (x,y)}{\partial(s,\theta)} \right| = z_{0}^{2} s \]

Hence

\begin{align*}
u(x_{0},y_{0},z_{0})& = \frac{z_{0}}{2\pi} \int_{0}^{2\pi} \int_{0}^{\infty} [z_{0}^{2}(s^{2} + 1)]^{-3/2} h(x,y) z_{0}^{2} s \d s \; \d \theta  \\
& = \frac{1}{2\pi} \int_{0}^{2\pi} \int_{0}^{\infty} s[s^{2} + 1]^{-3/2} h(x_{0} + z_{0} s \cos \theta, y_{0} + z_{0} s \sin \theta) \; \d s \; \d \theta \\
\end{align*}

First boundary condition,


\begin{align*}
u(x_{0},y_{0},0) & = \frac{1}{2\pi} \int_{0}^{2\pi} \int_{0}^{\infty} s[s^{2} + 1]^{-3/2} h(x_{0}, y_{0}) \; \d s \; \d \theta \\
& = h(x_{0},y_{0}) \int_{0}^{\infty} s[s^{2} + 1]^{-3/2} \; \d s \\
& = h(x_{0},y_{0}) \left[ - [s^{2} + 1]^{-3/2} \right]_{0}^{\infty} \\
& = h(x_{0},y_{0})
\end{align*}

as required.

Second boundary condition, as $ x^{2} + y^{2} \to \infty $, must also have $ s^{2} \to \infty $.

So (very loosely), the integral tends to zero. 









\section{QUESTION 7}

Consider the forced one-dimensional heat equation

\[ \partial_{t} \theta - D \partial_{xx} \theta = f(x,t) \quad 0 < x,t < \infty, \quad \theta(0,t) = h(t), \theta(x,0) = \Theta(x)  \]

Note the initial condition is inhomogenous. We consider instead that $ V(x,t) = \theta(x,t) - h(t) $ solves the equation, since the conditions are now homogeneous.

We seek a Green's function such that

\[ \frac{\partial }{\partial t}G(x,t ; y,\tau) - D \frac{\partial^{2} }{\partial x^{2}} G(x,t ; y,\tau) = \delta(t - \tau) \delta(x-y) \]

Taking the Fourier transform wrt $ x $ 

\[ \frac{\partial }{\partial t}\tilde{G}(k,t ; y,\tau) + k^{2} D \tilde{G}(k,t ; y,\tau) = \delta(t - \tau) e^{-iky}\]

Multiplying by $ e^{Dk^{2}t} $ we obtain

\[ \frac{\partial }{\partial t} \left[  e^{Dk^{2}t} \tilde{G}(k,t ; y,\tau)  \right] = e^{-iky + Dk^{2}t} \delta(t - \tau)  \]

subject to the initial condition $ \tilde{G}(k,0 ; y,\tau) = 0 $.

\begin{align*}
\tilde{G}(k,t ; y,\tau)  & =  e^{-iky - Dk^{2}t} \int_{0}^{t} e^{Dk^{2}u} \delta(u - \tau) \; \d u   \\
& = \begin{cases}  0  & \text{ if } t < \tau \\ e^{-iky - Dk^{2}(t-\tau)} & \text{ if } t > \tau \end{cases} \\
& = \Theta(t - \tau) e^{-iky - Dk^{2}(t-\tau)}
\end{align*}

where $  \Theta(t - \tau) $ is the step function. 

\section{QUESTION 8}

We seek a Green's function such that

\[ \left(  \frac{\partial^{2} }{\partial t^{2}} - c^{2} \frac{\partial^{2} }{\partial t^{2}} \right)  G(x,t ; y,\tau) = V \delta(t - \tau) \delta(x-y) \]

Take the Fourier transform wrt $ x $, obtaining 

\[ \frac{\partial^{2} }{\partial t^{2}} \tilde{G}(k,t ; y,\tau) + k^{2} c^{2} \tilde{G}(k,t ; y,\tau) = e^{-iky} V  \delta(t - \tau) \]

The solution is

\[ \tilde{G}(k,t ; y,\tau) = \Theta(t-\tau) V e^{-iky}  \frac{\sin(  kc(t-\tau) )  }{ck} \]

To recover the Green's function itself, we must compute the inverse Fourier transform 

\[ G(x,t ; y,\tau) = \frac{\Theta(t-\tau)}{2\pi} \int_{-\infty}^{\infty} V e^{-iky}  \frac{\sin(  kc(t-\tau) )  }{ck} \; \d k \]

Have that 

\[ \mathcal{F}^{-1} [ \sin ka] = \frac{i}{2} ( \delta(x+a) - \delta(x-a))    \]

Alright, I don't know what I'm doing anymore.









\section{QUESTION 9}
\section{QUESTION 10}

Let $ \Omega = \{  (x,y) \in \R^{2} \; : \; y \geq 0 \} $ and suppose $ \psi : \Omega \to \R $ solves Laplace's equation $ \nabla^{2} \psi = 0 $ inside $ \Omega $, subject to

\[ \psi(x,0) = f(x) \quad \text{ and } \quad \lim\limits_{| \mathbf{x} | \to \infty} \psi = 0 \]

\begin{enumerate}
	\item We must construct a Green's function that vanishes on $ \delta \Omega $. As well as vanishing on the $ x $-axis, we also require $ G $ vanishes as $ | \mathbf{x} | \to \infty $. We'll set $ \mathbf{x} = (x,y) $ and $ \mathbf{y} : = \mathbf{x}_{0}^{+} = (x_{0},y_{0}) $ in terms of Cartesian coordinates, with $ y_{0} > 0 $. We know that the free-space Green's function
	
	\[ G_{2}(\mathbf{x},\mathbf{x}_{0}^{+}) = \frac{1}{2\pi} \log |  \mathbf{x} - \mathbf{x_{0}^{+}} | \]
	
	satisfies all conditions except that
	
		\[ G_{2}(\mathbf{x},\mathbf{x}_{0}^{+}) |_{y = 0} = \frac{1}{2\pi} \log | [ (x - x_{0})^{2} + y_{0}^{2} ]^{1/2} | \neq 0 \quad (*)  \]
		
	We need to cancel the nonzero boundary value of $ G_{2} $ by adding on some function.
	
	Let $ \mathbf{x_{0}}^{-} $ be the point $ (x_{0},-y_{0}) $. The location $ \mathbf{x_{0}}^{-} \notin \Omega $, so the Green's function $ G_{2}(\mathbf{x},\mathbf{x}_{0}^{-}) $ is regular everywhere within $ \Omega $, and so obeys Laplace's equation everywhere in the upper half-space. Also,
	
	\[ G_{2}(\mathbf{x},\mathbf{x}_{0}^{-}) |_{y = 0} = G_{2}(\mathbf{x},\mathbf{x}_{0}^{+}) |_{y = 0} \]
	
	Thus the Dirichlet Green's function we seek is 
	
	\begin{align*}
	G(\mathbf{x} ; \mathbf{x}_{0} )& = G_{3} (\mathbf{x} ; \mathbf{x}_{0}^{+}) - G_{3} (\mathbf{x} ; \mathbf{x}_{0}^{-})   \\
	& =  \frac{1}{2\pi} \log |  \mathbf{x} - \mathbf{x_{0}^{+}} | -  \frac{1}{2\pi} \log |  \mathbf{x} - \mathbf{x_{0}^{-}} |
	\end{align*}
	
	Note this satisfies both conditions,
	
	\item Know that given a Green's function, the solution to our Laplace's equation problem is
	
	\[ \psi(\mathbf{y}) = \int_{\partial \Omega} f(\mathbf{x}) \mathbf{n} \cdot \nabla G(\mathbf{x} ; \mathbf{y}) \; \d S  \]
	
	
	Note that there is no contribution from the far field since $ \psi \to 0 $ asymptotically by our boundary condition. The \emph{outward} normal at $ y = 0 $ points in the negative $ y $-direction, so the only contribution to the formula above comes from the lower boundary:
	
	\begin{align*}
	(\mathbf{n} \cdot \nabla G )|_{y = 0} & = - \frac{\partial G }{\partial y} \Big|_{y = 0} \\
	& = \frac{1}{2\pi} \left(  \frac{y + y_{0}}{| \mathbf{x} - \mathbf{x}_{0}^{-} |^{2}} - \frac{y - y_{0}}{| \mathbf{x} - \mathbf{x}_{0}^{-} |^{2}} \right) \Big|_{y = 0} \\
	& = \frac{y_{0}}{\pi} [  (x - x_{0})^{2} + y_{0}^{2}   ]^{-1}
	\end{align*}
	
	\st{At last} the solution is given by
	
	\[ \psi(x_{0},y_{0}) = \frac{y_{0}}{\pi} \int_{-\infty}^{\infty} \frac{f(x)}{(x - x_{0})^{2} + y^{2} } \; \d x \]
	
	\item Next, take the Fourier transform of $ \nabla^{2}(x,y) = 0 $ with respect to $ x $;
	
	\[ \int_{-\infty}^{\infty} \left(  \frac{\partial^{2} \phi }{\partial x^{2}} + \frac{\partial^{2} \phi }{\partial y^{2}} \right) e^{-ikx} \; \d x = 0  \]
	
	\begin{align*}
	\Rightarrow 0 & = \int_{-\infty}^{\infty}  \frac{\partial^{2} \phi }{\partial x^{2}}  e^{-ikx} \; \d x  + \int_{-\infty}^{\infty} \frac{\partial^{2} \phi }{\partial y^{2}} e^{-ikx} \; \d x\\
	& = - k^{2} \tilde{\phi}(k) + \frac{\partial^{2} }{\partial y^{2}} \int_{-\infty}^{\infty}  \phi e^{- i k x} \; \d x\\
	& = - k^{2} \tilde{\phi}(k) +  \frac{\partial^{2} \tilde{\phi}(k) }{\partial y^{2}}
	\end{align*}
	
	Solving this second order ODE in $ k $ gives
	
	\[ \tilde{\phi}(k,y) = A(k) e^{ky} + B(k) e^{-ky}  \]
	
	We have the boundary conditions $ \tilde{\phi}(k,0) = \tilde{f}(k) $, and $ \tilde{\phi} \to 0 $ as $ y \to \infty $. Thus

	\[ \tilde{\phi}(k,y) = \tilde{f}(k) e^{-ky}  \]
	
	respectively. Solving gives $  A = - e^{-k}, B = e^{k} $. Thus, we take the inverse FT to finish.
	
	\begin{align*}
	\phi(x,y) &  = - \frac{1}{2\pi}  \int_{-1}^{1} e^{k(y-1)} - e^{-k(y-1)} \; \d k \\
	\end{align*}
	

	
\end{enumerate}




\section{QUESTION 11}

\section{QUESTION 12}

Consider $ u_{1} = 1/(R^{2} + z^{2})^{1/2} = (x^{2} + y^{2} + z^{2})^{-1/2} $.
This satisfies the boundary conditions. Also


\begin{align*}
\frac{\partial^{2} }{\partial x^{2}} (u_{1}) & = \frac{\partial }{\partial x} \left[ - x (x^{2} + y^{2} + z^{2})^{-3/2}   \right]  \\
& = 3x^{2} (x^{2} + y^{2} + z^{2})^{-5/2} - (x^{2} + y^{2} + z^{2})^{-3/2} 
\end{align*}

Hence 

\begin{align*}
\nabla^{2} u_{1} & = (3x^{2} + 3y^{2} + 3z^{2})(x^{2} + y^{2} + z^{2})^{-5/2} - 3 (x^{2} + y^{2} + z^{2})^{-3/2} \\
& = 0
\end{align*}

So $ u_{1} $ is a solution for $ r = 0 $.

Now suppose 

\[ \frac{1}{R} \frac{\partial }{\partial R} \left(  R \frac{\partial u_{2} }{\partial R} \right) + \frac{\partial^{2} Z }{\partial z^{2}}  = 0\]

Let $ u_{2} = \rho(R)Z(z) $.  Separating variables shows that

\[ \left(  \frac{\rho''}{\rho} + \frac{1}{R} \frac{\rho'}{\rho} \right) + \frac{Z''}{Z} = 0  \]

$ \frac{Z''}{Z} $ cannot vary as $ R $ held fixed, so it must be equal to some constant, $ \lambda^{2} $. Solving gives $ Z(z) = C e^{-\lambda| z |} $ (want solution to decay as $ z \to \infty $).

The radial equation, upon multiplying by $ R^{2} $ becomes

\[ R^{2} \rho'' + R \rho' + \lambda^{2} R^{2}  \rho = 0 \]

We cancel the factor of $ \lambda^{2} $ with the substitution $ x = \lambda R  $, and obtain 

\[ x^{2} \rho'' + x \rho' + x^{2} \rho = 0 \]

which we recognise as Bessel's equation of order zero, having solutions $ J_{0}(x), Y_{0}(x) $ which are regular and singular at the origin respectively. We want our solutions to be regular at the origin, hence we take $ \rho(R) = \text{const. }J_{0}(\text{ R}) $

Hence the solution is $ u_{\lambda} = f(\lambda) e^{-\lambda | z |} J_{0} (\lambda R) $. There where no restrictions on $ \lambda \in Z^{+} $, so this must hold for all possible values of $ \lambda $, so (can we say?) our solution is given by

\[ u = \int_{0}^{\infty} f(\lambda) e^{-\lambda | z |} J_{0} (\lambda R) \d \lambda  \]

Now, setting $ u_{1} = u_{2} $ on $ R = 0 $, and noting $ J_{0}(0) = 1 $ gives

\[ \frac{1}{| z |} = \int_{0}^{\infty} f(\lambda) e^{-\lambda | z |} \d \lambda \]

Multiplying by $ | z | $ and integrating by parts,

\begin{align*}
1 & = \int_{0}^{\infty} f(\lambda) | z |e^{-\lambda | z |} \d \lambda \\
& = \left[ - f(\lambda) e^{-\lambda | z |}  \right]_{0}^{\infty} - \int_{0}^{\infty} f'(\lambda) e^{-\lambda | z |} \d \lambda \\
& = f(0)  - \int_{0}^{\infty} f'(\lambda) e^{-\lambda | z |} \d \lambda
\end{align*}

ie.

\[  \int_{0}^{\infty} f'(\lambda) e^{-\lambda | z |} \d \lambda = f(0) - 1\]

Not sure about last bit.


\end{document}