\documentclass[a4paper]{article}
\usepackage{amsmath}
\def\npart {IB}
\def\nterm {Michaelmas}
\def\nyear {2017}
\def\nlecturer {Dr. Saxton}
\def\ncourse {Methods Example Sheet 4}

% Imports
\ifx \nauthor\undefined
  \def\nauthor{Christopher Turnbull}
\else
\fi

\author{Supervised by \nlecturer \\\small Solutions presented by \nauthor}
\date{\nterm\ \nyear}

\usepackage{alltt}
\usepackage{amsfonts}
\usepackage{amsmath}
\usepackage{amssymb}
\usepackage{amsthm}
\usepackage{booktabs}
\usepackage{caption}
\usepackage{enumitem}
\usepackage{fancyhdr}
\usepackage{graphicx}
\usepackage{mathdots}
\usepackage{mathtools}
\usepackage{microtype}
\usepackage{multirow}
\usepackage{pdflscape}
\usepackage{pgfplots}
\usepackage{siunitx}
\usepackage{slashed}
\usepackage{tabularx}
\usepackage{tikz}
\usepackage{tkz-euclide}
\usepackage[normalem]{ulem}
\usepackage[all]{xy}
\usepackage{imakeidx}

\makeindex[intoc, title=Index]
\indexsetup{othercode={\lhead{\emph{Index}}}}

\ifx \nextra \undefined
  \usepackage[pdftex,
    hidelinks,
    pdfauthor={Christopher Turnbull},
    pdfsubject={Cambridge Maths Notes: Part \npart\ - \ncourse},
    pdftitle={Part \npart\ - \ncourse},
  pdfkeywords={Cambridge Mathematics Maths Math \npart\ \nterm\ \nyear\ \ncourse}]{hyperref}
  \title{Part \npart\ --- \ncourse}
\else
  \usepackage[pdftex,
    hidelinks,
    pdfauthor={Christopher Turnbull},
    pdfsubject={Cambridge Maths Notes: Part \npart\ - \ncourse\ (\nextra)},
    pdftitle={Part \npart\ - \ncourse\ (\nextra)},
  pdfkeywords={Cambridge Mathematics Maths Math \npart\ \nterm\ \nyear\ \ncourse\ \nextra}]{hyperref}

  \title{Part \npart\ --- \ncourse \\ {\Large \nextra}}
  \renewcommand\printindex{}
\fi

\pgfplotsset{compat=1.12}

\pagestyle{fancyplain}
\lhead{\emph{\nouppercase{\leftmark}}}
\ifx \nextra \undefined
  \rhead{
    \ifnum\thepage=1
    \else
      \npart\ \ncourse
    \fi}
\else
  \rhead{
    \ifnum\thepage=1
    \else
      \npart\ \ncourse\ (\nextra)
    \fi}
\fi
\usetikzlibrary{arrows.meta}
\usetikzlibrary{decorations.markings}
\usetikzlibrary{decorations.pathmorphing}
\usetikzlibrary{positioning}
\usetikzlibrary{fadings}
\usetikzlibrary{intersections}
\usetikzlibrary{cd}

\newcommand*{\Cdot}{{\raisebox{-0.25ex}{\scalebox{1.5}{$\cdot$}}}}
\newcommand {\pd}[2][ ]{
  \ifx #1 { }
    \frac{\partial}{\partial #2}
  \else
    \frac{\partial^{#1}}{\partial #2^{#1}}
  \fi
}
\ifx \nhtml \undefined
\else
  \renewcommand\printindex{}
  \makeatletter
  \DisableLigatures[f]{family = *}
  \let\Contentsline\contentsline
  \renewcommand\contentsline[3]{\Contentsline{#1}{#2}{}}
  \renewcommand{\@dotsep}{10000}
  \newlength\currentparindent
  \setlength\currentparindent\parindent

  \newcommand\@minipagerestore{\setlength{\parindent}{\currentparindent}}
  \usepackage[active,tightpage,pdftex]{preview}
  \renewcommand{\PreviewBorder}{0.1cm}

  \newenvironment{stretchpage}%
  {\begin{preview}\begin{minipage}{\hsize}}%
    {\end{minipage}\end{preview}}
  \AtBeginDocument{\begin{stretchpage}}
  \AtEndDocument{\end{stretchpage}}

  \newcommand{\@@newpage}{\end{stretchpage}\begin{stretchpage}}

  \let\@real@section\section
  \renewcommand{\section}{\@@newpage\@real@section}
  \let\@real@subsection\subsection
  \renewcommand{\subsection}{\@@newpage\@real@subsection}
  \makeatother
\fi

% Theorems
\theoremstyle{definition}
\newtheorem*{aim}{Aim}
\newtheorem*{axiom}{Axiom}
\newtheorem*{claim}{Claim}
\newtheorem*{cor}{Corollary}
\newtheorem*{conjecture}{Conjecture}
\newtheorem*{defi}{Definition}
\newtheorem*{eg}{Example}
\newtheorem*{ex}{Exercise}
\newtheorem*{fact}{Fact}
\newtheorem*{law}{Law}
\newtheorem*{lemma}{Lemma}
\newtheorem*{notation}{Notation}
\newtheorem*{prop}{Proposition}
\newtheorem*{soln}{Solution}
\newtheorem*{thm}{Theorem}

\newtheorem*{remark}{Remark}
\newtheorem*{warning}{Warning}
\newtheorem*{exercise}{Exercise}

\newtheorem{nthm}{Theorem}[section]
\newtheorem{nlemma}[nthm]{Lemma}
\newtheorem{nprop}[nthm]{Proposition}
\newtheorem{ncor}[nthm]{Corollary}


\renewcommand{\labelitemi}{--}
\renewcommand{\labelitemii}{$\circ$}
\renewcommand{\labelenumi}{(\roman{*})}

\let\stdsection\section
\renewcommand\section{\newpage\stdsection}

% Strike through
\def\st{\bgroup \ULdepth=-.55ex \ULset}

% Maths symbols
\newcommand{\abs}[1]{\left\lvert #1\right\rvert}
\newcommand\ad{\mathrm{ad}}
\newcommand\AND{\mathsf{AND}}
\newcommand\Art{\mathrm{Art}}
\newcommand{\Bilin}{\mathrm{Bilin}}
\newcommand{\bket}[1]{\left\lvert #1\right\rangle}
\newcommand{\B}{\mathcal{B}}
\newcommand{\bolds}[1]{{\bfseries #1}}
\newcommand{\brak}[1]{\left\langle #1 \right\rvert}
\newcommand{\braket}[2]{\left\langle #1\middle\vert #2 \right\rangle}
\newcommand{\bra}{\langle}
\newcommand{\cat}[1]{\mathsf{#1}}
\newcommand{\C}{\mathbb{C}}
\newcommand{\CP}{\mathbb{CP}}
\newcommand{\cU}{\mathcal{U}}
\newcommand{\Der}{\mathrm{Der}}
\newcommand{\D}{\mathrm{D}}
\newcommand{\dR}{\mathrm{dR}}
\newcommand{\E}{\mathbb{E}}
\newcommand{\F}{\mathbb{F}}
\newcommand{\Frob}{\mathrm{Frob}}
\newcommand{\GG}{\mathbb{G}}
\newcommand{\gl}{\mathfrak{gl}}
\newcommand{\GL}{\mathrm{GL}}
\newcommand{\G}{\mathcal{G}}
\newcommand{\Gr}{\mathrm{Gr}}
\newcommand{\haut}{\mathrm{ht}}
\newcommand{\Id}{\mathrm{Id}}
\newcommand{\ket}{\rangle}
\newcommand{\lie}[1]{\mathfrak{#1}}
\newcommand{\Mat}{\mathrm{Mat}}
\newcommand{\N}{\mathbb{N}}
\newcommand{\norm}[1]{\left\lVert #1\right\rVert}
\newcommand{\normalorder}[1]{\mathop{:}\nolimits\!#1\!\mathop{:}\nolimits}
\newcommand\NOT{\mathsf{NOT}}
\newcommand{\Oc}{\mathcal{O}}
\newcommand{\Or}{\mathrm{O}}
\newcommand\OR{\mathsf{OR}}
\newcommand{\ort}{\mathfrak{o}}
\newcommand{\PGL}{\mathrm{PGL}}
\newcommand{\ph}{\,\cdot\,}
\newcommand{\pr}{\mathrm{pr}}
\newcommand{\Prob}{\mathbb{P}}
\newcommand{\PSL}{\mathrm{PSL}}
\newcommand{\Ps}{\mathcal{P}}
\newcommand{\PSU}{\mathrm{PSU}}
\newcommand{\pt}{\mathrm{pt}}
\newcommand{\qeq}{\mathrel{``{=}"}}
\newcommand{\Q}{\mathbb{Q}}
\newcommand{\R}{\mathbb{R}}
\newcommand{\RP}{\mathbb{RP}}
\newcommand{\Rs}{\mathcal{R}}
\newcommand{\SL}{\mathrm{SL}}
\newcommand{\so}{\mathfrak{so}}
\newcommand{\SO}{\mathrm{SO}}
\newcommand{\Spin}{\mathrm{Spin}}
\newcommand{\Sp}{\mathrm{Sp}}
\newcommand{\su}{\mathfrak{su}}
\newcommand{\SU}{\mathrm{SU}}
\newcommand{\term}[1]{\emph{#1}\index{#1}}
\newcommand{\T}{\mathbb{T}}
\newcommand{\tv}[1]{|#1|}
\newcommand{\U}{\mathrm{U}}
\newcommand{\uu}{\mathfrak{u}}
\newcommand{\Vect}{\mathrm{Vect}}
\newcommand{\wsto}{\stackrel{\mathrm{w}^*}{\to}}
\newcommand{\wt}{\mathrm{wt}}
\newcommand{\wto}{\stackrel{\mathrm{w}}{\to}}
\newcommand{\Z}{\mathbb{Z}}
\renewcommand{\d}{\mathrm{d}}
\renewcommand{\H}{\mathbb{H}}
\renewcommand{\P}{\mathbb{P}}
\renewcommand{\sl}{\mathfrak{sl}}
\renewcommand{\vec}[1]{\boldsymbol{\mathbf{#1}}}
%\renewcommand{\F}{\mathcal{F}}

\let\Im\relax
\let\Re\relax

\DeclareMathOperator{\adj}{adj}
\DeclareMathOperator{\Ann}{Ann}
\DeclareMathOperator{\area}{area}
\DeclareMathOperator{\Aut}{Aut}
\DeclareMathOperator{\Bernoulli}{Bernoulli}
\DeclareMathOperator{\betaD}{beta}
\DeclareMathOperator{\bias}{bias}
\DeclareMathOperator{\binomial}{binomial}
\DeclareMathOperator{\card}{card}
\DeclareMathOperator{\ccl}{ccl}
\DeclareMathOperator{\Char}{char}
\DeclareMathOperator{\ch}{ch}
\DeclareMathOperator{\cl}{cl}
\DeclareMathOperator{\cls}{\overline{\mathrm{span}}}
\DeclareMathOperator{\conv}{conv}
\DeclareMathOperator{\corr}{corr}
\DeclareMathOperator{\cosec}{cosec}
\DeclareMathOperator{\cosech}{cosech}
\DeclareMathOperator{\cov}{cov}
\DeclareMathOperator{\covol}{covol}
\DeclareMathOperator{\diag}{diag}
\DeclareMathOperator{\diam}{diam}
\DeclareMathOperator{\Diff}{Diff}
\DeclareMathOperator{\disc}{disc}
\DeclareMathOperator{\dom}{dom}
\DeclareMathOperator{\End}{End}
\DeclareMathOperator{\energy}{energy}
\DeclareMathOperator{\erfc}{erfc}
\DeclareMathOperator{\erf}{erf}
\DeclareMathOperator*{\esssup}{ess\,sup}
\DeclareMathOperator{\ev}{ev}
\DeclareMathOperator{\Ext}{Ext}
\DeclareMathOperator{\Fit}{Fit}
\DeclareMathOperator{\fix}{fix}
\DeclareMathOperator{\Frac}{Frac}
\DeclareMathOperator{\Gal}{Gal}
\DeclareMathOperator{\gammaD}{gamma}
\DeclareMathOperator{\gr}{gr}
\DeclareMathOperator{\hcf}{hcf}
\DeclareMathOperator{\Hom}{Hom}
\DeclareMathOperator{\id}{id}
\DeclareMathOperator{\image}{image}
\DeclareMathOperator{\im}{im}
\DeclareMathOperator{\Im}{Im}
\DeclareMathOperator{\Ind}{Ind}
\DeclareMathOperator{\Int}{Int}
\DeclareMathOperator{\Isom}{Isom}
\DeclareMathOperator{\lcm}{lcm}
\DeclareMathOperator{\length}{length}
\DeclareMathOperator{\Lie}{Lie}
\DeclareMathOperator{\like}{like}
\DeclareMathOperator{\Lk}{Lk}
\DeclareMathOperator{\mse}{mse}
\DeclareMathOperator{\multinomial}{multinomial}
\DeclareMathOperator{\orb}{orb}
\DeclareMathOperator{\ord}{ord}
\DeclareMathOperator{\otp}{otp}
\DeclareMathOperator{\Poisson}{Poisson}
\DeclareMathOperator{\poly}{poly}
\DeclareMathOperator{\rank}{rank}
\DeclareMathOperator{\rel}{rel}
\DeclareMathOperator{\Re}{Re}
\DeclareMathOperator*{\res}{res}
\DeclareMathOperator{\Res}{Res}
\DeclareMathOperator{\rk}{rk}
\DeclareMathOperator{\Root}{Root}
\DeclareMathOperator{\sech}{sech}
\DeclareMathOperator{\sgn}{sgn}
\DeclareMathOperator{\spn}{span}
\DeclareMathOperator{\stab}{stab}
\DeclareMathOperator{\St}{St}
\DeclareMathOperator{\supp}{supp}
\DeclareMathOperator{\Syl}{Syl}
\DeclareMathOperator{\Sym}{Sym}
\DeclareMathOperator{\tr}{tr}
\DeclareMathOperator{\Tr}{Tr}
\DeclareMathOperator{\var}{var}
\DeclareMathOperator{\vol}{vol}

\pgfarrowsdeclarecombine{twolatex'}{twolatex'}{latex'}{latex'}{latex'}{latex'}
\tikzset{->/.style = {decoration={markings,
                                  mark=at position 1 with {\arrow[scale=2]{latex'}}},
                      postaction={decorate}}}
\tikzset{<-/.style = {decoration={markings,
                                  mark=at position 0 with {\arrowreversed[scale=2]{latex'}}},
                      postaction={decorate}}}
\tikzset{<->/.style = {decoration={markings,
                                   mark=at position 0 with {\arrowreversed[scale=2]{latex'}},
                                   mark=at position 1 with {\arrow[scale=2]{latex'}}},
                       postaction={decorate}}}
\tikzset{->-/.style = {decoration={markings,
                                   mark=at position #1 with {\arrow[scale=2]{latex'}}},
                       postaction={decorate}}}
\tikzset{-<-/.style = {decoration={markings,
                                   mark=at position #1 with {\arrowreversed[scale=2]{latex'}}},
                       postaction={decorate}}}
\tikzset{->>/.style = {decoration={markings,
                                  mark=at position 1 with {\arrow[scale=2]{latex'}}},
                      postaction={decorate}}}
\tikzset{<<-/.style = {decoration={markings,
                                  mark=at position 0 with {\arrowreversed[scale=2]{twolatex'}}},
                      postaction={decorate}}}
\tikzset{<<->>/.style = {decoration={markings,
                                   mark=at position 0 with {\arrowreversed[scale=2]{twolatex'}},
                                   mark=at position 1 with {\arrow[scale=2]{twolatex'}}},
                       postaction={decorate}}}
\tikzset{->>-/.style = {decoration={markings,
                                   mark=at position #1 with {\arrow[scale=2]{twolatex'}}},
                       postaction={decorate}}}
\tikzset{-<<-/.style = {decoration={markings,
                                   mark=at position #1 with {\arrowreversed[scale=2]{twolatex'}}},
                       postaction={decorate}}}

\tikzset{circ/.style = {fill, circle, inner sep = 0, minimum size = 3}}
\tikzset{mstate/.style={circle, draw, blue, text=black, minimum width=0.7cm}}

\tikzset{commutative diagrams/.cd,cdmap/.style={/tikz/column 1/.append style={anchor=base east},/tikz/column 2/.append style={anchor=base west},row sep=tiny}}

\definecolor{mblue}{rgb}{0.2, 0.3, 0.8}
\definecolor{morange}{rgb}{1, 0.5, 0}
\definecolor{mgreen}{rgb}{0.1, 0.4, 0.2}
\definecolor{mred}{rgb}{0.5, 0, 0}

\def\drawcirculararc(#1,#2)(#3,#4)(#5,#6){%
    \pgfmathsetmacro\cA{(#1*#1+#2*#2-#3*#3-#4*#4)/2}%
    \pgfmathsetmacro\cB{(#1*#1+#2*#2-#5*#5-#6*#6)/2}%
    \pgfmathsetmacro\cy{(\cB*(#1-#3)-\cA*(#1-#5))/%
                        ((#2-#6)*(#1-#3)-(#2-#4)*(#1-#5))}%
    \pgfmathsetmacro\cx{(\cA-\cy*(#2-#4))/(#1-#3)}%
    \pgfmathsetmacro\cr{sqrt((#1-\cx)*(#1-\cx)+(#2-\cy)*(#2-\cy))}%
    \pgfmathsetmacro\cA{atan2(#2-\cy,#1-\cx)}%
    \pgfmathsetmacro\cB{atan2(#6-\cy,#5-\cx)}%
    \pgfmathparse{\cB<\cA}%
    \ifnum\pgfmathresult=1
        \pgfmathsetmacro\cB{\cB+360}%
    \fi
    \draw (#1,#2) arc (\cA:\cB:\cr);%
}
\newcommand\getCoord[3]{\newdimen{#1}\newdimen{#2}\pgfextractx{#1}{\pgfpointanchor{#3}{center}}\pgfextracty{#2}{\pgfpointanchor{#3}{center}}}

\def\Xint#1{\mathchoice
   {\XXint\displaystyle\textstyle{#1}}%
   {\XXint\textstyle\scriptstyle{#1}}%
   {\XXint\scriptstyle\scriptscriptstyle{#1}}%
   {\XXint\scriptscriptstyle\scriptscriptstyle{#1}}%
   \!\int}
\def\XXint#1#2#3{{\setbox0=\hbox{$#1{#2#3}{\int}$}
     \vcenter{\hbox{$#2#3$}}\kern-.5\wd0}}
\def\ddashint{\Xint=}
\def\dashint{\Xint-}

\newcommand\separator{{\centering\rule{2cm}{0.2pt}\vspace{2pt}\par}}

\newenvironment{own}{\color{gray!70!black}}{}

\newcommand\makecenter[1]{\raisebox{-0.5\height}{#1}}
\newtheorem*{soln}{Solution}

\renewcommand{\thesection}{}
\renewcommand{\thesubsection}{\arabic{section}.\arabic{subsection}}
\makeatletter
\def\@seccntformat#1{\csname #1ignore\expandafter\endcsname\csname the#1\endcsname\quad}
\let\sectionignore\@gobbletwo
\let\latex@numberline\numberline
\def\numberline#1{\if\relax#1\relax\else\latex@numberline{#1}\fi}
\makeatother


\begin{document}
	
\maketitle

\section{QUESTION 1}
\begin{enumerate}
	\item Along characteristic curves,
	
	\[ \frac{\d x}{\d s} = 1 \quad \frac{\d y}{\d s} = y  \]
	
	which has general solution $ x = s + c $ and $ y = Ae^{s} $ for some constants $ c,A $. 
	
	Cauchy data is $ B(t) = \{ (x = 0, y = t) \} $, intersect $ B $ at $ s = 0 $, thus characteristic curves are
	
	\[ x = s \quad y = t e^{s} \]
	
	$ \frac{\partial u}{\partial s} \big|_{t} = 0 $ so that $ u = \text{cst} $ along these characteristics.
	
	 and the Cauchy data fixes $ u(s,t) = t^{3} $ on the $ t^{\text{th}} $ curve. Inverting gives
	
	\[ s = x \quad t = y e^{-x} \]
	
	and therefore the solution to our problem is 
	
	\[ u(x,y) = y^{3} e^{-3x} \]
	
	throughout $ \R^{2} $
	
	\item Along characteristic curves
	
	\[ \frac{\d x}{\d s} = y \quad \frac{\d y}{\d s} = x  \]
	
	\[ \Rightarrow \frac{\d^{2} x }{\d s^{2}} = x \]
	
	\[ x = A \sinh s + B \cosh s \]
	
	Cauchy data is $ B(t) = \{ (x = 0, y = t) \} $, intersect at $ s = 0 $ gives $ B = 0 $, thus
	
	\[ x = t \sinh s \qquad y = t \cosh s \]
	
	Then $ \frac{\partial u }{\partial s} \Big|_{t} = 0  \Rightarrow u = \text{ cst. }$ along these characteristics, and Cauchy data fixes $ u(s,t) = e^{-t^{2}} $, and have that
	
	\[ t^{2} = (t \cosh s)^{2} - (t \sinh s)^{2} = y^{2} - x^{2} \]
	
	therefore solution is
	
	\[ u(x,y) = e^{x^{2} - y^{2}} \]
	
	which is uniquely defined only in the upper quadrant of the plane $ \{ (x,y) \subset R^{2} : x \geq 0, y \geq 0 \} $
	
	\item PDE is $ u_{x} + u_{y} = e^{x + 2y} - u $.
	
	Along characteristic curves
	
	\[ \frac{\d x}{\d s} = 1 \quad \frac{\d y}{\d s} = 1  \]
	
	
	\[ x = s + c \qquad y = s + d \]
	
	Cauchy data is $ B(t) = \{ (x = t, y = 0) \} $, intersect at $ s = 0 $, thus
	
	\[ x = s + t  \qquad y = s \]

	Then $ \frac{\partial u }{\partial s} \Big|_{t} = e^{2x + y} - u = e^{3s + t} - u $ along these characteristics, this is an ODE in $ s $ so multiplying by the integrating factor $ e^{s} $ gives
	
	\[ e^{s} \frac{\d u}{\d s} + e^{s} u = e^{4s + t}  \]
	
	\[ \Rightarrow e^{s} u = \frac{1}{4} e^{4s + t } + \text{cst.} \]
	
	Cauchy data  $ u(t,0) = 0 $ fixes $ \text{cst} = -  \frac{1}{4} e^{t} $, and have that
	
	\[ u(s,t) = \frac{1}{4} e^{t  + 3s} -  \frac{1}{4} e^{t-s}  \]


	inverting the relations
	
	\[ s = y \qquad t = x - y \]
	
	the solution is
	
	\begin{align*}
	u(x,y) & = \frac{1}{4}e^{x + 2y} - \frac{1}{4} e^{x - 2y}  \\
	& = \frac{1}{2}e^{x} \sinh 2y
	\end{align*}
	
\end{enumerate}

\section{QUESTION 2}

Separation of variables $ \Rightarrow u(x,t) = X(x) T(t) $, thus

\[ X''T + X \dot{T} = 0 \]
\[ \Rightarrow \frac{X''}{X} = - \frac{\dot{T}}{T}  \]

LHS independent of $ x $, RHS independent of $ t $, so both sides constant.
Setting $ \lambda = \dot{T} / T $ we have

\[ X'' + \lambda X = 0 \]
\[ X(x) = A \cos \sqrt{\lambda} x + B \sin \sqrt{\lambda} x \]

Boundary conditions $ X(0) = X(\pi) = 0 $ imply that $ A = 0 $, $ \lambda = n^{2} $. Then solving $ \dot{T} = \lambda T $ with condition $ T(0) = U(x) $ gives

\[ T(t) = U(x) e^{n^{2}t} \]

Thus the unnormalised eigenfunctions of our problem are

\[ u(x,t) = U(x) e^{n^{2}t} \sin n x  \]

For large n the solution then has oscillations with higher and higher wavenumber and larger and larger (indeed arbitrarily large) amplitude $ U(x) e^{n^{2}t} $, and so this problem is ill-posed.

\section{QUESTION 3}

\begin{enumerate}
	\item The principal part of the symbol of the differential operator is $ \mathbf{k}^{T} \mathbf{A} \mathbf{k} $, where 

	\[ A = \begin{pmatrix}
	1 & 0 \\
	0 & x
	\end{pmatrix} \]
	
	$ \det A  $ is product of eigenvalues, therefore we have 
	
	\begin{center}
		\begin{tabular}{rll}
			elliptic  & $ x > 0  $ \\
			parabolic  &  $ x = 0 $  \\
			hyperbolic &  $ x < 0$ 
		\end{tabular}
	\end{center}

	In the hyperbolic region the negative eigenvector points in the $ y $ direction. Thus, if $ f(x,y) = \text{cst.} $ is to be a characteristic surface, we need $ \partial_{y} f = \pm \sqrt{ \partial_{x} f  A_{xx} \partial_{x} f / x } = \pm \sqrt{x} \partial_{x} f  $. Letting $ p = 2 x^{1/2} $, this is $ (\partial_{y} \pm \partial_{p})f = 0 $, so the characteristic surfaces are the two curves of constant $ y \pm p $, that is
	
	\[ u = y + x^{1/2} \]
	\[ v = y - x^{1/2} \]
	
	for $ u $ constant and $ v $ constant.

	\item The PDE is hyperbolic in the $ y < 0 $ region. Here
	
	\[ \mathbf{A} = \text{diag}(1,y) \]
	
	and $ \mathbf{m} $ points in the $ y $-direction, with $ \mathbf{A} \mathbf{m} = y \mathbf{m} $. Thus, if $ f(x,y) = \text{cst.} $ is to be a characteristic surface, we need $ \partial_{y} f = \pm \sqrt{ \partial_{x} f  A_{xx} \partial_{x} f / y } = \pm \sqrt{y} \partial_{x} f  $. Letting $ \xi = x, \nu  = 2 y^{1/2} $, this is $ (\partial_{\xi} \pm \partial_{p})f = 0 $, so the characteristic surfaces are the two curves of constant $ y \pm p $, that is
	
	 
	
\end{enumerate}



\section{QUESTION 4}

Green's second identity is 

\[ \int_{\Omega}  \phi \nabla^{2} \psi  - \psi \nabla^{2} \phi \;  \d V = \int_{\delta \Omega} \phi ( \mathbf{n} \cdot \nabla \psi ) - \psi( \mathbf{n} \cdot \nabla \phi ) \d S   \]

where $ \Omega \subset \R^{n} $ is a compact set, and $ \phi, \psi : \Omega \to \R $ are a pair of functions on $ \Omega $ regular throughout $ \Omega $.

\section{QUESTION 5}

$ u(\mathbf{x}) $ is harmonic and therefore satisfies $ \nabla^{2} \mathbf{u} = 0 $. Consider a Dirichlet Green's function for the Laplace operator on $ D $; we have

\[ \nabla^{2} G ( \mathbf{r} ; \mathbf{r}_{0}) \]

It can be shown that

\[ G (\mathbf{x} ; \mathbf{x}_{0} ) = \frac{1}{2\pi} \log | \mathbf{x} - \mathbf{x}_{0} | \]

Using Green's second identity with $ \mathbf{u} $ and $ G $ we have

\[ \int_{\Omega}  u \nabla^{2} G  - G \nabla^{2} u \; \d V = \int_{\delta \Omega} G ( \mathbf{n} \cdot \nabla u ) - u ( \mathbf{n} \cdot \nabla G ) \d S   \]

For some reason, $ \mathbf{n} \cdot \nabla u = \frac{\partial u }{\partial n} $. Also, since $ G $ only depends on the outward normal, we have $ \mathbf{n} \cdot \nabla G =  \frac{\partial G }{\partial n} $.  

Thus the equation becomes 

\[ \int_{\Omega}  u \nabla^{2} G  - G \nabla^{2} u \; \d V = \int_{\delta \Omega} G ( \mathbf{n} \cdot \nabla u ) - u ( \mathbf{n} \cdot \nabla G ) \d S  \]



\section{QUESTION 6}



\section{QUESTION 7}



\section{QUESTION 8}
\section{QUESTION 9}
\section{QUESTION 10}

Let $ \Omega = \{  (x,y) \in \R^{2} \; : \; y \geq 0 \} $ and suppose $ \psi : \Omega \to \R $ solves Laplace's equation $ \nabla^{2} \psi = 0 $ inside $ \Omega $, subject to

\[ \psi(x,0) = f(x) \quad \text{ and } \quad \lim\limits_{| \mathbf{x} | \to \infty} \psi = 0 \]

\begin{enumerate}
	\item We must construct a Green's function that vanishes on $ \delta \Omega $. As well as vanishing on the $ x $-axis, we also require $ G $ vanishes as $ | \mathbf{x} | \to \infty $. We'll set $ \mathbf{x} = (x,y) $ and $ \mathbf{y} : = \mathbf{x}_{0}^{+} = (x_{0},y_{0}) $ in terms of Cartesian coordinates, with $ y_{0} > 0 $. We know that the free-space Green's function
	
	\[ G_{2}(\mathbf{x},\mathbf{x}_{0}^{+}) = \frac{1}{2\pi} \log |  \mathbf{x} - \mathbf{x_{0}^{+}} | + c_{2}  \]
	
	satisfies all conditions except that
	
		\[ G_{2}(\mathbf{x},\mathbf{x}_{0}^{+}) |_{y = 0} = \frac{1}{2\pi} \log | [ (x - x_{0})^{2} + y_{0}^{2} ]^{1/2} | + c_{2} \neq 0  \]
		
	We need to cancel the nonzero boundary value of $ G_{2} $ by adding on some function.
	
	Let $ \mathbf{x_{0}}^{-} $ be the point $ (x_{0},-y_{0}) $. The location $ \mathbf{x_{0}}^{-} \notin \Omega $, so the Green's function $ G_{2}(\mathbf{x},\mathbf{x}_{0}^{-}) $ is regular everywhere within $ \Omega $, and so obeys Laplace's equation everywhere in the upper half-space. Also,
	
	\[ G_{2}(\mathbf{x},\mathbf{x}_{0}^{-} |_{y = 0} = \frac{1}{2\pi} \log | [ (x - x_{0})^{2} + y_{0}^{2} ]^{1/2} | + c_{2}'  \]
	

	
\end{enumerate}




\section{QUESTION 11}
\section{QUESTION 12}




\end{document}