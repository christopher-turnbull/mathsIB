\documentclass[a4paper]{article}
\usepackage{amsmath}
\def\npart {IB}
\def\nterm {Michaelmas}
\def\nyear {2017}
\def\nlecturer {Dr. Saxton}
\def\ncourse {Methods Example Sheet 4}

\input{header}
\newtheorem*{soln}{Solution}

\renewcommand{\thesection}{}
\renewcommand{\thesubsection}{\arabic{section}.\arabic{subsection}}
\makeatletter
\def\@seccntformat#1{\csname #1ignore\expandafter\endcsname\csname the#1\endcsname\quad}
\let\sectionignore\@gobbletwo
\let\latex@numberline\numberline
\def\numberline#1{\if\relax#1\relax\else\latex@numberline{#1}\fi}
\makeatother


\begin{document}
	
\maketitle

\section{QUESTION 1}
\begin{enumerate}
	\item Along characteristic curves,
	
	\[ \frac{\d x}{\d s} = 1 \quad \frac{\d y}{\d s} = y  \]
	
	which has general solution $ x = s + c $ and $ y = Ae^{s} $ for some constants $ c,A $. 
	
	Cauchy data is $ B(t) = \{ (x = 0, y = t) \} $, intersect $ B $ at $ s = 0 $, thus characteristic curves are
	
	\[ x = s \quad y = t e^{s} \]
	
	$ \frac{\partial u}{\partial s} \big|_{t} = 0 $ so that $ u = \text{cst} $ along these characteristics.
	
	 and the Cauchy data fixes $ u(s,t) = t^{3} $ on the $ t^{\text{th}} $ curve. Inverting gives
	
	\[ s = x \quad t = y e^{-x} \]
	
	and therefore the solution to our problem is 
	
	\[ u(x,y) = y^{3} e^{-3x} \]
	
	throughout $ \R^{2} $
	
	\item Along characteristic curves
	
	\[ \frac{\d x}{\d s} = y \quad \frac{\d y}{\d s} = x  \]
	
	\[ \Rightarrow \frac{\d^{2} x }{\d s^{2}} = x \]
	
	\[ x = A \sinh s + B \cosh s \]
	
	Cauchy data is $ B(t) = \{ (x = 0, y = t) \} $, intersect at $ s = 0 $ gives $ B = 0 $, thus
	
	\[ x = t \sinh s \qquad y = t \cosh s \]
	
	Then $ \frac{\partial u }{\partial s} \Big|_{t} = 0  \Rightarrow u = \text{ cst. }$ along these characteristics, and Cauchy data fixes $ u(s,t) = e^{-t^{2}} $, and have that
	
	\[ t^{2} = (t \cosh s)^{2} - (t \sinh s)^{2} = y^{2} - x^{2} \]
	
	therefore solution is
	
	\[ u(x,y) = e^{x^{2} - y^{2}} \]
	
	which is uniquely defined only in the upper quadrant of the plane $ \{ (x,y) \subset R^{2} : x \geq 0, y \geq 0 \} $
	
	\item PDE is $ u_{x} + u_{y} = e^{x + 2y} - u $.
	
	Along characteristic curves
	
	\[ \frac{\d x}{\d s} = 1 \quad \frac{\d y}{\d s} = 1  \]
	
	
	\[ x = s + c \qquad y = s + d \]
	
	Cauchy data is $ B(t) = \{ (x = t, y = 0) \} $, intersect at $ s = 0 $, thus
	
	\[ x = s + t  \qquad y = s \]

	Then $ \frac{\partial u }{\partial s} \Big|_{t} = e^{2x + y} - u = e^{3s + t} - u $ along these characteristics, this is an ODE in $ s $ so multiplying by the integrating factor $ e^{s} $ gives
	
	\[ e^{s} \frac{\d u}{\d s} + e^{s} u = e^{4s + t}  \]
	
	\[ \Rightarrow e^{s} u = \frac{1}{4} e^{4s + t } + \text{cst.} \]
	
	Cauchy data  $ u(t,0) = 0 $ fixes $ \text{cst} = -  \frac{1}{4} e^{t} $, and have that
	
	\[ u(s,t) = \frac{1}{4} e^{t  + 3s} -  \frac{1}{4} e^{t-s}  \]


	inverting the relations
	
	\[ s = y \qquad t = x - y \]
	
	the solution is
	
	\begin{align*}
	u(x,y) & = \frac{1}{4}e^{x + 2y} - \frac{1}{4} e^{x - 2y}  \\
	& = \frac{1}{2}e^{x} \sinh 2y
	\end{align*}
	
\end{enumerate}

\section{QUESTION 2}

Separation of variables $ \Rightarrow u(x,t) = X(x) T(t) $, thus

\[ X''T + X \dot{T} = 0 \]
\[ \Rightarrow \frac{X''}{X} = - \frac{\dot{T}}{T}  \]

LHS independent of $ x $, RHS independent of $ t $, so both sides constant.
Setting $ \lambda = \dot{T} / T $ we have

\[ X'' + \lambda X = 0 \]
\[ X(x) = A \cos \sqrt{\lambda} x + B \sin \sqrt{\lambda} x \]

Boundary conditions $ X(0) = X(\pi) = 0 $ imply that $ A = 0 $, $ \lambda = n^{2} $. Then solving $ \dot{T} = \lambda T $ with condition $ T(0) = U(x) $ gives

\[ T(t) = U(x) e^{n^{2}t} \]

Thus the unnormalised eigenfunctions of our problem are

\[ u(x,t) = U(x) e^{n^{2}t} \sin n x  \]

For large n the solution then has oscillations with higher and higher wavenumber and larger and larger (indeed arbitrarily large) amplitude $ U(x) e^{n^{2}t} $, and so this problem is ill-posed.

\section{QUESTION 3}

\begin{enumerate}
	\item The principal part of the symbol of the differential operator is $ \mathbf{k}^{T} \mathbf{A} \mathbf{k} $, where 

	\[ A = \begin{pmatrix}
	1 & 0 \\
	0 & x
	\end{pmatrix} \]
	
	$ \det A  $ is product of eigenvalues, therefore we have 
	
	\begin{center}
		\begin{tabular}{rll}
			elliptic  & $ x > 0  $ \\
			parabolic  &  $ x = 0 $  \\
			hyperbolic &  $ x < 0$ 
		\end{tabular}
	\end{center}

	In the hyperbolic region the negative eigenvector points in the $ y $ direction. Thus, if $ f(x,y) = \text{cst.} $ is to be a characteristic surface, we need $ \partial_{y} f = \pm \sqrt{ \partial_{x} f  A_{xx} \partial_{x} f / x } = \pm \sqrt{x} \partial_{x} f  $. Letting $ p = 2 x^{1/2} $, this is $ (\partial_{y} \pm \partial_{p})f = 0 $, so the characteristic surfaces are the two curves of constant $ y \pm p $, that is
	
	\[ u = y + x^{1/2} \]
	\[ v = y - x^{1/2} \]
	
	for $ u $ constant and $ v $ constant.

	\item The PDE is hyperbolic in the $ y < 0 $ region. Here
	
	\[ \mathbf{A} = \text{diag}(1,y) \]
	
	and $ \mathbf{m} $ points in the $ y $-direction, with $ \mathbf{A} \mathbf{m} = y \mathbf{m} $. Thus, if $ f(x,y) = \text{cst.} $ is to be a characteristic surface, we need $ \partial_{y} f = \pm \sqrt{ \partial_{x} f  A_{xx} \partial_{x} f / y } = \pm \sqrt{y} \partial_{x} f  $. Letting $ \xi = x, \nu  = 2 y^{1/2} $, this is $ (\partial_{\xi} \pm \partial_{p})f = 0 $, so the characteristic surfaces are the two curves of constant $ y \pm p $, that is
	
	 
	
\end{enumerate}



\section{QUESTION 4}

Green's second identity is 

\[ \int_{\Omega}  \phi \nabla^{2} \psi  - \psi \nabla^{2} \phi \;  \d V = \int_{\delta \Omega} \phi ( \mathbf{n} \cdot \nabla \psi ) - \psi( \mathbf{n} \cdot \nabla \phi ) \d S   \]

where $ \Omega \subset \R^{n} $ is a compact set, and $ \phi, \psi : \Omega \to \R $ are a pair of functions on $ \Omega $ regular throughout $ \Omega $.

\section{QUESTION 5}

$ u(\mathbf{x}) $ is harmonic and therefore satisfies $ \nabla^{2} \mathbf{u} = 0 $. Consider a Dirichlet Green's function for the Laplace operator on $ D $; we have

\[ \nabla^{2} G ( \mathbf{r} ; \mathbf{r}_{0}) \]

It can be shown that

\[ G (\mathbf{x} ; \mathbf{x}_{0} ) = \frac{1}{2\pi} \log | \mathbf{x} - \mathbf{x}_{0} | \]

Using Green's second identity with $ \mathbf{u} $ and $ G $ we have

\[ \int_{\Omega}  u \nabla^{2} G  - G \nabla^{2} u \; \d V = \int_{\delta \Omega} G ( \mathbf{n} \cdot \nabla u ) - u ( \mathbf{n} \cdot \nabla G ) \d S   \]

For some reason, $ \mathbf{n} \cdot \nabla u = \frac{\partial u }{\partial n} $. Also, since $ G $ only depends on the outward normal, we have $ \mathbf{n} \cdot \nabla G =  \frac{\partial G }{\partial n} $.  

Thus the equation becomes 

\[ \int_{\Omega}  u \nabla^{2} G  - G \nabla^{2} u \; \d V = \int_{\delta \Omega} G ( \mathbf{n} \cdot \nabla u ) - u ( \mathbf{n} \cdot \nabla G ) \d S  \]



\section{QUESTION 6}



\section{QUESTION 7}



\section{QUESTION 8}
\section{QUESTION 9}
\section{QUESTION 10}

Let $ \Omega = \{  (x,y) \in \R^{2} \; : \; y \geq 0 \} $ and suppose $ \psi : \Omega \to \R $ solves Laplace's equation $ \nabla^{2} \psi = 0 $ inside $ \Omega $, subject to

\[ \psi(x,0) = f(x) \quad \text{ and } \quad \lim\limits_{| \mathbf{x} | \to \infty} \psi = 0 \]

\begin{enumerate}
	\item We must construct a Green's function that vanishes on $ \delta \Omega $. As well as vanishing on the $ x $-axis, we also require $ G $ vanishes as $ | \mathbf{x} | \to \infty $. We'll set $ \mathbf{x} = (x,y) $ and $ \mathbf{y} : = \mathbf{x}_{0}^{+} = (x_{0},y_{0}) $ in terms of Cartesian coordinates, with $ y_{0} > 0 $. We know that the free-space Green's function
	
	\[ G_{2}(\mathbf{x},\mathbf{x}_{0}^{+}) = \frac{1}{2\pi} \log |  \mathbf{x} - \mathbf{x_{0}^{+}} | + c_{2}  \]
	
	satisfies all conditions except that
	
		\[ G_{2}(\mathbf{x},\mathbf{x}_{0}^{+}) |_{y = 0} = \frac{1}{2\pi} \log | [ (x - x_{0})^{2} + y_{0}^{2} ]^{1/2} | + c_{2} \neq 0  \]
		
	We need to cancel the nonzero boundary value of $ G_{2} $ by adding on some function.
	
	Let $ \mathbf{x_{0}}^{-} $ be the point $ (x_{0},-y_{0}) $. The location $ \mathbf{x_{0}}^{-} \notin \Omega $, so the Green's function $ G_{2}(\mathbf{x},\mathbf{x}_{0}^{-}) $ is regular everywhere within $ \Omega $, and so obeys Laplace's equation everywhere in the upper half-space. Also,
	
	\[ G_{2}(\mathbf{x},\mathbf{x}_{0}^{-} |_{y = 0} = \frac{1}{2\pi} \log | [ (x - x_{0})^{2} + y_{0}^{2} ]^{1/2} | + c_{2}'  \]
	

	
\end{enumerate}




\section{QUESTION 11}
\section{QUESTION 12}




\end{document}