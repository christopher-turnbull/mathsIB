\documentclass[a4paper]{article}
\usepackage{amsmath}
\def\npart {IB}
\def\nterm {Michaelmas}
\def\nyear {2017}
\def\nlecturer {Dr. Saxton}
\def\ncourse {Methods Example Sheet 1}

\input{header}
\newtheorem*{soln}{Solution}

\renewcommand{\thesection}{}
\renewcommand{\thesubsection}{\arabic{section}.\arabic{subsection}}
\makeatletter
\def\@seccntformat#1{\csname #1ignore\expandafter\endcsname\csname the#1\endcsname\quad}
\let\sectionignore\@gobbletwo
\let\latex@numberline\numberline
\def\numberline#1{\if\relax#1\relax\else\latex@numberline{#1}\fi}
\makeatother


\begin{document}
	
\maketitle

\section{QUESTION 1}

We have $ \nabla^{2} \phi = 0 $ on $ 0 < x < a $, $ 0 < y < b $, $ 0 < z < c $ with $ \phi = 1 $ on the $ z $ surface and $ \phi = 0 $ on all other surfaces:

Assume $ \phi(x,y,z) = X(x) Y(y) Z(z) $, so we have

\[ \frac{X''}{X} + \frac{Y''}{Y} + \frac{Z''}{Z} = 0 \]

Solving $ X'' = - \lambda_{p} X $ such that $ X(0) = X(a) = 0 $ implies that

\[ \lambda_{p} = \frac{p^{2} \pi^{2}}{a^{2}}, X_{l} = \sqrt{\frac{2}{a}} \sin \left( \frac{p \pi x}{a} \right), l = 1,2,3,\cdots  \]

Similarly, solving $ Y'' = - \mu_{q} Y  $, such that $ Y(0) = Y(b) = 0 $ implies that

\[ \mu_{q} = \frac{q^{2} \pi^{2}}{b^{2}}, Y_{q} = \sqrt{\frac{2}{b}} \sin \left(  \frac{q \pi x}{b} \right), m = 1,2,3,\cdots  \]

Now solving for $ Z $ using the eigenvalues:

\[ Z'' = \left(   \frac{p^{2} \pi^{2}}{a^{2}}  + \frac{q^{2} \pi^{2}}{b^{2}} \right) Z, \]

\[ Z = \alpha \cosh \left[  \left(    \frac{p^{2}}{a^{2}} + \frac{q^{2}}{b^{2}} \right)^{1/2} \pi z  \right] + \beta \sinh \left[  \left(    \frac{p^{2}}{a^{2}} + \frac{q^{2}}{b^{2}} \right)^{1/2} \pi z  \right]  \]

Therefore, the general solution is 

\[ \psi(x,y,z) 
= \frac{2}{\sqrt{ab}} \sum_{p = 0} \sum_{q = 0} a_{pq} \sin \left(  \frac{p \pi}{a} x \right) \sin \left(  \frac{q \pi}{b} y \right)  \sinh \left[  \left(    \frac{p^{2}}{a^{2}} + \frac{q^{2}}{b^{2}} \right)^{1/2} \pi z  \right]     \]





\section{QUESTION 2}
\section{QUESTION 3}
\section{QUESTION 4}
\section{QUESTION 5}
\section{QUESTION 6}
\section{QUESTION 7}
\section{QUESTION 8}
\section{QUESTION 9}
\section{QUESTION 10}

\end{document}