\documentclass[a4paper]{article}
\usepackage{amsmath}
\def\npart {IB}
\def\nterm {Michaelmas}
\def\nyear {2017}
\def\nlecturer {Dr. Saxton}
\def\ncourse {Methods Example Sheet 1}

% Imports
\ifx \nauthor\undefined
  \def\nauthor{Christopher Turnbull}
\else
\fi

\author{Supervised by \nlecturer \\\small Solutions presented by \nauthor}
\date{\nterm\ \nyear}

\usepackage{alltt}
\usepackage{amsfonts}
\usepackage{amsmath}
\usepackage{amssymb}
\usepackage{amsthm}
\usepackage{booktabs}
\usepackage{caption}
\usepackage{enumitem}
\usepackage{fancyhdr}
\usepackage{graphicx}
\usepackage{mathdots}
\usepackage{mathtools}
\usepackage{microtype}
\usepackage{multirow}
\usepackage{pdflscape}
\usepackage{pgfplots}
\usepackage{siunitx}
\usepackage{slashed}
\usepackage{tabularx}
\usepackage{tikz}
\usepackage{tkz-euclide}
\usepackage[normalem]{ulem}
\usepackage[all]{xy}
\usepackage{imakeidx}

\makeindex[intoc, title=Index]
\indexsetup{othercode={\lhead{\emph{Index}}}}

\ifx \nextra \undefined
  \usepackage[pdftex,
    hidelinks,
    pdfauthor={Christopher Turnbull},
    pdfsubject={Cambridge Maths Notes: Part \npart\ - \ncourse},
    pdftitle={Part \npart\ - \ncourse},
  pdfkeywords={Cambridge Mathematics Maths Math \npart\ \nterm\ \nyear\ \ncourse}]{hyperref}
  \title{Part \npart\ --- \ncourse}
\else
  \usepackage[pdftex,
    hidelinks,
    pdfauthor={Christopher Turnbull},
    pdfsubject={Cambridge Maths Notes: Part \npart\ - \ncourse\ (\nextra)},
    pdftitle={Part \npart\ - \ncourse\ (\nextra)},
  pdfkeywords={Cambridge Mathematics Maths Math \npart\ \nterm\ \nyear\ \ncourse\ \nextra}]{hyperref}

  \title{Part \npart\ --- \ncourse \\ {\Large \nextra}}
  \renewcommand\printindex{}
\fi

\pgfplotsset{compat=1.12}

\pagestyle{fancyplain}
\lhead{\emph{\nouppercase{\leftmark}}}
\ifx \nextra \undefined
  \rhead{
    \ifnum\thepage=1
    \else
      \npart\ \ncourse
    \fi}
\else
  \rhead{
    \ifnum\thepage=1
    \else
      \npart\ \ncourse\ (\nextra)
    \fi}
\fi
\usetikzlibrary{arrows.meta}
\usetikzlibrary{decorations.markings}
\usetikzlibrary{decorations.pathmorphing}
\usetikzlibrary{positioning}
\usetikzlibrary{fadings}
\usetikzlibrary{intersections}
\usetikzlibrary{cd}

\newcommand*{\Cdot}{{\raisebox{-0.25ex}{\scalebox{1.5}{$\cdot$}}}}
\newcommand {\pd}[2][ ]{
  \ifx #1 { }
    \frac{\partial}{\partial #2}
  \else
    \frac{\partial^{#1}}{\partial #2^{#1}}
  \fi
}
\ifx \nhtml \undefined
\else
  \renewcommand\printindex{}
  \makeatletter
  \DisableLigatures[f]{family = *}
  \let\Contentsline\contentsline
  \renewcommand\contentsline[3]{\Contentsline{#1}{#2}{}}
  \renewcommand{\@dotsep}{10000}
  \newlength\currentparindent
  \setlength\currentparindent\parindent

  \newcommand\@minipagerestore{\setlength{\parindent}{\currentparindent}}
  \usepackage[active,tightpage,pdftex]{preview}
  \renewcommand{\PreviewBorder}{0.1cm}

  \newenvironment{stretchpage}%
  {\begin{preview}\begin{minipage}{\hsize}}%
    {\end{minipage}\end{preview}}
  \AtBeginDocument{\begin{stretchpage}}
  \AtEndDocument{\end{stretchpage}}

  \newcommand{\@@newpage}{\end{stretchpage}\begin{stretchpage}}

  \let\@real@section\section
  \renewcommand{\section}{\@@newpage\@real@section}
  \let\@real@subsection\subsection
  \renewcommand{\subsection}{\@@newpage\@real@subsection}
  \makeatother
\fi

% Theorems
\theoremstyle{definition}
\newtheorem*{aim}{Aim}
\newtheorem*{axiom}{Axiom}
\newtheorem*{claim}{Claim}
\newtheorem*{cor}{Corollary}
\newtheorem*{conjecture}{Conjecture}
\newtheorem*{defi}{Definition}
\newtheorem*{eg}{Example}
\newtheorem*{ex}{Exercise}
\newtheorem*{fact}{Fact}
\newtheorem*{law}{Law}
\newtheorem*{lemma}{Lemma}
\newtheorem*{notation}{Notation}
\newtheorem*{prop}{Proposition}
\newtheorem*{soln}{Solution}
\newtheorem*{thm}{Theorem}

\newtheorem*{remark}{Remark}
\newtheorem*{warning}{Warning}
\newtheorem*{exercise}{Exercise}

\newtheorem{nthm}{Theorem}[section]
\newtheorem{nlemma}[nthm]{Lemma}
\newtheorem{nprop}[nthm]{Proposition}
\newtheorem{ncor}[nthm]{Corollary}


\renewcommand{\labelitemi}{--}
\renewcommand{\labelitemii}{$\circ$}
\renewcommand{\labelenumi}{(\roman{*})}

\let\stdsection\section
\renewcommand\section{\newpage\stdsection}

% Strike through
\def\st{\bgroup \ULdepth=-.55ex \ULset}

% Maths symbols
\newcommand{\abs}[1]{\left\lvert #1\right\rvert}
\newcommand\ad{\mathrm{ad}}
\newcommand\AND{\mathsf{AND}}
\newcommand\Art{\mathrm{Art}}
\newcommand{\Bilin}{\mathrm{Bilin}}
\newcommand{\bket}[1]{\left\lvert #1\right\rangle}
\newcommand{\B}{\mathcal{B}}
\newcommand{\bolds}[1]{{\bfseries #1}}
\newcommand{\brak}[1]{\left\langle #1 \right\rvert}
\newcommand{\braket}[2]{\left\langle #1\middle\vert #2 \right\rangle}
\newcommand{\bra}{\langle}
\newcommand{\cat}[1]{\mathsf{#1}}
\newcommand{\C}{\mathbb{C}}
\newcommand{\CP}{\mathbb{CP}}
\newcommand{\cU}{\mathcal{U}}
\newcommand{\Der}{\mathrm{Der}}
\newcommand{\D}{\mathrm{D}}
\newcommand{\dR}{\mathrm{dR}}
\newcommand{\E}{\mathbb{E}}
\newcommand{\F}{\mathbb{F}}
\newcommand{\Frob}{\mathrm{Frob}}
\newcommand{\GG}{\mathbb{G}}
\newcommand{\gl}{\mathfrak{gl}}
\newcommand{\GL}{\mathrm{GL}}
\newcommand{\G}{\mathcal{G}}
\newcommand{\Gr}{\mathrm{Gr}}
\newcommand{\haut}{\mathrm{ht}}
\newcommand{\Id}{\mathrm{Id}}
\newcommand{\ket}{\rangle}
\newcommand{\lie}[1]{\mathfrak{#1}}
\newcommand{\Mat}{\mathrm{Mat}}
\newcommand{\N}{\mathbb{N}}
\newcommand{\norm}[1]{\left\lVert #1\right\rVert}
\newcommand{\normalorder}[1]{\mathop{:}\nolimits\!#1\!\mathop{:}\nolimits}
\newcommand\NOT{\mathsf{NOT}}
\newcommand{\Oc}{\mathcal{O}}
\newcommand{\Or}{\mathrm{O}}
\newcommand\OR{\mathsf{OR}}
\newcommand{\ort}{\mathfrak{o}}
\newcommand{\PGL}{\mathrm{PGL}}
\newcommand{\ph}{\,\cdot\,}
\newcommand{\pr}{\mathrm{pr}}
\newcommand{\Prob}{\mathbb{P}}
\newcommand{\PSL}{\mathrm{PSL}}
\newcommand{\Ps}{\mathcal{P}}
\newcommand{\PSU}{\mathrm{PSU}}
\newcommand{\pt}{\mathrm{pt}}
\newcommand{\qeq}{\mathrel{``{=}"}}
\newcommand{\Q}{\mathbb{Q}}
\newcommand{\R}{\mathbb{R}}
\newcommand{\RP}{\mathbb{RP}}
\newcommand{\Rs}{\mathcal{R}}
\newcommand{\SL}{\mathrm{SL}}
\newcommand{\so}{\mathfrak{so}}
\newcommand{\SO}{\mathrm{SO}}
\newcommand{\Spin}{\mathrm{Spin}}
\newcommand{\Sp}{\mathrm{Sp}}
\newcommand{\su}{\mathfrak{su}}
\newcommand{\SU}{\mathrm{SU}}
\newcommand{\term}[1]{\emph{#1}\index{#1}}
\newcommand{\T}{\mathbb{T}}
\newcommand{\tv}[1]{|#1|}
\newcommand{\U}{\mathrm{U}}
\newcommand{\uu}{\mathfrak{u}}
\newcommand{\Vect}{\mathrm{Vect}}
\newcommand{\wsto}{\stackrel{\mathrm{w}^*}{\to}}
\newcommand{\wt}{\mathrm{wt}}
\newcommand{\wto}{\stackrel{\mathrm{w}}{\to}}
\newcommand{\Z}{\mathbb{Z}}
\renewcommand{\d}{\mathrm{d}}
\renewcommand{\H}{\mathbb{H}}
\renewcommand{\P}{\mathbb{P}}
\renewcommand{\sl}{\mathfrak{sl}}
\renewcommand{\vec}[1]{\boldsymbol{\mathbf{#1}}}
%\renewcommand{\F}{\mathcal{F}}

\let\Im\relax
\let\Re\relax

\DeclareMathOperator{\adj}{adj}
\DeclareMathOperator{\Ann}{Ann}
\DeclareMathOperator{\area}{area}
\DeclareMathOperator{\Aut}{Aut}
\DeclareMathOperator{\Bernoulli}{Bernoulli}
\DeclareMathOperator{\betaD}{beta}
\DeclareMathOperator{\bias}{bias}
\DeclareMathOperator{\binomial}{binomial}
\DeclareMathOperator{\card}{card}
\DeclareMathOperator{\ccl}{ccl}
\DeclareMathOperator{\Char}{char}
\DeclareMathOperator{\ch}{ch}
\DeclareMathOperator{\cl}{cl}
\DeclareMathOperator{\cls}{\overline{\mathrm{span}}}
\DeclareMathOperator{\conv}{conv}
\DeclareMathOperator{\corr}{corr}
\DeclareMathOperator{\cosec}{cosec}
\DeclareMathOperator{\cosech}{cosech}
\DeclareMathOperator{\cov}{cov}
\DeclareMathOperator{\covol}{covol}
\DeclareMathOperator{\diag}{diag}
\DeclareMathOperator{\diam}{diam}
\DeclareMathOperator{\Diff}{Diff}
\DeclareMathOperator{\disc}{disc}
\DeclareMathOperator{\dom}{dom}
\DeclareMathOperator{\End}{End}
\DeclareMathOperator{\energy}{energy}
\DeclareMathOperator{\erfc}{erfc}
\DeclareMathOperator{\erf}{erf}
\DeclareMathOperator*{\esssup}{ess\,sup}
\DeclareMathOperator{\ev}{ev}
\DeclareMathOperator{\Ext}{Ext}
\DeclareMathOperator{\Fit}{Fit}
\DeclareMathOperator{\fix}{fix}
\DeclareMathOperator{\Frac}{Frac}
\DeclareMathOperator{\Gal}{Gal}
\DeclareMathOperator{\gammaD}{gamma}
\DeclareMathOperator{\gr}{gr}
\DeclareMathOperator{\hcf}{hcf}
\DeclareMathOperator{\Hom}{Hom}
\DeclareMathOperator{\id}{id}
\DeclareMathOperator{\image}{image}
\DeclareMathOperator{\im}{im}
\DeclareMathOperator{\Im}{Im}
\DeclareMathOperator{\Ind}{Ind}
\DeclareMathOperator{\Int}{Int}
\DeclareMathOperator{\Isom}{Isom}
\DeclareMathOperator{\lcm}{lcm}
\DeclareMathOperator{\length}{length}
\DeclareMathOperator{\Lie}{Lie}
\DeclareMathOperator{\like}{like}
\DeclareMathOperator{\Lk}{Lk}
\DeclareMathOperator{\mse}{mse}
\DeclareMathOperator{\multinomial}{multinomial}
\DeclareMathOperator{\orb}{orb}
\DeclareMathOperator{\ord}{ord}
\DeclareMathOperator{\otp}{otp}
\DeclareMathOperator{\Poisson}{Poisson}
\DeclareMathOperator{\poly}{poly}
\DeclareMathOperator{\rank}{rank}
\DeclareMathOperator{\rel}{rel}
\DeclareMathOperator{\Re}{Re}
\DeclareMathOperator*{\res}{res}
\DeclareMathOperator{\Res}{Res}
\DeclareMathOperator{\rk}{rk}
\DeclareMathOperator{\Root}{Root}
\DeclareMathOperator{\sech}{sech}
\DeclareMathOperator{\sgn}{sgn}
\DeclareMathOperator{\spn}{span}
\DeclareMathOperator{\stab}{stab}
\DeclareMathOperator{\St}{St}
\DeclareMathOperator{\supp}{supp}
\DeclareMathOperator{\Syl}{Syl}
\DeclareMathOperator{\Sym}{Sym}
\DeclareMathOperator{\tr}{tr}
\DeclareMathOperator{\Tr}{Tr}
\DeclareMathOperator{\var}{var}
\DeclareMathOperator{\vol}{vol}

\pgfarrowsdeclarecombine{twolatex'}{twolatex'}{latex'}{latex'}{latex'}{latex'}
\tikzset{->/.style = {decoration={markings,
                                  mark=at position 1 with {\arrow[scale=2]{latex'}}},
                      postaction={decorate}}}
\tikzset{<-/.style = {decoration={markings,
                                  mark=at position 0 with {\arrowreversed[scale=2]{latex'}}},
                      postaction={decorate}}}
\tikzset{<->/.style = {decoration={markings,
                                   mark=at position 0 with {\arrowreversed[scale=2]{latex'}},
                                   mark=at position 1 with {\arrow[scale=2]{latex'}}},
                       postaction={decorate}}}
\tikzset{->-/.style = {decoration={markings,
                                   mark=at position #1 with {\arrow[scale=2]{latex'}}},
                       postaction={decorate}}}
\tikzset{-<-/.style = {decoration={markings,
                                   mark=at position #1 with {\arrowreversed[scale=2]{latex'}}},
                       postaction={decorate}}}
\tikzset{->>/.style = {decoration={markings,
                                  mark=at position 1 with {\arrow[scale=2]{latex'}}},
                      postaction={decorate}}}
\tikzset{<<-/.style = {decoration={markings,
                                  mark=at position 0 with {\arrowreversed[scale=2]{twolatex'}}},
                      postaction={decorate}}}
\tikzset{<<->>/.style = {decoration={markings,
                                   mark=at position 0 with {\arrowreversed[scale=2]{twolatex'}},
                                   mark=at position 1 with {\arrow[scale=2]{twolatex'}}},
                       postaction={decorate}}}
\tikzset{->>-/.style = {decoration={markings,
                                   mark=at position #1 with {\arrow[scale=2]{twolatex'}}},
                       postaction={decorate}}}
\tikzset{-<<-/.style = {decoration={markings,
                                   mark=at position #1 with {\arrowreversed[scale=2]{twolatex'}}},
                       postaction={decorate}}}

\tikzset{circ/.style = {fill, circle, inner sep = 0, minimum size = 3}}
\tikzset{mstate/.style={circle, draw, blue, text=black, minimum width=0.7cm}}

\tikzset{commutative diagrams/.cd,cdmap/.style={/tikz/column 1/.append style={anchor=base east},/tikz/column 2/.append style={anchor=base west},row sep=tiny}}

\definecolor{mblue}{rgb}{0.2, 0.3, 0.8}
\definecolor{morange}{rgb}{1, 0.5, 0}
\definecolor{mgreen}{rgb}{0.1, 0.4, 0.2}
\definecolor{mred}{rgb}{0.5, 0, 0}

\def\drawcirculararc(#1,#2)(#3,#4)(#5,#6){%
    \pgfmathsetmacro\cA{(#1*#1+#2*#2-#3*#3-#4*#4)/2}%
    \pgfmathsetmacro\cB{(#1*#1+#2*#2-#5*#5-#6*#6)/2}%
    \pgfmathsetmacro\cy{(\cB*(#1-#3)-\cA*(#1-#5))/%
                        ((#2-#6)*(#1-#3)-(#2-#4)*(#1-#5))}%
    \pgfmathsetmacro\cx{(\cA-\cy*(#2-#4))/(#1-#3)}%
    \pgfmathsetmacro\cr{sqrt((#1-\cx)*(#1-\cx)+(#2-\cy)*(#2-\cy))}%
    \pgfmathsetmacro\cA{atan2(#2-\cy,#1-\cx)}%
    \pgfmathsetmacro\cB{atan2(#6-\cy,#5-\cx)}%
    \pgfmathparse{\cB<\cA}%
    \ifnum\pgfmathresult=1
        \pgfmathsetmacro\cB{\cB+360}%
    \fi
    \draw (#1,#2) arc (\cA:\cB:\cr);%
}
\newcommand\getCoord[3]{\newdimen{#1}\newdimen{#2}\pgfextractx{#1}{\pgfpointanchor{#3}{center}}\pgfextracty{#2}{\pgfpointanchor{#3}{center}}}

\def\Xint#1{\mathchoice
   {\XXint\displaystyle\textstyle{#1}}%
   {\XXint\textstyle\scriptstyle{#1}}%
   {\XXint\scriptstyle\scriptscriptstyle{#1}}%
   {\XXint\scriptscriptstyle\scriptscriptstyle{#1}}%
   \!\int}
\def\XXint#1#2#3{{\setbox0=\hbox{$#1{#2#3}{\int}$}
     \vcenter{\hbox{$#2#3$}}\kern-.5\wd0}}
\def\ddashint{\Xint=}
\def\dashint{\Xint-}

\newcommand\separator{{\centering\rule{2cm}{0.2pt}\vspace{2pt}\par}}

\newenvironment{own}{\color{gray!70!black}}{}

\newcommand\makecenter[1]{\raisebox{-0.5\height}{#1}}
\newtheorem*{soln}{Solution}

\renewcommand{\thesection}{}
\renewcommand{\thesubsection}{\arabic{section}.\arabic{subsection}}
\makeatletter
\def\@seccntformat#1{\csname #1ignore\expandafter\endcsname\csname the#1\endcsname\quad}
\let\sectionignore\@gobbletwo
\let\latex@numberline\numberline
\def\numberline#1{\if\relax#1\relax\else\latex@numberline{#1}\fi}
\makeatother


\begin{document}
	
\maketitle

\section{QUESTION 1}

We have $ \nabla^{2} \phi = 0 $ on $ 0 < x < a $, $ 0 < y < b $, $ 0 < z < c $ with $ \phi = 1 $ on the $ z $ surface and $ \phi = 0 $ on all other surfaces:

Assume $ \phi(x,y,z) = X(x) Y(y) Z(z) $, so we have

\[ \frac{X''}{X} + \frac{Y''}{Y} + \frac{Z''}{Z} = 0 \]

Solving $ X'' = - \lambda_{p} X $ such that $ X(0) = X(a) = 0 $ implies that

\[ \lambda_{p} = \frac{p^{2} \pi^{2}}{a^{2}}, X_{l} = \sqrt{\frac{2}{a}} \sin \left( \frac{p \pi x}{a} \right), p = 1,2,3,\cdots  \]

Similarly, solving $ Y'' = - \mu_{q} Y  $, such that $ Y(0) = Y(b) = 0 $ implies that

\[ \mu_{q} = \frac{q^{2} \pi^{2}}{b^{2}}, Y_{q} = \sqrt{\frac{2}{b}} \sin \left(  \frac{q \pi x}{b} \right), q = 1,2,3,\cdots  \]

Now solving for $ Z $ using the eigenvalues:

\[ Z'' = \left(   \frac{p^{2} \pi^{2}}{a^{2}}  + \frac{q^{2} \pi^{2}}{b^{2}} \right) Z, \]

\[ Z = \alpha \cosh \left[  \left(    \frac{p^{2}}{a^{2}} + \frac{q^{2}}{b^{2}} \right)^{1/2} \pi z  \right] + \beta \sinh \left[  \left(    \frac{p^{2}}{a^{2}} + \frac{q^{2}}{b^{2}} \right)^{1/2} \pi z  \right]  \]

We can rewrite the z-dependent part (why??) as


\[ Z = \alpha \cosh \left[ l(c-z)  \right] + \beta \sinh \left[ l(c-z)  \right]  \]

Using the boundary conditions, 

$ Z(c) = 0 \Rightarrow \alpha = 0 $, $ Z(0) = 1 \Rightarrow \beta= \frac{1}{\sinh cl}  $


Therefore, the general solution is 

\[ \psi(x,y,z) 
= \frac{2}{\sqrt{ab}} \sum_{p = 0} \sum_{q = 0} a_{pq} \sin \left(  \frac{p \pi}{a} x \right) \sin \left(  \frac{q \pi}{b} y \right)  \sinh \left[  l(c - z)\right] \frac{1}{\sinh cl}      \]


Now, using orthogonality on the surface $ z = 0 $

%





\section{QUESTION 2}

The general solution for Laplace's equation in polar coordinates is

\[ \phi(r,\theta) = c_{0} + d_{0} \log r + \sum_{n=1}^{\infty}  (a_{n} \cos n \theta  + b_{n} \sin n \theta )(c_{n}r^{n}  + d_{n}r^{-n} )  \]

by requiring regularity at the origin, $ d_{0}=0,d_{n} = 0 $, then absorb $ c_{n} $ as a general rescaling, we can write this solution as

\[ \psi(r,\theta)  = \frac{a_{0}}{2} + \sum_{n=1}^{\infty} (a_{n}  \cos n \theta + b_{n} \sin n \theta ) r^{n}  \]

Using boundary conditions,

\[ f(\theta) = \psi(1,\theta)  = \frac{a_{0}}{2} + \sum_{n=1}^{\infty} (a_{n}  \cos n \theta + b_{n} \sin n \theta ) \]


The function is odd, hence 

\[ f(\theta) = \psi(1,\theta)  = \sum_{n=1}^{\infty} b_{n} \sin n \theta \]

Then

\begin{align*}
\int_{0}^{2\pi} f(\theta) \sin m \theta \; \d \theta  & = \sum_{n=1}^{\infty} b_{n}  \int_{0}^{2\pi} \sin n \theta   \sin m \theta  \; \d \theta \\
\int_{0}^{\pi} \frac{\pi}{2}  \sin m \theta  \; \d \theta + \int_{\pi}^{2\pi} - \frac{\pi}{2}  \sin m \theta \; \d \theta & = b_{m}
\end{align*}

This gives $ b_{m} = \pi / m $, hence

\[ \psi(r,\theta)  =  \sum_{n=1}^{\infty}  \frac{\pi r^{n} \sin n \theta }{n}  \]










\section{QUESTION 3}


$ \psi(r,\theta) $ satisfies $ \nabla^{2} = 0 $, in spherical polars this is

\[ \frac{1}{r^{2}} \frac{\partial }{\partial r} \left(  r^{2}  \frac{\partial }{\partial r} \psi     \right) + \frac{1}{r^{2} \sin \theta}  \frac{\partial }{\partial \theta} \left(  \sin \theta \frac{\partial }{\partial \theta \psi }   \right) = 0    \]

and $ \psi(r,\theta) $ satisfies the boundary conditions

\[ \psi(r,\theta) = \begin{cases} V  & \text{ if } 0 \leq \theta < \frac{\pi}{2} \\ -V & \text{ if } \frac{\pi}{2} < \theta \leq \pi \end{cases} \]

Assume 

\[ \psi(r,\theta) = R(r) \Theta(\theta) \]

we obtain 

\begin{align*}
(\sin \theta \Theta'  )' + \lambda \sin \theta \Theta & = 0 \\
 (r^{2} R' )' - \lambda R & = 0 
\end{align*}

Making the substitution $ x = \cos \theta $ in the first equation yields Legendre's equation

\[ - \frac{\d }{\d x}  \left[   (1-x^{2}) \frac{\d }{\d x}  \Theta \right]   = \lambda \Theta \]

substituting  $ \Theta = \sum_{n=0}^{\infty} a_{n} x^{n}  $ yielding Legendre Polynomials as the answer:

\begin{align*}
\Theta_{e} & = a_{0}  \left[  1 + \frac{(-\lambda)x^{2}}{2!} \cdots  \right]   \\
\Theta_{o} & = a_{1} \left[   x + \frac{(2 - \lambda) x^{3} }{3!} + \cdots \right]  
\end{align*}

Solution must remain bounded at $ x = \pm 1 $, so $ \lambda = m(m+1) $ for some integer $ m $. Call this  $ \Theta_{n}(\theta) = P_{n}(x) = P_{n} (\cos \theta )  $, with $ \lambda= n(n+1) $. Then the DE for $ R $ becomes


\begin{align*}
(r^{2} R_{n}' )' - n(n+1) R_{n} & = 0 
\end{align*}

Have

\[ \psi_{n} (r,\theta) = (a_{n} r^{n}  + b_{n} r^{-(n+1)}    ) P_{n} (\cos \theta)   \]

So GS is

\[ \psi (r,\theta) =  \sum_{n=0}^{\infty} (a_{n} r^{n}  + b_{n} r^{-(n+1)}    ) P_{n} (\cos \theta)   \]


for solution to be regular at origin must have $ b_{n} = 0 $, so

\[ \psi (r,\theta) =  \sum_{n=0}^{\infty} a_{n} r^{n} P_{n} (\cos \theta)   \]

We can apply our boundary condition easiest by setting $ r=1 $, then

\begin{align*}
f(\theta) & = \sum_{n=0}^{\infty}  a_{n} P_{n} \cos (\theta), \quad 0 \leq \theta \leq \pi  \\
F(x) & = \sum_{n=0}^{\infty}  a_{n} P_{n} (x), \qquad x = \cos \theta, -1 \leq x \leq 1 \\
a_{n} & = \frac{(2n + 1)}{2} \int_{-1}^{1} F(x) P_{n}(x) \; \d x  
\end{align*}

For $ 0 \leq \theta < \pi / 2 $ we have $ f(\theta) = V $, ie. for $ 0 \leq x <1 $ we have $ F(x) = V $. Similarly $ F(x) = - V $ if $ -1 \leq x < 0  $. So

\[ a_{n} = \frac{(2n + 1)}{2} V \int_{-1}^{1} P_{n}(x) \; \d x \qquad \text{ for } 0 \leq x <1  \]
\[ a_{n} = - \frac{(2n + 1)}{2} V \int_{-1}^{1} P_{n}(x) \; \d x \qquad \text{ for } -1 \leq x <0  \]


\section{QUESTION 4}


$ y_{m} $ is an eigenfunction, hence satisfies the Sturm-Liouville equation, so we may write

\[ \frac{\d }{\d x} \left( p y_{m}' \right) = - ( \lambda_{m} - q)y_{m}     \]

Then, integrating by parts,

\begin{align*}
& = \\
& = 
\end{align*}


\section{QUESTION 5}


(a)
\begin{enumerate}
	\item 
	
	\[ q_{n} = \frac{1}{2^{n} n!} \frac{\d^{n} }{\d x^{n}} \underbrace{\left(  x^{2n} - \binom{n}{2} x^{2n - 2} + \cdots  \right)}_{(*)}    \]
	
	Differentiating (*) $ n $ times produces a polynomial with highest power $ \frac{\d^{n} }{\d x^{n} } (x^{2n}) = x^{n} $. Hence $ q_{n}(x) $ is of degree $ n $.
	
	\item By induction: 
	
	\begin{align*}
	 q_{1}(1) & = \frac{1}{2} \frac{\d }{\d x} (x^{2} - 1) \Big|_{x=1} \\
	& = \frac{1}{2} 2x \Big|_{x=1} \\
	& = 1
	\end{align*}
	
	True for $ n = 1 $. Now suppose $ q_{k}(1) = 1 $ for some $ k > 0 $.
	
	\begin{align*}
	q_{k+1}(x) & = \frac{1}{2^{k+1} (k+1)!} \frac{\d^{n} }{\d x^{k+1}} (x^{2} - 1)^{k+1}   \\
	& =  \frac{1}{2 (k+1)} \left[  \frac{1}{2^{k} k!} \frac{\d^{k} }{\d x^{k}} \left(  2(k+1)x(x^{2} - 1)^{k}  \right)   \right]  \\
	& =  \frac{1}{2^{k} k!} \frac{\d^{k} }{\d x^{k}} \left( x(x^{2} - 1)^{k}  \right) 
	\end{align*}
	
	
	
\end{enumerate}

\section{QUESTION 6}


$ y(x,t) $ satisfies the 1D wave equation

\[ \frac{\partial^{2} }{\partial t^2} y(x,t) = c^{2} \frac{\partial^{2} }{\partial x^{2}} y(x,t) \qquad c^{2} = \frac{T}{\mu} \]


Assume $ y(x,t) = X(x) T(t) $ and separate variables:

\begin{align*}
X \ddot{T} & = c^{2} X'' T \\
\Rightarrow \frac{1}{c^{2}} \frac{\ddot{T}}{T}  & = \frac{X''}{X} = - \lambda 
\end{align*}

for some $ \lambda > 0 $, so we have

\begin{align*}
X'' + \lambda X& =0 \\
\ddot{T}  + \lambda c^{2} T & = 0 
\end{align*}

Solving the spatial equation first,

\[ X = \alpha \cos ( \sqrt{\lambda} x ) + \beta \sin ( \sqrt{\lambda} x ) \]

Applying initial conditions

\begin{itemize}
	\item $ X(0) = 0 \Rightarrow \alpha = 0$
	\item $ X(l) = 0 \Rightarrow \beta \sin ( \sqrt{\lambda} x ) = 0 \Rightarrow \lambda = n^{2} \pi^{2} / l^{2} $, for integer $ n $
\end{itemize}

The normal modes are the associated eigenfunctions given by

\[ X_{n}(x) = \beta_{n} \sin \left(  \frac{n \pi x}{L} \right)  \]

The associated $ T_{n}(t) $ is given by

\begin{align*}
 \ddot{T}_{n} + \frac{n^{2} \pi^{2} c^{2} }{L^{2}}  T_{n} & = 0 \\
\Rightarrow T_{n}(t) & = \gamma_{n} \cos \left(  \frac{n \pi c t}{L} \right) + \delta_{n} \sin \left(  \frac{n \pi c t}{L} \right)  
\end{align*}

Hence the specific solution is 

\[ y_{n} = X(x) T(t)   = \sin \left(  \frac{n \pi x}{L} \right)  \left(  A_{n} \cos  \left(  \frac{n \pi ct}{L} \right) + B_{n} \sin  \left(  \frac{n \pi ct}{L} \right) \right)    \]


\section{QUESTION 7}


\begin{enumerate}
	\item \begin{itemize}
		\item Assume all displacements are sufficiently small $ ( y < < l) $
		\item Assume all displacements are vertical
		\item Consider two points $ x $ and $ x + \delta x $. The angle of the string to the horizontal at $ x $ is $ \theta_{1} $, and the angle at $ x + \delta x $ is $ \theta_{2} $.
		\item Resolving vertically 
		
		\begin{align*}
		T \sin \theta_{2} -  T \sin \theta_{1} - \mu g \delta x - 2k \mu \delta x \frac{\partial  }{\partial t}y  & = \mu \delta x \frac{\partial^{2} }{\partial t^{2}} y   \qquad (*) \\
		\end{align*} 
		
		
		\item Assume angles are small
		\[ \sin \theta_{2} \approx tan \theta_{2} =  \frac{\partial y }{\partial x} \Big|_{x + \delta x} \approx \frac{\partial y }{\partial x} \Big|_{x} + \delta x \frac{\partial^{2} y }{\partial x^{2}} \Big|_{x}   \]
		
		\[ \sin \theta_{1} \approx \tan \theta_{1} = \frac{\partial y }{\partial x} \Big|_{x} \]
	
		\item (*) becomes 
		
		\begin{align*}
		T \delta x \frac{\partial^{2} y }{\partial x^{2}}   - \mu g \delta x & - 2k \mu \delta x \frac{\partial  }{\partial t}y   = \mu \delta x \frac{\partial^{2} }{\partial t^{2}} y   \\
		\Rightarrow \frac{T}{\mu} \frac{\partial^{2} y }{\partial x^{2}}  & = \frac{\partial^{2} }{\partial t^{2}} y +  2k \frac{\partial  }{\partial t}y + g
		\end{align*} 
		
		\item Further assume the weight is insignificant $ (g \to 0) $
		
		\item Hence arrive at the equation of motion 
		
		\[ c^{2} \frac{\partial^{2} y }{\partial x^{2}}   = \frac{\partial^{2} }{\partial t^{2}} y +  2k \frac{\partial  }{\partial t}y \]
		
		where $ c^{2} = \frac{T}{\mu} $
		
	\end{itemize}
		
		Assume $ y(x,t) = X(x)T(t) $ and separating variables gives
		
		\begin{align*}
		c^{2} \frac{X''}{X} & = \frac{\ddot{T}}{T} + 2k \frac{\dot{T}}{T}    \\
		\Rightarrow  \frac{X''}{X} & = \frac{\ddot{T} + 2k \dot{T} }{T} = - \lambda 
		\end{align*}
		
		for some $ \lambda > 0 $, so we have
		
		\begin{align*}
		X'' + \lambda X& =0 \\
		\ddot{T}  + 2k \dot{T} + \lambda c^{2} T & = 0 
		\end{align*}
		
		Solving the spatial equation first,
		
		\[ X = \alpha \cos ( \sqrt{\lambda} x ) + \beta \sin ( \sqrt{\lambda} x ) \]
		
		Applying initial conditions
		
		\begin{itemize}
			\item $ X(0) = 0 \Rightarrow \alpha = 0$
			\item $ X(l) = 0 \Rightarrow \beta \sin ( \sqrt{\lambda} x ) = 0 \Rightarrow \lambda = n^{2} \pi^{2} / l^{2} $, for integer $ n $
		\end{itemize}
	
		These $ \lambda $ are eigenvalues, with associated eigenfunctions
		
		\[ X_{n}(x) = \beta_{n} \sin \left(  \frac{n \pi x}{L} \right)  \]
		
		The associated $ T_{n}(t) $ is given by
		
		\begin{align*}
		\qquad  & \ddot{T}_{n} + 2k \dot{T}_{n}  + \frac{n^{2} \pi^{2} c^{2} }{L^{2}}  T_{n} = 0 \qquad k = \frac{\pi c}{l}\\
		\Rightarrow T_{n}(t) & = e^{-kt} \left(    \gamma_{n} \cos \left(  \frac{\sqrt{n^{2} - 1} \pi c }{L} t \right) + \delta_{n} \sin \left(  \frac{\sqrt{n^{2} - 1} \pi c }{L} t \right)  \right)
		\end{align*}
		
		Hence the specific solution is 
		
		\[ y_{n}  = e^{-kt}\sin \left( \frac{\sqrt{n^{2} - 1} \pi c }{L} t \right)  \left(  A_{n} \cos  \left(  \frac{\sqrt{n^{2} - 1} \pi c }{L} t \right) + B_{n} \sin  \left(  \frac{\sqrt{n^{2} - 1} \pi c }{L} t \right) \right)    \]
		
		
\end{enumerate}
\section{QUESTION 8}
\section{QUESTION 9}
\section{QUESTION 10}

\end{document}