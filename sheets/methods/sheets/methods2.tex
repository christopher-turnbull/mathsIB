\documentclass[a4paper]{article}
\usepackage{amsmath}
\def\npart {IB}
\def\nterm {Michaelmas}
\def\nyear {2017}
\def\nlecturer {Dr. Saxton}
\def\ncourse {Methods Example Sheet 1}

% Imports
\ifx \nauthor\undefined
  \def\nauthor{Christopher Turnbull}
\else
\fi

\author{Supervised by \nlecturer \\\small Solutions presented by \nauthor}
\date{\nterm\ \nyear}

\usepackage{alltt}
\usepackage{amsfonts}
\usepackage{amsmath}
\usepackage{amssymb}
\usepackage{amsthm}
\usepackage{booktabs}
\usepackage{caption}
\usepackage{enumitem}
\usepackage{fancyhdr}
\usepackage{graphicx}
\usepackage{mathdots}
\usepackage{mathtools}
\usepackage{microtype}
\usepackage{multirow}
\usepackage{pdflscape}
\usepackage{pgfplots}
\usepackage{siunitx}
\usepackage{slashed}
\usepackage{tabularx}
\usepackage{tikz}
\usepackage{tkz-euclide}
\usepackage[normalem]{ulem}
\usepackage[all]{xy}
\usepackage{imakeidx}

\makeindex[intoc, title=Index]
\indexsetup{othercode={\lhead{\emph{Index}}}}

\ifx \nextra \undefined
  \usepackage[pdftex,
    hidelinks,
    pdfauthor={Christopher Turnbull},
    pdfsubject={Cambridge Maths Notes: Part \npart\ - \ncourse},
    pdftitle={Part \npart\ - \ncourse},
  pdfkeywords={Cambridge Mathematics Maths Math \npart\ \nterm\ \nyear\ \ncourse}]{hyperref}
  \title{Part \npart\ --- \ncourse}
\else
  \usepackage[pdftex,
    hidelinks,
    pdfauthor={Christopher Turnbull},
    pdfsubject={Cambridge Maths Notes: Part \npart\ - \ncourse\ (\nextra)},
    pdftitle={Part \npart\ - \ncourse\ (\nextra)},
  pdfkeywords={Cambridge Mathematics Maths Math \npart\ \nterm\ \nyear\ \ncourse\ \nextra}]{hyperref}

  \title{Part \npart\ --- \ncourse \\ {\Large \nextra}}
  \renewcommand\printindex{}
\fi

\pgfplotsset{compat=1.12}

\pagestyle{fancyplain}
\lhead{\emph{\nouppercase{\leftmark}}}
\ifx \nextra \undefined
  \rhead{
    \ifnum\thepage=1
    \else
      \npart\ \ncourse
    \fi}
\else
  \rhead{
    \ifnum\thepage=1
    \else
      \npart\ \ncourse\ (\nextra)
    \fi}
\fi
\usetikzlibrary{arrows.meta}
\usetikzlibrary{decorations.markings}
\usetikzlibrary{decorations.pathmorphing}
\usetikzlibrary{positioning}
\usetikzlibrary{fadings}
\usetikzlibrary{intersections}
\usetikzlibrary{cd}

\newcommand*{\Cdot}{{\raisebox{-0.25ex}{\scalebox{1.5}{$\cdot$}}}}
\newcommand {\pd}[2][ ]{
  \ifx #1 { }
    \frac{\partial}{\partial #2}
  \else
    \frac{\partial^{#1}}{\partial #2^{#1}}
  \fi
}
\ifx \nhtml \undefined
\else
  \renewcommand\printindex{}
  \makeatletter
  \DisableLigatures[f]{family = *}
  \let\Contentsline\contentsline
  \renewcommand\contentsline[3]{\Contentsline{#1}{#2}{}}
  \renewcommand{\@dotsep}{10000}
  \newlength\currentparindent
  \setlength\currentparindent\parindent

  \newcommand\@minipagerestore{\setlength{\parindent}{\currentparindent}}
  \usepackage[active,tightpage,pdftex]{preview}
  \renewcommand{\PreviewBorder}{0.1cm}

  \newenvironment{stretchpage}%
  {\begin{preview}\begin{minipage}{\hsize}}%
    {\end{minipage}\end{preview}}
  \AtBeginDocument{\begin{stretchpage}}
  \AtEndDocument{\end{stretchpage}}

  \newcommand{\@@newpage}{\end{stretchpage}\begin{stretchpage}}

  \let\@real@section\section
  \renewcommand{\section}{\@@newpage\@real@section}
  \let\@real@subsection\subsection
  \renewcommand{\subsection}{\@@newpage\@real@subsection}
  \makeatother
\fi

% Theorems
\theoremstyle{definition}
\newtheorem*{aim}{Aim}
\newtheorem*{axiom}{Axiom}
\newtheorem*{claim}{Claim}
\newtheorem*{cor}{Corollary}
\newtheorem*{conjecture}{Conjecture}
\newtheorem*{defi}{Definition}
\newtheorem*{eg}{Example}
\newtheorem*{ex}{Exercise}
\newtheorem*{fact}{Fact}
\newtheorem*{law}{Law}
\newtheorem*{lemma}{Lemma}
\newtheorem*{notation}{Notation}
\newtheorem*{prop}{Proposition}
\newtheorem*{soln}{Solution}
\newtheorem*{thm}{Theorem}

\newtheorem*{remark}{Remark}
\newtheorem*{warning}{Warning}
\newtheorem*{exercise}{Exercise}

\newtheorem{nthm}{Theorem}[section]
\newtheorem{nlemma}[nthm]{Lemma}
\newtheorem{nprop}[nthm]{Proposition}
\newtheorem{ncor}[nthm]{Corollary}


\renewcommand{\labelitemi}{--}
\renewcommand{\labelitemii}{$\circ$}
\renewcommand{\labelenumi}{(\roman{*})}

\let\stdsection\section
\renewcommand\section{\newpage\stdsection}

% Strike through
\def\st{\bgroup \ULdepth=-.55ex \ULset}

% Maths symbols
\newcommand{\abs}[1]{\left\lvert #1\right\rvert}
\newcommand\ad{\mathrm{ad}}
\newcommand\AND{\mathsf{AND}}
\newcommand\Art{\mathrm{Art}}
\newcommand{\Bilin}{\mathrm{Bilin}}
\newcommand{\bket}[1]{\left\lvert #1\right\rangle}
\newcommand{\B}{\mathcal{B}}
\newcommand{\bolds}[1]{{\bfseries #1}}
\newcommand{\brak}[1]{\left\langle #1 \right\rvert}
\newcommand{\braket}[2]{\left\langle #1\middle\vert #2 \right\rangle}
\newcommand{\bra}{\langle}
\newcommand{\cat}[1]{\mathsf{#1}}
\newcommand{\C}{\mathbb{C}}
\newcommand{\CP}{\mathbb{CP}}
\newcommand{\cU}{\mathcal{U}}
\newcommand{\Der}{\mathrm{Der}}
\newcommand{\D}{\mathrm{D}}
\newcommand{\dR}{\mathrm{dR}}
\newcommand{\E}{\mathbb{E}}
\newcommand{\F}{\mathbb{F}}
\newcommand{\Frob}{\mathrm{Frob}}
\newcommand{\GG}{\mathbb{G}}
\newcommand{\gl}{\mathfrak{gl}}
\newcommand{\GL}{\mathrm{GL}}
\newcommand{\G}{\mathcal{G}}
\newcommand{\Gr}{\mathrm{Gr}}
\newcommand{\haut}{\mathrm{ht}}
\newcommand{\Id}{\mathrm{Id}}
\newcommand{\ket}{\rangle}
\newcommand{\lie}[1]{\mathfrak{#1}}
\newcommand{\Mat}{\mathrm{Mat}}
\newcommand{\N}{\mathbb{N}}
\newcommand{\norm}[1]{\left\lVert #1\right\rVert}
\newcommand{\normalorder}[1]{\mathop{:}\nolimits\!#1\!\mathop{:}\nolimits}
\newcommand\NOT{\mathsf{NOT}}
\newcommand{\Oc}{\mathcal{O}}
\newcommand{\Or}{\mathrm{O}}
\newcommand\OR{\mathsf{OR}}
\newcommand{\ort}{\mathfrak{o}}
\newcommand{\PGL}{\mathrm{PGL}}
\newcommand{\ph}{\,\cdot\,}
\newcommand{\pr}{\mathrm{pr}}
\newcommand{\Prob}{\mathbb{P}}
\newcommand{\PSL}{\mathrm{PSL}}
\newcommand{\Ps}{\mathcal{P}}
\newcommand{\PSU}{\mathrm{PSU}}
\newcommand{\pt}{\mathrm{pt}}
\newcommand{\qeq}{\mathrel{``{=}"}}
\newcommand{\Q}{\mathbb{Q}}
\newcommand{\R}{\mathbb{R}}
\newcommand{\RP}{\mathbb{RP}}
\newcommand{\Rs}{\mathcal{R}}
\newcommand{\SL}{\mathrm{SL}}
\newcommand{\so}{\mathfrak{so}}
\newcommand{\SO}{\mathrm{SO}}
\newcommand{\Spin}{\mathrm{Spin}}
\newcommand{\Sp}{\mathrm{Sp}}
\newcommand{\su}{\mathfrak{su}}
\newcommand{\SU}{\mathrm{SU}}
\newcommand{\term}[1]{\emph{#1}\index{#1}}
\newcommand{\T}{\mathbb{T}}
\newcommand{\tv}[1]{|#1|}
\newcommand{\U}{\mathrm{U}}
\newcommand{\uu}{\mathfrak{u}}
\newcommand{\Vect}{\mathrm{Vect}}
\newcommand{\wsto}{\stackrel{\mathrm{w}^*}{\to}}
\newcommand{\wt}{\mathrm{wt}}
\newcommand{\wto}{\stackrel{\mathrm{w}}{\to}}
\newcommand{\Z}{\mathbb{Z}}
\renewcommand{\d}{\mathrm{d}}
\renewcommand{\H}{\mathbb{H}}
\renewcommand{\P}{\mathbb{P}}
\renewcommand{\sl}{\mathfrak{sl}}
\renewcommand{\vec}[1]{\boldsymbol{\mathbf{#1}}}
%\renewcommand{\F}{\mathcal{F}}

\let\Im\relax
\let\Re\relax

\DeclareMathOperator{\adj}{adj}
\DeclareMathOperator{\Ann}{Ann}
\DeclareMathOperator{\area}{area}
\DeclareMathOperator{\Aut}{Aut}
\DeclareMathOperator{\Bernoulli}{Bernoulli}
\DeclareMathOperator{\betaD}{beta}
\DeclareMathOperator{\bias}{bias}
\DeclareMathOperator{\binomial}{binomial}
\DeclareMathOperator{\card}{card}
\DeclareMathOperator{\ccl}{ccl}
\DeclareMathOperator{\Char}{char}
\DeclareMathOperator{\ch}{ch}
\DeclareMathOperator{\cl}{cl}
\DeclareMathOperator{\cls}{\overline{\mathrm{span}}}
\DeclareMathOperator{\conv}{conv}
\DeclareMathOperator{\corr}{corr}
\DeclareMathOperator{\cosec}{cosec}
\DeclareMathOperator{\cosech}{cosech}
\DeclareMathOperator{\cov}{cov}
\DeclareMathOperator{\covol}{covol}
\DeclareMathOperator{\diag}{diag}
\DeclareMathOperator{\diam}{diam}
\DeclareMathOperator{\Diff}{Diff}
\DeclareMathOperator{\disc}{disc}
\DeclareMathOperator{\dom}{dom}
\DeclareMathOperator{\End}{End}
\DeclareMathOperator{\energy}{energy}
\DeclareMathOperator{\erfc}{erfc}
\DeclareMathOperator{\erf}{erf}
\DeclareMathOperator*{\esssup}{ess\,sup}
\DeclareMathOperator{\ev}{ev}
\DeclareMathOperator{\Ext}{Ext}
\DeclareMathOperator{\Fit}{Fit}
\DeclareMathOperator{\fix}{fix}
\DeclareMathOperator{\Frac}{Frac}
\DeclareMathOperator{\Gal}{Gal}
\DeclareMathOperator{\gammaD}{gamma}
\DeclareMathOperator{\gr}{gr}
\DeclareMathOperator{\hcf}{hcf}
\DeclareMathOperator{\Hom}{Hom}
\DeclareMathOperator{\id}{id}
\DeclareMathOperator{\image}{image}
\DeclareMathOperator{\im}{im}
\DeclareMathOperator{\Im}{Im}
\DeclareMathOperator{\Ind}{Ind}
\DeclareMathOperator{\Int}{Int}
\DeclareMathOperator{\Isom}{Isom}
\DeclareMathOperator{\lcm}{lcm}
\DeclareMathOperator{\length}{length}
\DeclareMathOperator{\Lie}{Lie}
\DeclareMathOperator{\like}{like}
\DeclareMathOperator{\Lk}{Lk}
\DeclareMathOperator{\mse}{mse}
\DeclareMathOperator{\multinomial}{multinomial}
\DeclareMathOperator{\orb}{orb}
\DeclareMathOperator{\ord}{ord}
\DeclareMathOperator{\otp}{otp}
\DeclareMathOperator{\Poisson}{Poisson}
\DeclareMathOperator{\poly}{poly}
\DeclareMathOperator{\rank}{rank}
\DeclareMathOperator{\rel}{rel}
\DeclareMathOperator{\Re}{Re}
\DeclareMathOperator*{\res}{res}
\DeclareMathOperator{\Res}{Res}
\DeclareMathOperator{\rk}{rk}
\DeclareMathOperator{\Root}{Root}
\DeclareMathOperator{\sech}{sech}
\DeclareMathOperator{\sgn}{sgn}
\DeclareMathOperator{\spn}{span}
\DeclareMathOperator{\stab}{stab}
\DeclareMathOperator{\St}{St}
\DeclareMathOperator{\supp}{supp}
\DeclareMathOperator{\Syl}{Syl}
\DeclareMathOperator{\Sym}{Sym}
\DeclareMathOperator{\tr}{tr}
\DeclareMathOperator{\Tr}{Tr}
\DeclareMathOperator{\var}{var}
\DeclareMathOperator{\vol}{vol}

\pgfarrowsdeclarecombine{twolatex'}{twolatex'}{latex'}{latex'}{latex'}{latex'}
\tikzset{->/.style = {decoration={markings,
                                  mark=at position 1 with {\arrow[scale=2]{latex'}}},
                      postaction={decorate}}}
\tikzset{<-/.style = {decoration={markings,
                                  mark=at position 0 with {\arrowreversed[scale=2]{latex'}}},
                      postaction={decorate}}}
\tikzset{<->/.style = {decoration={markings,
                                   mark=at position 0 with {\arrowreversed[scale=2]{latex'}},
                                   mark=at position 1 with {\arrow[scale=2]{latex'}}},
                       postaction={decorate}}}
\tikzset{->-/.style = {decoration={markings,
                                   mark=at position #1 with {\arrow[scale=2]{latex'}}},
                       postaction={decorate}}}
\tikzset{-<-/.style = {decoration={markings,
                                   mark=at position #1 with {\arrowreversed[scale=2]{latex'}}},
                       postaction={decorate}}}
\tikzset{->>/.style = {decoration={markings,
                                  mark=at position 1 with {\arrow[scale=2]{latex'}}},
                      postaction={decorate}}}
\tikzset{<<-/.style = {decoration={markings,
                                  mark=at position 0 with {\arrowreversed[scale=2]{twolatex'}}},
                      postaction={decorate}}}
\tikzset{<<->>/.style = {decoration={markings,
                                   mark=at position 0 with {\arrowreversed[scale=2]{twolatex'}},
                                   mark=at position 1 with {\arrow[scale=2]{twolatex'}}},
                       postaction={decorate}}}
\tikzset{->>-/.style = {decoration={markings,
                                   mark=at position #1 with {\arrow[scale=2]{twolatex'}}},
                       postaction={decorate}}}
\tikzset{-<<-/.style = {decoration={markings,
                                   mark=at position #1 with {\arrowreversed[scale=2]{twolatex'}}},
                       postaction={decorate}}}

\tikzset{circ/.style = {fill, circle, inner sep = 0, minimum size = 3}}
\tikzset{mstate/.style={circle, draw, blue, text=black, minimum width=0.7cm}}

\tikzset{commutative diagrams/.cd,cdmap/.style={/tikz/column 1/.append style={anchor=base east},/tikz/column 2/.append style={anchor=base west},row sep=tiny}}

\definecolor{mblue}{rgb}{0.2, 0.3, 0.8}
\definecolor{morange}{rgb}{1, 0.5, 0}
\definecolor{mgreen}{rgb}{0.1, 0.4, 0.2}
\definecolor{mred}{rgb}{0.5, 0, 0}

\def\drawcirculararc(#1,#2)(#3,#4)(#5,#6){%
    \pgfmathsetmacro\cA{(#1*#1+#2*#2-#3*#3-#4*#4)/2}%
    \pgfmathsetmacro\cB{(#1*#1+#2*#2-#5*#5-#6*#6)/2}%
    \pgfmathsetmacro\cy{(\cB*(#1-#3)-\cA*(#1-#5))/%
                        ((#2-#6)*(#1-#3)-(#2-#4)*(#1-#5))}%
    \pgfmathsetmacro\cx{(\cA-\cy*(#2-#4))/(#1-#3)}%
    \pgfmathsetmacro\cr{sqrt((#1-\cx)*(#1-\cx)+(#2-\cy)*(#2-\cy))}%
    \pgfmathsetmacro\cA{atan2(#2-\cy,#1-\cx)}%
    \pgfmathsetmacro\cB{atan2(#6-\cy,#5-\cx)}%
    \pgfmathparse{\cB<\cA}%
    \ifnum\pgfmathresult=1
        \pgfmathsetmacro\cB{\cB+360}%
    \fi
    \draw (#1,#2) arc (\cA:\cB:\cr);%
}
\newcommand\getCoord[3]{\newdimen{#1}\newdimen{#2}\pgfextractx{#1}{\pgfpointanchor{#3}{center}}\pgfextracty{#2}{\pgfpointanchor{#3}{center}}}

\def\Xint#1{\mathchoice
   {\XXint\displaystyle\textstyle{#1}}%
   {\XXint\textstyle\scriptstyle{#1}}%
   {\XXint\scriptstyle\scriptscriptstyle{#1}}%
   {\XXint\scriptscriptstyle\scriptscriptstyle{#1}}%
   \!\int}
\def\XXint#1#2#3{{\setbox0=\hbox{$#1{#2#3}{\int}$}
     \vcenter{\hbox{$#2#3$}}\kern-.5\wd0}}
\def\ddashint{\Xint=}
\def\dashint{\Xint-}

\newcommand\separator{{\centering\rule{2cm}{0.2pt}\vspace{2pt}\par}}

\newenvironment{own}{\color{gray!70!black}}{}

\newcommand\makecenter[1]{\raisebox{-0.5\height}{#1}}
\newtheorem*{soln}{Solution}

\renewcommand{\thesection}{}
\renewcommand{\thesubsection}{\arabic{section}.\arabic{subsection}}
\makeatletter
\def\@seccntformat#1{\csname #1ignore\expandafter\endcsname\csname the#1\endcsname\quad}
\let\sectionignore\@gobbletwo
\let\latex@numberline\numberline
\def\numberline#1{\if\relax#1\relax\else\latex@numberline{#1}\fi}
\makeatother


\begin{document}
	
\maketitle

\section{QUESTION 1}


Let $ \Omega $ be the region

\[ \Omega = \{  (x,y,z) \in \R^{3} \; | \; 0 \leq x \leq a, 0 \leq y \leq b, 0 \leq z \leq c  \} \]

We have Laplace's equation $ \nabla^{2} \phi = 0 $ inside $ \Omega $, with the Dirichlet boundary conditions $ \phi = 1 $ on the $ z $ surface and $ \phi = 0 $ on all other surfaces:

Assume $ \phi(x,y,z) = X(x) Y(y) \sinh [ l(c - z) ] $, so we have

\[ \frac{X''}{X} + \frac{Y''}{Y} + l^{2} = 0 \]

Moving along the $ x $ direction for fixed $ y,z $,$ \frac{Y''}{Y}, l^{2} $ both constant, hence $ \frac{X''}{X} = - \frac{Y''}{Y} - l^{2} $ is also equal to some constant, $ - \lambda $.

Solving $ X'' = - \lambda X $ with boundary conditions $ X(0) = X(a) = 0 $ restricts values of $ \lambda $ to

\[ \lambda_{p} = \frac{p^{2} \pi^{2}}{a^{2}}, X_{p} = A \sin \left( \frac{p \pi x}{a} \right), p = 1,2,3,\cdots  \]

Similarly, solving $ Y'' = - \mu Y  $, such that $ Y(0) = Y(b) = 0 $ implies that

\[ \mu_{q} = \frac{q^{2} \pi^{2}}{b^{2}}, Y_{q} = B \sin \left(  \frac{q \pi x}{b} \right), q = 1,2,3,\cdots  \]

At this point we have a family of solutions

\[ \psi_{p,q}(x,y,z) : = \sin \left(  \frac{p \pi}{a} x \right) \sin \left(  \frac{q \pi}{b} y \right)  \sinh \left[  l(c - z)\right]  \]

Laplaces equation linear with boundary conditions homogenous, the general solution is 

\[ \psi(x,y,z) 
=  \sum_{p = 0}^{\infty} \sum_{q = 0}^{\infty} A_{p,q} \sin \left(  \frac{p \pi}{a} x \right) \sin \left(  \frac{q \pi}{b} y \right)  \sinh \left[  l(c - z)\right]    \]


for some constants $ A_{p,q} $. Setting $ z = 0 $ we require

\[ 1 
=  \sum_{p = 0}^{\infty} \sum_{q = 0}^{\infty} A_{p,q} \sin \left(  \frac{p \pi}{a} x \right) \sin \left(  \frac{q \pi}{b} y \right)  \sinh cl \]

Now using orthogonality relations

\[ \int_{0}^{a} \sin\left( \frac{p \pi x}{a} \right)\sin\left( \frac{q \pi x}{a} \right) \; \d x = \frac{a}{2} \delta_{p,q}  \]

we deduce

\begin{align*}
A_{p,q} & = \frac{4}{ab \sinh cl} \int_{0}^{a} \int_{0}^{b} \sin\left( \frac{q \pi x}{a} \right)\sin\left( \frac{p \pi y}{b} \right) \; \d x \d y \\
& = \frac{4}{ab \sinh cl} \left[   - \frac{a}{q \pi}  \cos \left(  \frac{q \pi x}{a} \right)  \right]_{0}^{a} \left[   - \frac{b}{p \pi}  \cos \left(  \frac{p \pi x}{b} \right)  \right]_{0}^{b} \\
&  =  \frac{4}{ab \sinh cl} \frac{ab}{\pi^{2} pq}\left(  (-1)^{q} - 1 \right)\left(  (-1)^{p} - 1 \right) \\
& = \begin{cases} \frac{16}{\pi^{2}pq \sinh cl}  & \text{ if } q \text{ and } p \text{ are both odd} \\ 0 & \text{ otherwise } \end{cases}
\end{align*}


Therefore, the solution satisfying these boundary conditions is

\[ \psi(x,y,z) 
=  \frac{16}{\pi^{2}} \sum_{p = 0}^{\infty} \sum_{q = 0}^{\infty} \frac{\sinh \left[  l(c - z)\right] \sin \left( (2p + 1)\pi x / a \right) \sin \left( (2q + 1)\pi y / b \right)   }{ (2p+1)(2q+1)\sinh cl   }   \]

as required.

As $ c  \to \infty $, note that 

\begin{align*}
\frac{\sinh(L(c-z))}{\sinh(Lc)} & = \frac{\exp[ L(c-z)]  - \exp[-L(c-z)]  }{\exp(Lc) - \exp(-Lc)  }\\
& \to \frac{\exp[ L(c-z)]}{\exp(Lc)} \\
& \to \exp(-Lz)
\end{align*}





\section{QUESTION 2}


The potential satisfies

\[ \nabla^{2} \phi = 0 = \frac{1}{r} \frac{\partial }{\partial r} \left(  r \frac{\partial \phi}{\partial r}  \right) + \frac{1}{r^{2}} \frac{\partial^{2} \phi}{\partial \theta^{2}}   \]

with Dirichlet boundary conditions

\[ \phi(r=1,\theta) = \begin{cases} \pi / 2  & \text{ if } 0 \leq \theta < \pi \\ - \pi / 2 & \text{ if } \pi \leq \theta < 2\pi \end{cases} \]

Separating variables by writing $ \phi(r,\theta) = R(r) \Theta(\theta) $,

\[ \frac{1}{r}  \frac{\partial }{\partial r} (  r R' \Theta )  + \frac{1}{r^{2}} R \Theta '' = 0   \]

\[ \Rightarrow \frac{r}{R}  \frac{\partial }{\partial r}  ( r R') + \frac{\Theta''}{\Theta}  = 0   \]

Keeping $ r $ fixed and varying $ \theta $, we see that $ \frac{\Theta''}{\Theta} = - \lambda $ constant.
Solving $ \Theta'' = - \Theta X $, if $ \phi $ single valued, must have $ \Theta(\theta + 2\pi) = \Theta(\theta)  $, thus $ \lambda = n^{2} $ for some integer $ n $, and 

\[ \Theta_{n}(\theta) = a_{n} \cos n \theta + b_{n} \sin n \theta  \]

Next we solve

\[ r \frac{\d }{\d r}(r R'_{n}) - n^{2} R_{n}= 0\]

For $ n \neq 0 $, assuming that $ R_{n} \propto r^{\beta} $, we have

\[ \beta^{2} - n^{2} = 0 \Rightarrow \beta = \pm n \]

Thus 

\[ R_{n}(r) = c_{n}r^{n} + d_{n} r^{-n}, \quad n = 1,2,3,\cdots \]

Therefore, the family of particular solutions is

\[ \phi_{n}(r,\theta) = (a_{n} \cos n \theta + b_{n} \sin n \theta)(c_{n}r^{n}  + d_{n} r^{-n} ) \quad n = 1,2,3,\cdots \]

For $ n = 0 $, we solve $ r \frac{\d }{\d r}(r R'_{0}) = 0 $, thus $ r R'_{0} = \text{constant} $, and we have $ R_{0} = d_{0} \log r + c_{0} $ 

Hence by linearity, the general solution for Laplace's equation in polar coordinates is

\[ \phi(r,\theta) = c_{0} + d_{0} \log r + \sum_{n=1}^{\infty}  (a_{n} \cos n \theta  + b_{n} \sin n \theta )(c_{n}r^{n}  + d_{n}r^{-n} )  \]

Now, requiring regularity at the origin, $ d_{0}=0,d_{n} = 0 $, then absorb $ c_{n} $ as a general rescaling, we can write this solution as

\[ \phi(r,\theta)  = \frac{a_{0}}{2} + \sum_{n=1}^{\infty} (a_{n}  \cos n \theta + b_{n} \sin n \theta ) r^{n}  \]

Using boundary conditions,

\[ f(\theta) = \phi(1,\theta)  = \frac{a_{0}}{2} + \sum_{n=1}^{\infty} (a_{n}  \cos n \theta + b_{n} \sin n \theta ) \]


The function is odd, hence 

\[ f(\theta) = \phi(1,\theta)  = \sum_{n=1}^{\infty} b_{n} \sin n \theta \]

By orthogonality of sines, 

\begin{align*}
b_{m} & = \frac{1}{\pi} \int_{0}^{2 \pi} f(\theta) \sin (m \theta) \; \d \theta\\
& = \frac{1}{\pi} \left(  \int_{0}^{\pi} \frac{\pi}{2}  \sin m \theta  \; \d \theta + \int_{\pi}^{2\pi} - \frac{\pi}{2}  \sin m \theta \; \d \theta \right)   \\
& = \frac{1}{2m}  \left(  \left[  - \cos m \theta  \right]_{0}^{\pi}  + \left[  \cos m \theta \right]_{\pi}^{2\pi}  \right)  \\
& = \frac{1}{m} ( -1 - (-1)^{m}) \\
& = \begin{cases} \frac{2}{m}  & \text{ if } m \text{ odd} \\ 0 & \text{ if } m \text{ even} \end{cases}
\end{align*}

This gives $ b_{m} = 2 / m $, hence

\[ \phi(r,\theta)  = 2 \sum_{n \text{ odd}}  \frac{ r^{n} \sin n \theta }{n}  \]

Next, under the substitution $ z = r e^{i \theta} $, we have $ z^{n} = r^{n} \cos n \theta + i \sin n \theta $

\begin{align*}
\phi(r,\theta) & = 2 \sum_{n \text{ odd}}  \frac{ r^{n} \sin n \theta }{n} \\
& = 2 \Im \left( \sum_{n \text{ odd}} \frac{z^{n}}{n}  \right) \\
& = 2 \Im \left( \sum_{n = 0}^{\infty} \frac{z^{2n + 1}}{2n + 1}  \right) \\
& = 2 \Im \left( \sum_{n= 0}^{\infty} \int_{0}^{z} t^{2n} \; \d t \right) 
\end{align*}

and, assuming we can swap the order of summation and integration, we have 

\begin{align*}
& = 2 \Im \left( \int_{0}^{z} \sum_{n= 0}^{\infty} t^{2n} \; \d t \right) \\
& = 2 \Im \int_{0}^{z} \frac{1}{1-t^{2}} \; \d t \\
& = \Im \int_{0}^{z} \left(   \frac{1}{1+t} + \frac{1}{1-t}  \right)  \; \d t \\
& = \Im \left[  \log(1+t) - \log(1-t) \right]_{0}^{z} \\
& = \Im \left[  \log(1+z) - \log(1-z) \right] \\
& = \arg(1 + z) - \arg(1-z)
\end{align*}

which is some angle, see $ w^{2} = \frac{1+z}{1-z} $, V and M sheet 1










\section{QUESTION 3}


The potential satisfies

\[ \nabla^{2} \psi = 0 = \frac{1}{r^{2}} \frac{\partial }{\partial r} \left( r^{2}  \frac{\partial \psi }{\partial r} \right) + \frac{1}{r^{2} \sin \theta} \frac{\partial}{\partial \theta} \left(  \sin \theta \frac{\partial \psi }{\partial \theta} \right)     \]

and $ \psi(r,\theta) $ satisfies the Dirichlet boundary conditions (hence solution existence and uniqueness)

\[ \psi(r,\theta) = \begin{cases} V  & \text{ if } 0 \leq \theta < \frac{\pi}{2} \\ -V & \text{ if } \frac{\pi}{2} < \theta \leq \pi \end{cases} \]

Separating variables by writing $ \phi(r,\theta) = R(r) \Theta(\theta) $, we have the two ODEs


\[ \frac{\d }{\d \theta} \left(  \sin \theta \frac{\d \Theta}{\d \theta} \right) + \lambda \sin \theta \; \Theta = 0  \]

\[ \frac{\d }{\d r} \left(  r^{2} \frac{\d R}{\d r} \right) - \lambda R = 0   \]

with $ \lambda \in \R $ separation constant. 

Making the substitution $ x = \cos \theta $ in the angular equation yields $ \frac{\d }{\d \theta} = - \sin \theta \frac{\d }{\d x} $, and 

\[ - \sin \theta \frac{\d }{\d x} \left[  \sin \theta \left(  - \sin \theta \frac{\d \Theta}{\d \theta} \right)  \right] + \lambda \sin \theta \; \Theta = 0  \]

which becomes Legendre's equation: 

\[ - \frac{\d }{\d x}  \left[   (1-x^{2}) \frac{\d }{\d x}  \Theta \right]   = \lambda \Theta \]

substituting  $ \Theta = \sum_{n=0}^{\infty} a_{n} x^{n}  $ (only non-negative powers as we want solution to be regular at origin) yields

\[ (1-x^{2}) \sum_{n=2}^{\infty} a_{n} n(n-1)x^{n-2} - 2 \sum_{n=1}^{\infty} a_{n} n x^{n} + \lambda \sum_{n=0}^{\infty} x^{n} = 0  \]

which must hold for each power separately, thus we obtain the recursion relation:

\[ 0 = a_{n+2}(n+2)(n+1) - a_{n} n(n-1) - 2a_{n} n + \lambda a_{n}  \]

yielding the recursion relation

\[ a_{n+2} = \left[  \frac{n(n+1) - \lambda}{(n+1)(n+2)} \right] a_{n}  \]

Thus we find two linearly independent (even and odd) solutions

\begin{align*}
\Theta_{e} & = a_{0}  \left[  1 + \frac{(-\lambda)x^{2}}{2!} \cdots  \right]   \\
\Theta_{o} & = a_{1} \left[   x + \frac{(2 - \lambda) x^{3} }{3!} + \cdots \right]  
\end{align*}

Solution must remain bounded at $ x = \pm 1 $, so $ \lambda = m(m+1) $ for some integer $ m $. These  $ \Theta_{n}(\theta) = P_{n}(x) = P_{n} (\cos \theta )  $, with $ \lambda= n(n+1) $, are Legendre polynomials of order $ n $, with the orthogonal property $ \int_{-1}^{1} P_{m}(z)P_{n}(z) \; \d z = \frac{2}{2n + 1} \delta_{mn} $. Then the DE for $ R $ becomes

\[ \frac{\d }{\d r} \left(  r^{2} \frac{\d R_{n}}{\d r} \right) - n(n+1) R_{n} = 0   \]

Assuming that $ R_{n} \propto r^{\beta} $,

\begin{align*}
\beta (\beta + 1) & = n(n+1) \\
\Rightarrow \beta & = n \text{ or } -(n+1) 
\end{align*}

Thus our general solution takes the from 


\[ \psi (r,\theta) =  \sum_{n=0}^{\infty} (a_{n} r^{n}  + b_{n} r^{-(n+1)}    ) P_{n} (\cos \theta)   \]


for solution to be regular at origin must have $ b_{n} = 0 $, so

\[ \psi (r,\theta) =  \sum_{n=0}^{\infty} a_{n} r^{n} P_{n} (\cos \theta)   \]

We can apply our boundary condition easiest by setting $ r=1 $, then

\begin{align*}
f(\theta) : = \psi(r=1,\theta) & = \sum_{n=0}^{\infty}  a_{n} P_{n} \cos (\theta), \quad 0 \leq \theta \leq \pi  \\
F(x) & : = \sum_{n=0}^{\infty}  a_{n} P_{n} (x), \qquad x = \cos \theta, -1 \leq x \leq 1 \\
\text{where } F(x) & = \begin{cases} V  & \text{ if } 0 \leq x < 1 \\ -V & \text{ if } -1 \leq x < 0 \end{cases} \\
\Rightarrow a_{n} & = \frac{(2n + 1)}{2} \int_{-1}^{1} F(x) P_{n}(x) \; \d x  \\
& = \frac{(2n + 1)}{2} \int_{0}^{1} V P_{n}(x) \; \d x + \frac{(2n + 1)}{2} \int_{-1}^{0} - V P_{n}(x) \; \d x \\
& = \frac{V}{2} \int_{0}^{1} P'_{n+1}(x) - P'_{n-1}(x) \; \d x - \frac{V}{2} \int_{-1}^{0}  P'_{n+1}(x) - P'_{n-1}(x) \; \d x
\end{align*}

Note that for $ n $ even, the integrals cancel out and we have $ a_{n} = 0 $. Otherwise, 

\begin{align*}
a_{n} & = V \int_{0}^{1}   P'_{n+1}(x) - P'_{n-1}(x) \; \d x \qquad \text{ as } n \text{ odd}  \\
& = V [ P_{n+1} (z) - P_{n-1}(z)  ]_{0}^{1} \\
& = V (  P_{n-1}(0) - P_{n+1}(0) ) \quad \text{ as } P_{n}(1) = 1 \; \forall \;  n
\end{align*}

Hence the potential inside the region is given by

\[ \psi (r,\theta) =  V \sum_{n=0}^{\infty} r^{n} (  P_{n-1}(0) - P_{n+1}(0) )   P_{n} (\cos \theta)   \]





\section{QUESTION 4}


$ y_{m} $ is an eigenfunction, hence satisfies the Sturm-Liouville equation, so we may write

\[ \frac{\d }{\d x} \left( p y_{m}' \right) = - ( \lambda_{m} - q)y_{m}     \]

Starting with the left hand side

\begin{align*}
\int_{a}^{b}  (p y'_{m}  y'_{n} + q y_{m}y_{n} ) \; \d x & = \int_{a}^{b} p y'_{m}  y'_{n} \; \d x + \int_{a}^{b} q y_{m}y_{n} \; \d x \\
\end{align*}

and integrating by parts

\begin{align*}
\int_{a}^{b} p y'_{m}  y'_{n} \; \d x & =  \left[  p y'_{m} y_{n} \right]_{a}^{b} - \int_{a}^{b} - ( \lambda_{m} - q)y_{m} y_{n} \; \d x   \\
\end{align*}

Choosing suitable boundary conditions such that

\[ \left[  p y'_{m} y_{n} \right]_{a}^{b} = 0 \]

we have 

\begin{align*}
\int_{a}^{b}  (p y'_{m}  y'_{n} + q y_{m}y_{n} ) \; \d x & =  \int_{a}^{b} ( \lambda_{m} - q)y_{m} y_{n} \; \d x  + \int_{a}^{b} q y_{m}y_{n} \; \d x \\
& =  \int_{a}^{b} \lambda_{m}y_{m} y_{n} \; \d x  \\
& = \lambda_{m} \delta_{mn} \quad \text{ by S-L orthogonality}
\end{align*}

Last part? Get solution of Zimaras. Note that 

\[ y_{n} = P_{n} \not\implies \int y_{n}^{2} \; \d x = 1 \]


\section{QUESTION 5}


(a)
\begin{enumerate}
	\item 
	
	\[ q_{n} = \frac{1}{2^{n} n!} \frac{\d^{n} }{\d x^{n}} \underbrace{\left(  x^{2n} - \binom{n}{2} x^{2n - 2} + \cdots  \right)}_{(*)}    \]
	
	Differentiating (*) $ n $ times produces a polynomial with highest power $ \frac{\d^{n} }{\d x^{n} } (x^{2n}) = x^{n} $. Hence $ q_{n}(x) $ is of degree $ n $.
	
	\item By induction: 
	
	\begin{align*}
	 q_{1}(1) & = \frac{1}{2} \frac{\d }{\d x} (x^{2} - 1) \Big|_{x=1} \\
	& = \frac{1}{2} 2x \Big|_{x=1} \\
	& = 1
	\end{align*}
	
	True for $ n = 1 $. Now suppose $ q_{k}(1) = 1 $ for some $ k > 0 $.
	
	\begin{align*}
	q_{k+1}(x) & = \frac{1}{2^{k+1} (k+1)!} \frac{\d^{n} }{\d x^{k+1}} (x^{2} - 1)^{k+1}   \\
	& =  \frac{1}{2 (k+1)} \left[  \frac{1}{2^{k} k!} \frac{\d^{k} }{\d x^{k}} \left(  2(k+1)x(x^{2} - 1)^{k}  \right)   \right]  \\
	& =  \frac{1}{2^{k} k!} \frac{\d^{k} }{\d x^{k}} \left( x(x^{2} - 1)^{k}  \right) 
	\end{align*}
	
	
	
\end{enumerate}


(b)

\begin{enumerate}
	\item $ P_{n} $ are polynomial solutions to Legendre's equation with $ \lambda_{n} = n(n+1) $, and $ q_{n} $ are polynomial solutions to Legendre's equation with $ \lambda_{n} = n(n+1) $, we have $ P_{n} \propto q_{n} $. But $ P(1) = q(1) = 1 $, so $ P_{n} = q_{n} $ (are there non poly solutions to worry about? See Skinners notes) 
	
	\item Hence we see that
	
	\[ P_{n}(x) = \frac{1}{2^{n} n!} \left(  \frac{\d }{\d x}   \right)^{n} (x^{2} - 1)^{n}   \]
	
	Using this result, by parts, 
	
	\begin{align*}
	\int_{-1}^{1} [  P_{n} (x)  ]^{2} \; \d x & = \frac{1}{2^{2n} (n!)^{2}} \int_{-1}^{1} \left(  \frac{\d }{\d x}   \right)^{n} (x^{2} - 1)^{n} \left(  \frac{\d }{\d x}   \right)^{n} (x^{2} - 1)^{n} \; \d x   \\
	& = \frac{1}{2^{2n} (n!)^{2}} \left( \left[  \left(  \frac{\d }{\d x}   \right)^{n-1} (x^{2} - 1)^{n} \left(  \frac{\d }{\d x}   \right)^{n} (x^{2} - 1)^{n} \right]_{-1}^{1} \right)  \\
	& \qquad - \frac{1}{2^{2n} (n!)^{2}} \left(     \int_{-1}^{1} \left(  \frac{\d }{\d x}   \right)^{n-1} (x^{2} - 1)^{n} \left(  \frac{\d }{\d x}   \right)^{n+1} (x^{2} - 1)^{n} \; \d x \right)  
	\end{align*}
	
	
	But we know that the boundary term vanishes since \[ \left( \frac{\d }{\d x} \right)^{m} (x^{2} - 1) \big|_{x= \pm 1} = 0 \text{ for } m < n   \]
	
	Thus, we integrate by parts iteratively, until 
	
	\begin{align*}
	\int_{-1}^{1} [  P_{n} (x)  ]^{2} \; \d x & = \cdots \\
	 & = \frac{(-1)^{n}}{2^{2n} (n!)^{2}} \int_{-1}^{1} (x^{2} - 1)^{n} \left(  \frac{\d }{\d x}   \right)^{2n} (x^{2} - 1)^{n} \; \d x   \\
	 & = \frac{(-1)^{n}(2n)!}{2^{2n} (n!)^{2}} \int_{-1}^{1} (x^{2} - 1)^{n}  \; \d x 
	\end{align*}
	
	Finally,
	
	\begin{align*}
	I_{n} & = \int_{-1}^{1}  (x^{2} - 1)^{n} \; \d x  \qquad (\text{by parts with } 1)\\
	& = \left[  x(x^{2} - 1)^{n} \right]_{-1}^{1} - \int_{-1}^{1} 2x^{2} n (x^{2} - 1)^{n-1} \; \d x\\
	& = - 2 n  \int_{-1}^{1} (x^{2} - 1) (x^{2} - 1)^{n-1} + (x^{2} - 1)^{n-1} \; \d x \\
	& = -2n (  I_{n}+ I_{n-1}) \\
	\Rightarrow I_{n} = \frac{2}{(2n+1)!!}
	\end{align*}
	
	Arrive at (?) result.
 	
\end{enumerate}

\section{QUESTION 6}


$ y(x,t) $ satisfies the 1D wave equation

\[ \frac{\partial^{2} }{\partial t^2} y(x,t) = c^{2} \frac{\partial^{2} }{\partial x^{2}} y(x,t) \qquad c^{2} = \frac{T}{\mu} \]


Assume $ y(x,t) = X(x) T(t) $ and separate variables:

\begin{align*}
X \ddot{T} & = c^{2} X'' T \\
\Rightarrow \frac{1}{c^{2}} \frac{\ddot{T}}{T}  & = \frac{X''}{X} = - \lambda 
\end{align*}

for some $ \lambda > 0 $, so we have

\begin{align*}
X'' + \lambda X& =0 \\
\ddot{T}  + \lambda c^{2} T & = 0 
\end{align*}

Solving the spatial equation first,

\[ X = \alpha \cos ( \sqrt{\lambda} x ) + \beta \sin ( \sqrt{\lambda} x ) \]

Applying initial conditions

\begin{itemize}
	\item $ X(0) = 0 \Rightarrow \alpha = 0$
	\item $ X(l) = 0 \Rightarrow \beta \sin ( \sqrt{\lambda} x ) = 0 \Rightarrow \lambda = n^{2} \pi^{2} / l^{2} $, for integer $ n $
\end{itemize}

The normal modes are the associated eigenfunctions given by

\[ X_{n}(x) = \beta_{n} \sin \left(  \frac{n \pi x}{l} \right)  \]

The associated $ T_{n}(t) $ is given by

\begin{align*}
 \ddot{T}_{n} + \frac{n^{2} \pi^{2} c^{2} }{l^{2}}  T_{n} & = 0 \\
\Rightarrow T_{n}(t) & = \gamma_{n} \cos \left(  \frac{n \pi c t}{l} \right) + \delta_{n} \sin \left(  \frac{n \pi c t}{l} \right)  
\end{align*}

Hence the general solution is

\[ y(x,t) = \sum_{n=1}^{\infty} \sin \left(  \frac{n \pi x}{l} \right)  \left(  A_{n} \cos  \left(  \frac{n \pi ct}{l} \right) + B_{n} \sin  \left(  \frac{n \pi ct}{l} \right) \right)    \]

Using BCs, $ y(x,0) = 0 \Rightarrow A_{n} = 0 \; \forall \; n $.

Next, have $ y_{t}(x,0) = \frac{4V}{l^{2}} x (l - x) $, so 


\[ \frac{4V}{l^{2}} x (l - x) = \sum_{n=1}^{\infty}  B_{n} \left( \frac{n \pi c}{l} \right)  \sin \left(  \frac{n \pi x}{l} \right)     \]

\begin{align*}
\Rightarrow B_{n} & = \frac{4V}{l^{2}} \left(  \frac{l}{n \pi c} \right)  \frac{2}{l}   \int_{0}^{l}  x(l-x) \sin \left(  \frac{n \pi x}{l} \right) \; \d x   \\
& = \frac{8V}{n \pi c l^{2}}  \int_{0}^{l}  x(l-x) \sin \left(  \frac{n \pi x}{l} \right) \; \d x  \\
& = \cdots \\
& = \frac{16 l V^{2}}{c \pi^{4} n^{4}} \left(  1 - (-1)^{n} \right)  
\end{align*}

(Think should be $ V $, and not $ V^{2} $, do check later.)

Hence

\[ y(x,t) = \frac{32 l V^{2}}{c \pi^{4}} \sum_{n \text{ odd}} \frac{1}{n^{4} }\sin \left(  \frac{n \pi x}{l} \right) \sin  \left(  \frac{n \pi ct}{l} \right)    \]

Next, kinetic energy is given by

\begin{align*}
K & = \int_{0}^{L} \frac{1}{2} \mu \left(  \frac{\partial y}{\partial t} \right)^{2} \; \d x   \\
\end{align*}


Have

\[y_{t}(x,t) = \frac{32V^{2}}{\pi^{3}} \sum_{n \text{ odd}} \frac{1}{n^{3}} \sin \left( \frac{n \pi x}{l} \right)  \cos \left( \frac{n \pi c t}{l}    \right)   \] 


Thus 

\begin{align*}
K & = \frac{512 V^{4}}{\pi^{6}} \mu \int_{0}^{l}  \sum_{n,m \text{ odd}} \frac{1}{n^{3}m^{3}} \sin \left( \frac{n \pi x}{l} \right)  \cos \left( \frac{n \pi c t}{l}\right)  \sin \left( \frac{m \pi x}{l} \right)  \cos \left( \frac{m \pi c t}{l} \right) \; \d x   \\
& = \frac{512 V^{4}}{\pi^{6}} \mu  \sum_{n,m \text{ odd}} \frac{1}{n^{3}m^{3}}  \cos \left( \frac{n \pi c t}{l}\right)   \cos \left( \frac{m \pi c t}{l} \right) \underbrace{\int_{0}^{l} \sin \left( \frac{n \pi x}{l} \right) \sin \left( \frac{m \pi x}{l} \right)   \; \d x  }_{= \frac{l}{2} \delta_{mn}} \\
& = \frac{256 V^{4} l}{\pi^{6}} \mu  \sum_{n\text{ odd}} \frac{1}{n^{6}}  \cos^{2} \left( \frac{n \pi c t}{l}\right) 
\end{align*}


Similarly, PE given by

\[ V = \int_{0}^{l} \frac{T}{2} \left(  \frac{\partial y}{\partial x} \right)^{2} \; \d x  \]

Have

\[ y_{x}(x,t) = \frac{32 V^{2}}{c \pi^{3}} \sum_{n \text{ odd}} \frac{1}{n^{3}} \cos \left(  \frac{n \pi x}{l} \right) \sin \left(  \frac{n \pi c t}{l} \right)   \]

Thus

\begin{align*}
V & = \frac{512 V^{4}}{c^{2} \pi^{6}}T \int_{0}^{l} \sum_{n,m \text{ odd}} \frac{1}{n^{3}m^{3}} \cos \left(  \frac{n \pi x}{l} \right) \cos \left(  \frac{m \pi x}{l} \right) \sin \left(  \frac{n \pi c t}{l} \right)  \sin \left(  \frac{m \pi c t}{l} \right)  \; \d x \\
& = \frac{512 V^{4}}{c^{2} \pi^{6}}T  \sum_{n,m \text{ odd}} \frac{1}{n^{3}m^{3}}  \sin \left(  \frac{n \pi c t}{l} \right)  \sin \left(  \frac{m \pi c t}{l} \right) \int_{0}^{l} \underbrace{\cos \left(  \frac{n \pi x}{l} \right) \cos \left(  \frac{m \pi x}{l} \right)   \; \d x}_{\frac{l}{2}\delta_{mn}} \\
& = \frac{256 V^{4}l}{c^{2} \pi^{6}}T  \sum_{n \text{ odd}} \frac{1}{n^{6}}  \sin^{2} \left(  \frac{n \pi c t}{l} \right)  
\end{align*}

Compare this with the initial energy

\begin{align*}
K & = \int_{0}^{l} \frac{1}{2} \mu \left(  \frac{\partial y}{\partial t} \Big|_{t=0} \right)^{2} \; \d x \\
& = \int_{0}^{l} \frac{1}{2} \mu \left(  \frac{4V}{l^{2}}x(l-x) \right)^{2} \; \d x  \\
& =  \frac{8V^{2} \mu}{l^{4}} \int_{0}^{l} x^{2}(l-x)^{2} \; \d x  \\
& = \frac{8V^{2} \mu}{l^{4}} \left( \frac{l^{5}}{30} \right) \\
& = \frac{4V^{2}\mu l}{15}
\end{align*}

Setting $ t = 0 $ in our previous result gives

\[ K = \frac{256 V^{4} l}{\pi^{6}} \mu  \sum_{n\text{ odd}} \frac{1}{n^{6}}  \]

Thus by comparison

\begin{align*}
\frac{256}{\pi^{6}}  \sum_{n\text{ odd}} \frac{1}{n^{6}} & = \frac{4}{15} \\
\Rightarrow \; \sum_{n \text{ odd }} \frac{1}{n^{6}} & = \frac{\pi^{6}}{960} 
\end{align*}

\section{QUESTION 7}


\begin{enumerate}
	\item \begin{itemize}
		\item Assume all displacements are sufficiently small $ ( y \ll l) $
		\item Assume all displacements are vertical
		\item Consider two points $ x $ and $ x + \delta x $. The angle of the string to the horizontal at $ x $ is $ \theta_{1} $, and the angle at $ x + \delta x $ is $ \theta_{2} $.
		\item Resolving horizontally shows that $ T(x)\cos \theta_{1} = T(x + \delta x) \cos \theta_{2} $, since $ \theta \ll 1 $, the tension is approximately constant.
		\item Resolving vertically 
		
		\begin{align*}
		T \sin \theta_{2} -  T \sin \theta_{1} - \mu g \delta x - 2k \mu \delta x \frac{\partial  }{\partial t}y  & = \mu \delta x \frac{\partial^{2} }{\partial t^{2}} y   \qquad (*) \\
		\end{align*} 
		
		
		\item Assume angles are small
		\[ \sin \theta_{2} \approx tan \theta_{2} =  \frac{\partial y }{\partial x} \Big|_{x + \delta x} \approx \frac{\partial y }{\partial x} \Big|_{x} + \delta x \frac{\partial^{2} y }{\partial x^{2}} \Big|_{x}   \]
		
		\[ \sin \theta_{1} \approx \tan \theta_{1} = \frac{\partial y }{\partial x} \Big|_{x} \]
	
		\item (*) becomes 
		
		\begin{align*}
		T \delta x \frac{\partial^{2} y }{\partial x^{2}}   - \mu g \delta x & - 2k \mu \delta x \frac{\partial  }{\partial t}y   = \mu \delta x \frac{\partial^{2} }{\partial t^{2}} y   \\
		\Rightarrow \frac{T}{\mu} \frac{\partial^{2} y }{\partial x^{2}}  & = \frac{\partial^{2} }{\partial t^{2}} y +  2k \frac{\partial  }{\partial t}y + g
		\end{align*} 
		
		\item Further assume the weight is insignificant $ (g \to 0) $
		
		\item Hence arrive at the equation of motion 
		
		\[ c^{2} \frac{\partial^{2} y }{\partial x^{2}}   = \frac{\partial^{2} }{\partial t^{2}} y +  2k \frac{\partial  }{\partial t}y \]
		
		where $ c^{2} = \frac{T}{\mu} $
		
	\end{itemize}
		
		Assume $ y(x,t) = X(x)T(t) $ and separating variables gives
		
		\begin{align*}
		c^{2} \frac{X''}{X} & = \frac{\ddot{T}}{T} + 2k \frac{\dot{T}}{T}    \\
		\Rightarrow  \frac{X''}{X} & = \frac{\ddot{T} + 2k \dot{T} }{T} = - \lambda 
		\end{align*}
		
		for some $ \lambda > 0 $, so we have
		
		\begin{align*}
		X'' + \lambda X& =0 \\
		\ddot{T}  + 2k \dot{T} + \lambda c^{2} T & = 0 
		\end{align*}
		
		Solving the spatial equation first,
		
		\[ X = \alpha \cos ( \sqrt{\lambda} x ) + \beta \sin ( \sqrt{\lambda} x ) \]
		
		Applying initial conditions
		
		\begin{itemize}
			\item $ X(0) = 0 \Rightarrow \alpha = 0$
			\item $ X(l) = 0 \Rightarrow \beta \sin ( \sqrt{\lambda} x ) = 0 \Rightarrow \lambda = n^{2} \pi^{2} / l^{2} $, for integer $ n $
		\end{itemize}
	
		These $ \lambda $ are eigenvalues, with associated eigenfunctions
		
		\[ X_{n}(x) = \sin \left(  \frac{n \pi x}{L} \right)  \]
		
		The associated $ T_{n}(t) $ is given by
		
		\begin{align*}
		\qquad  & \ddot{T}_{n} + 2k \dot{T}_{n}  + \frac{n^{2} \pi^{2} c^{2} }{L^{2}}  T_{n} = 0 \qquad k = \frac{\pi c}{l}\\
		\Rightarrow T_{n}(t) & = e^{-kt} \left(    \gamma_{n} \cos \left(  \sqrt{n^{2} - 1} k t \right) + \delta_{n} \sin \left(  \sqrt{n^{2} - 1} k t \right)  \right) \qquad (n \geq 2)
		\end{align*}
		
		Being careful with the $ n = 1 $ case, must have
		
		\[ T_{1}(t) = e^{-kt} \left( \gamma_{1} + \delta_{1} t \right)    \]
		
		Hence the general solution is 
		
		\begin{align*}
		y(x,t) = & e^{-kt} \sin \left( \frac{\pi x}{l} \right)  \left( A_{1} + B_{1} t \right) \\
		 & +  \sum_{n = 2}^{\infty} e^{-kt}\sin \left( \frac{n \pi x}{l} \right)  \left(  A_{n} \cos  \left( \sqrt{n^{2} - 1} k t \right) + B_{n} \sin  \left(  \sqrt{n^{2} - 1} k t \right) \right)  
		\end{align*}
		
		Using the boundary condition $ y(x,0) = A \sin (\pi x / l) $, we have
		
		\[ A \sin (\pi x / l)  =  A_{1} \sin \left(  \frac{\pi x}{l} \right)   + \sum_{n=2}^{\infty} A_{n} \sin \left( \frac{n \pi x}{l} \right)    \]
		
		Ah. So we conclude that $ A_{1} = A $, and $ A_{n} = 0 $ for $ n \geq 2 $, thus
		
		\[ y(x,t) = (A + B_{1} t) e^{-kt}\sin \left( \frac{\pi x}{l} \right)     + \sum_{n=2}^{\infty} e^{-kt}\sin \left( \frac{n \pi x}{l} \right)  B_{n} \sin  \left(  \sqrt{n^{2} - 1} k t \right)   \]
		
		Next, use the boundary condition $ y_{t}(x,0) = 0 $. We have that
		
		
		\begin{align*}
		y_{t}(x,t) & = (-kA + B_{1} - kB_{1} t) e^{-kt}\sin \left( \frac{\pi x}{l} \right)  \\
		& \; \; + \sum_{n=2}^{\infty} \sin \left( \frac{n \pi x}{l} \right) \left[ -kt e^{-kt}  B_{n} \sin  \left(  \sqrt{n^{2} - 1} k t \right) + ( \sqrt{n^{2} - 1}k   ) e^{-kt} B_{n} \cos  \left(  \sqrt{n^{2} - 1} k t \right)  \right]  \\
		\end{align*}
		
		Thus
		
		\begin{align*}
		0 = y_{t}(x,0)  & =  (- k A + B_{1}) \sin \left( \frac{\pi x}{l} \right)  + \sum_{n=2}^{\infty} B_{n} \sin \left( \frac{n \pi x}{l} \right) ( \sqrt{n^{2} - 1}k )  
		\end{align*}
		
		Thus we conclude that $ B_{1} = k A $ and $ B_{n} = 0 $ for $ n \geq 2 $. 
		
		Thus the general solution is given by
		
		\[ y(x,t) = A (1 + k t) e^{-kt}\sin \left( \frac{\pi x}{l} \right)    \]
		
		
	
		\item 
		
		
		
\end{enumerate}
\section{QUESTION 8}

	\begin{enumerate}
			\item \begin{itemize}
			\item Assume all displacements are sufficiently small $ ( y \ll l) $
			\item Assume all displacements are vertical
			\item With mass $ M $ at $ x = 0 $, consider the tension of the string acting on the two points $ - \varepsilon $ and $ \varepsilon $ either side, with the angle the string makes with the horizontal denoted by $ \theta_{- \varepsilon} $ and $ \theta_{+\varepsilon} $
			\item Resolving horizontally shows that the tension is constant.
			\item Using Newton's Second Law vertically, we have that 
			
			\begin{align*}
			M  \frac{\d^{2} y }{\d t^{2}} \Big|_{x=0} & = T \sin \theta_{\varepsilon} - T \sin \theta_{-\varepsilon} \\
			\end{align*} 
			
			
			\item Assume angles are small
			
			\[ \sin \theta_{\varepsilon} \approx \tan \theta_{\varepsilon} = \frac{\partial y }{\partial x} \Big|_{x = \varepsilon} \]
			
			\item Hence
			
			\begin{align*}
			M  \frac{\d^{2} y }{\d t^{2}} \Big|_{x=0} & = T \frac{\partial y }{\partial x} \Big|_{x = \varepsilon} - T \frac{\partial y }{\partial x} \Big|_{x = -\varepsilon} \\
			& = T \left[  \frac{\partial y }{\partial x} \right]_{x = 0_{-}}^{x = 0_{+}} \qquad \text{ as } \varepsilon \to 0  
			\end{align*} 
			
			
		\end{itemize}
	
	\item The incident wave $ W_{I} =  \exp(i \omega(t - x/c) )$ will produce a transmitted wave $ W_{T} = T \exp( i \omega(t - x / c)  ) $ and a reflected wave $ W_{R} = R\exp(i \omega(t + x/c)) $.  Know that
	
	\[ y(x,t) = \begin{cases} W_{I} + W_{R}  & \text{ if } x < 0 \\ W_{T} & \text{ if } x > 0 \end{cases} \]
	
	with boundary conditions continuity at zero ( $ [ y(0,t) ]_{x = 0_{-}}^{x = 0_{+}} = 0 $ ) and 
	
	\[  \frac{\d^{2} y }{\d t^{2}} \Big|_{x=0} = \frac{T}{M} \left[  \frac{\partial y }{\partial x} \right]_{x = 0_{-}}^{x = 0_{+}} \]
	
	
	\end{enumerate}




\section{QUESTION 9}


$ y(x,t) $ define on $ 0 \leq x \leq l $ satisfies the wave equation

\[ \frac{\partial^{2}  y}{\partial t^{2} } = c^{2} \frac{\partial^{2} y }{\partial x^{2}} \]

with $ y(0,t) = y(l,t) = 0 $ (fixed at endpoints). We can find the solution $ y(x,t) $ for $ t < 0 $, given $ y(x,0) = 0 $, and 

\[ \left[  \frac{\partial y }{\partial t} \right]_{t = 0_{-}}^{t = 0_{+}} = \lambda \delta \left( x - \frac{l}{2} \right) \] 



\section{QUESTION 10}

\end{document}