\documentclass[a4paper]{article}
\usepackage{amsmath}
\def\npart {IB}
\def\nterm {Michaelmas}
\def\nyear {2017}
\def\nlecturer {Dr. Saxton}
\def\ncourse {Methods Example Sheet 3}

% Imports
\ifx \nauthor\undefined
  \def\nauthor{Christopher Turnbull}
\else
\fi

\author{Supervised by \nlecturer \\\small Solutions presented by \nauthor}
\date{\nterm\ \nyear}

\usepackage{alltt}
\usepackage{amsfonts}
\usepackage{amsmath}
\usepackage{amssymb}
\usepackage{amsthm}
\usepackage{booktabs}
\usepackage{caption}
\usepackage{enumitem}
\usepackage{fancyhdr}
\usepackage{graphicx}
\usepackage{mathdots}
\usepackage{mathtools}
\usepackage{microtype}
\usepackage{multirow}
\usepackage{pdflscape}
\usepackage{pgfplots}
\usepackage{siunitx}
\usepackage{slashed}
\usepackage{tabularx}
\usepackage{tikz}
\usepackage{tkz-euclide}
\usepackage[normalem]{ulem}
\usepackage[all]{xy}
\usepackage{imakeidx}

\makeindex[intoc, title=Index]
\indexsetup{othercode={\lhead{\emph{Index}}}}

\ifx \nextra \undefined
  \usepackage[pdftex,
    hidelinks,
    pdfauthor={Christopher Turnbull},
    pdfsubject={Cambridge Maths Notes: Part \npart\ - \ncourse},
    pdftitle={Part \npart\ - \ncourse},
  pdfkeywords={Cambridge Mathematics Maths Math \npart\ \nterm\ \nyear\ \ncourse}]{hyperref}
  \title{Part \npart\ --- \ncourse}
\else
  \usepackage[pdftex,
    hidelinks,
    pdfauthor={Christopher Turnbull},
    pdfsubject={Cambridge Maths Notes: Part \npart\ - \ncourse\ (\nextra)},
    pdftitle={Part \npart\ - \ncourse\ (\nextra)},
  pdfkeywords={Cambridge Mathematics Maths Math \npart\ \nterm\ \nyear\ \ncourse\ \nextra}]{hyperref}

  \title{Part \npart\ --- \ncourse \\ {\Large \nextra}}
  \renewcommand\printindex{}
\fi

\pgfplotsset{compat=1.12}

\pagestyle{fancyplain}
\lhead{\emph{\nouppercase{\leftmark}}}
\ifx \nextra \undefined
  \rhead{
    \ifnum\thepage=1
    \else
      \npart\ \ncourse
    \fi}
\else
  \rhead{
    \ifnum\thepage=1
    \else
      \npart\ \ncourse\ (\nextra)
    \fi}
\fi
\usetikzlibrary{arrows.meta}
\usetikzlibrary{decorations.markings}
\usetikzlibrary{decorations.pathmorphing}
\usetikzlibrary{positioning}
\usetikzlibrary{fadings}
\usetikzlibrary{intersections}
\usetikzlibrary{cd}

\newcommand*{\Cdot}{{\raisebox{-0.25ex}{\scalebox{1.5}{$\cdot$}}}}
\newcommand {\pd}[2][ ]{
  \ifx #1 { }
    \frac{\partial}{\partial #2}
  \else
    \frac{\partial^{#1}}{\partial #2^{#1}}
  \fi
}
\ifx \nhtml \undefined
\else
  \renewcommand\printindex{}
  \makeatletter
  \DisableLigatures[f]{family = *}
  \let\Contentsline\contentsline
  \renewcommand\contentsline[3]{\Contentsline{#1}{#2}{}}
  \renewcommand{\@dotsep}{10000}
  \newlength\currentparindent
  \setlength\currentparindent\parindent

  \newcommand\@minipagerestore{\setlength{\parindent}{\currentparindent}}
  \usepackage[active,tightpage,pdftex]{preview}
  \renewcommand{\PreviewBorder}{0.1cm}

  \newenvironment{stretchpage}%
  {\begin{preview}\begin{minipage}{\hsize}}%
    {\end{minipage}\end{preview}}
  \AtBeginDocument{\begin{stretchpage}}
  \AtEndDocument{\end{stretchpage}}

  \newcommand{\@@newpage}{\end{stretchpage}\begin{stretchpage}}

  \let\@real@section\section
  \renewcommand{\section}{\@@newpage\@real@section}
  \let\@real@subsection\subsection
  \renewcommand{\subsection}{\@@newpage\@real@subsection}
  \makeatother
\fi

% Theorems
\theoremstyle{definition}
\newtheorem*{aim}{Aim}
\newtheorem*{axiom}{Axiom}
\newtheorem*{claim}{Claim}
\newtheorem*{cor}{Corollary}
\newtheorem*{conjecture}{Conjecture}
\newtheorem*{defi}{Definition}
\newtheorem*{eg}{Example}
\newtheorem*{ex}{Exercise}
\newtheorem*{fact}{Fact}
\newtheorem*{law}{Law}
\newtheorem*{lemma}{Lemma}
\newtheorem*{notation}{Notation}
\newtheorem*{prop}{Proposition}
\newtheorem*{soln}{Solution}
\newtheorem*{thm}{Theorem}

\newtheorem*{remark}{Remark}
\newtheorem*{warning}{Warning}
\newtheorem*{exercise}{Exercise}

\newtheorem{nthm}{Theorem}[section]
\newtheorem{nlemma}[nthm]{Lemma}
\newtheorem{nprop}[nthm]{Proposition}
\newtheorem{ncor}[nthm]{Corollary}


\renewcommand{\labelitemi}{--}
\renewcommand{\labelitemii}{$\circ$}
\renewcommand{\labelenumi}{(\roman{*})}

\let\stdsection\section
\renewcommand\section{\newpage\stdsection}

% Strike through
\def\st{\bgroup \ULdepth=-.55ex \ULset}

% Maths symbols
\newcommand{\abs}[1]{\left\lvert #1\right\rvert}
\newcommand\ad{\mathrm{ad}}
\newcommand\AND{\mathsf{AND}}
\newcommand\Art{\mathrm{Art}}
\newcommand{\Bilin}{\mathrm{Bilin}}
\newcommand{\bket}[1]{\left\lvert #1\right\rangle}
\newcommand{\B}{\mathcal{B}}
\newcommand{\bolds}[1]{{\bfseries #1}}
\newcommand{\brak}[1]{\left\langle #1 \right\rvert}
\newcommand{\braket}[2]{\left\langle #1\middle\vert #2 \right\rangle}
\newcommand{\bra}{\langle}
\newcommand{\cat}[1]{\mathsf{#1}}
\newcommand{\C}{\mathbb{C}}
\newcommand{\CP}{\mathbb{CP}}
\newcommand{\cU}{\mathcal{U}}
\newcommand{\Der}{\mathrm{Der}}
\newcommand{\D}{\mathrm{D}}
\newcommand{\dR}{\mathrm{dR}}
\newcommand{\E}{\mathbb{E}}
\newcommand{\F}{\mathbb{F}}
\newcommand{\Frob}{\mathrm{Frob}}
\newcommand{\GG}{\mathbb{G}}
\newcommand{\gl}{\mathfrak{gl}}
\newcommand{\GL}{\mathrm{GL}}
\newcommand{\G}{\mathcal{G}}
\newcommand{\Gr}{\mathrm{Gr}}
\newcommand{\haut}{\mathrm{ht}}
\newcommand{\Id}{\mathrm{Id}}
\newcommand{\ket}{\rangle}
\newcommand{\lie}[1]{\mathfrak{#1}}
\newcommand{\Mat}{\mathrm{Mat}}
\newcommand{\N}{\mathbb{N}}
\newcommand{\norm}[1]{\left\lVert #1\right\rVert}
\newcommand{\normalorder}[1]{\mathop{:}\nolimits\!#1\!\mathop{:}\nolimits}
\newcommand\NOT{\mathsf{NOT}}
\newcommand{\Oc}{\mathcal{O}}
\newcommand{\Or}{\mathrm{O}}
\newcommand\OR{\mathsf{OR}}
\newcommand{\ort}{\mathfrak{o}}
\newcommand{\PGL}{\mathrm{PGL}}
\newcommand{\ph}{\,\cdot\,}
\newcommand{\pr}{\mathrm{pr}}
\newcommand{\Prob}{\mathbb{P}}
\newcommand{\PSL}{\mathrm{PSL}}
\newcommand{\Ps}{\mathcal{P}}
\newcommand{\PSU}{\mathrm{PSU}}
\newcommand{\pt}{\mathrm{pt}}
\newcommand{\qeq}{\mathrel{``{=}"}}
\newcommand{\Q}{\mathbb{Q}}
\newcommand{\R}{\mathbb{R}}
\newcommand{\RP}{\mathbb{RP}}
\newcommand{\Rs}{\mathcal{R}}
\newcommand{\SL}{\mathrm{SL}}
\newcommand{\so}{\mathfrak{so}}
\newcommand{\SO}{\mathrm{SO}}
\newcommand{\Spin}{\mathrm{Spin}}
\newcommand{\Sp}{\mathrm{Sp}}
\newcommand{\su}{\mathfrak{su}}
\newcommand{\SU}{\mathrm{SU}}
\newcommand{\term}[1]{\emph{#1}\index{#1}}
\newcommand{\T}{\mathbb{T}}
\newcommand{\tv}[1]{|#1|}
\newcommand{\U}{\mathrm{U}}
\newcommand{\uu}{\mathfrak{u}}
\newcommand{\Vect}{\mathrm{Vect}}
\newcommand{\wsto}{\stackrel{\mathrm{w}^*}{\to}}
\newcommand{\wt}{\mathrm{wt}}
\newcommand{\wto}{\stackrel{\mathrm{w}}{\to}}
\newcommand{\Z}{\mathbb{Z}}
\renewcommand{\d}{\mathrm{d}}
\renewcommand{\H}{\mathbb{H}}
\renewcommand{\P}{\mathbb{P}}
\renewcommand{\sl}{\mathfrak{sl}}
\renewcommand{\vec}[1]{\boldsymbol{\mathbf{#1}}}
%\renewcommand{\F}{\mathcal{F}}

\let\Im\relax
\let\Re\relax

\DeclareMathOperator{\adj}{adj}
\DeclareMathOperator{\Ann}{Ann}
\DeclareMathOperator{\area}{area}
\DeclareMathOperator{\Aut}{Aut}
\DeclareMathOperator{\Bernoulli}{Bernoulli}
\DeclareMathOperator{\betaD}{beta}
\DeclareMathOperator{\bias}{bias}
\DeclareMathOperator{\binomial}{binomial}
\DeclareMathOperator{\card}{card}
\DeclareMathOperator{\ccl}{ccl}
\DeclareMathOperator{\Char}{char}
\DeclareMathOperator{\ch}{ch}
\DeclareMathOperator{\cl}{cl}
\DeclareMathOperator{\cls}{\overline{\mathrm{span}}}
\DeclareMathOperator{\conv}{conv}
\DeclareMathOperator{\corr}{corr}
\DeclareMathOperator{\cosec}{cosec}
\DeclareMathOperator{\cosech}{cosech}
\DeclareMathOperator{\cov}{cov}
\DeclareMathOperator{\covol}{covol}
\DeclareMathOperator{\diag}{diag}
\DeclareMathOperator{\diam}{diam}
\DeclareMathOperator{\Diff}{Diff}
\DeclareMathOperator{\disc}{disc}
\DeclareMathOperator{\dom}{dom}
\DeclareMathOperator{\End}{End}
\DeclareMathOperator{\energy}{energy}
\DeclareMathOperator{\erfc}{erfc}
\DeclareMathOperator{\erf}{erf}
\DeclareMathOperator*{\esssup}{ess\,sup}
\DeclareMathOperator{\ev}{ev}
\DeclareMathOperator{\Ext}{Ext}
\DeclareMathOperator{\Fit}{Fit}
\DeclareMathOperator{\fix}{fix}
\DeclareMathOperator{\Frac}{Frac}
\DeclareMathOperator{\Gal}{Gal}
\DeclareMathOperator{\gammaD}{gamma}
\DeclareMathOperator{\gr}{gr}
\DeclareMathOperator{\hcf}{hcf}
\DeclareMathOperator{\Hom}{Hom}
\DeclareMathOperator{\id}{id}
\DeclareMathOperator{\image}{image}
\DeclareMathOperator{\im}{im}
\DeclareMathOperator{\Im}{Im}
\DeclareMathOperator{\Ind}{Ind}
\DeclareMathOperator{\Int}{Int}
\DeclareMathOperator{\Isom}{Isom}
\DeclareMathOperator{\lcm}{lcm}
\DeclareMathOperator{\length}{length}
\DeclareMathOperator{\Lie}{Lie}
\DeclareMathOperator{\like}{like}
\DeclareMathOperator{\Lk}{Lk}
\DeclareMathOperator{\mse}{mse}
\DeclareMathOperator{\multinomial}{multinomial}
\DeclareMathOperator{\orb}{orb}
\DeclareMathOperator{\ord}{ord}
\DeclareMathOperator{\otp}{otp}
\DeclareMathOperator{\Poisson}{Poisson}
\DeclareMathOperator{\poly}{poly}
\DeclareMathOperator{\rank}{rank}
\DeclareMathOperator{\rel}{rel}
\DeclareMathOperator{\Re}{Re}
\DeclareMathOperator*{\res}{res}
\DeclareMathOperator{\Res}{Res}
\DeclareMathOperator{\rk}{rk}
\DeclareMathOperator{\Root}{Root}
\DeclareMathOperator{\sech}{sech}
\DeclareMathOperator{\sgn}{sgn}
\DeclareMathOperator{\spn}{span}
\DeclareMathOperator{\stab}{stab}
\DeclareMathOperator{\St}{St}
\DeclareMathOperator{\supp}{supp}
\DeclareMathOperator{\Syl}{Syl}
\DeclareMathOperator{\Sym}{Sym}
\DeclareMathOperator{\tr}{tr}
\DeclareMathOperator{\Tr}{Tr}
\DeclareMathOperator{\var}{var}
\DeclareMathOperator{\vol}{vol}

\pgfarrowsdeclarecombine{twolatex'}{twolatex'}{latex'}{latex'}{latex'}{latex'}
\tikzset{->/.style = {decoration={markings,
                                  mark=at position 1 with {\arrow[scale=2]{latex'}}},
                      postaction={decorate}}}
\tikzset{<-/.style = {decoration={markings,
                                  mark=at position 0 with {\arrowreversed[scale=2]{latex'}}},
                      postaction={decorate}}}
\tikzset{<->/.style = {decoration={markings,
                                   mark=at position 0 with {\arrowreversed[scale=2]{latex'}},
                                   mark=at position 1 with {\arrow[scale=2]{latex'}}},
                       postaction={decorate}}}
\tikzset{->-/.style = {decoration={markings,
                                   mark=at position #1 with {\arrow[scale=2]{latex'}}},
                       postaction={decorate}}}
\tikzset{-<-/.style = {decoration={markings,
                                   mark=at position #1 with {\arrowreversed[scale=2]{latex'}}},
                       postaction={decorate}}}
\tikzset{->>/.style = {decoration={markings,
                                  mark=at position 1 with {\arrow[scale=2]{latex'}}},
                      postaction={decorate}}}
\tikzset{<<-/.style = {decoration={markings,
                                  mark=at position 0 with {\arrowreversed[scale=2]{twolatex'}}},
                      postaction={decorate}}}
\tikzset{<<->>/.style = {decoration={markings,
                                   mark=at position 0 with {\arrowreversed[scale=2]{twolatex'}},
                                   mark=at position 1 with {\arrow[scale=2]{twolatex'}}},
                       postaction={decorate}}}
\tikzset{->>-/.style = {decoration={markings,
                                   mark=at position #1 with {\arrow[scale=2]{twolatex'}}},
                       postaction={decorate}}}
\tikzset{-<<-/.style = {decoration={markings,
                                   mark=at position #1 with {\arrowreversed[scale=2]{twolatex'}}},
                       postaction={decorate}}}

\tikzset{circ/.style = {fill, circle, inner sep = 0, minimum size = 3}}
\tikzset{mstate/.style={circle, draw, blue, text=black, minimum width=0.7cm}}

\tikzset{commutative diagrams/.cd,cdmap/.style={/tikz/column 1/.append style={anchor=base east},/tikz/column 2/.append style={anchor=base west},row sep=tiny}}

\definecolor{mblue}{rgb}{0.2, 0.3, 0.8}
\definecolor{morange}{rgb}{1, 0.5, 0}
\definecolor{mgreen}{rgb}{0.1, 0.4, 0.2}
\definecolor{mred}{rgb}{0.5, 0, 0}

\def\drawcirculararc(#1,#2)(#3,#4)(#5,#6){%
    \pgfmathsetmacro\cA{(#1*#1+#2*#2-#3*#3-#4*#4)/2}%
    \pgfmathsetmacro\cB{(#1*#1+#2*#2-#5*#5-#6*#6)/2}%
    \pgfmathsetmacro\cy{(\cB*(#1-#3)-\cA*(#1-#5))/%
                        ((#2-#6)*(#1-#3)-(#2-#4)*(#1-#5))}%
    \pgfmathsetmacro\cx{(\cA-\cy*(#2-#4))/(#1-#3)}%
    \pgfmathsetmacro\cr{sqrt((#1-\cx)*(#1-\cx)+(#2-\cy)*(#2-\cy))}%
    \pgfmathsetmacro\cA{atan2(#2-\cy,#1-\cx)}%
    \pgfmathsetmacro\cB{atan2(#6-\cy,#5-\cx)}%
    \pgfmathparse{\cB<\cA}%
    \ifnum\pgfmathresult=1
        \pgfmathsetmacro\cB{\cB+360}%
    \fi
    \draw (#1,#2) arc (\cA:\cB:\cr);%
}
\newcommand\getCoord[3]{\newdimen{#1}\newdimen{#2}\pgfextractx{#1}{\pgfpointanchor{#3}{center}}\pgfextracty{#2}{\pgfpointanchor{#3}{center}}}

\def\Xint#1{\mathchoice
   {\XXint\displaystyle\textstyle{#1}}%
   {\XXint\textstyle\scriptstyle{#1}}%
   {\XXint\scriptstyle\scriptscriptstyle{#1}}%
   {\XXint\scriptscriptstyle\scriptscriptstyle{#1}}%
   \!\int}
\def\XXint#1#2#3{{\setbox0=\hbox{$#1{#2#3}{\int}$}
     \vcenter{\hbox{$#2#3$}}\kern-.5\wd0}}
\def\ddashint{\Xint=}
\def\dashint{\Xint-}

\newcommand\separator{{\centering\rule{2cm}{0.2pt}\vspace{2pt}\par}}

\newenvironment{own}{\color{gray!70!black}}{}

\newcommand\makecenter[1]{\raisebox{-0.5\height}{#1}}
\newtheorem*{soln}{Solution}

\renewcommand{\thesection}{}
\renewcommand{\thesubsection}{\arabic{section}.\arabic{subsection}}
\makeatletter
\def\@seccntformat#1{\csname #1ignore\expandafter\endcsname\csname the#1\endcsname\quad}
\let\sectionignore\@gobbletwo
\let\latex@numberline\numberline
\def\numberline#1{\if\relax#1\relax\else\latex@numberline{#1}\fi}
\makeatother


\begin{document}
	
\maketitle

\section{QUESTION 1}

\begin{itemize}
	\item We are given
	
	\[ \underbrace{\ddot{\theta} + 2p \dot{\theta} + (p^{2} + q^{2})}_{\mathcal{L}}\theta = f(t) \]
	
	with $ \theta(0) = \dot{\theta}(0) = 0 $, and $ p > 0, q \neq 0 $.
	
	Want to find $ G $ such that $ \mathcal{L} G = \delta(t - \tau) $, so that for each value of $ \tau $, the Green's function will solve the homogeneous equation $ LG = 0 $ whenever $ t \neq \tau $.
	
	We construct $ G $ for $ 0 \leq t < \tau $ as a general solution of the homogeneous equation, so that $ G = Ay_{1}(t) + B y_{2}(t) $. 
	Have
	
	\[ G(t,\tau) = \Theta(t - \tau) e^{-p(t-\tau)}[ C(\tau)\cos(q(t-\tau)) + D(\tau) \sin(q(t-\tau))   ] \]
	
	where $ \Theta $ is the Heaviside step function. Continuity demands $ G(\tau,\tau) = 0 $ so $ C(\tau) = 0 $. The jump condition (with $ \alpha(\tau) = 1 $) enforces $ D(\tau) = \frac{1}{q} $. Therefore the Green's function is
	
	\[ G(t,\tau) = \Theta(t - \tau) \frac{1}{q} e^{-p(t - \tau)}\sin(q(t - \tau)) \]
	
	and the general solution to $ \mathcal{L}\theta = f(t) $ obeying $ \theta(0) = \dot{\theta}(0) = 0 $ is 
	
	\[ \theta(t) = \frac{1}{q} \int_{0}^{t} e^{-p(t - \tau)}\sin(q(t - \tau)) f(\tau) \; \d \tau  \]
	
	\item Next we use Fourier Transforms. Taking the Fourier transform of the differential equation gives 
	
	\[ (i \omega)^{2} \tilde{\theta} + 2 i p \omega \tilde{\theta} + (p^{2} + q^{2}) \tilde{\theta} = \tilde{f} \]
	
	and so
	
	\[ \tilde{\theta} = \frac{\tilde{f}}{-\omega^{2} + 2 i p \omega + (p^{2} + q^{2}) } = : \tilde{R}(\omega) \tilde{f}(\omega) \]
	
	which solves the equation algebraically in Fourier space. Note that
	
	\begin{align*}
	\tilde{R}(\omega) & = \frac{1}{-\omega^{2} + 2 i p \omega + (p^{2} + q^{2}) } \\
	& = \frac{1}{(i \omega  + p)^{2} - (q i)^{2}  } \\
	& = \frac{1}{2qi} \left[  \frac{1}{ i \omega + p - qi } - \frac{1}{ i \omega  + p + qi }    \right] 
	\end{align*}
	
	To solve for $ \theta $ in real space we take the inverse Fourier transform to find 
	
	\begin{align*}
	\theta(t) & = \int_{0}^{t} R(t - u)f(u) \; \d u \\
	& = \int_{0}^{t} \left[  \frac{1}{2 \pi} \int_{- \infty}^{\infty} \tilde{R}(\omega) e^{i\omega(t-u)} \; \d \omega   \right] f(u) \; \d u \\
	& = \frac{1}{q} \int_{0}^{t} \underbrace{\left[  \frac{1}{4 i \pi}  \int_{- \infty}^{\infty}   e^{i\omega(t-u)}  \left[  \frac{1}{ i \omega + p - qi } - \frac{1}{ i \omega  + p + qi }    \right]  \; \d \omega   \right]}_{(*)} f(u) \; \d u 
	\end{align*}
	
	which agrees with the first result, if we can show that 
	
	\[ (*) = e^{-p(t-u)} \sin[ q(t - u) ] \]
	
	but not sure how to evaluate the integral.
	
	
\end{itemize}

\section{QUESTION 2}

The general homogeneous solution is $ c_{1} \sinh \lambda x + c_{2} \cosh \lambda x $ so we can take $ y_{1}(x) = \sinh \lambda x $ and $ y_{2}(x) = \sinh[\lambda(1-x)] $ as our homogeneous solutions satisfying the boundary conditions at $ x = 0 $ and $ x = 1 $ respectively. Then 

\[ G(x; \varepsilon) = \begin{cases} A(\varepsilon) \sinh( \lambda x)  & \text{ when } 0 \leq x  < \varepsilon \\  B(\varepsilon) \sinh[\lambda(1-x)]  & \text{ when } \varepsilon < x \leq 1 \end{cases} \]

Applying the continuity condition we get

\[ A \sinh \lambda \varepsilon = B \sinh[\lambda(1-\varepsilon)] \]
 
while the jump condition gives 

\[ B(- \lambda \cosh[\lambda(1-\varepsilon)] ) - A \lambda \cosh( \lambda \varepsilon) = -1 \]

Solving the for $ A $ and $ B $ gives us the Green's function

\[ G(x;\varepsilon) = - \frac{1}{ \lambda \sinh \lambda} \left[  \Theta( \varepsilon - x) \sinh[  \lambda(1 - \varepsilon) ] \sinh \lambda x  + \Theta(x - \varepsilon) \sinh(\varepsilon \lambda) \sinh[\lambda(1-x)]  \right]   \]

Using this Green's function we are immediately able to write down the complete solution as

\[ y = - \frac{1}{\lambda \sinh \lambda} \left\{  \sinh \lambda x \int_{x}^{1}f(\varepsilon) \sinh \lambda(1-\varepsilon) \; \d \varepsilon + \sinh \lambda(1-x)\int_{0}^{x} f(\varepsilon) \sinh \lambda \varepsilon \; \d \varepsilon \right\}  \]


\section{QUESTION 3}

Use the substitution $ y = z / x $, (quotient rule speeds thing up)

\begin{align*}
L_{x}y = - \frac{1}{x} \frac{\d^{2} z}{\d x^{2}} + \frac{z}{x} 
\end{align*}

So

\begin{align*}
y & = \frac{1}{x} \left(  c_{1} \cosh x + c_{2} \sinh x \right)  \\
y & = \frac{1}{x} \left( c_{1}'e^{x} + c_{2}'e^{-x} \right) 
\end{align*} 

For solutions that are (a) bounded as $ x \to 0 $, we have $ y = A \frac{1}{x} \sinh x $. For solutions that are (b) bounded as $ x \to \infty $, have $ y = B \frac{1}{x} e^{-x} $. 

Green's function satisfying $ L_{x} G = \delta(x - \xi) $ and both conditions (a), (b) is given by 

\[ G(x;xi) =  \begin{cases} A(\xi) \frac{\sinh x}{x}  & \text{ if } 0 < x < \xi \\ B(\xi) \frac{e^{-x} }{x}  & \text{ if } \xi < x \end{cases}   \]

Continuity $ \Rightarrow $

\[ A \sinh \xi  = B e^{-\xi}   \]

Jump ($ \alpha =-1 $) $  \Rightarrow $ 

\[ Be^{\xi}(\xi + 1) + A( \xi \cosh \xi - \sinh \xi) = \xi^{2}  \]

Solving,

\[ A = \xi e^{-\xi} \qquad B = \xi \sinh \xi  \]


Next, to solve 

\[ L_{x}y(x) = \begin{cases} 1  & \text{ if } 0 \leq x \leq R \\ 0 & \text{ if } x > R  \end{cases} \]

Using this $ G $,

\[ y = x^{-1}e^{-x} \int_{0}^{x} \xi \sinh \xi \; \d \xi + x^{-1} \sinh x \int_{x}^{R} \xi e^{-xi} \; \d \xi  \]

For $ 0 \leq x \leq R $, solve. (things combine into a sinh) For $ x > R $ note the RHS integral vanishes, so solve. 



\section{QUESTION 4}

Can use a similar argument to the one in lectures to show that $ G(t,\tau) = 0 $ whenever $ t \in [0,\tau] $.
Following the usual procedure we get

\[ G(t,\tau)  =  \Theta ( t - \tau)[A(\tau) \cos[k(t - \tau)] + B(\tau)\sin[k(t - \tau)] + C(\tau) (t - \tau) + D(\tau)   ]  \]

Continuity demands $ G(\tau,\tau) = G'(\tau,\tau) = G''(\tau,\tau) = 0 $, yielding

\begin{align*}
& A(\tau) + D(\tau) = 0 \\
& kB(\tau) + C(\tau) = 0\\
& A(\tau) = 0 
\end{align*}

respectively. The third equation also implies $ D(\tau) = 0 $, and the jump condition on $ G''' $ gives

\[  - k^{3} B(\tau) = 1  \]

$ \Rightarrow  B(\tau) = -k^{-3} \Rightarrow C(\tau) = k^{-2}  $, showing the Green's function is indeed what is required. 

Solution to $ \mathcal{L}y = e^{-t} $ is given by

\begin{align*}
y(t) & = \int_{0}^{t} [k^{-2} (t - \tau) - k^{-3} \sin k(t-\tau) ]e^{-\tau} \; \d \tau    \\
\end{align*}




\section{QUESTION 5}

Under the substitution $ y = \phi[x] $, hence $ \frac{\d x}{\d y} =  \frac{1}{| \phi'(x) |} $ (monotone increasing) the result becomes easy to show

\begin{align*}
\int_{a}^{b} f(x) \delta[\phi(x)]  \; \d x & = \int_{\phi(a)}^{\phi(b)} f(\phi^{-1}(y)) \frac{1}{| \phi'(\phi^{-1}(y)) |} \delta(y) \d y \\
& = f(\phi^{-1}(0)) \frac{1}{| \phi'(\phi^{-1}(0)) |} \\
& = \frac{f(c)}{| \phi'(c) |} \qquad \text{ as } \phi^{-1}(0) = c
\end{align*}

If monotone decreasing, $ \frac{\d x}{\d y} =  - \frac{1}{| \phi'(x) |} $, but $ \phi(b) < \phi(a) $, so $ \int_{\phi(a)}^{\phi(b)} = - \int_{\phi(b)}^{\phi(a)} $ and the result holds.

Hence for a general $ \phi(x) $ with simple zeros at $ c_{1},c_{2},c_{3}\cdots,c_{N} $ we have

\[ \int_{a}^{b} f(x) \delta[\phi(x)]  \; \d x = \sum_{i=1}^{N}  \frac{f(c_{i})}{| \phi'(c_{i}) |} \qquad (*) \]

Next for any $ a \in \R \setminus \{0 \} $

\[ \int_{\R} \delta(at) \phi(t) \; \d t = \frac{1}{| a |} \int_{\R}  \delta(y) \phi (y \ a) \; \d y = \frac{1}{| a |} \phi(0) \]

so we may write $ \delta(at) = \delta(t) / | a | $.

Lastly, we apply $ (*) $ for $ f(x) = | x | $ and $ \phi(x) = x^{2} - a^{2} $, which has simple zeros at $ x = a $ and $ x = -a $. Get


\begin{align*}
\int_{-\infty}^{\infty} | x | \delta( x^{2} - a^{2} ) \; \d x & = \frac{f(a)}{| \phi'(a) |} + \frac{f(-a)}{| \phi'(-a) |} \\
& = \frac{| a |}{2|  a |} + \frac{| a |}{2| a |} \\
& = 1
\end{align*}

as required





\section{QUESTION 6}
\section{QUESTION 7}

\begin{enumerate}
	\item $ f(x) = 1 $, $ | x | < c $.
	
	\begin{align*}
	\tilde{f}(k)  & = \int_{-c}^{c} e^{-ikx} \; \d x   \\
	& = \frac{i}{k} \left[    e^{-ikx}  \right]_{-c}^{c} \\
	& =  \frac{i}{k}(- 2 i \sin k c) \\
	& = \frac{2 \sin kc}{k}
	\end{align*}
	
	\item \[ \mathcal{F}[ e^{iax}f(x) ] = \tilde{f}(k-a) \]
	
	we have for $ f(x) = e^{iax} $ and using our previous answer,
	
	\[ \tilde{f}(k) = \frac{2\sin[(k-a)c]}{k-a} \]
	
	\item $ f(x) = \sin (ax) $
	
	Noting that this is the imaginary part of what we've just done (but not seeing how to make use of that... so stuck with calculating ) :
	
	No idiot, use $ \sin(ax) = \frac{1}{2i (e^{iax} - e^{-iax})} $ and linearity of Fourier Transform...
	
	\begin{align*}
	\tilde{f}(k)  & = \int_{-c}^{c} \sin(ax) e^{-ikx} \; \d x   \\
	& = \frac{i}{k} \left[ \sin(ax)e^{-ikx}  \right]_{-c}^{c} - \frac{ia}{k} \int_{-c}^{c} \cos(ax)e^{-ikx} \; \d x  \\
	& = \frac{i}{k} 2\sin(ax)\cos(kc) - \frac{ia}{k} \left(   \frac{i}{k} \left[ \cos(ax)e^{-ikx}  \right]_{-c}^{c} + \frac{ia}{k} \int_{-c}^{c} \sin(ax)e^{-ikx} \; \d x \right) \\
	& = \frac{i}{k} 2\sin(ax)\cos(kc) + \frac{a}{k^{2}} \left(  -2\cos(ax)\sin(kc) + a \tilde{f}(k) \right) 
	\end{align*}
	
	Thus multiplying upon rearranging, we get
	
	\begin{align*}
	\tilde{f}(k) & = \frac{2i \left(  - k \cos[kc]\sin[ac] + a \cos[ac]\sin[kc]   \right)}{a^{2} - k^{2}} \\
	\end{align*}
	
	\item Next, by the differentiating property of Fourier Transforms, know that
	
	\[ \mathcal{F}[a\cos(ax)] = ik\tilde{f}(k) \]
	
	where $ \tilde{f}(k) $ is the FT of $ f(x) = \sin(ax) $. Then, by scaling,
	
	\[ \mathcal{F}[\cos(ax)] = | a | ik\tilde{f}(k) \]
	
	Thus the Fourier transform of $ \cos(ax) $ is given by
	
	\[ \tilde{f}(k) = \frac{-2ka \left(  - k \cos[kc]\sin[ac] + a \cos[ac]\sin[kc]   \right)}{a^{2} - k^{2}}  \]
	
	
	
\end{enumerate}

\section{QUESTION 8}

Have that

\[ f(x) = \begin{cases} 0  & \text{ if } x < 0 \\ e^{-x} & \text{ if } x > 0 \end{cases} \]

The Fourier transform is given by 

\begin{align*}
\tilde{f}(k) & = \int_{0}^{\infty} e^{-(ik + 1)x} \; \d x   \\
& = \frac{1}{-ik - 1}\left[  e^{-(ik + 1)x} \right]_{0}^{\infty} \\
& = \frac{1}{1 + ik } \\
& = \frac{1 - ik}{1 + k^{2}}
 \end{align*}
 
The inverse Fourier transform is given by

\[ f(x) = \frac{1}{2\pi} \int_{-\infty}^{\infty} e^{ikx} \tilde{f}(k) \; \d k  \]

Thus

\begin{align*}
f(0) & = \frac{1}{2\pi} \int_{-\infty}^{\infty}\frac{1 - ik}{1 + k^{2}} \; \d k   \\
& = \frac{1}{2\pi} \int_{-\infty}^{\infty} \frac{1}{1 + k^{2}} \; \d k  -  \frac{i}{2\pi} \int_{-\infty}^{\infty}  \frac{k}{1 + k^{2}} \; \d k   \\
& = \frac{1}{2\pi} \left[  \arctan k \right]_{-\infty}^{\infty} - \frac{i}{4\pi} \underbrace{\left[  \log (1 + k^{2}) \right]_{-\infty}^{\infty}}_{=0}  \\
& = \frac{1}{2\pi} \left(  \frac{\pi}{2} + \frac{\pi}{2} \right) \\
& = \frac{1}{2}
\end{align*}
Wait. Note that 

\section{QUESTION 9}

This is easy. Define $ g(x) = e^{-x^{2}} $, calculate $ \tilde{g}(k) $, then do rephasing and scaling in any order. Is good practice. Should get

\[ \tilde{f}(k) = e^{-ik\mu} \frac{1}{n} \sqrt{\pi} e^{-k^{2}/4 n^{2}} \]


\begin{align*}
f(x) & = e^{-n^{2}(x - \mu)^{2}} \\
f(x + \mu) & = e^{-n^{2}x^{2}} \\
f'(x +  \mu) & = -2n^{2} xe^{-n^{2}x^{2}}  
\end{align*}

Calculating the Fourier Transform of the last function gives

\begin{align*}
\tilde{f}(k) & = -2n^{2} \int_{-\infty}^{\infty}  xe^{-n^{2}x^{2}}   e^{-ikx} \; \d x  \\
& = 
\end{align*}

\section{QUESTION 10}


Let us suppose we have $ N $ measurements of a function $ h(t) $, where $ N $ is even, with constant sampling interval $ \Delta $, ie. we have the set of measurements

\[ h_{m} = h(t_{m}), \quad t_{m} = m \Delta, \; m = 0,1,\cdots,N-1 \]

Parseval's theorem for DFT is

\[ \sum_{m=0}^{N-1} | h(t_{m}) |^{2} = \frac{1}{N} \sum_{n=0}^{N-1} | \tilde{h}_{d}(f_{n}) |^{2} \qquad (*) \]

where $ h_{d} $ is the \emph{Discrete Fourier transform} 

\[ \tilde{h}_{d}(f_{n}) = \sum_{n=0}^{N-1} h_{m} \exp\left[ \frac{-2\pi i}{N} mn \right]   \]

and 

\[ f_{n} = \frac{n}{N \Delta}, n = -N / 2,\cdots,N/2,  \]

To prove (*), consider a fixed $ h(t_{m}) $ on the LSH. Appyling the inversion formula, and making a Riemann approximation to the integral, we obtain

\begin{align*}
h(t_{m}) & = \frac{1}{2\pi} \int_{-\infty}^{\infty} e^{i \omega t_{m}} \tilde{h}(\omega) \; \d \omega  \\
& = \int_{-\infty}^{\infty} e^{2 \pi i f t_{m}} \hbar(f) \; \d f  \qquad  ( 2\pi f = \omega)\\
& \approx \frac{\Delta}{\Delta N} \sum_{n=0}^{N-1} \tilde{h}_{d} (f_{n}) e^{2 \pi i f_{n} t_{m}} \\
& = \frac{1}{N} \sum_{n=0}^{N-1} \tilde{h}_{d}(f_{n}) \exp \left[  \frac{2 \pi i}{N} mn \right] \\
& = \frac{1}{N} \sum_{n=0}^{N-1} \tilde{h}_{d}(f_{n}) \kappa^{mn}
\end{align*}

where $ \kappa = e^{2\pi i / N} $ is an $ N $th root of unity.

Now

\begin{align*}
| h(t_{m}) |^{2} = (h(t_{m}))^{*} h(t_{m}) & = \frac{1}{N^{2}} \sum_{p,q = 0}^{N - 1}  \tilde{h}_{d}^{*}(f_{p}) \kappa^{-mp}  \tilde{h}_{d}(f_{q}) \kappa^{mq}   \\
& = \frac{1}{N^{2}} \sum_{p,q = 0}^{N - 1}  \tilde{h}_{d}^{*}(f_{p})	 \tilde{h}_{d}(f_{q})  \delta_{pq} \\
& = \frac{1}{N^{2}} \sum_{p = 0}^{N - 1}  \tilde{h}_{d}^{*}(f_{p})	 \tilde{h}_{d}(f_{p}) \\
& = \frac{1}{N} \sum_{p=0}^{N-1} | h_{d}(f_{p}) |^{2} 
\end{align*}

noting the independence from $ m $. Hence, summing over $ N $ on both sides gives $ (*) $ as required, as the right hand side just gains a factor of $ N $.

\section{QUESTION 11}


	Thus the Fourier transform of $ \cos(x) $ (Q7 (iv)) is given by

\begin{align*}
\tilde{f}(k) & = \frac{-2k \left(  - k \cos[k \frac{\pi}{2}]\sin[ \frac{\pi}{2} ] + \cos[ \frac{\pi}{2}]\sin[k \frac{\pi}{2}]   \right)}{1 - k^{2}}  \\
& =  \frac{2}{1-k^{2}} \cos \left(    \frac{k \pi}{2} \right) 
\end{align*}

And the Fourier transform of the derivative $ -\sin(x) $ is given by

\[ \tilde{f}(k) = - i \frac{2k}{1-k^{2}} \cos \left(   \frac{k \pi}{2} \right) \]





\section{QUESTION 12}

Inverse Fourier transform of $ \tilde{f}(k) $ is

\begin{align*}
f(x) & =  \frac{1}{2\pi} \int_{-1}^{1} e^{ikx} e^{k} - e^{-k} \; \d k \\
& = \frac{1}{\pi} \int_{-1}^{1} e^{ikx} \sinh k \; \d k \\
& = \frac{1}{\pi} \left[ - \frac{i}{x} e^{ikx} \sinh k   \right]_{-1}^{1} + \frac{i}{x\pi} \int_{-1}^{1} e^{ikx} \cosh k \; \d k  \\
& = \frac{-i}{\pi x} \left(  e^{ix}\sinh 1 + e^{-ix}\sinh 1   \right) + \frac{i}{x \pi} \left( \left[ - \frac{i}{x} e^{ikx} \cosh k   \right]_{-1}^{1} + \frac{i}{x} \int_{-1}^{1} e^{ikx} \sinh k \; \d k \right)  \\
& =  \frac{-2i}{\pi x} \left( \cos x \sinh 1   \right) + \frac{i}{x \pi} \left( - \frac{2i}{x} i \sin x \cosh 1 + \frac{i}{x} f(x) \right)  \\
\Rightarrow \pi x^{2} f(x) & = -2i x \cos x \sinh 1 + 2 i \sin x \cosh 1 - f(x)
\end{align*}

Hence rearranging gives

\[ f(x) = \frac{2i}{\pi(1+x^{2})} (  \cosh 1 \sin x - x \cos x \sinh 1 ) \]

(a long algebraic exercise. good for me?)

Next, take the Fourier transform of $ \nabla^{2}(x,y) = 0 $ with respect to $ x $;

\[ \int_{-\infty}^{\infty} \left(  \frac{\partial^{2} \phi }{\partial x^{2}} + \frac{\partial^{2} \phi }{\partial y^{2}} \right) e^{-ikx} \; \d x = 0  \]

\begin{align*}
\Rightarrow 0 & = \int_{-\infty}^{\infty}  \frac{\partial^{2} \phi }{\partial x^{2}}  e^{-ikx} \; \d x  + \int_{-\infty}^{\infty} \frac{\partial^{2} \phi }{\partial y^{2}} e^{-ikx} \; \d x\\
& = - k^{2} \tilde{\phi}(k) + \frac{\partial^{2} }{\partial y^{2}} \int_{-\infty}^{\infty}  \phi e^{- i k x} \; \d x\\
& = - k^{2} \tilde{\phi}(k) +  \frac{\partial^{2} \tilde{\phi}(k) }{\partial y^{2}}
\end{align*}

Solving this second order ODE in $ k $ gives

\[ \tilde{\phi}(k,y) = A(k) e^{ky} + B(k) e^{-ky}  \]

We have the boundary conditions $ \tilde{\phi}(k,0) = \tilde{f}(k) $, and $ \tilde{\phi}(k,1) = 0 $. This gives

\begin{align*}
A + B  & = e^{k} - e^{-k} \\
Ae^{k} + Be^{-k}  & = 0 
\end{align*}

respectively. Solving gives $  A = - e^{-k}, B = e^{k} $. Thus, we take the inverse FT to finish.

\begin{align*}
\phi(x,y) &  = - \frac{1}{2\pi}  \int_{-1}^{1} e^{k(y-1)} - e^{-k(y-1)} \; \d k \\
\end{align*}

Do int by substitution, should work out satisfying the bcs like you want!



\end{document}