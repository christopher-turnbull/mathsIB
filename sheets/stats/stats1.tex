\documentclass[a4paper]{article}
\usepackage{amsmath}
\def\npart {IB}
\def\nterm {Lent}
\def\nyear {2018}
\def\nlecturer {Mr. Higson}
\def\ncourse {Statistics Example Sheet 1}

% Imports
\ifx \nauthor\undefined
  \def\nauthor{Christopher Turnbull}
\else
\fi

\author{Supervised by \nlecturer \\\small Solutions presented by \nauthor}
\date{\nterm\ \nyear}

\usepackage{alltt}
\usepackage{amsfonts}
\usepackage{amsmath}
\usepackage{amssymb}
\usepackage{amsthm}
\usepackage{booktabs}
\usepackage{caption}
\usepackage{enumitem}
\usepackage{fancyhdr}
\usepackage{graphicx}
\usepackage{mathdots}
\usepackage{mathtools}
\usepackage{microtype}
\usepackage{multirow}
\usepackage{pdflscape}
\usepackage{pgfplots}
\usepackage{siunitx}
\usepackage{slashed}
\usepackage{tabularx}
\usepackage{tikz}
\usepackage{tkz-euclide}
\usepackage[normalem]{ulem}
\usepackage[all]{xy}
\usepackage{imakeidx}

\makeindex[intoc, title=Index]
\indexsetup{othercode={\lhead{\emph{Index}}}}

\ifx \nextra \undefined
  \usepackage[pdftex,
    hidelinks,
    pdfauthor={Christopher Turnbull},
    pdfsubject={Cambridge Maths Notes: Part \npart\ - \ncourse},
    pdftitle={Part \npart\ - \ncourse},
  pdfkeywords={Cambridge Mathematics Maths Math \npart\ \nterm\ \nyear\ \ncourse}]{hyperref}
  \title{Part \npart\ --- \ncourse}
\else
  \usepackage[pdftex,
    hidelinks,
    pdfauthor={Christopher Turnbull},
    pdfsubject={Cambridge Maths Notes: Part \npart\ - \ncourse\ (\nextra)},
    pdftitle={Part \npart\ - \ncourse\ (\nextra)},
  pdfkeywords={Cambridge Mathematics Maths Math \npart\ \nterm\ \nyear\ \ncourse\ \nextra}]{hyperref}

  \title{Part \npart\ --- \ncourse \\ {\Large \nextra}}
  \renewcommand\printindex{}
\fi

\pgfplotsset{compat=1.12}

\pagestyle{fancyplain}
\lhead{\emph{\nouppercase{\leftmark}}}
\ifx \nextra \undefined
  \rhead{
    \ifnum\thepage=1
    \else
      \npart\ \ncourse
    \fi}
\else
  \rhead{
    \ifnum\thepage=1
    \else
      \npart\ \ncourse\ (\nextra)
    \fi}
\fi
\usetikzlibrary{arrows.meta}
\usetikzlibrary{decorations.markings}
\usetikzlibrary{decorations.pathmorphing}
\usetikzlibrary{positioning}
\usetikzlibrary{fadings}
\usetikzlibrary{intersections}
\usetikzlibrary{cd}

\newcommand*{\Cdot}{{\raisebox{-0.25ex}{\scalebox{1.5}{$\cdot$}}}}
\newcommand {\pd}[2][ ]{
  \ifx #1 { }
    \frac{\partial}{\partial #2}
  \else
    \frac{\partial^{#1}}{\partial #2^{#1}}
  \fi
}
\ifx \nhtml \undefined
\else
  \renewcommand\printindex{}
  \makeatletter
  \DisableLigatures[f]{family = *}
  \let\Contentsline\contentsline
  \renewcommand\contentsline[3]{\Contentsline{#1}{#2}{}}
  \renewcommand{\@dotsep}{10000}
  \newlength\currentparindent
  \setlength\currentparindent\parindent

  \newcommand\@minipagerestore{\setlength{\parindent}{\currentparindent}}
  \usepackage[active,tightpage,pdftex]{preview}
  \renewcommand{\PreviewBorder}{0.1cm}

  \newenvironment{stretchpage}%
  {\begin{preview}\begin{minipage}{\hsize}}%
    {\end{minipage}\end{preview}}
  \AtBeginDocument{\begin{stretchpage}}
  \AtEndDocument{\end{stretchpage}}

  \newcommand{\@@newpage}{\end{stretchpage}\begin{stretchpage}}

  \let\@real@section\section
  \renewcommand{\section}{\@@newpage\@real@section}
  \let\@real@subsection\subsection
  \renewcommand{\subsection}{\@@newpage\@real@subsection}
  \makeatother
\fi

% Theorems
\theoremstyle{definition}
\newtheorem*{aim}{Aim}
\newtheorem*{axiom}{Axiom}
\newtheorem*{claim}{Claim}
\newtheorem*{cor}{Corollary}
\newtheorem*{conjecture}{Conjecture}
\newtheorem*{defi}{Definition}
\newtheorem*{eg}{Example}
\newtheorem*{ex}{Exercise}
\newtheorem*{fact}{Fact}
\newtheorem*{law}{Law}
\newtheorem*{lemma}{Lemma}
\newtheorem*{notation}{Notation}
\newtheorem*{prop}{Proposition}
\newtheorem*{soln}{Solution}
\newtheorem*{thm}{Theorem}

\newtheorem*{remark}{Remark}
\newtheorem*{warning}{Warning}
\newtheorem*{exercise}{Exercise}

\newtheorem{nthm}{Theorem}[section]
\newtheorem{nlemma}[nthm]{Lemma}
\newtheorem{nprop}[nthm]{Proposition}
\newtheorem{ncor}[nthm]{Corollary}


\renewcommand{\labelitemi}{--}
\renewcommand{\labelitemii}{$\circ$}
\renewcommand{\labelenumi}{(\roman{*})}

\let\stdsection\section
\renewcommand\section{\newpage\stdsection}

% Strike through
\def\st{\bgroup \ULdepth=-.55ex \ULset}

% Maths symbols
\newcommand{\abs}[1]{\left\lvert #1\right\rvert}
\newcommand\ad{\mathrm{ad}}
\newcommand\AND{\mathsf{AND}}
\newcommand\Art{\mathrm{Art}}
\newcommand{\Bilin}{\mathrm{Bilin}}
\newcommand{\bket}[1]{\left\lvert #1\right\rangle}
\newcommand{\B}{\mathcal{B}}
\newcommand{\bolds}[1]{{\bfseries #1}}
\newcommand{\brak}[1]{\left\langle #1 \right\rvert}
\newcommand{\braket}[2]{\left\langle #1\middle\vert #2 \right\rangle}
\newcommand{\bra}{\langle}
\newcommand{\cat}[1]{\mathsf{#1}}
\newcommand{\C}{\mathbb{C}}
\newcommand{\CP}{\mathbb{CP}}
\newcommand{\cU}{\mathcal{U}}
\newcommand{\Der}{\mathrm{Der}}
\newcommand{\D}{\mathrm{D}}
\newcommand{\dR}{\mathrm{dR}}
\newcommand{\E}{\mathbb{E}}
\newcommand{\F}{\mathbb{F}}
\newcommand{\Frob}{\mathrm{Frob}}
\newcommand{\GG}{\mathbb{G}}
\newcommand{\gl}{\mathfrak{gl}}
\newcommand{\GL}{\mathrm{GL}}
\newcommand{\G}{\mathcal{G}}
\newcommand{\Gr}{\mathrm{Gr}}
\newcommand{\haut}{\mathrm{ht}}
\newcommand{\Id}{\mathrm{Id}}
\newcommand{\ket}{\rangle}
\newcommand{\lie}[1]{\mathfrak{#1}}
\newcommand{\Mat}{\mathrm{Mat}}
\newcommand{\N}{\mathbb{N}}
\newcommand{\norm}[1]{\left\lVert #1\right\rVert}
\newcommand{\normalorder}[1]{\mathop{:}\nolimits\!#1\!\mathop{:}\nolimits}
\newcommand\NOT{\mathsf{NOT}}
\newcommand{\Oc}{\mathcal{O}}
\newcommand{\Or}{\mathrm{O}}
\newcommand\OR{\mathsf{OR}}
\newcommand{\ort}{\mathfrak{o}}
\newcommand{\PGL}{\mathrm{PGL}}
\newcommand{\ph}{\,\cdot\,}
\newcommand{\pr}{\mathrm{pr}}
\newcommand{\Prob}{\mathbb{P}}
\newcommand{\PSL}{\mathrm{PSL}}
\newcommand{\Ps}{\mathcal{P}}
\newcommand{\PSU}{\mathrm{PSU}}
\newcommand{\pt}{\mathrm{pt}}
\newcommand{\qeq}{\mathrel{``{=}"}}
\newcommand{\Q}{\mathbb{Q}}
\newcommand{\R}{\mathbb{R}}
\newcommand{\RP}{\mathbb{RP}}
\newcommand{\Rs}{\mathcal{R}}
\newcommand{\SL}{\mathrm{SL}}
\newcommand{\so}{\mathfrak{so}}
\newcommand{\SO}{\mathrm{SO}}
\newcommand{\Spin}{\mathrm{Spin}}
\newcommand{\Sp}{\mathrm{Sp}}
\newcommand{\su}{\mathfrak{su}}
\newcommand{\SU}{\mathrm{SU}}
\newcommand{\term}[1]{\emph{#1}\index{#1}}
\newcommand{\T}{\mathbb{T}}
\newcommand{\tv}[1]{|#1|}
\newcommand{\U}{\mathrm{U}}
\newcommand{\uu}{\mathfrak{u}}
\newcommand{\Vect}{\mathrm{Vect}}
\newcommand{\wsto}{\stackrel{\mathrm{w}^*}{\to}}
\newcommand{\wt}{\mathrm{wt}}
\newcommand{\wto}{\stackrel{\mathrm{w}}{\to}}
\newcommand{\Z}{\mathbb{Z}}
\renewcommand{\d}{\mathrm{d}}
\renewcommand{\H}{\mathbb{H}}
\renewcommand{\P}{\mathbb{P}}
\renewcommand{\sl}{\mathfrak{sl}}
\renewcommand{\vec}[1]{\boldsymbol{\mathbf{#1}}}
%\renewcommand{\F}{\mathcal{F}}

\let\Im\relax
\let\Re\relax

\DeclareMathOperator{\adj}{adj}
\DeclareMathOperator{\Ann}{Ann}
\DeclareMathOperator{\area}{area}
\DeclareMathOperator{\Aut}{Aut}
\DeclareMathOperator{\Bernoulli}{Bernoulli}
\DeclareMathOperator{\betaD}{beta}
\DeclareMathOperator{\bias}{bias}
\DeclareMathOperator{\binomial}{binomial}
\DeclareMathOperator{\card}{card}
\DeclareMathOperator{\ccl}{ccl}
\DeclareMathOperator{\Char}{char}
\DeclareMathOperator{\ch}{ch}
\DeclareMathOperator{\cl}{cl}
\DeclareMathOperator{\cls}{\overline{\mathrm{span}}}
\DeclareMathOperator{\conv}{conv}
\DeclareMathOperator{\corr}{corr}
\DeclareMathOperator{\cosec}{cosec}
\DeclareMathOperator{\cosech}{cosech}
\DeclareMathOperator{\cov}{cov}
\DeclareMathOperator{\covol}{covol}
\DeclareMathOperator{\diag}{diag}
\DeclareMathOperator{\diam}{diam}
\DeclareMathOperator{\Diff}{Diff}
\DeclareMathOperator{\disc}{disc}
\DeclareMathOperator{\dom}{dom}
\DeclareMathOperator{\End}{End}
\DeclareMathOperator{\energy}{energy}
\DeclareMathOperator{\erfc}{erfc}
\DeclareMathOperator{\erf}{erf}
\DeclareMathOperator*{\esssup}{ess\,sup}
\DeclareMathOperator{\ev}{ev}
\DeclareMathOperator{\Ext}{Ext}
\DeclareMathOperator{\Fit}{Fit}
\DeclareMathOperator{\fix}{fix}
\DeclareMathOperator{\Frac}{Frac}
\DeclareMathOperator{\Gal}{Gal}
\DeclareMathOperator{\gammaD}{gamma}
\DeclareMathOperator{\gr}{gr}
\DeclareMathOperator{\hcf}{hcf}
\DeclareMathOperator{\Hom}{Hom}
\DeclareMathOperator{\id}{id}
\DeclareMathOperator{\image}{image}
\DeclareMathOperator{\im}{im}
\DeclareMathOperator{\Im}{Im}
\DeclareMathOperator{\Ind}{Ind}
\DeclareMathOperator{\Int}{Int}
\DeclareMathOperator{\Isom}{Isom}
\DeclareMathOperator{\lcm}{lcm}
\DeclareMathOperator{\length}{length}
\DeclareMathOperator{\Lie}{Lie}
\DeclareMathOperator{\like}{like}
\DeclareMathOperator{\Lk}{Lk}
\DeclareMathOperator{\mse}{mse}
\DeclareMathOperator{\multinomial}{multinomial}
\DeclareMathOperator{\orb}{orb}
\DeclareMathOperator{\ord}{ord}
\DeclareMathOperator{\otp}{otp}
\DeclareMathOperator{\Poisson}{Poisson}
\DeclareMathOperator{\poly}{poly}
\DeclareMathOperator{\rank}{rank}
\DeclareMathOperator{\rel}{rel}
\DeclareMathOperator{\Re}{Re}
\DeclareMathOperator*{\res}{res}
\DeclareMathOperator{\Res}{Res}
\DeclareMathOperator{\rk}{rk}
\DeclareMathOperator{\Root}{Root}
\DeclareMathOperator{\sech}{sech}
\DeclareMathOperator{\sgn}{sgn}
\DeclareMathOperator{\spn}{span}
\DeclareMathOperator{\stab}{stab}
\DeclareMathOperator{\St}{St}
\DeclareMathOperator{\supp}{supp}
\DeclareMathOperator{\Syl}{Syl}
\DeclareMathOperator{\Sym}{Sym}
\DeclareMathOperator{\tr}{tr}
\DeclareMathOperator{\Tr}{Tr}
\DeclareMathOperator{\var}{var}
\DeclareMathOperator{\vol}{vol}

\pgfarrowsdeclarecombine{twolatex'}{twolatex'}{latex'}{latex'}{latex'}{latex'}
\tikzset{->/.style = {decoration={markings,
                                  mark=at position 1 with {\arrow[scale=2]{latex'}}},
                      postaction={decorate}}}
\tikzset{<-/.style = {decoration={markings,
                                  mark=at position 0 with {\arrowreversed[scale=2]{latex'}}},
                      postaction={decorate}}}
\tikzset{<->/.style = {decoration={markings,
                                   mark=at position 0 with {\arrowreversed[scale=2]{latex'}},
                                   mark=at position 1 with {\arrow[scale=2]{latex'}}},
                       postaction={decorate}}}
\tikzset{->-/.style = {decoration={markings,
                                   mark=at position #1 with {\arrow[scale=2]{latex'}}},
                       postaction={decorate}}}
\tikzset{-<-/.style = {decoration={markings,
                                   mark=at position #1 with {\arrowreversed[scale=2]{latex'}}},
                       postaction={decorate}}}
\tikzset{->>/.style = {decoration={markings,
                                  mark=at position 1 with {\arrow[scale=2]{latex'}}},
                      postaction={decorate}}}
\tikzset{<<-/.style = {decoration={markings,
                                  mark=at position 0 with {\arrowreversed[scale=2]{twolatex'}}},
                      postaction={decorate}}}
\tikzset{<<->>/.style = {decoration={markings,
                                   mark=at position 0 with {\arrowreversed[scale=2]{twolatex'}},
                                   mark=at position 1 with {\arrow[scale=2]{twolatex'}}},
                       postaction={decorate}}}
\tikzset{->>-/.style = {decoration={markings,
                                   mark=at position #1 with {\arrow[scale=2]{twolatex'}}},
                       postaction={decorate}}}
\tikzset{-<<-/.style = {decoration={markings,
                                   mark=at position #1 with {\arrowreversed[scale=2]{twolatex'}}},
                       postaction={decorate}}}

\tikzset{circ/.style = {fill, circle, inner sep = 0, minimum size = 3}}
\tikzset{mstate/.style={circle, draw, blue, text=black, minimum width=0.7cm}}

\tikzset{commutative diagrams/.cd,cdmap/.style={/tikz/column 1/.append style={anchor=base east},/tikz/column 2/.append style={anchor=base west},row sep=tiny}}

\definecolor{mblue}{rgb}{0.2, 0.3, 0.8}
\definecolor{morange}{rgb}{1, 0.5, 0}
\definecolor{mgreen}{rgb}{0.1, 0.4, 0.2}
\definecolor{mred}{rgb}{0.5, 0, 0}

\def\drawcirculararc(#1,#2)(#3,#4)(#5,#6){%
    \pgfmathsetmacro\cA{(#1*#1+#2*#2-#3*#3-#4*#4)/2}%
    \pgfmathsetmacro\cB{(#1*#1+#2*#2-#5*#5-#6*#6)/2}%
    \pgfmathsetmacro\cy{(\cB*(#1-#3)-\cA*(#1-#5))/%
                        ((#2-#6)*(#1-#3)-(#2-#4)*(#1-#5))}%
    \pgfmathsetmacro\cx{(\cA-\cy*(#2-#4))/(#1-#3)}%
    \pgfmathsetmacro\cr{sqrt((#1-\cx)*(#1-\cx)+(#2-\cy)*(#2-\cy))}%
    \pgfmathsetmacro\cA{atan2(#2-\cy,#1-\cx)}%
    \pgfmathsetmacro\cB{atan2(#6-\cy,#5-\cx)}%
    \pgfmathparse{\cB<\cA}%
    \ifnum\pgfmathresult=1
        \pgfmathsetmacro\cB{\cB+360}%
    \fi
    \draw (#1,#2) arc (\cA:\cB:\cr);%
}
\newcommand\getCoord[3]{\newdimen{#1}\newdimen{#2}\pgfextractx{#1}{\pgfpointanchor{#3}{center}}\pgfextracty{#2}{\pgfpointanchor{#3}{center}}}

\def\Xint#1{\mathchoice
   {\XXint\displaystyle\textstyle{#1}}%
   {\XXint\textstyle\scriptstyle{#1}}%
   {\XXint\scriptstyle\scriptscriptstyle{#1}}%
   {\XXint\scriptscriptstyle\scriptscriptstyle{#1}}%
   \!\int}
\def\XXint#1#2#3{{\setbox0=\hbox{$#1{#2#3}{\int}$}
     \vcenter{\hbox{$#2#3$}}\kern-.5\wd0}}
\def\ddashint{\Xint=}
\def\dashint{\Xint-}

\newcommand\separator{{\centering\rule{2cm}{0.2pt}\vspace{2pt}\par}}

\newenvironment{own}{\color{gray!70!black}}{}

\newcommand\makecenter[1]{\raisebox{-0.5\height}{#1}}

\newtheorem*{soln}{Solution}

\renewcommand{\thesection}{}
\renewcommand{\thesubsection}{\arabic{section}.\arabic{subsection}}
\makeatletter
\def\@seccntformat#1{\csname #1ignore\expandafter\endcsname\csname the#1\endcsname\quad}
\let\sectionignore\@gobbletwo
\let\latex@numberline\numberline
\def\numberline#1{\if\relax#1\relax\else\latex@numberline{#1}\fi}
\makeatother


\begin{document}
	
\maketitle

%Lectured by Richard Samworth


\section{QUESTION 2}

If $ X \sim \text{Exp}(\lambda), Y \sim \text{Exp}(\mu)  $, $ X,Y $ independent, we can derive the standard result that the minimum of exponentials is exponential:
	\begin{align*}
	\P (\min[X,Y] < t) & =  1 - \P (\min[X,Y] \geq t) \\
	& = 1 - \int_{0}^{\infty} \int_{0}^{\infty} I(\lambda e^{-\lambda x_{1}} \geq t, \mu e^{-\mu x_{2}} \geq t ) \; \d x_{2} \d x_{1} \\
	& = 1 - \int_{t}^{\infty} \lambda e^{-\lambda x_{1}} \; \d x_{1} \int_{t}^{\infty} \mu e^{-\mu x_{2}} \; \d x_{2} \\
	& = 1 - e^{-(\lambda + \mu) t}, \; \text{i.e.} \min[X,Y] \sim \text{Exp}(\lambda + \mu)
	\end{align*}
	
	
 


	Next, suppose $ X \sim \Gamma(\alpha,\lambda) $, $ Y \sim \Gamma(\beta,\lambda)  $. We want to find the joint PDF of
	
	\[ U = X + Y \text{, \quad and \quad } V = X / (X + Y) \]
	
	Consider the map 
	
	\[ T: (x,y) \mapsto (u,v), \text{ \; where \;} u = x + y, \; v = \frac{x}{x + y} \]
	
	where $ x,y,u \geq 0 $, $ 0 \leq v \leq 1 $ The inverse map $ T^{-1} $ acts by
	
	\[ T^{-1}: (u,v) \mapsto (x,y), \text{ where } x = uv, \; y = u(1-v) \]
	
	and has the Jacobian
	
	\begin{align*}
	J(u,v) & = \det \begin{pmatrix}
	v & u \\
	1 - v & -u 
	\end{pmatrix}  \\
	& = -u
	\end{align*}
	
	Then the joint PDF
	
	\[ f_{U,V}(u,v) = f_{X,Y}(uv,u(1-v))\left| -u \right|  \]
	
	Substituting in $ f_{X,Y}(x,y) = \frac{\lambda^{\alpha} x^{\alpha - 1} e^{-\lambda x}}{\Gamma(\alpha)} \frac{\lambda^{\beta} y^{\beta - 1} e^{-\lambda y}}{\Gamma(\beta)}, x,y \geq 0  $, yields
	
	\begin{align*}
	f_{U,V}(u,v) & = \frac{\lambda^{\alpha + \beta}}{\Gamma(\alpha) \Gamma(\beta)} (uv)^{\alpha - 1}(u(1-v))^{\beta - 1}  u e^{-\lambda u}, \; u \geq 0, \;  0 \leq v \leq 1 \\
	& = \frac{\lambda^{\alpha + \beta}}{\Gamma(\alpha) \Gamma(\beta)} u^{\alpha + \beta - 1} v^{\alpha - 1} (1-v)^{\beta - 1} e^{-\lambda u} \\
	& = \text{Beta}(v ; \alpha,\beta) \frac{\lambda^{\alpha + \beta}}{\Gamma(a+b)} u^{\alpha + \beta - 1} e^{-\lambda u} \\
	& = \text{Beta}(v ; \alpha,\beta) \text{Gamma}(u ; \alpha + \beta)
	\end{align*}
	
	This factorises, so the respective marginal PDFs are
	
	\[ f_{U}(u) = \text{Gamma}(u ; \alpha + \beta), \quad f_{V}(v) = \text{Beta}(v ; \alpha,\beta)  \]

	
	



\section{QUESTION 3}

The factorization criterion states that a statistic $ T = t(\mathbf{x}) $ is sufficient for $ \theta $ iff 

\[ f_{\mathbf{X}}(\mathbf{x} \; ; \theta) = g(t(\mathbf{x}), \theta) h(\mathbf{x}) \]

We have proved the discrete case in lectures. The continuous case is similar:

\begin{proof}
	Suppose we are given the factorization $ f_{\mathbf{X}}(\mathbf{x} \; ; \theta) = g(t(\mathbf{x}), \theta) h(\mathbf{x}) $. If $ T = u $, then
	
	\begin{align*}
	f_{\mathbf{X}| T = u}(\mathbf{x} \; ; u) & = \frac{ g(t(\mathbf{x}), \theta) h(\mathbf{x})}{\int_{\mathbf{y} \; ; T(\mathbf{y}) = u} g(t(\mathbf{y}),\theta) h(\mathbf{y}) \; \d \mathbf{y}} \\
	& = \frac{ g(u, \theta) h(\mathbf{x})}{ g(u,\theta)   \int_{\mathbf{y} \; ; T(\mathbf{y}) = u}  h(\mathbf{y}) \; \d \mathbf{y}} \\
	& = \frac{h(\mathbf{x})}{ \int_{\mathbf{y}}  h(\mathbf{y}) \; \d \mathbf{y}}
	\end{align*}
	
	
	which does not depend on $ \theta $; thus $ T $ is sufficient for $ \theta $.
	
	The other direction is the same as the discrete case: Suppose $ T $ is sufficient for $ \theta $, ie. the conditional distribution of $ \mathbf{X} \; | T = u $ does not depend on $ \theta $. Then
	
	\[ \P_{\theta}( \mathbf{X} = \mathbf{x}) = \P_{\theta}( \mathbf{X} = \mathbf{x}, T = T(\mathbf{x})) = \P_{\theta} (\mathbf{X} = \mathbf{x} \; | T = T(\mathbf{x})) \P_{\theta}(T = T(\mathbf{x}) )  \]
	
	The first factor does not depend on $ \theta $ by assumption; call it $ h(\mathbf{x}) $. Let the second factor be $ g(t,\theta) $, and so we have the required factorisation. 
	
\end{proof}

 





\section{QUESTION 4}

\begin{enumerate}[label = (\alph*)]
	\item Let $ X_{1},\cdots,X_{n} $ be independent Po($ i\theta $). So
	
	\begin{align*}
	f_{\mathbf{X}}(\mathbf{x} \; | \; \theta) & =  \prod_{i=1}^{n} \frac{e^{-i\theta}(i \theta)^{x_{i}}}{x_{i}!} \\
	& = \underbrace{\exp  \left(- \frac{n(n+1)}{2} \theta   \right)  \theta^{\sum x_{i}}}_{g(t(\mathbf{x}),\theta)} \cdot \underbrace{\prod_{i=1}^{n} \frac{i^{x_{i}}}{x_{i}!}}_{h(\mathbf{x})}
	\end{align*}
	
	Using the factorization criterion, $ T = t(\mathbf{x}) = \sum_{i=1}^{n} x_{i} $ is a sufficient statistic, and $ T \sim \text{Po }(n(n+1)\theta/2) $
	
	The log-likelihood is 
	
	\begin{align*}
	l(\theta) &  = - \frac{n(n+1)}{2} \theta   + \sum x_{i} \log \theta + \log \left(  \prod_{i=1}^{n} \frac{i^{x_{i}}}{x_{i}!} \right)    \\
	\end{align*}
	
	and this is maximised when $ \frac{\d l}{\d \theta} = 0 $;
	
	\[  - \frac{n(n+1)}{2}  + \frac{1}{\theta} \sum x_{i}  = 0 \Rightarrow \hat{\theta} = \frac{2 \sum x_{i}}{n(n+1)}   \]
	
	Thus the MLE $ \hat{\theta} $ is a function of $ T = \sum x_{i} $, and is unbiased:
	
	\[ \E_{\theta}(\hat{\theta}) = \frac{2}{n(n+1)} \cdot \frac{n(n+1)}{2} \theta = \theta  \]
	
	\item Let $ X_{1},\cdots,X_{n} \sim $ iid Exp($ \theta $). Then
	
	
	\begin{align*}
	\frac{	f_{\mathbf{X}}(\mathbf{x} \; | \; \theta)}{	f_{\mathbf{X}}(\mathbf{y} \; | \; \theta)}& = \frac{\lambda^{n} e^{-\lambda \sum x_{i} }}{\lambda^{n} e^{-\lambda \sum y_{i} }} \\
	& = \exp\left\{ - \lambda \left( \sum x_{i} - \sum y_{i}  \right)   \right\} 
	\end{align*}
	
	This is constant as a function of $ \lambda $ iff $  \sum x_{i} = \sum y_{i} $. Hence $ T = \sum_{i=1}^{n} x_{i} $ is minimal sufficient, with $ T \sim \Gamma(n,\lambda) $. 
	
	The log-likelihood is  
	
	\[ l(\theta) = n \log \lambda - \lambda \sum x_{i} \]
	
	and this is maximised when $ \frac{\d l}{\d \lambda} = 0 $;
	
	\[   \frac{n}{\lambda} - \sum x_{i} = 0 \Rightarrow \hat{\lambda} = \frac{n}{\sum x_{i}}  \]
	
	Thus the MLE $ \hat{\lambda} $ is a function of $ T = \sum x_{i} $, and 
	
	\[ \E_{\lambda}(\hat{\lambda}) = n \cdot \left( \frac{n}{\lambda} \right)^{-1} = \lambda  \]
	
	so it is unbiased?
	
	
	
	
	
\end{enumerate}

\section{QUESTION 5}

Given $ \tilde{\theta} = \frac{2}{3} X_{1} $, we have

\begin{align*}
\E_{\theta}(\tilde{\theta}) & = \frac{2}{3} \frac{1}{2}(\theta + 2 \theta)  = \theta
\end{align*}

so $ \tilde{\theta} $ is unbiased. 

We have

\begin{align*}
\frac{	f_{\mathbf{X}}(\mathbf{x} \; | \; \theta)}{	f_{\mathbf{X}}(\mathbf{y} \; | \; \theta)}& = \frac{ \frac{1}{\theta^{n}} \mathbb{I}_{\{ \max x_{i} < 2 \theta \}}  \mathbb{I}_{\{ \min x_{i} > \theta \}} }{ \frac{1}{\theta^{n}} \mathbb{I}_{\{ \max y_{i} < 2 \theta \}}  \mathbb{I}_{\{ \min y_{i} > \theta \}} }
\end{align*}

Hence we can see $ T = \Delta x : =  \lfloor  \frac{\min x_{i} + \max x_{i}}{2}  \rfloor  $ is minimal sufficient, and

\begin{align*}
\E_{\theta}(\tilde{\theta} | T = u) & = \frac{2}{3} \E_{\theta} ( X_{1} |  \lfloor  \frac{\min x_{i} + \max x_{i}}{2}  \rfloor  = u ) \\
& = \frac{2}{3} \E_{\theta} ( X_{1} |   \Delta x   = u, X_{1} =  \Delta x  ) \P_{\theta}( X_{1} = \Delta x  \; | \;  \Delta x  = u ) \\
& + \frac{2}{3} \E_{\theta} ( X_{1} |   \Delta x   = u, X_{1} \neq  \Delta x  ) \P_{\theta}( X_{1} \neq \Delta x  \; | \;  \Delta x  = u ) \\
& = \frac{2}{3} \left(  u \times \frac{1}{n} + \frac{u}{2} \times \frac{n-1}{n}  \right) \\
& = \frac{1}{3} \frac{n+1}{n} u 
\end{align*}

So the Rao-Blackwell estimator is $ \frac{1}{3} \frac{n+1}{n}  \lfloor  \frac{\min x_{i} + \max x_{i}}{2}  \rfloor $

Note sure about this as I don't think the probability that $ X_{1} $ takes the value $ \Delta x $ is $ \frac{1}{n} $. Not sure how to incorporate both $ \min x_{i} $ and $ \max x_{i} $ into a sufficient statistic.  
 


\section{QUESTION 6}

Have

\[ L(\theta) = f_{\mathbf{X}} (\mathbf{x} \; ; \; \theta) = \frac{1}{\theta^{n}} \mathbb{I}_{\{  \max x_{i} < \theta \}} \mathbb{I}_{\{  \min x_{i} > 0 \}} \]

So for $ \theta \geq \max x_{i} $, $ L(\theta) = \frac{1}{\theta^{n}} $ and is decreasing as $ \theta $ increases, while for $ \theta < \max x_{i} $, $ L(\theta) = 0 $. Hence the value $ \hat{\theta} = \max x_{i} $ maximizes the likelihood.

Now, $ \P_{\theta}(\theta \geq \max x_{i}) = 1 $. So we can construct a one-sided $ 100(1-\alpha) $ \% confidence interval with lower bound $ \hat{\theta} $, and upper bound $ b(\hat{\theta}) $, such that $ \P(\theta \leq b(\hat{\theta})) = 1 - \alpha $, for some function $ b $ to be determined.

Have that 

\[ 1 - \alpha = \P (\theta \leq b(\hat{\theta})) = 1 - \P(b(\hat{\theta}) < \theta) = 1 - \P(\hat{\theta} < b^{-1}(\theta))  \]

Hence $ \P(\hat{\theta} < b^{-1}(\theta)) = \alpha  $. For $ 0 \leq t \leq \theta $ the cumulative distribution function of $ \hat{\theta}  $ is

\[ F_{\hat{\theta}}(t) = \P (\hat{\theta} \leq t) = \P ( X_{i} \leq t \text{ for all } i) = ( \P(X_{i}\leq t))^{n} = \left( \frac{t}{n} \right)^{n}  \]

We obtain

\[ \frac{(b^{-1}(\theta))^{n}}{\theta^{n}} = \alpha, \; \text{ whence } b^{-1}(\theta) = \theta \alpha^{1/n} \]

This gives inverse function $ b(\hat{\theta}) = \hat{\theta} / \alpha^{1/n} $.

Hence, the $ (1 - \alpha) $ \% CI for $ \theta $ is $ (\hat{\theta},\hat{\theta}/\alpha^{1/n}) $.



\section{QUESTION 7}


If we take $ z_{-} < z_{+} $ such that $ \Phi(z_{+}) - \Phi(z_{-}) = \sqrt{0.95} $, then the equation

\[ \P \left(    z_{-} < X_{1} - \theta_{1} < z_{+}, z_{-} < X_{2} - \theta_{2} < z_{+} \right) = 0.95  \]

determines a $ 95 $ \% confidence set for $ (\theta_{1},\theta_{2}) $.
 
Owing to independence we can model this as a square:


\[  \P \left(    z_{-} < X_{1} - \theta_{1} < z_{+} \right) = \sqrt{0.95}  \]

which can be written as 

\[  \P \left(    X_{1} - z_{+} < \theta_{1} < X_{1} - z_{-} \right) = \sqrt{0.95}  \]

ie. gives the interval 

\[ \left(   X_{1} - z_{+}, X_{1} - z_{-} \right)  \]

centred at $ X_{1} - (z_{+} + z_{-})/2   $ with width $ (z_{+} - z_{-}) $.

Choosing $ z_{+} = - z_{-} := z $ gives the interval

\[ \left(  X_{1} - z, X_{1} - z  \right)  \]

and $ z $ will be the upper $ (1 - \sqrt{0.95})/2 $ point of the standard normal distribution, ie. $ \Phi(a) = 1 - ( 1 - \sqrt{0.95}) /2  = (1 + \sqrt{0.95})/a  $. By the hint, we see $ a = 2.236 $; thus the confidence set can be written as


\[ S = \{  (\theta_{1},\theta_{2}) : | \theta_{1} - X_{1} | \leq 2.236, | \theta_{2} - X_{2} | \leq 2.236   \}  \]


Similarly, we try model this as a circle:

\[ \P  \left(  (\theta_{1} - X_{1})^{2} + (\theta_{2} - X_{2})^{2} < z_{+}^{2} \right) 0.95   \]

Not sure how to finish. 



%
%
%If $ X_{1} \sim N (\theta_{1},1) $ then $ \theta_{1} - X_{1} \sim N(0,1) $.
%
%The condition $ | \theta_{1} - X_{1} | \leq 2.236 $ is equivalent to 
%
%\[ - \Phi^{-1} \left( \frac{1 + \sqrt{0.95}}{2} \right)  \leq \theta_{1} - X_{1}  \leq \Phi^{-1} \left( \frac{1 + \sqrt{0.95}}{2} \right)  \]

\section{QUESTION 8}

The likelihood is written as 


\begin{align*}
f_{\mathbf{X}}(\mathbf{x} \; | \; \lambda) & =  \prod_{i=1}^{n} \frac{e^{-\lambda}\lambda^{x_{i}}}{x_{i}!} \\
& = \exp ( - n \lambda ) \lambda^{\sum x_{i}} \cdot \prod_{i=1}^{n} \frac{1}{x_{i}!}
\end{align*}

Here $ n = 5 $, $ \sum x_{i} = 16 $, $ \prod_{i=1}^{n} \frac{1}{x_{i}!} = 207,360 $. Calculating,


\begin{align*}
f_{X}(x) & = f_{X}(x | 1)  \pi_{\lambda}(1) +  f_{X}(x | 1.5)  \pi_{\lambda}(1.5) \\
& = \exp(-5) \cdot \frac{1}{207,360} \cdot 0.4 + \exp(-7.5) 1.5^{16} \cdot  \frac{1}{207,360} \cdot 0.6 \\
& = 3.249396 \times 10^{-8} \cdot 0.4 + 1.75197 \times 10^{-6} \times 0.6 \\
& = 7.20284 \times 10^{-7}
\end{align*}
	
Whence

\begin{align*}
\pi_{\lambda| X}(1|x)& = \frac{f_{X}(x | 1) \cdot \pi_{\lambda}(1)}{f_{X}(x)} \\
& = \frac{3.249396 \times 10^{-8} \cdot 0.4}{7.20284 \times 10^{-7}} \\
& = 0.0270676
\end{align*}

and

\begin{align*}
\pi_{\lambda| X}(1.5|x)& = \frac{f_{X}(x | 1.5) \cdot \pi_{\lambda}(1.5)}{f_{X}(x)} \\
& = \frac{1.75197 \times 10^{-6} \times 0.6}{7.20284 \times 10^{-7}} \\
& = 0.972932
\end{align*}

Getting different answers than those on the sheet but not sure where the error is. 



\section{QUESTION 9}

By independence, $ f_{X|\theta}(x|\theta) = \theta^{n} (x_{1}x_{2}\cdots x_{n})^{\theta-1}, 0 < x < 1 $, and so given a Gamma prior, the posterior is 

\begin{align*}
\pi_{\theta | X}(\theta | x) & \propto f_{X|\theta}(x|\theta) \pi_{\theta}(\theta) \\
& = \theta^{n} (x_{1}x_{2}\cdots x_{n})^{\theta-1} \cdot \frac{\lambda^{\alpha}\theta^{\alpha - 1} e^{- \lambda \theta}}{\Gamma (\alpha)}, \quad 0 < x < 1
\end{align*}

which is $ \Gamma(  n+\alpha,\lambda ) $ with the appropriate proportionality constant. 

For quadratic loss, the Bayesian point estimator of $ \theta $ is just the posterior mean, which is given as $ \frac{n + \alpha}{\lambda} $.


\section{QUESTION 10}

We have that $ S_{n} \sim \text{Bin }(n,p_{n}) $, so as $ np_{n} \to \lambda $, $ n \to \infty $

\begin{align*}
\P (S_{n} = x) & = \binom{n}{x}p_{n}^{x}(1-p_{n})^{n-x} \\
& = \frac{1}{x!} \frac{n(n-1)\cdots(n-x+1)}{n^{x}} (np_{n})^{x} \left( 1 - \frac{np_{n}}{x} \right)^{n-x} \\
& \to \frac{1}{x!} \lambda^{x} e^{-\lambda}   \\
& =   \P(Y=x) \text{ where } Y \sim   \text{Po }(\lambda)
\end{align*}

since $ (1-a/n)^{n} \to e^{-a} $

\section{QUESTION 11}

$ f_{X_{1}}(x_{1} | \theta) \sim N(0,1) $, so $ X_{2} = \theta X_{1} + (1- \theta^{2})^{1/2} \varepsilon_{2} $, and

\begin{align*}
f_{X_{2}}(x_{2} | \theta) & \sim N(0,\theta^{2}) + N(0,1-\theta^{2}) \\
& \sim N(0,1)
\end{align*}

Similarly $ f_{X_{i}}(x_{i} | \theta) \sim N(0,1) $ for $ i = 1,\cdots,n $.

Hence

\begin{align*}
f_{\mathbf{X}}(\mathbf{x} | \theta)& = (2\pi)^{-n/2} \exp\left( -\sum x_{i}^{2} \right)  \\
& = 
\end{align*}

Shouldn't be independent of $ \theta $?

\section{QUESTION 12}



\end{document}