\documentclass[a4paper]{article}
\usepackage{amsmath}
\def\npart {IB}
\def\nterm {Michaelmas}
\def\nyear {2017}
\def\nlecturer {Dr Warnick}
\def\ncourse {Quantum Mechanics Example Sheet 2}

% Imports
\ifx \nauthor\undefined
  \def\nauthor{Christopher Turnbull}
\else
\fi

\author{Supervised by \nlecturer \\\small Solutions presented by \nauthor}
\date{\nterm\ \nyear}

\usepackage{alltt}
\usepackage{amsfonts}
\usepackage{amsmath}
\usepackage{amssymb}
\usepackage{amsthm}
\usepackage{booktabs}
\usepackage{caption}
\usepackage{enumitem}
\usepackage{fancyhdr}
\usepackage{graphicx}
\usepackage{mathdots}
\usepackage{mathtools}
\usepackage{microtype}
\usepackage{multirow}
\usepackage{pdflscape}
\usepackage{pgfplots}
\usepackage{siunitx}
\usepackage{slashed}
\usepackage{tabularx}
\usepackage{tikz}
\usepackage{tkz-euclide}
\usepackage[normalem]{ulem}
\usepackage[all]{xy}
\usepackage{imakeidx}

\makeindex[intoc, title=Index]
\indexsetup{othercode={\lhead{\emph{Index}}}}

\ifx \nextra \undefined
  \usepackage[pdftex,
    hidelinks,
    pdfauthor={Christopher Turnbull},
    pdfsubject={Cambridge Maths Notes: Part \npart\ - \ncourse},
    pdftitle={Part \npart\ - \ncourse},
  pdfkeywords={Cambridge Mathematics Maths Math \npart\ \nterm\ \nyear\ \ncourse}]{hyperref}
  \title{Part \npart\ --- \ncourse}
\else
  \usepackage[pdftex,
    hidelinks,
    pdfauthor={Christopher Turnbull},
    pdfsubject={Cambridge Maths Notes: Part \npart\ - \ncourse\ (\nextra)},
    pdftitle={Part \npart\ - \ncourse\ (\nextra)},
  pdfkeywords={Cambridge Mathematics Maths Math \npart\ \nterm\ \nyear\ \ncourse\ \nextra}]{hyperref}

  \title{Part \npart\ --- \ncourse \\ {\Large \nextra}}
  \renewcommand\printindex{}
\fi

\pgfplotsset{compat=1.12}

\pagestyle{fancyplain}
\lhead{\emph{\nouppercase{\leftmark}}}
\ifx \nextra \undefined
  \rhead{
    \ifnum\thepage=1
    \else
      \npart\ \ncourse
    \fi}
\else
  \rhead{
    \ifnum\thepage=1
    \else
      \npart\ \ncourse\ (\nextra)
    \fi}
\fi
\usetikzlibrary{arrows.meta}
\usetikzlibrary{decorations.markings}
\usetikzlibrary{decorations.pathmorphing}
\usetikzlibrary{positioning}
\usetikzlibrary{fadings}
\usetikzlibrary{intersections}
\usetikzlibrary{cd}

\newcommand*{\Cdot}{{\raisebox{-0.25ex}{\scalebox{1.5}{$\cdot$}}}}
\newcommand {\pd}[2][ ]{
  \ifx #1 { }
    \frac{\partial}{\partial #2}
  \else
    \frac{\partial^{#1}}{\partial #2^{#1}}
  \fi
}
\ifx \nhtml \undefined
\else
  \renewcommand\printindex{}
  \makeatletter
  \DisableLigatures[f]{family = *}
  \let\Contentsline\contentsline
  \renewcommand\contentsline[3]{\Contentsline{#1}{#2}{}}
  \renewcommand{\@dotsep}{10000}
  \newlength\currentparindent
  \setlength\currentparindent\parindent

  \newcommand\@minipagerestore{\setlength{\parindent}{\currentparindent}}
  \usepackage[active,tightpage,pdftex]{preview}
  \renewcommand{\PreviewBorder}{0.1cm}

  \newenvironment{stretchpage}%
  {\begin{preview}\begin{minipage}{\hsize}}%
    {\end{minipage}\end{preview}}
  \AtBeginDocument{\begin{stretchpage}}
  \AtEndDocument{\end{stretchpage}}

  \newcommand{\@@newpage}{\end{stretchpage}\begin{stretchpage}}

  \let\@real@section\section
  \renewcommand{\section}{\@@newpage\@real@section}
  \let\@real@subsection\subsection
  \renewcommand{\subsection}{\@@newpage\@real@subsection}
  \makeatother
\fi

% Theorems
\theoremstyle{definition}
\newtheorem*{aim}{Aim}
\newtheorem*{axiom}{Axiom}
\newtheorem*{claim}{Claim}
\newtheorem*{cor}{Corollary}
\newtheorem*{conjecture}{Conjecture}
\newtheorem*{defi}{Definition}
\newtheorem*{eg}{Example}
\newtheorem*{ex}{Exercise}
\newtheorem*{fact}{Fact}
\newtheorem*{law}{Law}
\newtheorem*{lemma}{Lemma}
\newtheorem*{notation}{Notation}
\newtheorem*{prop}{Proposition}
\newtheorem*{soln}{Solution}
\newtheorem*{thm}{Theorem}

\newtheorem*{remark}{Remark}
\newtheorem*{warning}{Warning}
\newtheorem*{exercise}{Exercise}

\newtheorem{nthm}{Theorem}[section]
\newtheorem{nlemma}[nthm]{Lemma}
\newtheorem{nprop}[nthm]{Proposition}
\newtheorem{ncor}[nthm]{Corollary}


\renewcommand{\labelitemi}{--}
\renewcommand{\labelitemii}{$\circ$}
\renewcommand{\labelenumi}{(\roman{*})}

\let\stdsection\section
\renewcommand\section{\newpage\stdsection}

% Strike through
\def\st{\bgroup \ULdepth=-.55ex \ULset}

% Maths symbols
\newcommand{\abs}[1]{\left\lvert #1\right\rvert}
\newcommand\ad{\mathrm{ad}}
\newcommand\AND{\mathsf{AND}}
\newcommand\Art{\mathrm{Art}}
\newcommand{\Bilin}{\mathrm{Bilin}}
\newcommand{\bket}[1]{\left\lvert #1\right\rangle}
\newcommand{\B}{\mathcal{B}}
\newcommand{\bolds}[1]{{\bfseries #1}}
\newcommand{\brak}[1]{\left\langle #1 \right\rvert}
\newcommand{\braket}[2]{\left\langle #1\middle\vert #2 \right\rangle}
\newcommand{\bra}{\langle}
\newcommand{\cat}[1]{\mathsf{#1}}
\newcommand{\C}{\mathbb{C}}
\newcommand{\CP}{\mathbb{CP}}
\newcommand{\cU}{\mathcal{U}}
\newcommand{\Der}{\mathrm{Der}}
\newcommand{\D}{\mathrm{D}}
\newcommand{\dR}{\mathrm{dR}}
\newcommand{\E}{\mathbb{E}}
\newcommand{\F}{\mathbb{F}}
\newcommand{\Frob}{\mathrm{Frob}}
\newcommand{\GG}{\mathbb{G}}
\newcommand{\gl}{\mathfrak{gl}}
\newcommand{\GL}{\mathrm{GL}}
\newcommand{\G}{\mathcal{G}}
\newcommand{\Gr}{\mathrm{Gr}}
\newcommand{\haut}{\mathrm{ht}}
\newcommand{\Id}{\mathrm{Id}}
\newcommand{\ket}{\rangle}
\newcommand{\lie}[1]{\mathfrak{#1}}
\newcommand{\Mat}{\mathrm{Mat}}
\newcommand{\N}{\mathbb{N}}
\newcommand{\norm}[1]{\left\lVert #1\right\rVert}
\newcommand{\normalorder}[1]{\mathop{:}\nolimits\!#1\!\mathop{:}\nolimits}
\newcommand\NOT{\mathsf{NOT}}
\newcommand{\Oc}{\mathcal{O}}
\newcommand{\Or}{\mathrm{O}}
\newcommand\OR{\mathsf{OR}}
\newcommand{\ort}{\mathfrak{o}}
\newcommand{\PGL}{\mathrm{PGL}}
\newcommand{\ph}{\,\cdot\,}
\newcommand{\pr}{\mathrm{pr}}
\newcommand{\Prob}{\mathbb{P}}
\newcommand{\PSL}{\mathrm{PSL}}
\newcommand{\Ps}{\mathcal{P}}
\newcommand{\PSU}{\mathrm{PSU}}
\newcommand{\pt}{\mathrm{pt}}
\newcommand{\qeq}{\mathrel{``{=}"}}
\newcommand{\Q}{\mathbb{Q}}
\newcommand{\R}{\mathbb{R}}
\newcommand{\RP}{\mathbb{RP}}
\newcommand{\Rs}{\mathcal{R}}
\newcommand{\SL}{\mathrm{SL}}
\newcommand{\so}{\mathfrak{so}}
\newcommand{\SO}{\mathrm{SO}}
\newcommand{\Spin}{\mathrm{Spin}}
\newcommand{\Sp}{\mathrm{Sp}}
\newcommand{\su}{\mathfrak{su}}
\newcommand{\SU}{\mathrm{SU}}
\newcommand{\term}[1]{\emph{#1}\index{#1}}
\newcommand{\T}{\mathbb{T}}
\newcommand{\tv}[1]{|#1|}
\newcommand{\U}{\mathrm{U}}
\newcommand{\uu}{\mathfrak{u}}
\newcommand{\Vect}{\mathrm{Vect}}
\newcommand{\wsto}{\stackrel{\mathrm{w}^*}{\to}}
\newcommand{\wt}{\mathrm{wt}}
\newcommand{\wto}{\stackrel{\mathrm{w}}{\to}}
\newcommand{\Z}{\mathbb{Z}}
\renewcommand{\d}{\mathrm{d}}
\renewcommand{\H}{\mathbb{H}}
\renewcommand{\P}{\mathbb{P}}
\renewcommand{\sl}{\mathfrak{sl}}
\renewcommand{\vec}[1]{\boldsymbol{\mathbf{#1}}}
%\renewcommand{\F}{\mathcal{F}}

\let\Im\relax
\let\Re\relax

\DeclareMathOperator{\adj}{adj}
\DeclareMathOperator{\Ann}{Ann}
\DeclareMathOperator{\area}{area}
\DeclareMathOperator{\Aut}{Aut}
\DeclareMathOperator{\Bernoulli}{Bernoulli}
\DeclareMathOperator{\betaD}{beta}
\DeclareMathOperator{\bias}{bias}
\DeclareMathOperator{\binomial}{binomial}
\DeclareMathOperator{\card}{card}
\DeclareMathOperator{\ccl}{ccl}
\DeclareMathOperator{\Char}{char}
\DeclareMathOperator{\ch}{ch}
\DeclareMathOperator{\cl}{cl}
\DeclareMathOperator{\cls}{\overline{\mathrm{span}}}
\DeclareMathOperator{\conv}{conv}
\DeclareMathOperator{\corr}{corr}
\DeclareMathOperator{\cosec}{cosec}
\DeclareMathOperator{\cosech}{cosech}
\DeclareMathOperator{\cov}{cov}
\DeclareMathOperator{\covol}{covol}
\DeclareMathOperator{\diag}{diag}
\DeclareMathOperator{\diam}{diam}
\DeclareMathOperator{\Diff}{Diff}
\DeclareMathOperator{\disc}{disc}
\DeclareMathOperator{\dom}{dom}
\DeclareMathOperator{\End}{End}
\DeclareMathOperator{\energy}{energy}
\DeclareMathOperator{\erfc}{erfc}
\DeclareMathOperator{\erf}{erf}
\DeclareMathOperator*{\esssup}{ess\,sup}
\DeclareMathOperator{\ev}{ev}
\DeclareMathOperator{\Ext}{Ext}
\DeclareMathOperator{\Fit}{Fit}
\DeclareMathOperator{\fix}{fix}
\DeclareMathOperator{\Frac}{Frac}
\DeclareMathOperator{\Gal}{Gal}
\DeclareMathOperator{\gammaD}{gamma}
\DeclareMathOperator{\gr}{gr}
\DeclareMathOperator{\hcf}{hcf}
\DeclareMathOperator{\Hom}{Hom}
\DeclareMathOperator{\id}{id}
\DeclareMathOperator{\image}{image}
\DeclareMathOperator{\im}{im}
\DeclareMathOperator{\Im}{Im}
\DeclareMathOperator{\Ind}{Ind}
\DeclareMathOperator{\Int}{Int}
\DeclareMathOperator{\Isom}{Isom}
\DeclareMathOperator{\lcm}{lcm}
\DeclareMathOperator{\length}{length}
\DeclareMathOperator{\Lie}{Lie}
\DeclareMathOperator{\like}{like}
\DeclareMathOperator{\Lk}{Lk}
\DeclareMathOperator{\mse}{mse}
\DeclareMathOperator{\multinomial}{multinomial}
\DeclareMathOperator{\orb}{orb}
\DeclareMathOperator{\ord}{ord}
\DeclareMathOperator{\otp}{otp}
\DeclareMathOperator{\Poisson}{Poisson}
\DeclareMathOperator{\poly}{poly}
\DeclareMathOperator{\rank}{rank}
\DeclareMathOperator{\rel}{rel}
\DeclareMathOperator{\Re}{Re}
\DeclareMathOperator*{\res}{res}
\DeclareMathOperator{\Res}{Res}
\DeclareMathOperator{\rk}{rk}
\DeclareMathOperator{\Root}{Root}
\DeclareMathOperator{\sech}{sech}
\DeclareMathOperator{\sgn}{sgn}
\DeclareMathOperator{\spn}{span}
\DeclareMathOperator{\stab}{stab}
\DeclareMathOperator{\St}{St}
\DeclareMathOperator{\supp}{supp}
\DeclareMathOperator{\Syl}{Syl}
\DeclareMathOperator{\Sym}{Sym}
\DeclareMathOperator{\tr}{tr}
\DeclareMathOperator{\Tr}{Tr}
\DeclareMathOperator{\var}{var}
\DeclareMathOperator{\vol}{vol}

\pgfarrowsdeclarecombine{twolatex'}{twolatex'}{latex'}{latex'}{latex'}{latex'}
\tikzset{->/.style = {decoration={markings,
                                  mark=at position 1 with {\arrow[scale=2]{latex'}}},
                      postaction={decorate}}}
\tikzset{<-/.style = {decoration={markings,
                                  mark=at position 0 with {\arrowreversed[scale=2]{latex'}}},
                      postaction={decorate}}}
\tikzset{<->/.style = {decoration={markings,
                                   mark=at position 0 with {\arrowreversed[scale=2]{latex'}},
                                   mark=at position 1 with {\arrow[scale=2]{latex'}}},
                       postaction={decorate}}}
\tikzset{->-/.style = {decoration={markings,
                                   mark=at position #1 with {\arrow[scale=2]{latex'}}},
                       postaction={decorate}}}
\tikzset{-<-/.style = {decoration={markings,
                                   mark=at position #1 with {\arrowreversed[scale=2]{latex'}}},
                       postaction={decorate}}}
\tikzset{->>/.style = {decoration={markings,
                                  mark=at position 1 with {\arrow[scale=2]{latex'}}},
                      postaction={decorate}}}
\tikzset{<<-/.style = {decoration={markings,
                                  mark=at position 0 with {\arrowreversed[scale=2]{twolatex'}}},
                      postaction={decorate}}}
\tikzset{<<->>/.style = {decoration={markings,
                                   mark=at position 0 with {\arrowreversed[scale=2]{twolatex'}},
                                   mark=at position 1 with {\arrow[scale=2]{twolatex'}}},
                       postaction={decorate}}}
\tikzset{->>-/.style = {decoration={markings,
                                   mark=at position #1 with {\arrow[scale=2]{twolatex'}}},
                       postaction={decorate}}}
\tikzset{-<<-/.style = {decoration={markings,
                                   mark=at position #1 with {\arrowreversed[scale=2]{twolatex'}}},
                       postaction={decorate}}}

\tikzset{circ/.style = {fill, circle, inner sep = 0, minimum size = 3}}
\tikzset{mstate/.style={circle, draw, blue, text=black, minimum width=0.7cm}}

\tikzset{commutative diagrams/.cd,cdmap/.style={/tikz/column 1/.append style={anchor=base east},/tikz/column 2/.append style={anchor=base west},row sep=tiny}}

\definecolor{mblue}{rgb}{0.2, 0.3, 0.8}
\definecolor{morange}{rgb}{1, 0.5, 0}
\definecolor{mgreen}{rgb}{0.1, 0.4, 0.2}
\definecolor{mred}{rgb}{0.5, 0, 0}

\def\drawcirculararc(#1,#2)(#3,#4)(#5,#6){%
    \pgfmathsetmacro\cA{(#1*#1+#2*#2-#3*#3-#4*#4)/2}%
    \pgfmathsetmacro\cB{(#1*#1+#2*#2-#5*#5-#6*#6)/2}%
    \pgfmathsetmacro\cy{(\cB*(#1-#3)-\cA*(#1-#5))/%
                        ((#2-#6)*(#1-#3)-(#2-#4)*(#1-#5))}%
    \pgfmathsetmacro\cx{(\cA-\cy*(#2-#4))/(#1-#3)}%
    \pgfmathsetmacro\cr{sqrt((#1-\cx)*(#1-\cx)+(#2-\cy)*(#2-\cy))}%
    \pgfmathsetmacro\cA{atan2(#2-\cy,#1-\cx)}%
    \pgfmathsetmacro\cB{atan2(#6-\cy,#5-\cx)}%
    \pgfmathparse{\cB<\cA}%
    \ifnum\pgfmathresult=1
        \pgfmathsetmacro\cB{\cB+360}%
    \fi
    \draw (#1,#2) arc (\cA:\cB:\cr);%
}
\newcommand\getCoord[3]{\newdimen{#1}\newdimen{#2}\pgfextractx{#1}{\pgfpointanchor{#3}{center}}\pgfextracty{#2}{\pgfpointanchor{#3}{center}}}

\def\Xint#1{\mathchoice
   {\XXint\displaystyle\textstyle{#1}}%
   {\XXint\textstyle\scriptstyle{#1}}%
   {\XXint\scriptstyle\scriptscriptstyle{#1}}%
   {\XXint\scriptscriptstyle\scriptscriptstyle{#1}}%
   \!\int}
\def\XXint#1#2#3{{\setbox0=\hbox{$#1{#2#3}{\int}$}
     \vcenter{\hbox{$#2#3$}}\kern-.5\wd0}}
\def\ddashint{\Xint=}
\def\dashint{\Xint-}

\newcommand\separator{{\centering\rule{2cm}{0.2pt}\vspace{2pt}\par}}

\newenvironment{own}{\color{gray!70!black}}{}

\newcommand\makecenter[1]{\raisebox{-0.5\height}{#1}}
\newtheorem*{soln}{Solution}

\renewcommand{\thesection}{}
\renewcommand{\thesubsection}{\arabic{section}.\arabic{subsection}}
\makeatletter
\def\@seccntformat#1{\csname #1ignore\expandafter\endcsname\csname the#1\endcsname\quad}
\let\sectionignore\@gobbletwo
\let\latex@numberline\numberline
\def\numberline#1{\if\relax#1\relax\else\latex@numberline{#1}\fi}
\makeatother


\begin{document}
	
\maketitle

\section{QUESTION 1}


The potential $ V(x) = 0 $ so our time independent SE is:

\[ -\frac{\hbar^{2}}{2m} \frac{\d^{2} \psi}{\d x^{2}} = E \psi  \]

\[ \iff \psi'' + k^{2} \psi = 0 \]

setting $ E = k^{2} \hbar^{2} / 2m $, thus 

\[ \psi(x) = A \cos k x + B \sin k x \]

Using BCs, $ \psi(0) = 0 \Rightarrow A = 0 $

$ \psi(a) = 0 \Rightarrow \sin k a = 0 \Rightarrow k a =  n \pi  $ for integer $ n $, thus energy eigenvalues are $ E_{n} = n^{2} \pi^{2} \hbar^{2} / 2m a^{2} $ with corresponding energy eigenstates $ B_{n} \sin k_{n} x = B_{n} \sin ( \sqrt{2 m a^{2 }E_{n} / \hbar^{2} } x ) $, and 

\begin{align*}
1 & = \int_{0}^{a} |  \psi(x) |^{2} \; \d x \\
& = \int_{0}^{a} B_{n}^{2} \sin^{2} ( k_{n} x ) \; \d x  \\
& = B_{n}^{2} \left[  \frac{x}{2}  - \frac{1}{4} \sin ( 2 k_{n} x ) \right]_{0}^{a} \\
& = B_{n}^{2} \frac{a}{2}
\end{align*}

\begin{align*}
\Rightarrow \; & B_{n}  = \sqrt{\frac{2}{a}} \\
\Rightarrow & \text{ norm. states are } \psi_{n}(x) = \sqrt{\frac{2}{a}} \sin \left(    \sqrt{\frac{2m a^{2} E_{n}}{\hbar^{2}} } x \right)
\end{align*} 

Let $ \psi_{n} $ denote the expectation value of $ \hat{x} $ in state $ \psi_{n} $, then

\begin{align*}
\langle \hat{x} \rangle_{n} & = (\psi_{n},\hat{x} \psi_{n} )  \\
& = \int_{0}^{a} \psi_{n}^{*} x \psi_{n} \; \d x \\
& = \frac{2}{a} \int_{0}^{a}  x \sin^{2} k_{n} x \; \d x \\
\end{align*}

By parts,

\begin{align*}
\int_{0}^{a}  x \sin^{2} k_{n} x \; \d x & = \left[  \frac{x^{2}}{2}  - \frac{1}{4 k_{n}} x \sin ( 2 k_{n} x ) \right]_{0}^{a} - \int_{0}^{a} \frac{x}{2}  - \frac{1}{4 k_{n}} \sin ( 2 k_{n} x ) \; \d x \\
& = \frac{a^{2}}{2} - \left[  \frac{x^{2}}{4}  + \frac{1}{8 k_{n}^{2}} \cos(2 k_{n} x)  \right]_{0}^{a} \\
& = \frac{a^{2}}{2} - \left[  \frac{x^{2}}{4}  + \frac{1}{8 k_{n}^{2}} \left( 1 - \sin^{2} (k_{n} x )   \right) \right]_{0}^{a}\\
& = \frac{a^{2}}{2} -  \frac{a^{2}}{4} \\
& = \frac{a^{2}}{4}
\end{align*}
  
Thus $ \langle \hat{x} \rangle_{n} = a / 2 $ as required.

Next, uncertainty of measurement of $ \hat{x} $ in state $ \psi $ given by

\begin{align*}
(\Delta x)_{n}^{2} & = \langle \hat{x}^{2} \rangle_{\psi} -  \langle \hat{x} \rangle_{\psi}^{2} \\
& = \frac{2}{a} \int_{0}^{a} x^{2} \sin^{2} (k_{n} x ) \; \d x - \frac{a^{2}}{4}
\end{align*}

By parts,

\begin{align*}
\int_{0}^{a} x^{2} \sin^{2} (k_{n} x ) \; \d x & = \left[  \frac{x^{3}}{2} - \frac{1}{4 k_{n}} x^{2} \sin(2k_{n} x)     \right]_{0}^{a} - \int_{0}^{a} x^{2} - \frac{1}{2 k_{n} } x \sin(2 k_{n} x) \; \d x    \\
& = \frac{a^{3}}{2} - \frac{a^{3}}{3} + \frac{1}{2 k_{n}} \int_{0}^{a} x \sin(2 k_{n} x) \;  \d x \\
& = \frac{a^{3}}{6} + \frac{1}{2 k_{n}} \left(  \left[  - \frac{1}{2 k_{n}} x \cos(2k _{n} x)  \right]_{0}^{a} + \frac{1}{2k_{n}} \int_{0}^{a} \frac{1}{2k_{n}} \cos(2k_{n} x) \; \d x  \right) \\
& = \frac{a^{3}}{6} - \frac{a}{4k_{n}^{2}} \cos (2 k_{n} a) \\
& = \frac{a^{3}}{6} - \frac{a}{4k_{n}^{2}} \left(  1 - \sin^{2} k_{n} a \right) \\
& =  \frac{a^{3}}{6}  - \frac{a^{3}}{4 n^{2}\pi^{2}}
\end{align*}

Hence

\begin{align*}
(\Delta x)_{n}^{2} & = \frac{a^{2}}{3}  - \frac{a^{2}}{2 n^{2}\pi^{2}}  - \frac{a^{2}}{4} \\
& = \frac{a^{2}}{12} \left( 1 - \frac{6}{\pi^{2} n^{2}} \right) 
\end{align*}

as required.



\section{QUESTION 2}


Harmonic oscillator, mass $ m $, frequency $ \omega $, has potential $ V(x) = \frac{1}{2} m \omega^{2} x^{2} $, hence Hamiltonian is given by

\[ H \psi = - \frac{\hbar^{2}}{2m} \psi'' + \frac{1}{2} m \omega^{2} x^{2} \psi  \]

Writing $ H $ in terms of momentum and position operators, we show that

\begin{align*}
\langle H \rangle_{\psi} & = ( \psi, H \psi )  \\
& = ( \psi,  \frac{1}{2m} \hat{p}^{2} \psi + \frac{1}{2} m \omega^{2} \hat{x}^{2} \psi   ) \\
& = \frac{1}{2m} ( \psi, \hat{p}^{2} \psi) + \frac{1}{2} m \omega^{2} (\psi, \hat{x}^{2} \psi) \\
& = \frac{1}{2m} \left(  (\Delta p)_{\psi}^{2} + \langle \hat{p} \rangle_{\psi}^{2}  \right)  + \frac{1}{2} m \omega^{2} \left(  ( \Delta x)_{\psi}^{2} + \langle \hat{x} \rangle_{\psi}^{2}  \right)
\end{align*}

Energy eigenvalues given by


\begin{align*}
\langle H \rangle_{\psi} & \geq \frac{1}{2m} (\Delta p)_{\psi}^{2} +   + \frac{1}{2} m \omega^{2}( \Delta x)_{\psi}^{2} \\
& = \frac{1}{2m} \left( (\Delta p)_{\psi}^{2} +   + m^{2} \omega^{2}( \Delta x)_{\psi}^{2}  \right)  \\
& \geq  \frac{1}{2m} \left(  \hbar m \omega \right) = \frac{\hbar \omega}{2} 
\end{align*}

where the last inequality follows from the uncertainty relation.
\section{QUESTION 3}

Let $  \Psi(x,t) $ be a solution of the time-dependent SE for a free particle ie. $ \Psi(x,t) $ satisfies

\[ i \hbar \dot{\Psi} = - \frac{\hbar^{2}}{2m} \Psi'' \]

Define $ \Phi(x,t) = \Psi(x-ut,t)e^{ikx}e^{-i \omega t} $, and setting $ \Psi(\xi,\eta) = \Psi(x-ut,t) $,

\begin{align*}
\frac{\partial }{\partial t} \Psi(\xi,\eta) & = \frac{\partial \xi }{\partial t} \frac{\partial \Psi }{\partial \xi} + \frac{\partial \eta }{\partial  t}\frac{\partial \Psi}{\partial \eta}  \\
& = - u \frac{\partial \Psi }{\partial \xi} +\frac{\partial \Psi}{\partial \eta} 
\end{align*}

and 

\begin{align*}
\frac{\partial }{\partial x} \Psi(\xi,\eta) & = \frac{\partial \xi }{\partial x} \frac{\partial \Psi }{\partial \xi} + \frac{\partial \eta }{\partial  x}\frac{\partial \Psi}{\partial \eta}  \\
& = \frac{\partial \Psi }{\partial \xi} 
\end{align*}

similarly the second derivative is

\[ \frac{\partial^{2} \Psi }{\partial x^{2}} = \frac{\partial^{2} \Psi }{\partial \xi^{2}}  \]


First calculate time derivatives,

\begin{align*}
\dot{\Phi} &  = e^{ikx} \left(  e^{-i \omega t} \frac{\partial }{\partial t} \Psi(\xi,\eta) - i \omega e^{-i \omega t} \Psi(\xi,\eta)  \right)   \\
& = e^{ikx}e^{-i \omega t} \left(  - u \frac{\partial \Psi }{\partial \xi} +\frac{\partial \Psi}{\partial \eta}   - i \omega \Psi \right) 
\end{align*}

Next, spatial 

\begin{align*}
\Phi'& = e^{- i \omega t} \left(  e^{i k x } \frac{\partial }{\partial x} \Psi(x - ut,t) + i k e^{i k x} \Psi(x-ut,t)  \right)   \\ 
\end{align*}

\begin{align*}
\Phi'' & = e^{- i \omega t} \left(  e^{i k x } \frac{\partial^{2} }{\partial x^{2}} \Psi(x - ut,t) + 2 i k e^{i k x} \frac{\partial }{\partial x}\Psi(x-ut,t)  - k^{2} e^{i k x} \Psi(x-ut,t)  \right)   \\ 
& = e^{- i \omega t}  e^{i k x } \left( \frac{\partial^{2} \Psi }{\partial \xi^{2}} + 2 i k \frac{\partial \Psi }{\partial \xi}  - k^{2} \Psi  \right)   \\ 
\end{align*}


So time-dependent SE becomes

\begin{align*}
i \hbar \left(  - u \frac{\partial \Psi }{\partial \xi} +\frac{\partial \Psi}{\partial \eta}   - i \omega \Psi \right)  & = - \frac{\hbar^{2}}{2m} \left( \frac{\partial^{2} \Psi }{\partial \xi^{2}} + 2 i k \frac{\partial \Psi }{\partial \xi}  - k^{2} \Psi  \right) 
\end{align*}

Linear independence allows us to compare the coefficients of $ \frac{\partial \Psi }{\partial \xi}  $ and $ \Psi $ to obtain

\[ - i \hbar u = - \frac{\hbar^{2}}{2m} 2 i k  \qquad \qquad  \hbar \omega =  \frac{\hbar^{2} k^{2}}{2m} \]

\[  m u = \hbar k \qquad \qquad 2m \omega = \hbar k^{2} \]

Thus $ \Phi $ is a solution if $ k = \frac{m u}{\hbar} $ and $ \omega = \frac{\hbar}{2m} k^{2} = \frac{m u^{2}}{2 \hbar} $

Next, comparing expectation values. 

Note

\begin{align*}
\langle \hat{x} \rangle_{\Psi} & =  (\Psi, \hat{x} \Psi)\\
& = \int_{-\infty}^{\infty} x | \Psi |^{2} \; \d x
\end{align*}

Clearly

\[ | \Phi |^{2} = | \Psi |^{2}  \]

and so

\[ \langle \hat{x} \rangle_{\Phi} = \langle \hat{x} \rangle_{\Psi} \]

But

\begin{align*}
\langle \hat{p} \rangle_{\Phi}& = \int_{-\infty}^{\infty} \Phi^{*} ( - i \hbar \Phi') \; \d x   \\
& = \int_{-\infty}^{\infty} \Psi^{*} ( - i \hbar \Psi') \; \d x  + \int_{-\infty}^{\infty} \Psi^{*} \Psi \; \d x  \\
& = \langle \hat{p} \rangle_{\Psi} + \hbar k
\end{align*}

To show consistency with Ehrenfest's Thm, want to check

\[ \frac{\d }{\d t} \langle \hat{x} \rangle_{\Phi}  = \frac{1}{m} \langle \hat{p}  \rangle_{\Psi}  \]

However since $ \langle \hat{x} \rangle_{\Phi} = \langle \hat{x} \rangle_{\Psi} $, their derivatives must also be equal; this cannot happen as $ \langle \hat{p} \rangle_{\Phi} \neq \langle \hat{p} \rangle_{\Psi} $, so the first part of Ehrenfest's does not hold. 

The next part 

\[ \frac{\d }{\d t} \langle \hat{p} \rangle_{\Phi}  = - \langle V'(\hat{x})  \rangle_{\Psi}  \]

does hold, as the difference in momenta does not depend on $ t $.

\section{QUESTION 4}

We have

\[ H \psi_{n}(x) = E_{n} \psi_{n}(x)  \]

For energy levels $ E_{n} = (n + \frac{1}{2}) \hbar \omega $ with corresponding energy eigenstates $ \psi_{n}(x) = h_{n}(y) e^{-y^{2}/2} $ where $ y = (m \omega / \hbar )^{1/2} x $ and $ h_{n} $ is a polynomial of degree $ n $ with $ h_{n}(-y) = (-1)^{n} h_{n}(y) $, for $ n = 0,1,2,\cdots $

First, $ \psi_{0}(x) = a_{0} e^{- y^{2} / 2} $ for some constant $ a_{0} $.

Know that $ \psi_{2}(x) = a_{2}(y) e^{-y^{2}/2}  $, $ h_{2}(-y) = h_{2}(y) $ even function so $ h_{2}(y) $ is of the from $ Ay^{2} + B $. By orthogonality,

\begin{align*}
0 & = (\psi_{0},\psi_{2}) \\
& = \int_{- \infty}^{\infty} \psi_{0}^{*} \psi_{2} \; \d y  \\
& = \int_{- \infty}^{\infty} a_{0} \left( Ay^{2} + B \right)e^{- y^{2} / 2}    \; \d y \\
& = a_{0} \left( A \sqrt{2 \pi} + B \sqrt{2 \pi}  \right) 
\end{align*}

Thus $ A = - B $, and $ \psi_{2}(x) = a_2(y^{2} - 1) $ for some constant $ a_{2} $. Similarly, can write $ h_{3} = C y^{3} + D y $ as $ h_{3} $ odd function. Letting $ h_{1}(y) = a_{1} y $ for some constant $ a_{1} $ we have:

\begin{align*}
0 & = (\psi_{1},\psi_{3}) \\
& = \int_{- \infty}^{\infty} \psi_{1}^{*} \psi_{3} \; \d y  \\
& = \int_{- \infty}^{\infty} a_{1} y \left( Cy^{3} + D y \right)e^{- y^{2} / 2}    \; \d y \\
& = a_{1} \left( 3 C   \sqrt{2 \pi} + B \sqrt{2 \pi}  \right) 
\end{align*}


Thus $ B = - 3 C $, and we can write $ \psi_{3}(x) = a_{3}(  y^{3} - 3 y ) $. 

Next, if the initial state can be written as $ \Psi(x,0) = \sum_{n=0}^{\infty} c_{n} \psi_{n}(x)  $, then

\begin{align*}
\Psi(x,t) & = \sum_{n=0}^{\infty} c_{n} \psi_{n}(x) e^{-iE_{n}t / \hbar} \\
& = \sum_{n=0}^{\infty} c_{n} \psi_{n}(x) e^{-i(n + \frac{1}{2}) \omega t} \\
& = 
\end{align*}



\section{QUESTION 5}

SE is

\[ - \frac{\hbar^{2}}{2m} \psi''(x) + V(x) \psi(x) = E \psi(x) \]

Probability current given by

\[ J(x) = - \frac{i \hbar }{2m} (\psi^{*} \psi' - (\psi^{*})' \psi )  \]

Differentiating with respect to $ x $,

\begin{align*}
\frac{\d J}{\d x}  & = - \frac{i \hbar }{2m}\left[ (\psi^{*})'\psi' + \psi^{*} \psi'' - (\psi^{*})'' \psi - (\psi^{*})' \psi'  \right]     \\
& =  - \frac{i \hbar }{2m}\left[ \psi^{*} \psi'' - (\psi^{*})'' \psi \right]     \\
& = - \frac{i \hbar }{2m}\left[ \psi^{*} \left(  - \frac{2m}{\hbar^{2}}(E - V) \psi \right)  - \left(  - \frac{2m}{\hbar^{2}}(E - V) \psi^{*} \right) \psi \right] \\
& = 0
\end{align*}

Probability current as $ x \to - \infty $, $ \psi(x) \sim e^{ikx} + Be^{-ikx} $ given by:

\begin{align*}
J & = - \frac{i \hbar }{2m} \left[   ( e^{-ikx} + B^{*} e^{ikx} )(ik e^{ikx} - i k B e^{-ikx}  )       - ( - i k e^{-i k x} + i k B^{*} e^{ikx}    )   ( e^{ikx} + Be^{-ikx}) \right] \\
& = - \frac{i \hbar }{2m} \left[  ik - ikBe^{-2ikx} + ikB^{*} e^{2ikx} - i k | B |^{2}  - \left(  - ik - ik B e^{-2ikx} + i k B^{*} e^{2ikx} + i k | B |^{2}     \right)    \right] \\
& = - \frac{i \hbar }{2m} \left[  2ik  - 2i k | B |^{2}    \right] \\
& =  \frac{\hbar k}{m} ( 1  - | B |^{2} )
\end{align*}

Probability current as $ x \to \infty $, $ \psi(x) C e^{ikx} $ given by:

\begin{align*}
J & = - \frac{i \hbar }{2m} \left[   (C^{*} e^{-ikx}  )(ik C e^{ikx}  )   - ( - i k C^{*} e^{-i k x})(C e^{ikx})  \right] \\
& = - \frac{i \hbar }{2m} \left[ 2ik | C |^{2}  \right] \\
& = \frac{\hbar k}{m} | C |^{2} 
\end{align*}

As independent of $ x $ these two expressions are equal, thus $ | B |^{2} + | C |^{2} = 1 $





\section{QUESTION 6}

Take the potential to be 

\begin{center}
	\begin{tikzpicture}
	\draw [->] (-3, 0) -- (5.5, 0) node [right] {$x$};
	\draw [->] (2, 0) -- (2, 3) node [above] {$V$};
	\node [left] at (2, 2) {$U$};
	\node [below] at (2, 0) {$0$};
	
	\node [below] at (3, 0) {$a$};
	\draw [mred, semithick] (-3, 0) -- (2, 0) -- (2, 2) -- (3, 2) -- (3, 0) -- (5, 0);
	\end{tikzpicture}
\end{center}

\[ V(x) = \begin{cases} U  & \text{ if } 0 < x < a \\ 0  & \text{ othewise } \end{cases} \]

where $ U = 2E $. Set the constant

\[
E = \frac{\hbar^2 k^2}{2m}
\]

Then the Schr\"odinger equations become
\begin{align*}
\psi'' + k^2 \psi &= 0 && x < 0\\
\psi'' - k^2 \psi &= 0 && 0 < x < a\\
\psi'' + k^2 \psi &= 0 && x > a
\end{align*}
So we get
\begin{align*}
\psi &= I e^{ikx} + Re^{-ikx}&& x < 0\\
\psi &= Ae^{k x} + Be^{-k x}&& 0 < x < a\\
\psi &= Te^{ikx} && x > a
\end{align*}
(no $ e^{-ikx} $ term $ \iff $ no particles sent from right)

Matching $\psi$ and $\psi'$ at $x = 0$ and $a$ gives the equations
\begin{align*}
I + R &= A + B\\
ik(I - R) &= k (A - B)\\
A e^{k a} + Be^{- k a} &= Te^{ika}\\
k(Ae^{k a} - Be^{k a}) &= ik Te^{ika}.
\end{align*}
We can solve these to obtain
\begin{align*}
I + \frac{k - ik}{k + ik}R &= Te^{ika} e^{-k a}\\
I + \frac{k + ik}{k - ik}R &= Te^{ika} e^{k a}.
\end{align*}
After \st{lots of} \emph{some} algebra, we obtain
\begin{align*}
T &= I e^{-ika}\left(\cosh k a\right)^{-1}
\end{align*}
To interpret this, we use the currents
\[
j = j_{\mathrm{inc}} + j_{\mathrm{ref}} = (|I|^2 - |R|^2) \frac{\hbar k}{m}
\]
for $x < 0$. On the other hand, we have
\[
j = j_{\mathrm{tr}} = |T|^2 \frac{\hbar k}{m}
\]
for $x > a$. We can use these to find the transmission probability, and it turns out to be
\[
P_{\mathrm{tr}} = \frac{|j_{\mathrm{tr}j}|}{|j_{\mathrm{inc}}|} = \frac{|T|^2}{|I|^2} = \left[ \cosh^{2} ka \right]^{-1}.
\]
This demonstrates \emph{quantum tunneling}. There is a non-zero probability that the particles can pass through the potential barrier even though it classically does not have enough energy.

\section{QUESTION 7}


Time independent SE is 

\[ - \frac{\hbar^{2}}{2m} \psi'' - U \delta(x) \psi = E \psi  \]




\section{QUESTION 8}

\begin{enumerate}
	\item \begin{align*}
	1 & = \int_{0}^{a} | \Psi |^{2} \; \d x  \\
	& = C^{2} \int_{0}^{a} x^{2} (a-x)^{2} \; \d x\\
	& = C^{2} \frac{a^{5}}{30} 
	\end{align*}
	
	So $ C = \sqrt{30 } a^{-5/2} $
	
	\item 
	
\end{enumerate}


\section{QUESTION 9}


$ Q \psi_{n} = 0 \; \forall \;  n > 2 \Rightarrow  $ zero is an eigenvalue. We are also given

\[ Q \psi_{1} = \psi_{2}, Q \psi_{2} = \psi_{1} \]

Adding (and using linearity) gives $ Q(\psi_{1} + \psi_{2})  = \psi_{1} + \psi_{2} $, thus $ 1 $ is an eigenvalue of $ Q $. Similarly subtracting shows that $ Q (\psi_{1} - \psi_{2}) = - (\psi_{1} - \psi_{2}) $, ie. $ -1 $ is an eigenvalue. 

To find normalised eigenstates, 

\begin{align*}
1 & = \int | C |^{2} | \psi_{1} + \psi_{2} |^{2} \; \d x  \\
& = | C^{2} | \int  (\psi_{1} + \psi_{2})^{*} ( \psi_{1} + \psi_{2}) \; \d x\\
& = |  C^{2} | \int | \psi_{1} |^{2} + | \psi_{2} |^{2} \qquad ( \text{by orthoganilty of eigenstates})\\
& = 2 | C^{2} | \quad \text{ as } \psi_{n} \text{ normalised}
\end{align*}

Thus $ \chi_{+} = \frac{1}{\sqrt{2}} (\psi_{1} + \psi_{2})  $, and similarly $ \chi_{-} = \frac{1}{\sqrt{2}} ( \psi_{1} - \psi_{2})  $


\begin{align*}
\langle H \rangle_{\chi_{\pm}} & = ( \chi_{\pm}, H \chi_{\pm} )  \\
& = \frac{1}{\sqrt{2}} (  (\psi_{1} \pm \psi_{2}),  H( \psi_{1} \pm \psi_{2} ) \\
& = \frac{1}{\sqrt{2}} (  (\psi_{1}, H \psi_{1}) \pm  ( \psi_{2}, H \psi_{2} ) \\
& = \frac{1}{\sqrt{2}} ( E_{1} \pm E_{2} )
\end{align*}

Measurement axioms $ \Rightarrow  $ at time zero, $ Q $ is in state $ \chi_{+} $. 
Have

\[ \Psi(0) = \alpha_{+} \chi_{+} + \alpha_{-} \chi_{-} \qquad (\alpha_{\pm} = (\chi_{\pm},\Psi(0) )) \]

By linearity, the solution of the t-dep SE is

\begin{align*}
\Psi(t) & = \alpha_{+} \chi_{+} e^{- i E_{+} t / \hbar } + \alpha_{-} \chi_{-} e^{- i E_{-} t / \hbar }  \\
& = 
\end{align*}



\section{QUESTION 10}


\begin{align*}
\langle [H,A] \rangle_{\psi} & = \langle HA - AH \rangle_{\psi}  \\
& = (  \psi, (HA - AH)\psi ) \\
& = ( \psi, HA \psi) - ( \psi, AH \psi) \\
& = ( H \psi, A \psi) - ( \psi, AH \psi)
\end{align*}

\section{QUESTION 11}






\end{document}