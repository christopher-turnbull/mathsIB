\documentclass[a4paper]{article}
\usepackage{amsmath}
\def\npart {IB}
\def\nterm {Michaelmas}
\def\nyear {2017}
\def\nlecturer {?}
\def\ncourse {Quantum Mechanics Example Sheet 1}

\input{header}
\newtheorem*{soln}{Solution}

\renewcommand{\thesection}{}
\renewcommand{\thesubsection}{\arabic{section}.\arabic{subsection}}
\makeatletter
\def\@seccntformat#1{\csname #1ignore\expandafter\endcsname\csname the#1\endcsname\quad}
\let\sectionignore\@gobbletwo
\let\latex@numberline\numberline
\def\numberline#1{\if\relax#1\relax\else\latex@numberline{#1}\fi}
\makeatother


\begin{document}
	
\maketitle

\section{QUESTION 1}

The first electron has wavelength $ \lambda_{1} = 3 \times 10^{-7} $ m, and moves at the speed of light, so frequency $ \nu_{1} $ is given by

\[ \nu_{1} = \frac{c}{\lambda_{1}} = \frac{3.00 \times 10^{8}}{3 \times 10^{-7}} = 1 \times 10^{15} \text{ s}^{-1}  \]

Similarly $ \nu_{2} =  0.6 \times 10^{15} $. If $ W $ is the minimum energy needed to liberate an electron from Potassium then

\[ K_{1} = h \nu_{1} - W \]
\[  K_{2} = h \nu_{2} - W \]

where $ K_{1},K_{2} $ are the maximum kinetic energy of the liberated electrons. 

Thus the value of $ h $ is given by

\[ h = \frac{K_{1} - K_{2}}{\nu_{1} - \nu_{2}} = \frac{1.6 \times (1.60 \times 10^{-19})}{0.4 \times 10^{15}} = 6.4 \times 10^{-34}\]

Thus 


\begin{align*}
W & = h\nu_{1} - K_{1} \\
& = 6.4 \times 10^{-19} - 2.1 \times (1.60 \times 10^{-19}) \\
& = 3.04 \times 10^{-19} \text{ J} \\
& = 1.9 \text{ eV}
\end{align*}

\section{QUESTION 2}

Let the light have energy flux $ E = 10^{-10} $, with the wavelength $ \lambda = 5 \times 10^{-7} $. These are related by

\[ E = h \nu = \frac{h c}{\lambda}  \]





\section{QUESTION 3}
\section{QUESTION 4}
\section{QUESTION 5}






\end{document}