\documentclass[a4paper]{article}
\usepackage{amsmath}
\def\npart {IB}
\def\nterm {Michaelmas}
\def\nyear {2017}
\def\nlecturer {Dr Warnick}
\def\ncourse {Quantum Mechanics Example Sheet 1}

% Imports
\ifx \nauthor\undefined
  \def\nauthor{Christopher Turnbull}
\else
\fi

\author{Supervised by \nlecturer \\\small Solutions presented by \nauthor}
\date{\nterm\ \nyear}

\usepackage{alltt}
\usepackage{amsfonts}
\usepackage{amsmath}
\usepackage{amssymb}
\usepackage{amsthm}
\usepackage{booktabs}
\usepackage{caption}
\usepackage{enumitem}
\usepackage{fancyhdr}
\usepackage{graphicx}
\usepackage{mathdots}
\usepackage{mathtools}
\usepackage{microtype}
\usepackage{multirow}
\usepackage{pdflscape}
\usepackage{pgfplots}
\usepackage{siunitx}
\usepackage{slashed}
\usepackage{tabularx}
\usepackage{tikz}
\usepackage{tkz-euclide}
\usepackage[normalem]{ulem}
\usepackage[all]{xy}
\usepackage{imakeidx}

\makeindex[intoc, title=Index]
\indexsetup{othercode={\lhead{\emph{Index}}}}

\ifx \nextra \undefined
  \usepackage[pdftex,
    hidelinks,
    pdfauthor={Christopher Turnbull},
    pdfsubject={Cambridge Maths Notes: Part \npart\ - \ncourse},
    pdftitle={Part \npart\ - \ncourse},
  pdfkeywords={Cambridge Mathematics Maths Math \npart\ \nterm\ \nyear\ \ncourse}]{hyperref}
  \title{Part \npart\ --- \ncourse}
\else
  \usepackage[pdftex,
    hidelinks,
    pdfauthor={Christopher Turnbull},
    pdfsubject={Cambridge Maths Notes: Part \npart\ - \ncourse\ (\nextra)},
    pdftitle={Part \npart\ - \ncourse\ (\nextra)},
  pdfkeywords={Cambridge Mathematics Maths Math \npart\ \nterm\ \nyear\ \ncourse\ \nextra}]{hyperref}

  \title{Part \npart\ --- \ncourse \\ {\Large \nextra}}
  \renewcommand\printindex{}
\fi

\pgfplotsset{compat=1.12}

\pagestyle{fancyplain}
\lhead{\emph{\nouppercase{\leftmark}}}
\ifx \nextra \undefined
  \rhead{
    \ifnum\thepage=1
    \else
      \npart\ \ncourse
    \fi}
\else
  \rhead{
    \ifnum\thepage=1
    \else
      \npart\ \ncourse\ (\nextra)
    \fi}
\fi
\usetikzlibrary{arrows.meta}
\usetikzlibrary{decorations.markings}
\usetikzlibrary{decorations.pathmorphing}
\usetikzlibrary{positioning}
\usetikzlibrary{fadings}
\usetikzlibrary{intersections}
\usetikzlibrary{cd}

\newcommand*{\Cdot}{{\raisebox{-0.25ex}{\scalebox{1.5}{$\cdot$}}}}
\newcommand {\pd}[2][ ]{
  \ifx #1 { }
    \frac{\partial}{\partial #2}
  \else
    \frac{\partial^{#1}}{\partial #2^{#1}}
  \fi
}
\ifx \nhtml \undefined
\else
  \renewcommand\printindex{}
  \makeatletter
  \DisableLigatures[f]{family = *}
  \let\Contentsline\contentsline
  \renewcommand\contentsline[3]{\Contentsline{#1}{#2}{}}
  \renewcommand{\@dotsep}{10000}
  \newlength\currentparindent
  \setlength\currentparindent\parindent

  \newcommand\@minipagerestore{\setlength{\parindent}{\currentparindent}}
  \usepackage[active,tightpage,pdftex]{preview}
  \renewcommand{\PreviewBorder}{0.1cm}

  \newenvironment{stretchpage}%
  {\begin{preview}\begin{minipage}{\hsize}}%
    {\end{minipage}\end{preview}}
  \AtBeginDocument{\begin{stretchpage}}
  \AtEndDocument{\end{stretchpage}}

  \newcommand{\@@newpage}{\end{stretchpage}\begin{stretchpage}}

  \let\@real@section\section
  \renewcommand{\section}{\@@newpage\@real@section}
  \let\@real@subsection\subsection
  \renewcommand{\subsection}{\@@newpage\@real@subsection}
  \makeatother
\fi

% Theorems
\theoremstyle{definition}
\newtheorem*{aim}{Aim}
\newtheorem*{axiom}{Axiom}
\newtheorem*{claim}{Claim}
\newtheorem*{cor}{Corollary}
\newtheorem*{conjecture}{Conjecture}
\newtheorem*{defi}{Definition}
\newtheorem*{eg}{Example}
\newtheorem*{ex}{Exercise}
\newtheorem*{fact}{Fact}
\newtheorem*{law}{Law}
\newtheorem*{lemma}{Lemma}
\newtheorem*{notation}{Notation}
\newtheorem*{prop}{Proposition}
\newtheorem*{soln}{Solution}
\newtheorem*{thm}{Theorem}

\newtheorem*{remark}{Remark}
\newtheorem*{warning}{Warning}
\newtheorem*{exercise}{Exercise}

\newtheorem{nthm}{Theorem}[section]
\newtheorem{nlemma}[nthm]{Lemma}
\newtheorem{nprop}[nthm]{Proposition}
\newtheorem{ncor}[nthm]{Corollary}


\renewcommand{\labelitemi}{--}
\renewcommand{\labelitemii}{$\circ$}
\renewcommand{\labelenumi}{(\roman{*})}

\let\stdsection\section
\renewcommand\section{\newpage\stdsection}

% Strike through
\def\st{\bgroup \ULdepth=-.55ex \ULset}

% Maths symbols
\newcommand{\abs}[1]{\left\lvert #1\right\rvert}
\newcommand\ad{\mathrm{ad}}
\newcommand\AND{\mathsf{AND}}
\newcommand\Art{\mathrm{Art}}
\newcommand{\Bilin}{\mathrm{Bilin}}
\newcommand{\bket}[1]{\left\lvert #1\right\rangle}
\newcommand{\B}{\mathcal{B}}
\newcommand{\bolds}[1]{{\bfseries #1}}
\newcommand{\brak}[1]{\left\langle #1 \right\rvert}
\newcommand{\braket}[2]{\left\langle #1\middle\vert #2 \right\rangle}
\newcommand{\bra}{\langle}
\newcommand{\cat}[1]{\mathsf{#1}}
\newcommand{\C}{\mathbb{C}}
\newcommand{\CP}{\mathbb{CP}}
\newcommand{\cU}{\mathcal{U}}
\newcommand{\Der}{\mathrm{Der}}
\newcommand{\D}{\mathrm{D}}
\newcommand{\dR}{\mathrm{dR}}
\newcommand{\E}{\mathbb{E}}
\newcommand{\F}{\mathbb{F}}
\newcommand{\Frob}{\mathrm{Frob}}
\newcommand{\GG}{\mathbb{G}}
\newcommand{\gl}{\mathfrak{gl}}
\newcommand{\GL}{\mathrm{GL}}
\newcommand{\G}{\mathcal{G}}
\newcommand{\Gr}{\mathrm{Gr}}
\newcommand{\haut}{\mathrm{ht}}
\newcommand{\Id}{\mathrm{Id}}
\newcommand{\ket}{\rangle}
\newcommand{\lie}[1]{\mathfrak{#1}}
\newcommand{\Mat}{\mathrm{Mat}}
\newcommand{\N}{\mathbb{N}}
\newcommand{\norm}[1]{\left\lVert #1\right\rVert}
\newcommand{\normalorder}[1]{\mathop{:}\nolimits\!#1\!\mathop{:}\nolimits}
\newcommand\NOT{\mathsf{NOT}}
\newcommand{\Oc}{\mathcal{O}}
\newcommand{\Or}{\mathrm{O}}
\newcommand\OR{\mathsf{OR}}
\newcommand{\ort}{\mathfrak{o}}
\newcommand{\PGL}{\mathrm{PGL}}
\newcommand{\ph}{\,\cdot\,}
\newcommand{\pr}{\mathrm{pr}}
\newcommand{\Prob}{\mathbb{P}}
\newcommand{\PSL}{\mathrm{PSL}}
\newcommand{\Ps}{\mathcal{P}}
\newcommand{\PSU}{\mathrm{PSU}}
\newcommand{\pt}{\mathrm{pt}}
\newcommand{\qeq}{\mathrel{``{=}"}}
\newcommand{\Q}{\mathbb{Q}}
\newcommand{\R}{\mathbb{R}}
\newcommand{\RP}{\mathbb{RP}}
\newcommand{\Rs}{\mathcal{R}}
\newcommand{\SL}{\mathrm{SL}}
\newcommand{\so}{\mathfrak{so}}
\newcommand{\SO}{\mathrm{SO}}
\newcommand{\Spin}{\mathrm{Spin}}
\newcommand{\Sp}{\mathrm{Sp}}
\newcommand{\su}{\mathfrak{su}}
\newcommand{\SU}{\mathrm{SU}}
\newcommand{\term}[1]{\emph{#1}\index{#1}}
\newcommand{\T}{\mathbb{T}}
\newcommand{\tv}[1]{|#1|}
\newcommand{\U}{\mathrm{U}}
\newcommand{\uu}{\mathfrak{u}}
\newcommand{\Vect}{\mathrm{Vect}}
\newcommand{\wsto}{\stackrel{\mathrm{w}^*}{\to}}
\newcommand{\wt}{\mathrm{wt}}
\newcommand{\wto}{\stackrel{\mathrm{w}}{\to}}
\newcommand{\Z}{\mathbb{Z}}
\renewcommand{\d}{\mathrm{d}}
\renewcommand{\H}{\mathbb{H}}
\renewcommand{\P}{\mathbb{P}}
\renewcommand{\sl}{\mathfrak{sl}}
\renewcommand{\vec}[1]{\boldsymbol{\mathbf{#1}}}
%\renewcommand{\F}{\mathcal{F}}

\let\Im\relax
\let\Re\relax

\DeclareMathOperator{\adj}{adj}
\DeclareMathOperator{\Ann}{Ann}
\DeclareMathOperator{\area}{area}
\DeclareMathOperator{\Aut}{Aut}
\DeclareMathOperator{\Bernoulli}{Bernoulli}
\DeclareMathOperator{\betaD}{beta}
\DeclareMathOperator{\bias}{bias}
\DeclareMathOperator{\binomial}{binomial}
\DeclareMathOperator{\card}{card}
\DeclareMathOperator{\ccl}{ccl}
\DeclareMathOperator{\Char}{char}
\DeclareMathOperator{\ch}{ch}
\DeclareMathOperator{\cl}{cl}
\DeclareMathOperator{\cls}{\overline{\mathrm{span}}}
\DeclareMathOperator{\conv}{conv}
\DeclareMathOperator{\corr}{corr}
\DeclareMathOperator{\cosec}{cosec}
\DeclareMathOperator{\cosech}{cosech}
\DeclareMathOperator{\cov}{cov}
\DeclareMathOperator{\covol}{covol}
\DeclareMathOperator{\diag}{diag}
\DeclareMathOperator{\diam}{diam}
\DeclareMathOperator{\Diff}{Diff}
\DeclareMathOperator{\disc}{disc}
\DeclareMathOperator{\dom}{dom}
\DeclareMathOperator{\End}{End}
\DeclareMathOperator{\energy}{energy}
\DeclareMathOperator{\erfc}{erfc}
\DeclareMathOperator{\erf}{erf}
\DeclareMathOperator*{\esssup}{ess\,sup}
\DeclareMathOperator{\ev}{ev}
\DeclareMathOperator{\Ext}{Ext}
\DeclareMathOperator{\Fit}{Fit}
\DeclareMathOperator{\fix}{fix}
\DeclareMathOperator{\Frac}{Frac}
\DeclareMathOperator{\Gal}{Gal}
\DeclareMathOperator{\gammaD}{gamma}
\DeclareMathOperator{\gr}{gr}
\DeclareMathOperator{\hcf}{hcf}
\DeclareMathOperator{\Hom}{Hom}
\DeclareMathOperator{\id}{id}
\DeclareMathOperator{\image}{image}
\DeclareMathOperator{\im}{im}
\DeclareMathOperator{\Im}{Im}
\DeclareMathOperator{\Ind}{Ind}
\DeclareMathOperator{\Int}{Int}
\DeclareMathOperator{\Isom}{Isom}
\DeclareMathOperator{\lcm}{lcm}
\DeclareMathOperator{\length}{length}
\DeclareMathOperator{\Lie}{Lie}
\DeclareMathOperator{\like}{like}
\DeclareMathOperator{\Lk}{Lk}
\DeclareMathOperator{\mse}{mse}
\DeclareMathOperator{\multinomial}{multinomial}
\DeclareMathOperator{\orb}{orb}
\DeclareMathOperator{\ord}{ord}
\DeclareMathOperator{\otp}{otp}
\DeclareMathOperator{\Poisson}{Poisson}
\DeclareMathOperator{\poly}{poly}
\DeclareMathOperator{\rank}{rank}
\DeclareMathOperator{\rel}{rel}
\DeclareMathOperator{\Re}{Re}
\DeclareMathOperator*{\res}{res}
\DeclareMathOperator{\Res}{Res}
\DeclareMathOperator{\rk}{rk}
\DeclareMathOperator{\Root}{Root}
\DeclareMathOperator{\sech}{sech}
\DeclareMathOperator{\sgn}{sgn}
\DeclareMathOperator{\spn}{span}
\DeclareMathOperator{\stab}{stab}
\DeclareMathOperator{\St}{St}
\DeclareMathOperator{\supp}{supp}
\DeclareMathOperator{\Syl}{Syl}
\DeclareMathOperator{\Sym}{Sym}
\DeclareMathOperator{\tr}{tr}
\DeclareMathOperator{\Tr}{Tr}
\DeclareMathOperator{\var}{var}
\DeclareMathOperator{\vol}{vol}

\pgfarrowsdeclarecombine{twolatex'}{twolatex'}{latex'}{latex'}{latex'}{latex'}
\tikzset{->/.style = {decoration={markings,
                                  mark=at position 1 with {\arrow[scale=2]{latex'}}},
                      postaction={decorate}}}
\tikzset{<-/.style = {decoration={markings,
                                  mark=at position 0 with {\arrowreversed[scale=2]{latex'}}},
                      postaction={decorate}}}
\tikzset{<->/.style = {decoration={markings,
                                   mark=at position 0 with {\arrowreversed[scale=2]{latex'}},
                                   mark=at position 1 with {\arrow[scale=2]{latex'}}},
                       postaction={decorate}}}
\tikzset{->-/.style = {decoration={markings,
                                   mark=at position #1 with {\arrow[scale=2]{latex'}}},
                       postaction={decorate}}}
\tikzset{-<-/.style = {decoration={markings,
                                   mark=at position #1 with {\arrowreversed[scale=2]{latex'}}},
                       postaction={decorate}}}
\tikzset{->>/.style = {decoration={markings,
                                  mark=at position 1 with {\arrow[scale=2]{latex'}}},
                      postaction={decorate}}}
\tikzset{<<-/.style = {decoration={markings,
                                  mark=at position 0 with {\arrowreversed[scale=2]{twolatex'}}},
                      postaction={decorate}}}
\tikzset{<<->>/.style = {decoration={markings,
                                   mark=at position 0 with {\arrowreversed[scale=2]{twolatex'}},
                                   mark=at position 1 with {\arrow[scale=2]{twolatex'}}},
                       postaction={decorate}}}
\tikzset{->>-/.style = {decoration={markings,
                                   mark=at position #1 with {\arrow[scale=2]{twolatex'}}},
                       postaction={decorate}}}
\tikzset{-<<-/.style = {decoration={markings,
                                   mark=at position #1 with {\arrowreversed[scale=2]{twolatex'}}},
                       postaction={decorate}}}

\tikzset{circ/.style = {fill, circle, inner sep = 0, minimum size = 3}}
\tikzset{mstate/.style={circle, draw, blue, text=black, minimum width=0.7cm}}

\tikzset{commutative diagrams/.cd,cdmap/.style={/tikz/column 1/.append style={anchor=base east},/tikz/column 2/.append style={anchor=base west},row sep=tiny}}

\definecolor{mblue}{rgb}{0.2, 0.3, 0.8}
\definecolor{morange}{rgb}{1, 0.5, 0}
\definecolor{mgreen}{rgb}{0.1, 0.4, 0.2}
\definecolor{mred}{rgb}{0.5, 0, 0}

\def\drawcirculararc(#1,#2)(#3,#4)(#5,#6){%
    \pgfmathsetmacro\cA{(#1*#1+#2*#2-#3*#3-#4*#4)/2}%
    \pgfmathsetmacro\cB{(#1*#1+#2*#2-#5*#5-#6*#6)/2}%
    \pgfmathsetmacro\cy{(\cB*(#1-#3)-\cA*(#1-#5))/%
                        ((#2-#6)*(#1-#3)-(#2-#4)*(#1-#5))}%
    \pgfmathsetmacro\cx{(\cA-\cy*(#2-#4))/(#1-#3)}%
    \pgfmathsetmacro\cr{sqrt((#1-\cx)*(#1-\cx)+(#2-\cy)*(#2-\cy))}%
    \pgfmathsetmacro\cA{atan2(#2-\cy,#1-\cx)}%
    \pgfmathsetmacro\cB{atan2(#6-\cy,#5-\cx)}%
    \pgfmathparse{\cB<\cA}%
    \ifnum\pgfmathresult=1
        \pgfmathsetmacro\cB{\cB+360}%
    \fi
    \draw (#1,#2) arc (\cA:\cB:\cr);%
}
\newcommand\getCoord[3]{\newdimen{#1}\newdimen{#2}\pgfextractx{#1}{\pgfpointanchor{#3}{center}}\pgfextracty{#2}{\pgfpointanchor{#3}{center}}}

\def\Xint#1{\mathchoice
   {\XXint\displaystyle\textstyle{#1}}%
   {\XXint\textstyle\scriptstyle{#1}}%
   {\XXint\scriptstyle\scriptscriptstyle{#1}}%
   {\XXint\scriptscriptstyle\scriptscriptstyle{#1}}%
   \!\int}
\def\XXint#1#2#3{{\setbox0=\hbox{$#1{#2#3}{\int}$}
     \vcenter{\hbox{$#2#3$}}\kern-.5\wd0}}
\def\ddashint{\Xint=}
\def\dashint{\Xint-}

\newcommand\separator{{\centering\rule{2cm}{0.2pt}\vspace{2pt}\par}}

\newenvironment{own}{\color{gray!70!black}}{}

\newcommand\makecenter[1]{\raisebox{-0.5\height}{#1}}
\newtheorem*{soln}{Solution}

\renewcommand{\thesection}{}
\renewcommand{\thesubsection}{\arabic{section}.\arabic{subsection}}
\makeatletter
\def\@seccntformat#1{\csname #1ignore\expandafter\endcsname\csname the#1\endcsname\quad}
\let\sectionignore\@gobbletwo
\let\latex@numberline\numberline
\def\numberline#1{\if\relax#1\relax\else\latex@numberline{#1}\fi}
\makeatother


\begin{document}
	
\maketitle

\section{QUESTION 1}

The first electron has wavelength $ \lambda_{1} = 3 \times 10^{-7} $ m, and moves at the speed of light, so frequency $ \nu_{1} $ is given by

\[ \nu_{1} = \frac{c}{\lambda_{1}} = \frac{3.00 \times 10^{8}}{3 \times 10^{-7}} = 1 \times 10^{15} \text{ s}^{-1}  \]

Similarly $ \nu_{2} =  0.6 \times 10^{15} $. If $ W $ is the minimum energy needed to liberate an electron from Potassium then

\[ K_{1} = h \nu_{1} - W \]
\[  K_{2} = h \nu_{2} - W \]

where $ K_{1},K_{2} $ are the maximum kinetic energy of the liberated electrons. 

Thus the value of $ h $ is given by

\[ h = \frac{K_{1} - K_{2}}{\nu_{1} - \nu_{2}} = \frac{1.6 \times (1.60 \times 10^{-19})}{0.4 \times 10^{15}} = 6.4 \times 10^{-34}\]

Thus 


\begin{align*}
W & = h\nu_{1} - K_{1} \\
& = 6.4 \times 10^{-19} - 2.1 \times (1.60 \times 10^{-19}) \\
& = 3.04 \times 10^{-19} \text{ J} \\
& = 1.9 \text{ eV}
\end{align*}

\section{QUESTION 2}

Lol don't do to so many significant figures m8. only as many as you have in the question. 

Let the light have energy flux $ E = 10^{-10} \text{Jm}^{-2}\text{s}^{-1} $ , with the wavelength $ \lambda = 5 \times 10^{-7} $. The energy of one photon is given by

\begin{align*}
E_{p} & = \frac{h c}{\lambda} \\
& = \frac{6.63 \times 10^{-34} \times 3 \times 10^{8} }{5 \times 10^{-7}}\\
& = 3.987 \times 10^{-19} \text{ J}
\end{align*}

Take human eye to have area $ 1 \text{ cm}^{2} = 10^{-4} \text{ m}^{2}$, so energy flux $ E_{e} $ entering human eye is  $ E_{e} = 10^{-14} \text{Jm}^{-2}\text{s}^{-1} $.

Thus number of photons entering the eye $ N $ is given as

\[ N = \frac{E_{e}}{E_{p}} \approx 2.51 \times 10^{5}  \]





\section{QUESTION 3}

Classical equations of motion imply that the total energy $ E_{n} $ for an electron at level $ n $ ( $ n = 1,2,\cdots $) for the electron must be constant, and is given by

\[ E_{n} = \frac{1}{2} m v_{n}^{2} - \frac{e^{2}}{4 \pi \varepsilon_{0}} \frac{1}{r_{n}} \]

Resolving radial acceleration gives

\[ \frac{m v_{n}^{2}}{r_{n}}  =  \frac{e^{2}}{4 \pi \varepsilon_{0}} \frac{1}{r_{n}^{2}} \quad (1) \]

which simplifies are expression for energy levels to 

\[ E_{n} = - \frac{1}{2} m v_{n}^{2} \]

Next, angular momentum quantisation yields 
\[ m v_{n} r_{n} = n\hbar \quad (2) \]

Rearranging (1) and (2)

\[ r_{n} = \frac{e^{2}}{4 \pi \varepsilon_{0}} \frac{1}{m v_{n}^{2}} \qquad r_{n} = \frac{n \hbar}{m v_{n}} \]

Thus setting equal and solving for $ v_{n} $ gives

\[ v_{n} = \frac{1}{n \hbar } \frac{e^{2}}{4 \pi \varepsilon_{0}} = c \alpha \frac{1}{n} \]

where $ \alpha = e^{2} / 4 \pi \varepsilon_{0} \hbar c $ is the fine structure constant. 


Substituting this into the expression for energy gives

\begin{align*}
 E_{n} & = - \frac{1}{2} m c^{2} \alpha^{2} \frac{1}{n^{2}}
\end{align*}


\begin{enumerate}
	\item It is consistent. $ \alpha \approx \frac{1}{137} $, so the highest speed an electron can have is $ \frac{1}{137}  c$ (when $ n = 1 $, this decreases for larger $ n $), which is less that $ 1 \% $ of the speed of light.

	
	
	\item Suppose th electron makes a transition between levels $ n' $ and $ n $, (with $ n' > n $ say), accompanied by emission or a photon of frequency $ \nu $. Then
	
	\[ h \nu = E_{n'} - E_{n} = \frac{1}{2} m c^{2} \alpha^{2} \left(  \frac{1}{n^{2}} - \frac{1}{n'^{2}} \right)  \]
	
	  $ E = \frac{hc}{\lambda} $, so smallest wavelength $ \Rightarrow $ most amount of energy, which is emitted when the electron falls from `infinity' to level 1, ie. $ (1/n^{2} - 1/n'^{2} ) = 1 $. ie.
	  
	  \begin{align*}
	  \lambda & = \frac{hc}{E}  \\
	  & = \frac{2hc}{mc^{2}\alpha^{2} } \\
	  & = \frac{4\pi \hbar }{mc \alpha^{2}} \\
	  & =  \frac{e^{2}}{\varepsilon_{0} c}\frac{4\pi \varepsilon_{0} \hbar c }{e^{2}} \frac{1}{mc \alpha^{2}} \\
	  & = \frac{e^{2}}{\varepsilon_{0} m c^{2} \alpha }
	  \end{align*}
	  
	  Bohr radius is given by:
	  
	  \[ r_{1} = \frac{4 \pi \varepsilon_{0} \hbar^{2}}{m e^{2}} \]
	
\end{enumerate}








\section{QUESTION 4}

Bohr radius $ r_{1} $ and corresponding ($ n=1 $) radius $ r_{1}' $ of muon are given by 

\[ r_{1} = \frac{\hbar}{m_{e} c \alpha} \qquad r_{1}'  = \frac{\hbar}{m_{m} c \alpha} \]

where $ m_{e} $ is the mass of the electron, and $ m_{m} = 207 m_{e} $ is the mass of the muon.

So the radius the $ n=1 $ state of muonic Hydrogen is 207 times smaller than normal Hydrogen.

\section{QUESTION 5}

Given $ \psi_{0}(x) = C_{0} e^{-x^{2}/2\alpha} $, we calculate 

\[ \psi_{0}'(x) = - \frac{x}{\alpha} \psi_{0}(x) \text{ and } \psi_{0}''(x) = - \frac{1}{\alpha} \psi_{0}(x) + \frac{x^{2}}{\alpha^{2}} \psi_{0}(x) \]


Substituting into the time-indep SE gives

\[ - \frac{\hbar^{2}}{2m} \left(  -\frac{1}{\alpha} + \frac{x^{2}}{\alpha^{2}} \right) \psi_{0}  + \frac{1}{2} K x^{2} \psi_{0} = E_{0} \psi_{0} \]

Comparing constants and $ x^{2} $ coefficients respectively

\[ \frac{\hbar^{2}}{2m\alpha} = E_{0} \quad \text{ and } \quad \frac{\hbar^{2}}{2m} \frac{1}{\alpha^{2}} = \frac{1}{2} K \]

Thus $ \alpha = \sqrt{\frac{\hbar^{2}}{Km}} $ and hence energy eigenvalue $ E_{0}  = \frac{\hbar}{2} \sqrt{ \frac{K}{m}} $.

Similarly, given $ \psi_{1}(x) = C_{1}x e^{-x^{2}/2\alpha} = x \phi(x)$, where $ \phi(x) =C_{1} e^{-x^{2}/2\alpha}  $,

\[ \psi_{1}'(x) = \phi(x) - \frac{x^{2}}{\alpha} \phi(x) \quad \text{ and } \quad \psi_{1}''(x) = \frac{-x}{\alpha} \phi(x) - \frac{2x}{\alpha} \phi(x) + \frac{x^{3}}{\alpha^{2}} \phi(x)  \]

Substituting into time-indep SE yields,

\[ - \frac{\hbar^{2}}{2m} \left(  -\frac{3x}{\alpha} + \frac{x^{3}}{\alpha^{2}} \right) \phi(x)  + \frac{1}{2} K x^{2} x \phi(x) = E_{1} x \phi(x) \]

Comparing $ x $ and $ x^{3} $ coefficients respectively,

\[  \frac{3\hbar^{2}}{2m\alpha} = E_{1}  \quad \text{ and } \quad  \frac{\hbar^{2}}{2m} \frac{1}{\alpha^{2}} = \frac{1}{2} K    \]

Hence as before $ \alpha = \sqrt{\frac{\hbar^{2}}{Km}} $ and $ E_{1} = 3 E_{0} = \frac{3 \hbar}{2} \sqrt{ \frac{K}{m}} $

\begin{center}
	%label axis x,y as x, |psi(x)|^{2}
	\begin{tikzpicture}[yscale=0.3]
	\draw [->] (-3, 0) -- (3, 0) node [right] {$x$};
	\draw [->](0, -9)  -- (0, 9) node [above] {$ \psi_0(x) $};
	\draw [domain=-3:3,samples=50, mblue] plot (\x, {\x *\x});
	\end{tikzpicture}
\end{center}


\begin{center}
	%label axis x,y as x, |psi(x)|^{2}
	\begin{tikzpicture}[yscale=3.5]
	\draw [->] (-3, 0) -- (3, 0) node [right] {$x$};
	\draw [->](0, -1.1)  -- (0, 1.1) node [above] {$ \psi_1(x) $};
	\draw [mred] (-0.5, 0)  -- (-0.5, -0.5) node [below] {$  - \sqrt{\alpha} $};
	\draw (0.5, 0)  -- (0.5, 0.5) node [above] {$ \sqrt{\alpha} $};
	\draw [domain=-3:3,samples=50, mblue] plot (\x, {2.5*\x*exp(-2*\x *\x)});
	\end{tikzpicture}
\end{center}

\section{QUESTION 6}

Wavefunction $ \Psi(x,t) $ under one-dimensional harmonic oscillator potential $ V(x) = \frac{1}{2}Kx^{2} $ has time-dependent Schr\"odinger equation:

\[ i \hbar \frac{\partial \Psi }{\partial t} = - \frac{\hbar^{2}}{2m} \frac{\partial^{2} \Psi }{\partial x^{2}} + \frac{1}{2}Kx^{2} \Psi \]

For separable $ \Psi(x,t) = \psi(x)f(t) $, have solutions of type

\[ \Psi(x,t) = \psi(x)e^{-iEt/\hbar} \quad \text{ with } \quad H \psi = E \psi \]

\begin{enumerate}
	\item $ \Psi(x,0) = \psi_{0}(x) \Rightarrow \psi(x) = \psi_{0}(x)$ and $ E = \frac{\hbar}{2}\sqrt{\frac{K}{m}} $ from question 5, so we have 
	
	\[ \Psi(x,t) = C_{0} e^{-x^{2}/2\alpha} \exp\left( - \frac{i}{2}\sqrt{\frac{K}{m}} t \right)  \] 
	
	For some normalisation constant $ C_{0} $
	
	\item Similarly $ \psi(x) = \psi_{1}(x) $ and again using results for $ E $ from question 5,
	
	\[ \Psi(x,t) = C_{1} x e^{-x^{2}/2\alpha} \exp\left( - \frac{3i}{2}\sqrt{\frac{K}{m}} t \right)  \] 
	
	For some normalisation constant $ C_{1} $
	
	\item For $ \Psi(x,0)  = \frac{1}{2}  ( \sqrt{3} \psi_{0}(x) - i \psi_{1}(x)) = \psi(x) $, and solving $ H \psi = E \psi $ for $ E $, $ \psi_{0} $ part only has constant and $ x^{2} $ terms, whereas $ \psi_{1} $ only contains $ x $ and $ x^{3} $ terms. Thus from parts (i) and (ii) we have
	
	
	\begin{align*}
	E & = \frac{1}{2}(\sqrt{3} E_{0} - i E_{1} )  \\
	& = \frac{\sqrt{3} - 3i}{2} \left( \frac{\hbar}{2} \sqrt{\frac{K}{m}} \right) 
	\end{align*}
	
	Thus \[ \Psi(x,t) =  \frac{1}{2}  ( \sqrt{3} \psi_{0}(x) - i \psi_{1}(x)) \exp \left( - \frac{3 - i\sqrt{3}}{4} \sqrt{\frac{K}{m}} t   \right)  \]
 	
\end{enumerate}


\section{QUESTION 7}

Note that $ \Psi(x,t) = C \gamma(t)^{-1/2} \exp(-x^{2}/2\gamma(t)) $ is not separable, $ V = 0 $.


$ t $ -dep SE 

\[ i \hbar \frac{\partial \Psi }{\partial t} = - \frac{\hbar^{2}}{2m} \frac{\partial^{2} \Psi }{\partial x^{2}} + \frac{1}{2}Kx^{2} \Psi \]

Calculating the partial derivatives

\begin{align*}
\frac{\partial \Psi }{\partial t}& = -\frac{1}{2} \gamma' C \gamma^{-3/2} \exp(-x^{2}/2\gamma)  + C \gamma^{-1/2}  \left(\frac{x^{2}}{2 \gamma^{2}} \gamma' \exp(-x^{2}/2\gamma) \right)    \\
& = \left( - \frac{1}{2} \frac{\gamma'}{\gamma} + \gamma' \frac{x^{2}}{2 \gamma^{2}} \right) C \gamma^{-1/2} \exp(-x^{2}/2\gamma) \\
& = \left( - \frac{1}{2} \frac{\gamma'}{\gamma} + \gamma' \frac{x^{2}}{2 \gamma^{2}} \right) \Psi
\end{align*}  
and 
\begin{align*}
\frac{\partial^{2} \Psi }{\partial x^{2}} & = \frac{\partial }{\partial x} \left[ -  \frac{x}{\gamma} \Psi \right] \\
& =  -  \frac{1}{\gamma} \Psi + \frac{x^{2}}{\gamma^{2}} \Psi   \\
\end{align*}      

So t-dep SE becomes

\[ i \hbar \left( - \frac{1}{2} \frac{\gamma'}{\gamma} + \gamma '\frac{x^{2}}{2 \gamma^{2}} \right) \Psi = - \frac{\hbar^{2}}{2m} \left(   -\frac{1}{\gamma} + \frac{x^{2}}{\gamma^{2}} \right) \Psi  \]    

Comparing constant and $ x^{2} $ coefficients repectively,        

\[ - i \hbar \frac{1}{2} \frac{\gamma'}{\gamma} = \frac{\hbar^{2}}{2m} \frac{1}{\gamma} \quad \text{ and } \quad i\hbar \frac{\gamma'}{2\gamma^{2}} = - \frac{\hbar^{2}}{2m} \frac{1}{\gamma^{2}}   \]

Both reveal that



\[ \gamma' = \frac{i \hbar}{m} \quad \text{ and hence } \quad \gamma = \alpha + \frac{i \hbar}{m} t  \]

where $ \alpha = \gamma(0) $


Probability density given as 

\begin{align*}
|   \Psi(x,t)  |^{2} & = \frac{| C |^{2}}{| \gamma(t) |} e^{- \alpha x^{2} / | \gamma(t) |^{2} } \\
\end{align*}


localised around $ x = 0 $ on scale 

\[ \frac{| \gamma(t) |}{\sqrt{\alpha}} \qquad \text{ solution "diffuses"}\]

how quickly does solution ``diffuse''?

Time scale $ m \alpha / \hbar  $

For $ m = m_{e} $, and $ \sqrt{\alpha} = 10^{-12} $ m $ \Rightarrow $ time scale $ \sim 10^{-20} $ s

For $ m = 10^{-6} $ kg, $ \sqrt{\alpha} = 10^{-6} $ m, $ \Rightarrow $ time scale $ \sim 10^{16} $ s. 

\section{QUESTION 8}


\begin{enumerate}
	\item \[ H \psi_{1} = E \psi_{1} \]
	\[ H \psi_{2} = E \psi_{2} \]
	
	Time indep SE with $ V(x) \to 0 $ rapidly, is
	
	Time-indep SE is
	
	\[ - \frac{\hbar^{2}}{2m}\psi''(x)  = E \psi(x) \]
	
	Note that 
	
	\[ \det \begin{pmatrix}
	\psi_{1} & \psi_{2} \\
	\psi'_{1} & \psi_{2}'
	\end{pmatrix} = \psi_{1} \psi_{2}'  - \psi_{2} \psi_{1}'  \] 
	
	\item 
	\item
	
\end{enumerate}


\section{QUESTION 9}  

Time dependent SE is    

\[ i \hbar \frac{\partial \Psi }{\partial t} = - \frac{\hbar^{2}}{2m} \frac{\partial^{2} \Psi }{\partial x^{2}} + -U \delta(x) \Psi \]

Integrating from $ x - \varepsilon $ to $  x + \varepsilon $ gives

\[ \int_{x-\varepsilon}^{x+\varepsilon}  i \hbar \frac{\partial \Psi }{\partial t} \; \d x= - \frac{\hbar^{2}}{2m} \left[ \frac{\partial \Psi }{\partial x} \right]_{x - \varepsilon}^{x + \varepsilon} -  U \Psi  \]

Taking $ \frac{\partial \Psi }{\partial t} $ to be sufficiently smooth, LHS $ = 0 $ and we have

\[  \left[ \frac{\partial \Psi }{\partial x} \right]_{x - \varepsilon}^{x + \varepsilon} = -  \frac{2mU }{\hbar^{2}}  \Psi   \]



\section{QUESTION 10}

	\begin{center}
	\begin{tikzpicture}
	\draw [->] (-2.5, 0) -- (2.5, 0) node [right] {$x$};
	\draw [->] (0, -0.5) -- (0, 2) node [above] {$V(x)$};
	\draw [mred, semithick, dashed] (-1.5, 0) node [above] {$-a$} -- (-1.5, -1.5);
	\draw [mred, semithick, dashed] (1.5, 0) node [above] {$a$} -- (1.5, -1.5);
	\draw [mred, semithick] (-1.5, -1.5) -- (1.5, -1.5) node[right] {$ -  U $}  ;
	\end{tikzpicture}
\end{center}

Seek energy functions and eigenvalues given by

\[ -\frac{\hbar^{2}}{2m} \psi''  + V(x) \psi = E \psi \]

with energies in range $ -U < E < 0 $

SE becomes 

\[ \underbrace{-\frac{\hbar^{2}}{2m} \psi'' = (E+U) \psi}_{| x | < a} \text{ and } \underbrace{-\frac{\hbar^{2}}{2m} \psi'' = E \psi}_{| x | > a}  \]

Set 

\[ U + E = \frac{\hbar^{2}k^{2}}{2m} \text{ and } E = \frac{-\hbar^{2}\kappa^{2}}{2m}   \]

\[ k > 0 \qquad \text{ and } \qquad \kappa > 0 \]

then SE is 

\[ \psi''  + k^{2} \psi = 0 \text{ and } \psi'' - \kappa^{2} \psi = 0 \]

\[ | x | < a \qquad \text{ and } \qquad | x | > a \]

At $ x = \pm a $, $ \psi,\psi' $ continuous ($ \psi'' $ discontinuous, matching step in $ V(x) $ )

[ Integrate SE from $ a - \varepsilon $ to $ a + \varepsilon $, then provided $ U $, $ \psi $ bounded, find $ [\psi']_{a - \varepsilon}^{a + \varepsilon} \to 0 $ as $ \varepsilon \to 0^{+} $ ]

Consider \emph{odd parity} solutions, ie. those with $ \psi(-x) = -\psi(x) $,

\[ \psi = \begin{cases} A \sin k x  & \text{ if } | x | < a \\ Be^{-\kappa x} & \text{ if } x > a \end{cases} \]

Note the solution for $ x < -a $ fixed by parity. Matching at $ x = a $,

\[ \psi \text{ cts }: A \sin k a = B e^{-\kappa a} \]
\[ \psi' \text{ cts }: k A \cos k a = - B \kappa e^{-\kappa a} \]

These equations give same solution for $ A $ or $ B $ iff:

\[ \kappa \tan k a = - k  \]


To find when solutions exists it is convenient to set

\[ \xi = a k, \quad \eta = a \kappa \quad \text{ dimensionless and positive} \]

So 

\[ \eta = - \frac{\xi}{\tan \xi} \]

but also 

\[ \xi^{2} +\eta^{2}= \frac{2ma^{2}U}{\hbar^{2}} \quad \text{ from definitions of } k \text{ and } \kappa \]

Intersection of $ \nu = \xi \tan \xi $ with circle of $ (\text{radius})^{2} = \frac{2ma^{2} U}{\hbar^{2}} $. Have energy eigenstate for each point of intersection ($ a,U $ fixed parameters, determining $ \xi,\nu $ determines $ E $)

We can look for solutions by plotting these two equations. We first plot the curve $\eta =  - \frac{\xi}{\tan \xi}$:

\begin{center}
	\begin{tikzpicture}
	\draw [->] (0, 0) -- (6, 0) node [right] {$\xi$};
	\draw [->] (0, 0) -- (0, 4) node [right] {$\eta$};
	\end{tikzpicture}
\end{center}

The other equation is the equation of a circle. Depending on the size of the constant $2ma^2 U/\hbar^2$, there will be a different number of points of intersections.

Can see that circle must have radius $ \geq \frac{\pi}{2} $ for intersection; anything smaller will produce no intersections, ie. no intersections if

\[ 2ma^2 U/\hbar^2 < \left( \frac{\pi}{2}\right)^{2} \]

which rearranges to

\[  a U^{2}  < (\pi \hbar)^{2} / 8m \]

as required


\section{QUESTION 11}                                  

Potential $ V(x) = - \frac{\hbar^{2}}{2m} \sech^{2} x $

\begin{center}
	%draw lines 
	\begin{tikzpicture}[yscale=1.5]
	\draw [->] (-3, 0) -- (3, 0) node [right] {$x$};
	\draw [->](0, -1.3)   -- (0, 1.3) node [above] {$V(x)$};
	\draw [domain=-3:3,samples=50, mblue] plot (\x, - {exp(-\x * \x)});
	\end{tikzpicture}
\end{center}

Time-indep SE is

\[ - \frac{\hbar^{2}}{2m}\psi''(x)  - \frac{\hbar^{2}}{m} \sech^{2}(x) \psi(x) = E \psi(x) \]

which is

\[ -\psi''(x) - 2 \sech^{2}(x) \psi(x) =  \varepsilon \psi(x) \quad (*) \]

where $ \varepsilon = 2mE / \hbar^{2}  $.

\begin{align*}
A^{\dagger} A \psi & = A^{\dagger} \left[ \psi'(x) + \tanh(x) \psi(x)  \right]   \\
& = -\psi''(x) - \sech^{2}(x) \psi(x) - \tanh(x) \psi'(x) + \tanh(x) \psi'(x) + \tanh^{2}(x) \psi(x) \\
& = - \psi'' -2 \sech^{2}(x) \psi + \psi \qquad \text{ using } \tanh^{2}(x) = 1 - \sech^{2}(x)
\end{align*}

Hence, adding $ \psi(x) $ to both sides of $ (*) $ it is rewritten as

\[ A^{\dagger} A \psi = (\varepsilon + 1)\psi \]

\section{MEME}


\[ \frac{1}{2!} \frac{\partial^{2} f (x_{0},y_{0}) }{\partial y^{2} } (y-y_{0})^{2}  \]


\[  \qquad \]

\[ \frac{1}{2}  \partial_{yy} f (\mathbf{x_{0}}) (y-y_{0})^{2}  \]


\[ \qquad  \]

\[ \frac{1}{2} f_{yy}\Big|_{\mathbf{x}_{0}}  (\delta y)^{2} \]





\end{document}