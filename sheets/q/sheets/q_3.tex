\documentclass[a4paper]{article}
\usepackage{amsmath}
\def\npart {IB}
\def\nterm {Michaelmas}
\def\nyear {2017}
\def\nlecturer {Dr Warnick}
\def\ncourse {Quantum Mechanics Example Sheet 3}

% Imports
\ifx \nauthor\undefined
  \def\nauthor{Christopher Turnbull}
\else
\fi

\author{Supervised by \nlecturer \\\small Solutions presented by \nauthor}
\date{\nterm\ \nyear}

\usepackage{alltt}
\usepackage{amsfonts}
\usepackage{amsmath}
\usepackage{amssymb}
\usepackage{amsthm}
\usepackage{booktabs}
\usepackage{caption}
\usepackage{enumitem}
\usepackage{fancyhdr}
\usepackage{graphicx}
\usepackage{mathdots}
\usepackage{mathtools}
\usepackage{microtype}
\usepackage{multirow}
\usepackage{pdflscape}
\usepackage{pgfplots}
\usepackage{siunitx}
\usepackage{slashed}
\usepackage{tabularx}
\usepackage{tikz}
\usepackage{tkz-euclide}
\usepackage[normalem]{ulem}
\usepackage[all]{xy}
\usepackage{imakeidx}

\makeindex[intoc, title=Index]
\indexsetup{othercode={\lhead{\emph{Index}}}}

\ifx \nextra \undefined
  \usepackage[pdftex,
    hidelinks,
    pdfauthor={Christopher Turnbull},
    pdfsubject={Cambridge Maths Notes: Part \npart\ - \ncourse},
    pdftitle={Part \npart\ - \ncourse},
  pdfkeywords={Cambridge Mathematics Maths Math \npart\ \nterm\ \nyear\ \ncourse}]{hyperref}
  \title{Part \npart\ --- \ncourse}
\else
  \usepackage[pdftex,
    hidelinks,
    pdfauthor={Christopher Turnbull},
    pdfsubject={Cambridge Maths Notes: Part \npart\ - \ncourse\ (\nextra)},
    pdftitle={Part \npart\ - \ncourse\ (\nextra)},
  pdfkeywords={Cambridge Mathematics Maths Math \npart\ \nterm\ \nyear\ \ncourse\ \nextra}]{hyperref}

  \title{Part \npart\ --- \ncourse \\ {\Large \nextra}}
  \renewcommand\printindex{}
\fi

\pgfplotsset{compat=1.12}

\pagestyle{fancyplain}
\lhead{\emph{\nouppercase{\leftmark}}}
\ifx \nextra \undefined
  \rhead{
    \ifnum\thepage=1
    \else
      \npart\ \ncourse
    \fi}
\else
  \rhead{
    \ifnum\thepage=1
    \else
      \npart\ \ncourse\ (\nextra)
    \fi}
\fi
\usetikzlibrary{arrows.meta}
\usetikzlibrary{decorations.markings}
\usetikzlibrary{decorations.pathmorphing}
\usetikzlibrary{positioning}
\usetikzlibrary{fadings}
\usetikzlibrary{intersections}
\usetikzlibrary{cd}

\newcommand*{\Cdot}{{\raisebox{-0.25ex}{\scalebox{1.5}{$\cdot$}}}}
\newcommand {\pd}[2][ ]{
  \ifx #1 { }
    \frac{\partial}{\partial #2}
  \else
    \frac{\partial^{#1}}{\partial #2^{#1}}
  \fi
}
\ifx \nhtml \undefined
\else
  \renewcommand\printindex{}
  \makeatletter
  \DisableLigatures[f]{family = *}
  \let\Contentsline\contentsline
  \renewcommand\contentsline[3]{\Contentsline{#1}{#2}{}}
  \renewcommand{\@dotsep}{10000}
  \newlength\currentparindent
  \setlength\currentparindent\parindent

  \newcommand\@minipagerestore{\setlength{\parindent}{\currentparindent}}
  \usepackage[active,tightpage,pdftex]{preview}
  \renewcommand{\PreviewBorder}{0.1cm}

  \newenvironment{stretchpage}%
  {\begin{preview}\begin{minipage}{\hsize}}%
    {\end{minipage}\end{preview}}
  \AtBeginDocument{\begin{stretchpage}}
  \AtEndDocument{\end{stretchpage}}

  \newcommand{\@@newpage}{\end{stretchpage}\begin{stretchpage}}

  \let\@real@section\section
  \renewcommand{\section}{\@@newpage\@real@section}
  \let\@real@subsection\subsection
  \renewcommand{\subsection}{\@@newpage\@real@subsection}
  \makeatother
\fi

% Theorems
\theoremstyle{definition}
\newtheorem*{aim}{Aim}
\newtheorem*{axiom}{Axiom}
\newtheorem*{claim}{Claim}
\newtheorem*{cor}{Corollary}
\newtheorem*{conjecture}{Conjecture}
\newtheorem*{defi}{Definition}
\newtheorem*{eg}{Example}
\newtheorem*{ex}{Exercise}
\newtheorem*{fact}{Fact}
\newtheorem*{law}{Law}
\newtheorem*{lemma}{Lemma}
\newtheorem*{notation}{Notation}
\newtheorem*{prop}{Proposition}
\newtheorem*{soln}{Solution}
\newtheorem*{thm}{Theorem}

\newtheorem*{remark}{Remark}
\newtheorem*{warning}{Warning}
\newtheorem*{exercise}{Exercise}

\newtheorem{nthm}{Theorem}[section]
\newtheorem{nlemma}[nthm]{Lemma}
\newtheorem{nprop}[nthm]{Proposition}
\newtheorem{ncor}[nthm]{Corollary}


\renewcommand{\labelitemi}{--}
\renewcommand{\labelitemii}{$\circ$}
\renewcommand{\labelenumi}{(\roman{*})}

\let\stdsection\section
\renewcommand\section{\newpage\stdsection}

% Strike through
\def\st{\bgroup \ULdepth=-.55ex \ULset}

% Maths symbols
\newcommand{\abs}[1]{\left\lvert #1\right\rvert}
\newcommand\ad{\mathrm{ad}}
\newcommand\AND{\mathsf{AND}}
\newcommand\Art{\mathrm{Art}}
\newcommand{\Bilin}{\mathrm{Bilin}}
\newcommand{\bket}[1]{\left\lvert #1\right\rangle}
\newcommand{\B}{\mathcal{B}}
\newcommand{\bolds}[1]{{\bfseries #1}}
\newcommand{\brak}[1]{\left\langle #1 \right\rvert}
\newcommand{\braket}[2]{\left\langle #1\middle\vert #2 \right\rangle}
\newcommand{\bra}{\langle}
\newcommand{\cat}[1]{\mathsf{#1}}
\newcommand{\C}{\mathbb{C}}
\newcommand{\CP}{\mathbb{CP}}
\newcommand{\cU}{\mathcal{U}}
\newcommand{\Der}{\mathrm{Der}}
\newcommand{\D}{\mathrm{D}}
\newcommand{\dR}{\mathrm{dR}}
\newcommand{\E}{\mathbb{E}}
\newcommand{\F}{\mathbb{F}}
\newcommand{\Frob}{\mathrm{Frob}}
\newcommand{\GG}{\mathbb{G}}
\newcommand{\gl}{\mathfrak{gl}}
\newcommand{\GL}{\mathrm{GL}}
\newcommand{\G}{\mathcal{G}}
\newcommand{\Gr}{\mathrm{Gr}}
\newcommand{\haut}{\mathrm{ht}}
\newcommand{\Id}{\mathrm{Id}}
\newcommand{\ket}{\rangle}
\newcommand{\lie}[1]{\mathfrak{#1}}
\newcommand{\Mat}{\mathrm{Mat}}
\newcommand{\N}{\mathbb{N}}
\newcommand{\norm}[1]{\left\lVert #1\right\rVert}
\newcommand{\normalorder}[1]{\mathop{:}\nolimits\!#1\!\mathop{:}\nolimits}
\newcommand\NOT{\mathsf{NOT}}
\newcommand{\Oc}{\mathcal{O}}
\newcommand{\Or}{\mathrm{O}}
\newcommand\OR{\mathsf{OR}}
\newcommand{\ort}{\mathfrak{o}}
\newcommand{\PGL}{\mathrm{PGL}}
\newcommand{\ph}{\,\cdot\,}
\newcommand{\pr}{\mathrm{pr}}
\newcommand{\Prob}{\mathbb{P}}
\newcommand{\PSL}{\mathrm{PSL}}
\newcommand{\Ps}{\mathcal{P}}
\newcommand{\PSU}{\mathrm{PSU}}
\newcommand{\pt}{\mathrm{pt}}
\newcommand{\qeq}{\mathrel{``{=}"}}
\newcommand{\Q}{\mathbb{Q}}
\newcommand{\R}{\mathbb{R}}
\newcommand{\RP}{\mathbb{RP}}
\newcommand{\Rs}{\mathcal{R}}
\newcommand{\SL}{\mathrm{SL}}
\newcommand{\so}{\mathfrak{so}}
\newcommand{\SO}{\mathrm{SO}}
\newcommand{\Spin}{\mathrm{Spin}}
\newcommand{\Sp}{\mathrm{Sp}}
\newcommand{\su}{\mathfrak{su}}
\newcommand{\SU}{\mathrm{SU}}
\newcommand{\term}[1]{\emph{#1}\index{#1}}
\newcommand{\T}{\mathbb{T}}
\newcommand{\tv}[1]{|#1|}
\newcommand{\U}{\mathrm{U}}
\newcommand{\uu}{\mathfrak{u}}
\newcommand{\Vect}{\mathrm{Vect}}
\newcommand{\wsto}{\stackrel{\mathrm{w}^*}{\to}}
\newcommand{\wt}{\mathrm{wt}}
\newcommand{\wto}{\stackrel{\mathrm{w}}{\to}}
\newcommand{\Z}{\mathbb{Z}}
\renewcommand{\d}{\mathrm{d}}
\renewcommand{\H}{\mathbb{H}}
\renewcommand{\P}{\mathbb{P}}
\renewcommand{\sl}{\mathfrak{sl}}
\renewcommand{\vec}[1]{\boldsymbol{\mathbf{#1}}}
%\renewcommand{\F}{\mathcal{F}}

\let\Im\relax
\let\Re\relax

\DeclareMathOperator{\adj}{adj}
\DeclareMathOperator{\Ann}{Ann}
\DeclareMathOperator{\area}{area}
\DeclareMathOperator{\Aut}{Aut}
\DeclareMathOperator{\Bernoulli}{Bernoulli}
\DeclareMathOperator{\betaD}{beta}
\DeclareMathOperator{\bias}{bias}
\DeclareMathOperator{\binomial}{binomial}
\DeclareMathOperator{\card}{card}
\DeclareMathOperator{\ccl}{ccl}
\DeclareMathOperator{\Char}{char}
\DeclareMathOperator{\ch}{ch}
\DeclareMathOperator{\cl}{cl}
\DeclareMathOperator{\cls}{\overline{\mathrm{span}}}
\DeclareMathOperator{\conv}{conv}
\DeclareMathOperator{\corr}{corr}
\DeclareMathOperator{\cosec}{cosec}
\DeclareMathOperator{\cosech}{cosech}
\DeclareMathOperator{\cov}{cov}
\DeclareMathOperator{\covol}{covol}
\DeclareMathOperator{\diag}{diag}
\DeclareMathOperator{\diam}{diam}
\DeclareMathOperator{\Diff}{Diff}
\DeclareMathOperator{\disc}{disc}
\DeclareMathOperator{\dom}{dom}
\DeclareMathOperator{\End}{End}
\DeclareMathOperator{\energy}{energy}
\DeclareMathOperator{\erfc}{erfc}
\DeclareMathOperator{\erf}{erf}
\DeclareMathOperator*{\esssup}{ess\,sup}
\DeclareMathOperator{\ev}{ev}
\DeclareMathOperator{\Ext}{Ext}
\DeclareMathOperator{\Fit}{Fit}
\DeclareMathOperator{\fix}{fix}
\DeclareMathOperator{\Frac}{Frac}
\DeclareMathOperator{\Gal}{Gal}
\DeclareMathOperator{\gammaD}{gamma}
\DeclareMathOperator{\gr}{gr}
\DeclareMathOperator{\hcf}{hcf}
\DeclareMathOperator{\Hom}{Hom}
\DeclareMathOperator{\id}{id}
\DeclareMathOperator{\image}{image}
\DeclareMathOperator{\im}{im}
\DeclareMathOperator{\Im}{Im}
\DeclareMathOperator{\Ind}{Ind}
\DeclareMathOperator{\Int}{Int}
\DeclareMathOperator{\Isom}{Isom}
\DeclareMathOperator{\lcm}{lcm}
\DeclareMathOperator{\length}{length}
\DeclareMathOperator{\Lie}{Lie}
\DeclareMathOperator{\like}{like}
\DeclareMathOperator{\Lk}{Lk}
\DeclareMathOperator{\mse}{mse}
\DeclareMathOperator{\multinomial}{multinomial}
\DeclareMathOperator{\orb}{orb}
\DeclareMathOperator{\ord}{ord}
\DeclareMathOperator{\otp}{otp}
\DeclareMathOperator{\Poisson}{Poisson}
\DeclareMathOperator{\poly}{poly}
\DeclareMathOperator{\rank}{rank}
\DeclareMathOperator{\rel}{rel}
\DeclareMathOperator{\Re}{Re}
\DeclareMathOperator*{\res}{res}
\DeclareMathOperator{\Res}{Res}
\DeclareMathOperator{\rk}{rk}
\DeclareMathOperator{\Root}{Root}
\DeclareMathOperator{\sech}{sech}
\DeclareMathOperator{\sgn}{sgn}
\DeclareMathOperator{\spn}{span}
\DeclareMathOperator{\stab}{stab}
\DeclareMathOperator{\St}{St}
\DeclareMathOperator{\supp}{supp}
\DeclareMathOperator{\Syl}{Syl}
\DeclareMathOperator{\Sym}{Sym}
\DeclareMathOperator{\tr}{tr}
\DeclareMathOperator{\Tr}{Tr}
\DeclareMathOperator{\var}{var}
\DeclareMathOperator{\vol}{vol}

\pgfarrowsdeclarecombine{twolatex'}{twolatex'}{latex'}{latex'}{latex'}{latex'}
\tikzset{->/.style = {decoration={markings,
                                  mark=at position 1 with {\arrow[scale=2]{latex'}}},
                      postaction={decorate}}}
\tikzset{<-/.style = {decoration={markings,
                                  mark=at position 0 with {\arrowreversed[scale=2]{latex'}}},
                      postaction={decorate}}}
\tikzset{<->/.style = {decoration={markings,
                                   mark=at position 0 with {\arrowreversed[scale=2]{latex'}},
                                   mark=at position 1 with {\arrow[scale=2]{latex'}}},
                       postaction={decorate}}}
\tikzset{->-/.style = {decoration={markings,
                                   mark=at position #1 with {\arrow[scale=2]{latex'}}},
                       postaction={decorate}}}
\tikzset{-<-/.style = {decoration={markings,
                                   mark=at position #1 with {\arrowreversed[scale=2]{latex'}}},
                       postaction={decorate}}}
\tikzset{->>/.style = {decoration={markings,
                                  mark=at position 1 with {\arrow[scale=2]{latex'}}},
                      postaction={decorate}}}
\tikzset{<<-/.style = {decoration={markings,
                                  mark=at position 0 with {\arrowreversed[scale=2]{twolatex'}}},
                      postaction={decorate}}}
\tikzset{<<->>/.style = {decoration={markings,
                                   mark=at position 0 with {\arrowreversed[scale=2]{twolatex'}},
                                   mark=at position 1 with {\arrow[scale=2]{twolatex'}}},
                       postaction={decorate}}}
\tikzset{->>-/.style = {decoration={markings,
                                   mark=at position #1 with {\arrow[scale=2]{twolatex'}}},
                       postaction={decorate}}}
\tikzset{-<<-/.style = {decoration={markings,
                                   mark=at position #1 with {\arrowreversed[scale=2]{twolatex'}}},
                       postaction={decorate}}}

\tikzset{circ/.style = {fill, circle, inner sep = 0, minimum size = 3}}
\tikzset{mstate/.style={circle, draw, blue, text=black, minimum width=0.7cm}}

\tikzset{commutative diagrams/.cd,cdmap/.style={/tikz/column 1/.append style={anchor=base east},/tikz/column 2/.append style={anchor=base west},row sep=tiny}}

\definecolor{mblue}{rgb}{0.2, 0.3, 0.8}
\definecolor{morange}{rgb}{1, 0.5, 0}
\definecolor{mgreen}{rgb}{0.1, 0.4, 0.2}
\definecolor{mred}{rgb}{0.5, 0, 0}

\def\drawcirculararc(#1,#2)(#3,#4)(#5,#6){%
    \pgfmathsetmacro\cA{(#1*#1+#2*#2-#3*#3-#4*#4)/2}%
    \pgfmathsetmacro\cB{(#1*#1+#2*#2-#5*#5-#6*#6)/2}%
    \pgfmathsetmacro\cy{(\cB*(#1-#3)-\cA*(#1-#5))/%
                        ((#2-#6)*(#1-#3)-(#2-#4)*(#1-#5))}%
    \pgfmathsetmacro\cx{(\cA-\cy*(#2-#4))/(#1-#3)}%
    \pgfmathsetmacro\cr{sqrt((#1-\cx)*(#1-\cx)+(#2-\cy)*(#2-\cy))}%
    \pgfmathsetmacro\cA{atan2(#2-\cy,#1-\cx)}%
    \pgfmathsetmacro\cB{atan2(#6-\cy,#5-\cx)}%
    \pgfmathparse{\cB<\cA}%
    \ifnum\pgfmathresult=1
        \pgfmathsetmacro\cB{\cB+360}%
    \fi
    \draw (#1,#2) arc (\cA:\cB:\cr);%
}
\newcommand\getCoord[3]{\newdimen{#1}\newdimen{#2}\pgfextractx{#1}{\pgfpointanchor{#3}{center}}\pgfextracty{#2}{\pgfpointanchor{#3}{center}}}

\def\Xint#1{\mathchoice
   {\XXint\displaystyle\textstyle{#1}}%
   {\XXint\textstyle\scriptstyle{#1}}%
   {\XXint\scriptstyle\scriptscriptstyle{#1}}%
   {\XXint\scriptscriptstyle\scriptscriptstyle{#1}}%
   \!\int}
\def\XXint#1#2#3{{\setbox0=\hbox{$#1{#2#3}{\int}$}
     \vcenter{\hbox{$#2#3$}}\kern-.5\wd0}}
\def\ddashint{\Xint=}
\def\dashint{\Xint-}

\newcommand\separator{{\centering\rule{2cm}{0.2pt}\vspace{2pt}\par}}

\newenvironment{own}{\color{gray!70!black}}{}

\newcommand\makecenter[1]{\raisebox{-0.5\height}{#1}}
\newtheorem*{soln}{Solution}

\renewcommand{\thesection}{}
\renewcommand{\thesubsection}{\arabic{section}.\arabic{subsection}}
\makeatletter
\def\@seccntformat#1{\csname #1ignore\expandafter\endcsname\csname the#1\endcsname\quad}
\let\sectionignore\@gobbletwo
\let\latex@numberline\numberline
\def\numberline#1{\if\relax#1\relax\else\latex@numberline{#1}\fi}
\makeatother


\begin{document}
	
\maketitle

\section{QUESTION 1}

Time independent Schr\"odinger Equation is 

\[ - \frac{\hbar^{2}}{2m} \nabla^{2} \psi = E \psi \]

\[  - \frac{\hbar^{2}}{2m} \left(  \frac{X''}{X} +  \frac{Y''}{Y} +  \frac{Z''}{Z} \right)  = E  \]

Can split this into 3 equations, eg.



\[ - \frac{\hbar^{2}}{2m}\frac{X''}{X} = E_{1} \]
\[ X'' + k^{2}X = 0  \qquad \text{ where } E_{1} = \frac{\hbar^{2}k^{2}}{2m}  \]

Note we take $ E_{i} > 0 $, as boundary conditions mean $ E_{i} < 0 $ has no eigenstate solutions.

$ X(0) = X(a) = 0 \Rightarrow k = n_{1} \pi / a  $
Repeat for $ Y $ and $ Z $. 

\begin{align*}
 E & = E_{1} + E_{2} + E_{3} \\
  & = \frac{\hbar^{2} \pi^{2}}{2m} \left(  \frac{n_{1}^{2}}{a^{2}} + \frac{n_{2}^{2}}{b^{2}} + \frac{n_{3}^{2}}{c^{2}}  \right)
\end{align*}
 
 With $ a = b = c $, ground state is $  E = \frac{3 \hbar^{2} \pi^{2}}{2ma^{2}} $ where $ n_{1} = n_{2} = n_{3} = 1 $, and next when $ \sum_{i} n_{i} = 4 $ (which happens in 3 different ways) we have $  E = \frac{2\hbar^{2} \pi^{2}}{ma^{2}} $, so first excited state has degeneracy 3. 

\section{QUESTION 2}

Time independent Schr\"odinger Equation is 


\[ - \frac{\hbar^{2}}{2m} \nabla^{2} \psi  + \frac{1}{2} m \omega^{2} (x_{1}^{2} + x_{2}^{2} + x_{3}^{2}) \psi  = E \psi \]

The Hamiltonian splits into $ H = H_{1} + H_{2} + H_{3} $

Seek solutions of the form $ \psi = X(x_{1})Y(x_{2})Z(x_{3}) $.
Separating variables shows

\[  - \frac{\hbar^{2}}{2m} \left(  \frac{X''}{X} +  \frac{Y''}{Y} +  \frac{Z''}{Z} \right)  + \frac{1}{2} m \omega^{2} (x_{1}^{2} + x_{2}^{2} + x_{3}^{2}) = E  \]


As $ X $ cannot vary for fixed $ Y,Z $, we have

\[ - \frac{\hbar^{2}}{2m} \frac{X''}{X}  + \frac{1}{2} m \omega^{2} x_{1}^{2}  = E_{1} \]

This is the one dimensional harmonic oscillator equation; with eigenstates and eigenvalues

\[ X_{n_{1}}(x_{1}) = h_{n_{1}}(y_{1})   \exp \left(   - y_{1}^{2}/2 \right), \qquad E_{1} = \hbar \omega (n_{1} + \frac{1}{2})    \]

\[ y_{1} = \left( \left(  \frac{m \omega}{\hbar}  \right)^{1/2} x_{1} \right) \]

for $ n_{1} = 0,1,2,\cdots $

Similarly, recover that $ E_{i} = \hbar \omega (n_{i} + \frac{1}{2}) $

\begin{align*}
 E &= E_{1} + E_{2} + E_{3} \\
& = \hbar \omega \left(     n_{1} + n_{2} + n_{3} + \frac{3}{2} \right)
\end{align*}

where $ n_{i} = 0,1,2,\cdots $


To count the number of linearly independent eigenstates corresponding to energy $ E = (N + \frac{3}{2} ) \hbar \omega $, need $ n_{1} + n_{2} + n_{3} = N $. With $ n_{1} = 0 $, need $ n_{2} + n_{3} = N $, which can happen in $ N + 1 $ ways. Then $ n_{1} = 1 $, have $ N $ more states. So the total number of states is given by
\begin{align*}
\text{Degeneracy }& = (N + 1) + N + \cdots + 2 + 1  \\
& = (N + 2)(N + 1)/2
\end{align*}

Now have

\[ \psi(\mathbf{x}) = h_{n_{1}}(y_{1})h_{n_{2}}(y_{2})h_{n_{3}}(y_{3})  \exp \left(  -(y_{1}^{2} + y_{2}^{2} + y_{3}^{2})/2 \right) \]


Note $  \exp \left(  -(y_{1}^{2} + y_{2}^{2} + y_{3}^{2})/2 \right) = \exp(- \alpha r^{2}) $ for some constant $ \alpha $, ie. this term is spherically symmetrical. We just need to look at the hermite polynomials.


For $ N := n_{1} = n_{2} = n_{3} = 0 $ (ground state), $ h_{0}(y_{i}) = \text{constant } $, so this is spherically symmetric. For a solution with $ N = 2 $, consider

\begin{align*}
 \psi(\mathbf{x}) & = \psi_{0}(x_{1})\psi_{0}(x_{2}) \psi_{0}(x_{3}) \\
& = A (1 - 2y_{3}^{2}) e^{- r^{2}/2}
\end{align*}

Now adding similar solutions gives

\begin{align*}
 \psi(\mathbf{x}) & = A ( 1 - 2y_{1}^{2}   - 2y_{2}^{2} - 2y_{3}^{2}  ) e^{- r^{2}/2} \\
 & = A ( 1 - 2r^{2} ) e^{- r^{2}/2}
\end{align*}









 
 

\section{QUESTION 3}

\section{QUESTION 4}

\section{QUESTION 5}

Laplacian for a spherically symmetric potential is

\begin{align*}
\nabla^{2} \psi & = \frac{1}{r^{2}} \frac{\d }{\d r} \left( r^{2} \frac{\d \psi }{\d r} \right)  \\
& = \psi'' + \frac{2}{r} \psi'
\end{align*}

For $ \psi(r) = C e^{-r/a}$,

\begin{align*}
\nabla^{2} \psi & = \frac{1}{r^{2}} \left(  \frac{r^{2}}{a^{2}} - \frac{1}{a} \right) \psi + \frac{2}{r} \left( \frac{-r}{a} \psi \right)  \\
& = \left(  \frac{1}{a^{2}}  - \frac{2}{a} \right) \psi - \frac{1}{r^{2}a}\psi  \\
& = 
\end{align*} 





\section{QUESTION 6}

For any any spherically symmetric wavefunction $ \phi(r) $, we have that $ L_{3} \phi = 0$.

\begin{align*}
L_{3} \phi(r) & = - i \hbar \left(  x_{1} \frac{\partial \phi(r)}{\partial x_{2}}   - x_{2}  \frac{\partial \phi(r) }{\partial x_{1} }\right)      \\
& = - i \hbar \left(  x_{1} \frac{\partial r}{\partial x_{2}} \phi'(r)   - x_{2}  \frac{\partial r }{\partial x_{1} } \phi'(r) \right) \\
& = - i \hbar \left(  x_{1} \frac{x_{2}}{r} \phi'(r)   - x_{2}  \frac{x_{1}}{r} \phi'(r) \right) \\
& = 0                                                            
\end{align*}   

Note that $ \frac{\partial \phi }{\partial x_{i}} = \frac{\phi'(r)}{r} x_{i} $.

Now,

\begin{align*}
L_{3}[ x_{1} \phi(r) ] & = - i \hbar \left(  x_{1} \frac{\partial [ x_{1} \phi(r) ]}{\partial x_{2}}   - x_{2}  \frac{\partial [ x_{1} \phi(r) ]}{\partial x_{1} }\right)      \\
& = - i \hbar \left(  x_{1}^{2} x_{2} \frac{\phi'(r)}{r}   - x_{2} \phi(r) -  x_{1}^{2} x_{2} \frac{\phi'(r)}{r}  \right) \\
& =  i \hbar x_{2} \phi(r)                                                           
\end{align*} 

Similarly,

\[ L_{3}[ x_{2} \phi(r) ] = -i \hbar x_{1} \phi(r), \qquad L_{3}[x_{3}\phi(r)] = 0   \]


We can use these results we calculate $ L_{3}^{2} $

\begin{align*}
L_{3}^{2} [ x_{1} \phi(r) ] & = i \hbar  L [x_{2} \phi(r) ]\\
& = i \hbar (-i \hbar x_{1} \phi(r))\\
& = \hbar^{2} x_{1} \phi(r)
\end{align*}

Similarly,

\[ L_{3}^{2}[ x_{2} \phi(r) ] = \hbar^{2} x_{2} \phi(r), \qquad L_{3}^{2}[x_{3}\phi(r)] = 0   \]

The total angular momentum operator is 

\[ L^{2} = L_{1}^{2} + L_{2}^{2} + L_{3}^{2} \]

We can use symmetry to deduce that

\[ L_{i}^{2}[ x_{j} \phi(r) ] = \begin{cases} \hbar^{2} x_{j} \phi(r)  & \text{ if } i \neq j \\ 0 &  \text{ if } i = j \end{cases}  \]

Thus

\[ L^{2}[ x_{j} \phi(r) ] = 2  \hbar^{2} x_{j} \phi(r) \]

ie. $ \psi_{i}(\mathbf{x}) = x_{j} \phi(r) $ is an eigenfunction of $ L^{2}  $ with eigenvalue $ 2\hbar^{2} $.

Also, letting $ \psi_{\pm}(\mathbf{x}) = x_{1} \phi(r) \pm  x_{2} \phi(r) $

\begin{align*}
L_{3} [ x_{1} \phi(r) \pm  x_{2} \phi(r) ] & = i \hbar [x_{2} \phi(r) ] \mp i \hbar [x_{1} \phi(r) ] \\
& = \pm i \hbar \psi_{\pm}(\mathbf{x})
\end{align*}

ie. $ \psi_{\pm}(\mathbf{x}) $ are eigenvalues of $ L_{3} $ with eigenvalues $ \pm 1 $.






                                                     
                                                                    
\section{QUESTION 7}

\section{QUESTION 8}

By the Leibnitz property

\begin{align*}
[L_{i},\mathbf{L}] & = [L_{i},L_{jj}]\\
& =  [L_{i},L_{j}] L_{j} + L_{j}[L_{i},L_{j}]  \\
& = i \hbar \varepsilon_{ijk} (L_{k}L_{j} + L_{j}L_{k}  ) \\
& = 0
\end{align*}

for $ i = 1,2,3, $ and we get 0 since we are contracting the antisymmetric tensor $ \varepsilon_{ijk} $ with the symmetric tensor $ L_{k}L_{j} + L_{j}L_{k} $.

\section{QUESTION 9}

Calculation shows

\begin{align*}
[S_{1},S_{2}]& = i \hbar S_{3} \\ 
[S_{2},S_{3}]& = i \hbar S_{1} \\ 
[S_{3},S_{1}]& = i \hbar S_{2}
\end{align*}

ie. $ [S_{i},S_{j}] = \varepsilon_{ijk} i\hbar S_{k} $

Also find that

\begin{align*}
S^{2} & = S_{1}^{2}  + S_{2}^{2} + S_{3}^{2} \\
& = \frac{3 \hbar}{2} \begin{pmatrix}
1 & 0 \\
0 & 1
\end{pmatrix}
\end{align*}





\end{document}
