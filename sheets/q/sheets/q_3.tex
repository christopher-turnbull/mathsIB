\documentclass[a4paper]{article}
\usepackage{amsmath}
\def\npart {IB}
\def\nterm {Michaelmas}
\def\nyear {2017}
\def\nlecturer {Dr Warnick}
\def\ncourse {Quantum Mechanics Example Sheet 3}

\input{header}
\newtheorem*{soln}{Solution}

\renewcommand{\thesection}{}
\renewcommand{\thesubsection}{\arabic{section}.\arabic{subsection}}
\makeatletter
\def\@seccntformat#1{\csname #1ignore\expandafter\endcsname\csname the#1\endcsname\quad}
\let\sectionignore\@gobbletwo
\let\latex@numberline\numberline
\def\numberline#1{\if\relax#1\relax\else\latex@numberline{#1}\fi}
\makeatother


\begin{document}
	
\maketitle

\section{QUESTION 1}

Time independent Schr\"odinger Equation is 

\[ - \frac{\hbar^{2}}{2m} \nabla^{2} \psi = E \psi \]

\[  - \frac{\hbar^{2}}{2m} \left(  \frac{X''}{X} +  \frac{Y''}{Y} +  \frac{Z''}{Z} \right)  = E  \]

Can split this into 3 equations, eg.



\[ - \frac{\hbar^{2}}{2m}\frac{X''}{X} = E_{1} \]
\[ X'' + k^{2}X = 0  \qquad \text{ where } E_{1} = \frac{\hbar^{2}k^{2}}{2m}  \]

Note we take $ E_{i} > 0 $, as boundary conditions mean $ E_{i} < 0 $ has no eigenstate solutions.

$ X(0) = X(a) = 0 \Rightarrow k = n_{1} \pi / a  $
Repeat for $ Y $ and $ Z $. 

\begin{align*}
 E & = E_{1} + E_{2} + E_{3} \\
  & = \frac{\hbar^{2} \pi^{2}}{2m} \left(  \frac{n_{1}^{2}}{a^{2}} + \frac{n_{2}^{2}}{b^{2}} + \frac{n_{3}^{2}}{c^{2}}  \right)
\end{align*}
 
 With $ a = b = c $, ground state is $  E = \frac{3 \hbar^{2} \pi^{2}}{2ma^{2}} $ where $ n_{1} = n_{2} = n_{3} = 1 $, and next when $ \sum_{i} n_{i} = 4 $ (which happens in 3 different ways) we have $  E = \frac{2\hbar^{2} \pi^{2}}{ma^{2}} $, so first excited state has degeneracy 3. 

\section{QUESTION 2}

Time independent Schr\"odinger Equation is 


\[ - \frac{\hbar^{2}}{2m} \nabla^{2} \psi  + \frac{1}{2} m \omega^{2} (x_{1}^{2} + x_{2}^{2} + x_{3}^{2}) \psi  = E \psi \]

The Hamiltonian splits into $ H = H_{1} + H_{2} + H_{3} $

Seek solutions of the form $ \psi = X(x_{1})Y(x_{2})Z(x_{3}) $.
Separating variables shows

\[  - \frac{\hbar^{2}}{2m} \left(  \frac{X''}{X} +  \frac{Y''}{Y} +  \frac{Z''}{Z} \right)  + \frac{1}{2} m \omega^{2} (x_{1}^{2} + x_{2}^{2} + x_{3}^{2}) = E  \]


As $ X $ cannot vary for fixed $ Y,Z $, we have

\[ - \frac{\hbar^{2}}{2m} \frac{X''}{X}  + \frac{1}{2} m \omega^{2} x_{1}^{2}  = E_{1} \]

This is the one dimensional harmonic oscillator equation; with eigenstates and eigenvalues

\[ X_{n_{1}}(x_{1}) = h_{n_{1}}(y_{1})   \exp \left(   - y_{1}^{2}/2 \right), \qquad E_{1} = \hbar \omega (n_{1} + \frac{1}{2})    \]

\[ y_{1} = \left( \left(  \frac{m \omega}{\hbar}  \right)^{1/2} x_{1} \right) \]

for $ n_{1} = 0,1,2,\cdots $

Similarly, recover that $ E_{i} = \hbar \omega (n_{i} + \frac{1}{2}) $

\begin{align*}
 E &= E_{1} + E_{2} + E_{3} \\
& = \hbar \omega \left(     n_{1} + n_{2} + n_{3} + \frac{3}{2} \right)
\end{align*}

where $ n_{i} = 0,1,2,\cdots $


To count the number of linearly independent eigenstates corresponding to energy $ E = (N + \frac{3}{2} ) \hbar \omega $, need $ n_{1} + n_{2} + n_{3} = N $. With $ n_{1} = 0 $, need $ n_{2} + n_{3} = N $, which can happen in $ N + 1 $ ways. Then $ n_{1} = 1 $, have $ N $ more states. So the total number of states is given by
\begin{align*}
\text{Degeneracy }& = (N + 1) + N + \cdots + 2 + 1  \\
& = (N + 2)(N + 1)/2
\end{align*}

Now have

\[ \psi(\mathbf{x}) = h_{n_{1}}(y_{1})h_{n_{2}}(y_{2})h_{n_{3}}(y_{3})  \exp \left(  -(y_{1}^{2} + y_{2}^{2} + y_{3}^{2})/2 \right) \]


Note $  \exp \left(  -(y_{1}^{2} + y_{2}^{2} + y_{3}^{2})/2 \right) = \exp(- \alpha r^{2}) $ for some constant $ \alpha $, ie. this term is spherically symmetrical. We just need to look at the hermite polynomials.


For $ N := n_{1} = n_{2} = n_{3} = 0 $ (ground state), $ h_{0}(y_{i}) = \text{constant } $, so this is spherically symmetric. For a solution with $ N = 2 $, consider

\begin{align*}
 \psi(\mathbf{x}) & = \psi_{0}(x_{1})\psi_{0}(x_{2}) \psi_{0}(x_{3}) \\
& = A (1 - 2y_{3}^{2}) e^{- r^{2}/2}
\end{align*}

Now adding similar solutions gives

\begin{align*}
 \psi(\mathbf{x}) & = A ( 1 - 2y_{1}^{2}   - 2y_{2}^{2} - 2y_{3}^{2}  ) e^{- r^{2}/2} \\
 & = A ( 1 - 2r^{2} ) e^{- r^{2}/2}
\end{align*}









 
 

\section{QUESTION 3}

\section{QUESTION 4}

\section{QUESTION 5}

Laplacian for a spherically symmetric potential is

\begin{align*}
\nabla^{2} \psi & = \frac{1}{r^{2}} \frac{\d }{\d r} \left( r^{2} \frac{\d \psi }{\d r} \right)  \\
& = \psi'' + \frac{2}{r} \psi'
\end{align*}

For $ \psi(r) = C e^{-r/a}$,

\begin{align*}
\nabla^{2} \psi & = \frac{1}{r^{2}} \left(  \frac{r^{2}}{a^{2}} - \frac{1}{a} \right) \psi + \frac{2}{r} \left( \frac{-r}{a} \psi \right)  \\
& = \left(  \frac{1}{a^{2}}  - \frac{2}{a} \right) \psi - \frac{1}{r^{2}a}\psi  \\
& = 
\end{align*} 





\section{QUESTION 6}

For any any spherically symmetric wavefunction $ \phi(r) $, we have that $ L_{3} \phi = 0$.

\begin{align*}
L_{3} \phi(r) & = - i \hbar \left(  x_{1} \frac{\partial \phi(r)}{\partial x_{2}}   - x_{2}  \frac{\partial \phi(r) }{\partial x_{1} }\right)      \\
& = - i \hbar \left(  x_{1} \frac{\partial r}{\partial x_{2}} \phi'(r)   - x_{2}  \frac{\partial r }{\partial x_{1} } \phi'(r) \right) \\
& = - i \hbar \left(  x_{1} \frac{x_{2}}{r} \phi'(r)   - x_{2}  \frac{x_{1}}{r} \phi'(r) \right) \\
& = 0                                                            
\end{align*}   

Note that $ \frac{\partial \phi }{\partial x_{i}} = \frac{\phi'(r)}{r} x_{i} $.

Now,

\begin{align*}
L_{3}[ x_{1} \phi(r) ] & = - i \hbar \left(  x_{1} \frac{\partial [ x_{1} \phi(r) ]}{\partial x_{2}}   - x_{2}  \frac{\partial [ x_{1} \phi(r) ]}{\partial x_{1} }\right)      \\
& = - i \hbar \left(  x_{1}^{2} x_{2} \frac{\phi'(r)}{r}   - x_{2} \phi(r) -  x_{1}^{2} x_{2} \frac{\phi'(r)}{r}  \right) \\
& =  i \hbar x_{2} \phi(r)                                                           
\end{align*} 

Similarly,

\[ L_{3}[ x_{2} \phi(r) ] = -i \hbar x_{1} \phi(r), \qquad L_{3}[x_{3}\phi(r)] = 0   \]


We can use these results we calculate $ L_{3}^{2} $

\begin{align*}
L_{3}^{2} [ x_{1} \phi(r) ] & = i \hbar  L [x_{2} \phi(r) ]\\
& = i \hbar (-i \hbar x_{1} \phi(r))\\
& = \hbar^{2} x_{1} \phi(r)
\end{align*}

Similarly,

\[ L_{3}^{2}[ x_{2} \phi(r) ] = \hbar^{2} x_{2} \phi(r), \qquad L_{3}^{2}[x_{3}\phi(r)] = 0   \]

The total angular momentum operator is 

\[ L^{2} = L_{1}^{2} + L_{2}^{2} + L_{3}^{2} \]

We can use symmetry to deduce that

\[ L_{i}^{2}[ x_{j} \phi(r) ] = \begin{cases} \hbar^{2} x_{j} \phi(r)  & \text{ if } i \neq j \\ 0 &  \text{ if } i = j \end{cases}  \]

Thus

\[ L^{2}[ x_{j} \phi(r) ] = 2  \hbar^{2} x_{j} \phi(r) \]

ie. $ \psi_{i}(\mathbf{x}) = x_{j} \phi(r) $ is an eigenfunction of $ L^{2}  $ with eigenvalue $ 2\hbar^{2} $.

Also, letting $ \psi_{\pm}(\mathbf{x}) = x_{1} \phi(r) \pm  x_{2} \phi(r) $

\begin{align*}
L_{3} [ x_{1} \phi(r) \pm  x_{2} \phi(r) ] & = i \hbar [x_{2} \phi(r) ] \mp i \hbar [x_{1} \phi(r) ] \\
& = \pm i \hbar \psi_{\pm}(\mathbf{x})
\end{align*}

ie. $ \psi_{\pm}(\mathbf{x}) $ are eigenvalues of $ L_{3} $ with eigenvalues $ \pm 1 $.






                                                     
                                                                    
\section{QUESTION 7}

\section{QUESTION 8}

By the Leibnitz property

\begin{align*}
[L_{i},\mathbf{L}] & = [L_{i},L_{jj}]\\
& =  [L_{i},L_{j}] L_{j} + L_{j}[L_{i},L_{j}]  \\
& = i \hbar \varepsilon_{ijk} (L_{k}L_{j} + L_{j}L_{k}  ) \\
& = 0
\end{align*}

for $ i = 1,2,3, $ and we get 0 since we are contracting the antisymmetric tensor $ \varepsilon_{ijk} $ with the symmetric tensor $ L_{k}L_{j} + L_{j}L_{k} $.

\section{QUESTION 9}

Calculation shows

\begin{align*}
[S_{1},S_{2}]& = i \hbar S_{3} \\ 
[S_{2},S_{3}]& = i \hbar S_{1} \\ 
[S_{3},S_{1}]& = i \hbar S_{2}
\end{align*}

ie. $ [S_{i},S_{j}] = \varepsilon_{ijk} i\hbar S_{k} $

Also find that

\begin{align*}
S^{2} & = S_{1}^{2}  + S_{2}^{2} + S_{3}^{2} \\
& = \frac{3 \hbar}{2} \begin{pmatrix}
1 & 0 \\
0 & 1
\end{pmatrix}
\end{align*}





\end{document}
