\documentclass[a4paper]{article}
\usepackage{amsmath}
\def\npart {IB}
\def\nterm {Lent}
\def\nyear {2018}
\def\nlecturer {Prof. Haynes (P.H.Haynes@damtp.cam.ac.uk)}
\def\ncourse {Fluid Dynamics Example Sheet 1}

% Imports
\ifx \nauthor\undefined
  \def\nauthor{Christopher Turnbull}
\else
\fi

\author{Supervised by \nlecturer \\\small Solutions presented by \nauthor}
\date{\nterm\ \nyear}

\usepackage{alltt}
\usepackage{amsfonts}
\usepackage{amsmath}
\usepackage{amssymb}
\usepackage{amsthm}
\usepackage{booktabs}
\usepackage{caption}
\usepackage{enumitem}
\usepackage{fancyhdr}
\usepackage{graphicx}
\usepackage{mathdots}
\usepackage{mathtools}
\usepackage{microtype}
\usepackage{multirow}
\usepackage{pdflscape}
\usepackage{pgfplots}
\usepackage{siunitx}
\usepackage{slashed}
\usepackage{tabularx}
\usepackage{tikz}
\usepackage{tkz-euclide}
\usepackage[normalem]{ulem}
\usepackage[all]{xy}
\usepackage{imakeidx}

\makeindex[intoc, title=Index]
\indexsetup{othercode={\lhead{\emph{Index}}}}

\ifx \nextra \undefined
  \usepackage[pdftex,
    hidelinks,
    pdfauthor={Christopher Turnbull},
    pdfsubject={Cambridge Maths Notes: Part \npart\ - \ncourse},
    pdftitle={Part \npart\ - \ncourse},
  pdfkeywords={Cambridge Mathematics Maths Math \npart\ \nterm\ \nyear\ \ncourse}]{hyperref}
  \title{Part \npart\ --- \ncourse}
\else
  \usepackage[pdftex,
    hidelinks,
    pdfauthor={Christopher Turnbull},
    pdfsubject={Cambridge Maths Notes: Part \npart\ - \ncourse\ (\nextra)},
    pdftitle={Part \npart\ - \ncourse\ (\nextra)},
  pdfkeywords={Cambridge Mathematics Maths Math \npart\ \nterm\ \nyear\ \ncourse\ \nextra}]{hyperref}

  \title{Part \npart\ --- \ncourse \\ {\Large \nextra}}
  \renewcommand\printindex{}
\fi

\pgfplotsset{compat=1.12}

\pagestyle{fancyplain}
\lhead{\emph{\nouppercase{\leftmark}}}
\ifx \nextra \undefined
  \rhead{
    \ifnum\thepage=1
    \else
      \npart\ \ncourse
    \fi}
\else
  \rhead{
    \ifnum\thepage=1
    \else
      \npart\ \ncourse\ (\nextra)
    \fi}
\fi
\usetikzlibrary{arrows.meta}
\usetikzlibrary{decorations.markings}
\usetikzlibrary{decorations.pathmorphing}
\usetikzlibrary{positioning}
\usetikzlibrary{fadings}
\usetikzlibrary{intersections}
\usetikzlibrary{cd}

\newcommand*{\Cdot}{{\raisebox{-0.25ex}{\scalebox{1.5}{$\cdot$}}}}
\newcommand {\pd}[2][ ]{
  \ifx #1 { }
    \frac{\partial}{\partial #2}
  \else
    \frac{\partial^{#1}}{\partial #2^{#1}}
  \fi
}
\ifx \nhtml \undefined
\else
  \renewcommand\printindex{}
  \makeatletter
  \DisableLigatures[f]{family = *}
  \let\Contentsline\contentsline
  \renewcommand\contentsline[3]{\Contentsline{#1}{#2}{}}
  \renewcommand{\@dotsep}{10000}
  \newlength\currentparindent
  \setlength\currentparindent\parindent

  \newcommand\@minipagerestore{\setlength{\parindent}{\currentparindent}}
  \usepackage[active,tightpage,pdftex]{preview}
  \renewcommand{\PreviewBorder}{0.1cm}

  \newenvironment{stretchpage}%
  {\begin{preview}\begin{minipage}{\hsize}}%
    {\end{minipage}\end{preview}}
  \AtBeginDocument{\begin{stretchpage}}
  \AtEndDocument{\end{stretchpage}}

  \newcommand{\@@newpage}{\end{stretchpage}\begin{stretchpage}}

  \let\@real@section\section
  \renewcommand{\section}{\@@newpage\@real@section}
  \let\@real@subsection\subsection
  \renewcommand{\subsection}{\@@newpage\@real@subsection}
  \makeatother
\fi

% Theorems
\theoremstyle{definition}
\newtheorem*{aim}{Aim}
\newtheorem*{axiom}{Axiom}
\newtheorem*{claim}{Claim}
\newtheorem*{cor}{Corollary}
\newtheorem*{conjecture}{Conjecture}
\newtheorem*{defi}{Definition}
\newtheorem*{eg}{Example}
\newtheorem*{ex}{Exercise}
\newtheorem*{fact}{Fact}
\newtheorem*{law}{Law}
\newtheorem*{lemma}{Lemma}
\newtheorem*{notation}{Notation}
\newtheorem*{prop}{Proposition}
\newtheorem*{soln}{Solution}
\newtheorem*{thm}{Theorem}

\newtheorem*{remark}{Remark}
\newtheorem*{warning}{Warning}
\newtheorem*{exercise}{Exercise}

\newtheorem{nthm}{Theorem}[section]
\newtheorem{nlemma}[nthm]{Lemma}
\newtheorem{nprop}[nthm]{Proposition}
\newtheorem{ncor}[nthm]{Corollary}


\renewcommand{\labelitemi}{--}
\renewcommand{\labelitemii}{$\circ$}
\renewcommand{\labelenumi}{(\roman{*})}

\let\stdsection\section
\renewcommand\section{\newpage\stdsection}

% Strike through
\def\st{\bgroup \ULdepth=-.55ex \ULset}

% Maths symbols
\newcommand{\abs}[1]{\left\lvert #1\right\rvert}
\newcommand\ad{\mathrm{ad}}
\newcommand\AND{\mathsf{AND}}
\newcommand\Art{\mathrm{Art}}
\newcommand{\Bilin}{\mathrm{Bilin}}
\newcommand{\bket}[1]{\left\lvert #1\right\rangle}
\newcommand{\B}{\mathcal{B}}
\newcommand{\bolds}[1]{{\bfseries #1}}
\newcommand{\brak}[1]{\left\langle #1 \right\rvert}
\newcommand{\braket}[2]{\left\langle #1\middle\vert #2 \right\rangle}
\newcommand{\bra}{\langle}
\newcommand{\cat}[1]{\mathsf{#1}}
\newcommand{\C}{\mathbb{C}}
\newcommand{\CP}{\mathbb{CP}}
\newcommand{\cU}{\mathcal{U}}
\newcommand{\Der}{\mathrm{Der}}
\newcommand{\D}{\mathrm{D}}
\newcommand{\dR}{\mathrm{dR}}
\newcommand{\E}{\mathbb{E}}
\newcommand{\F}{\mathbb{F}}
\newcommand{\Frob}{\mathrm{Frob}}
\newcommand{\GG}{\mathbb{G}}
\newcommand{\gl}{\mathfrak{gl}}
\newcommand{\GL}{\mathrm{GL}}
\newcommand{\G}{\mathcal{G}}
\newcommand{\Gr}{\mathrm{Gr}}
\newcommand{\haut}{\mathrm{ht}}
\newcommand{\Id}{\mathrm{Id}}
\newcommand{\ket}{\rangle}
\newcommand{\lie}[1]{\mathfrak{#1}}
\newcommand{\Mat}{\mathrm{Mat}}
\newcommand{\N}{\mathbb{N}}
\newcommand{\norm}[1]{\left\lVert #1\right\rVert}
\newcommand{\normalorder}[1]{\mathop{:}\nolimits\!#1\!\mathop{:}\nolimits}
\newcommand\NOT{\mathsf{NOT}}
\newcommand{\Oc}{\mathcal{O}}
\newcommand{\Or}{\mathrm{O}}
\newcommand\OR{\mathsf{OR}}
\newcommand{\ort}{\mathfrak{o}}
\newcommand{\PGL}{\mathrm{PGL}}
\newcommand{\ph}{\,\cdot\,}
\newcommand{\pr}{\mathrm{pr}}
\newcommand{\Prob}{\mathbb{P}}
\newcommand{\PSL}{\mathrm{PSL}}
\newcommand{\Ps}{\mathcal{P}}
\newcommand{\PSU}{\mathrm{PSU}}
\newcommand{\pt}{\mathrm{pt}}
\newcommand{\qeq}{\mathrel{``{=}"}}
\newcommand{\Q}{\mathbb{Q}}
\newcommand{\R}{\mathbb{R}}
\newcommand{\RP}{\mathbb{RP}}
\newcommand{\Rs}{\mathcal{R}}
\newcommand{\SL}{\mathrm{SL}}
\newcommand{\so}{\mathfrak{so}}
\newcommand{\SO}{\mathrm{SO}}
\newcommand{\Spin}{\mathrm{Spin}}
\newcommand{\Sp}{\mathrm{Sp}}
\newcommand{\su}{\mathfrak{su}}
\newcommand{\SU}{\mathrm{SU}}
\newcommand{\term}[1]{\emph{#1}\index{#1}}
\newcommand{\T}{\mathbb{T}}
\newcommand{\tv}[1]{|#1|}
\newcommand{\U}{\mathrm{U}}
\newcommand{\uu}{\mathfrak{u}}
\newcommand{\Vect}{\mathrm{Vect}}
\newcommand{\wsto}{\stackrel{\mathrm{w}^*}{\to}}
\newcommand{\wt}{\mathrm{wt}}
\newcommand{\wto}{\stackrel{\mathrm{w}}{\to}}
\newcommand{\Z}{\mathbb{Z}}
\renewcommand{\d}{\mathrm{d}}
\renewcommand{\H}{\mathbb{H}}
\renewcommand{\P}{\mathbb{P}}
\renewcommand{\sl}{\mathfrak{sl}}
\renewcommand{\vec}[1]{\boldsymbol{\mathbf{#1}}}
%\renewcommand{\F}{\mathcal{F}}

\let\Im\relax
\let\Re\relax

\DeclareMathOperator{\adj}{adj}
\DeclareMathOperator{\Ann}{Ann}
\DeclareMathOperator{\area}{area}
\DeclareMathOperator{\Aut}{Aut}
\DeclareMathOperator{\Bernoulli}{Bernoulli}
\DeclareMathOperator{\betaD}{beta}
\DeclareMathOperator{\bias}{bias}
\DeclareMathOperator{\binomial}{binomial}
\DeclareMathOperator{\card}{card}
\DeclareMathOperator{\ccl}{ccl}
\DeclareMathOperator{\Char}{char}
\DeclareMathOperator{\ch}{ch}
\DeclareMathOperator{\cl}{cl}
\DeclareMathOperator{\cls}{\overline{\mathrm{span}}}
\DeclareMathOperator{\conv}{conv}
\DeclareMathOperator{\corr}{corr}
\DeclareMathOperator{\cosec}{cosec}
\DeclareMathOperator{\cosech}{cosech}
\DeclareMathOperator{\cov}{cov}
\DeclareMathOperator{\covol}{covol}
\DeclareMathOperator{\diag}{diag}
\DeclareMathOperator{\diam}{diam}
\DeclareMathOperator{\Diff}{Diff}
\DeclareMathOperator{\disc}{disc}
\DeclareMathOperator{\dom}{dom}
\DeclareMathOperator{\End}{End}
\DeclareMathOperator{\energy}{energy}
\DeclareMathOperator{\erfc}{erfc}
\DeclareMathOperator{\erf}{erf}
\DeclareMathOperator*{\esssup}{ess\,sup}
\DeclareMathOperator{\ev}{ev}
\DeclareMathOperator{\Ext}{Ext}
\DeclareMathOperator{\Fit}{Fit}
\DeclareMathOperator{\fix}{fix}
\DeclareMathOperator{\Frac}{Frac}
\DeclareMathOperator{\Gal}{Gal}
\DeclareMathOperator{\gammaD}{gamma}
\DeclareMathOperator{\gr}{gr}
\DeclareMathOperator{\hcf}{hcf}
\DeclareMathOperator{\Hom}{Hom}
\DeclareMathOperator{\id}{id}
\DeclareMathOperator{\image}{image}
\DeclareMathOperator{\im}{im}
\DeclareMathOperator{\Im}{Im}
\DeclareMathOperator{\Ind}{Ind}
\DeclareMathOperator{\Int}{Int}
\DeclareMathOperator{\Isom}{Isom}
\DeclareMathOperator{\lcm}{lcm}
\DeclareMathOperator{\length}{length}
\DeclareMathOperator{\Lie}{Lie}
\DeclareMathOperator{\like}{like}
\DeclareMathOperator{\Lk}{Lk}
\DeclareMathOperator{\mse}{mse}
\DeclareMathOperator{\multinomial}{multinomial}
\DeclareMathOperator{\orb}{orb}
\DeclareMathOperator{\ord}{ord}
\DeclareMathOperator{\otp}{otp}
\DeclareMathOperator{\Poisson}{Poisson}
\DeclareMathOperator{\poly}{poly}
\DeclareMathOperator{\rank}{rank}
\DeclareMathOperator{\rel}{rel}
\DeclareMathOperator{\Re}{Re}
\DeclareMathOperator*{\res}{res}
\DeclareMathOperator{\Res}{Res}
\DeclareMathOperator{\rk}{rk}
\DeclareMathOperator{\Root}{Root}
\DeclareMathOperator{\sech}{sech}
\DeclareMathOperator{\sgn}{sgn}
\DeclareMathOperator{\spn}{span}
\DeclareMathOperator{\stab}{stab}
\DeclareMathOperator{\St}{St}
\DeclareMathOperator{\supp}{supp}
\DeclareMathOperator{\Syl}{Syl}
\DeclareMathOperator{\Sym}{Sym}
\DeclareMathOperator{\tr}{tr}
\DeclareMathOperator{\Tr}{Tr}
\DeclareMathOperator{\var}{var}
\DeclareMathOperator{\vol}{vol}

\pgfarrowsdeclarecombine{twolatex'}{twolatex'}{latex'}{latex'}{latex'}{latex'}
\tikzset{->/.style = {decoration={markings,
                                  mark=at position 1 with {\arrow[scale=2]{latex'}}},
                      postaction={decorate}}}
\tikzset{<-/.style = {decoration={markings,
                                  mark=at position 0 with {\arrowreversed[scale=2]{latex'}}},
                      postaction={decorate}}}
\tikzset{<->/.style = {decoration={markings,
                                   mark=at position 0 with {\arrowreversed[scale=2]{latex'}},
                                   mark=at position 1 with {\arrow[scale=2]{latex'}}},
                       postaction={decorate}}}
\tikzset{->-/.style = {decoration={markings,
                                   mark=at position #1 with {\arrow[scale=2]{latex'}}},
                       postaction={decorate}}}
\tikzset{-<-/.style = {decoration={markings,
                                   mark=at position #1 with {\arrowreversed[scale=2]{latex'}}},
                       postaction={decorate}}}
\tikzset{->>/.style = {decoration={markings,
                                  mark=at position 1 with {\arrow[scale=2]{latex'}}},
                      postaction={decorate}}}
\tikzset{<<-/.style = {decoration={markings,
                                  mark=at position 0 with {\arrowreversed[scale=2]{twolatex'}}},
                      postaction={decorate}}}
\tikzset{<<->>/.style = {decoration={markings,
                                   mark=at position 0 with {\arrowreversed[scale=2]{twolatex'}},
                                   mark=at position 1 with {\arrow[scale=2]{twolatex'}}},
                       postaction={decorate}}}
\tikzset{->>-/.style = {decoration={markings,
                                   mark=at position #1 with {\arrow[scale=2]{twolatex'}}},
                       postaction={decorate}}}
\tikzset{-<<-/.style = {decoration={markings,
                                   mark=at position #1 with {\arrowreversed[scale=2]{twolatex'}}},
                       postaction={decorate}}}

\tikzset{circ/.style = {fill, circle, inner sep = 0, minimum size = 3}}
\tikzset{mstate/.style={circle, draw, blue, text=black, minimum width=0.7cm}}

\tikzset{commutative diagrams/.cd,cdmap/.style={/tikz/column 1/.append style={anchor=base east},/tikz/column 2/.append style={anchor=base west},row sep=tiny}}

\definecolor{mblue}{rgb}{0.2, 0.3, 0.8}
\definecolor{morange}{rgb}{1, 0.5, 0}
\definecolor{mgreen}{rgb}{0.1, 0.4, 0.2}
\definecolor{mred}{rgb}{0.5, 0, 0}

\def\drawcirculararc(#1,#2)(#3,#4)(#5,#6){%
    \pgfmathsetmacro\cA{(#1*#1+#2*#2-#3*#3-#4*#4)/2}%
    \pgfmathsetmacro\cB{(#1*#1+#2*#2-#5*#5-#6*#6)/2}%
    \pgfmathsetmacro\cy{(\cB*(#1-#3)-\cA*(#1-#5))/%
                        ((#2-#6)*(#1-#3)-(#2-#4)*(#1-#5))}%
    \pgfmathsetmacro\cx{(\cA-\cy*(#2-#4))/(#1-#3)}%
    \pgfmathsetmacro\cr{sqrt((#1-\cx)*(#1-\cx)+(#2-\cy)*(#2-\cy))}%
    \pgfmathsetmacro\cA{atan2(#2-\cy,#1-\cx)}%
    \pgfmathsetmacro\cB{atan2(#6-\cy,#5-\cx)}%
    \pgfmathparse{\cB<\cA}%
    \ifnum\pgfmathresult=1
        \pgfmathsetmacro\cB{\cB+360}%
    \fi
    \draw (#1,#2) arc (\cA:\cB:\cr);%
}
\newcommand\getCoord[3]{\newdimen{#1}\newdimen{#2}\pgfextractx{#1}{\pgfpointanchor{#3}{center}}\pgfextracty{#2}{\pgfpointanchor{#3}{center}}}

\def\Xint#1{\mathchoice
   {\XXint\displaystyle\textstyle{#1}}%
   {\XXint\textstyle\scriptstyle{#1}}%
   {\XXint\scriptstyle\scriptscriptstyle{#1}}%
   {\XXint\scriptscriptstyle\scriptscriptstyle{#1}}%
   \!\int}
\def\XXint#1#2#3{{\setbox0=\hbox{$#1{#2#3}{\int}$}
     \vcenter{\hbox{$#2#3$}}\kern-.5\wd0}}
\def\ddashint{\Xint=}
\def\dashint{\Xint-}

\newcommand\separator{{\centering\rule{2cm}{0.2pt}\vspace{2pt}\par}}

\newenvironment{own}{\color{gray!70!black}}{}

\newcommand\makecenter[1]{\raisebox{-0.5\height}{#1}}

\newtheorem*{soln}{Solution}

\renewcommand{\thesection}{}
\renewcommand{\thesubsection}{\arabic{section}.\arabic{subsection}}
\makeatletter
\def\@seccntformat#1{\csname #1ignore\expandafter\endcsname\csname the#1\endcsname\quad}
\let\sectionignore\@gobbletwo
\let\latex@numberline\numberline
\def\numberline#1{\if\relax#1\relax\else\latex@numberline{#1}\fi}
\makeatother


\begin{document}
	
\maketitle

\section{QUESTION 1}

 \begin{center}
	\begin{tikzpicture}
	\draw (0, 0) -- (6,0);
	\draw (0.5,4) -- (0.5,1) -- (6,1);
	\draw [dashed] (0.5, 0.5) -- (6, 0.5);
	\draw [<->] (0.5, -0.5) -- (6, -0.5) node [pos=0.5, fill=white] {$L$};
%	\node [left] at (0, 1) {$y = 0$};
%	
	\fill [mblue, opacity=0.5] (0, 3.5) -- (0, 0) -- (6, 0) -- (6, 1) -- (0.5,1) -- (0.5, 3.5) -- (0, 3.5);
	
	\draw [<->] (-0.5, 0) -- (-0.5, 3.5) node [pos=0.5, left] {$h$};
	\end{tikzpicture}
	
\end{center}

We apply time-dependent Bernoulli equation:

\[
\rho \frac{\partial\phi}{\partial t} + \frac{1}{2} \rho |\nabla \phi|^2 + p + \chi = f(t),
\]

along the centre of the horizontal tube, so gravity ($ \chi $) cancels out.

At the left end of the dashed line, the open vessel implies $ \phi = 0 $, and $ p = p_{\text{atm}} + \rho g h $. At the right most end, the water will move only horizontally (since the diameter is small). Hence 

\[ \phi = u x, \frac{\partial \phi }{\partial t} = \dot{u} x \]

Applying the equation, taking $ x = 0 $ at the left end and $ x = L $ at the right gives 

\[  p_{\text{atm}} + \rho g h = \rho \dot{u} L + \frac{1}{2} \rho u^{2} +  p_{\text{atm}}  \]

which simplifies to 

\[ L \dot{u} + \frac{1}{2} u^{2} = g h \]

Not sure how to solve this first order ODE, but we can ansatz the given form of the solution as $ u = A \tanh (k t) $, which yields

\[ k L A \sech^{2}(kt) + \frac{1}{2} A^{2}\tanh^{2} (k t) = g h \]

\[ k L A ( 1  - \tanh^{2} (k t) ) + \frac{1}{2} A^2 \tanh^{2} (k t) = g h \]

Comparing constants and coefficients of $ \tanh $ shows that

\[ klA = gh, \quad \frac{1}{2} A^{2} = k l A \]

which give $ A = \sqrt{2gh} $ and $ k = \frac{\sqrt{2gh}}{2L} $ as required. 

Given $ L \sim 10  $ m $ h \sim 5 $ m, we have $ \frac{\sqrt{2gh}}{2L} \sim \frac{1}{2}$.

Given $ \tanh 2 \approx 0.96$, we are pretty much there after $ 4 $ seconds.  





\section{QUESTION 2}

\begin{center}
	\begin{tikzpicture}
	\fill [gray, path fading=south] (-3, 0) rectangle (3, -1);
	
	
	\fill [mblue, opacity=0.5] (-3, 0) -- (-3,1) -- (3,1) -- (3,0) ;
	
	\fill [mblue, opacity=0.5, path fading=north] (-3, 1) -- (-3,1.5) -- (3,1.5) -- (3,1) ;
	
	\draw (-3, 1) -- (3,1);
	
	\draw [->] (0, 1) -- (0, 2);
	
	\draw [->] (-5, 0.5) -- +(1, 0) node [right] {$r$};
	\draw [->] (-5, 0.5) -- +(0, 1) node [above] {$z$};
	
	\draw (-1.5, 2.3) node [right] {$h(t)$} edge [out=180, in=90, -latex] (-2, 1);
	
	\end{tikzpicture}
\end{center}

Incompressibility $ \implies u = \nabla \phi $ ?

No penetration conditions imply that $ \frac{\partial \phi }{\partial z} = 0 $ on $ z = 0 $ and the dynamic boundary condition $ \frac{\partial \phi }{\partial z} = \dot{h} $ on $ z = h $.

Not sure how to derive the results: can show the given results satisfy boundary conditions though.

To find pressure distribution $ p(r,z,t) $ under the disc, applying Bernoulli at $ z = h(t) $ and $ z = \infty $ gives

\begin{align*}
(2z^{2} - r^{2}) \left(  \frac{\ddot{h}}{4h} - \frac{\dot{h}^{2}}{4h^{2}} \right)  + \frac{1}{2} \left( \frac{\dot{h}^{2}}{h^{2}} \right) \left(  \frac{r^{2}}{4} + z^{2} \right) + \frac{p}{\rho} =  \frac{p_{\infty}}{\rho} \\ 
\end{align*}

Thus

\[ p(r,z,t) = p_{\infty} - \frac{\dot{h}^{2}}{h^{2}} \left(  \frac{3}{8} r^{2} + z^{2} \right) - \frac{\dot{h}}{4h}(2z^{2} - r^{2})  \]

Force on disc from below is given by surface integral of this pressure distribution, but not sure how to calculate force from above...

\section{QUESTION 3}





\begin{center}
	\begin{tikzpicture}
	\draw [fill=black] circle [radius=1];
	\draw [fill = mblue, opacity=0.5] circle [radius=2];
	

%	
	\draw [->] (0, 1) -- (0, 1.5) node [pos=0.5, right] {$U(t)$};
%	\draw [<->] (2.2, 2.5) -- (2.2, 3.3) node [pos=0.5, left] {$\zeta$};
%	

%	\node [right] at (3, 2) {$A_{2}$};
	\end{tikzpicture}
\end{center}

Since the fluid is inviscid and irrational, between the two spheres we have 

\[ \nabla^{2} \phi = 0  \]

with kinematic boundary conditions 

\[ \frac{\partial \phi }{\partial r} \Big|_{r = b} = 0, \quad \frac{\partial \phi }{\partial r} \Big|_{r = a} = \frac{D a}{D t }  \approx U(t) \cos \theta \]

that is, no flow through the boundary of the outer sphere, and fluid on the inner surface of the sphere stays there. 

Assuming the motion is asymmetrical, the solution of Laplace's equation with these boundary conditions becomes

\[ \phi(r,\theta) = \sum_{l=0}^{\infty} \left(  A_{l} r^{l} + \frac{B_{l}}{r^{l+1}} \right)  P_{l} (\cos \theta) \]

where $ P_{l} $ is the $ l^{\text{th}} $ Lagrange polynomial. Neglecting quadratic terms in amplitude means the solution can be approximated by

\[ \phi(r,\theta) = \frac{B_{0}}{r} + \left(  A_{l} r + \frac{B_{1}}{r^{2}} \right) (\cos \theta) \] 

since $ P_{0}(\cos \theta) = 1 $, $ P_{1}(\cos \theta) = \cos \theta $ and wlog $ A_{0} = 0 $. We then apply our boundary condition to determine the potential but I feel like I've forgotten something. 

Then applying time dependent Bernoulli to get the pressure should be no problem...

The dynamic force will be given by the surface integral of the pressure on the inner sphere, ie.

\[ F = \int_{\theta = 0}^{\pi} \int_{\varphi = 0}^{2\pi} p_{0}(t) + \frac{a^{3} + \frac{1}{2} b^{3}}{b^{3} - a^{3}} \rho \dot{U} a \cos \theta \; \d \varphi \d \theta \]

The outer sphere has a larger surface area so the force will be more spread out. Not sure on the case of a tight fit; the pressure will shoot of to infinity?


\section{QUESTION 4}

\begin{center}
	\begin{tikzpicture}
	\draw (0, 4) -- (0, 0) -- (3, 0) -- (3, 4);
	\draw (0.5, 4) -- (0.5, 1) -- (2.5, 1) -- (2.5, 4);
	\draw [dashed] (-0.5, 2.5) -- (3.5, 2.5);
	
	\fill [mblue, opacity=0.5] (0, 1.7) -- (0, 0) -- (3, 0) -- (3, 3.3) arc(0:-180:0.25 and 0.05) -- (2.5, 1) -- (0.5, 1) -- (0.5, 1.7) arc(0:-180:0.25 and 0.05);
	
	\draw [<->] (0.8, 1.7) -- (0.8, 2.5) node [pos=0.5, right] {$ - \frac{A_{2}}{A_{1}}\zeta$};
	\draw [<->] (2.2, 2.5) -- (2.2, 3.3) node [pos=0.5, left] {$\zeta$};
	
	\node [left] at (0, 2) {$A_{1}$};
	\node [right] at (3, 2) {$A_{2}$};
	\end{tikzpicture}
\end{center}

Equal volumes of water are displaced, so if the water moves up by $ \zeta(t) $ in $ A_{2} $, it must move down by $ \frac{A_{2}}{A_{1}} \zeta $ in $ A_{1} $. On the right, we have

\[ \phi = \dot{\zeta} z, \quad \frac{\partial \phi }{\partial t} = \ddot{\zeta} z \]

and on the left,

\[ \phi = - \frac{A_{2}}{A_{1}} \dot{\zeta} z, \quad  \frac{\partial \phi }{\partial t} = \frac{A_{2}}{A_{1}} \ddot{\zeta} z   \]

Applying time-dependent Bernoulli thus yields

\begin{align*}
- & \rho \frac{A_{2}}{A_{1}} \ddot{\zeta} (h - \frac{A_{2}}{A_{1}} \zeta) +  \frac{1}{2} \rho \frac{A_{2}^{2}}{A_{1}^{2}} \dot{\zeta}^{2} + g \rho ( h - \frac{A_{2}}{A_{1}} \zeta) + p_{\text{atm}} \\
& = \rho \ddot{\zeta} (h + \zeta) + \frac{1}{2} \rho \dot{\zeta}^{2} + g \rho ( h + \zeta) + p_{\text{atm}}
\end{align*}

which simplifies to

\[ \rho \ddot{\zeta}\left(  h + \zeta + \frac{A_{2}}{A_{1}} h - \frac{A_{2}^{2}}{A_{1}^{2}} \zeta  \right) + \frac{1}{2} \rho \dot{\zeta}^{2} \left(  1 - \frac{A_{2}^{2}}{A_{1}^{2}} \right) + g \rho \zeta \left(  1 + \frac{A_{2}}{A_{1}}\right) = 0      \]

where we divide by $ \rho \left( 1 + \frac{A_{2}}{A_{1}} \right)  $ to achieve the required result.



\section{QUESTION 5}


\begin{center}
	\begin{tikzpicture}
	\draw (0, 3) -- (0, 0) -- (3, 0) -- (3, 3);
	\draw [dashed] (0, 2.5) -- (3, 2.5);
	
	\fill [mblue, opacity=0.5] (0, 3) -- (0,0) -- (3,0) -- (3,3) arc(0:-180:1.5 and 0.2);
	
	%	\draw [<->] (0.8, 1.7) -- (0.8, 2.5) node [pos=0.5, right] {$ - \frac{A_{2}}{A_{1}}\zeta$};
	%	\draw [<->] (2.2, 2.5) -- (2.2, 3.3) node [pos=0.5, left] {$\zeta$};
	%	
	%	\node [left] at (0, 2) {$A_{1}$};
	\node [right] at (3, 2.5) {$z = 0 $};
	\node [left] at (0, 2.5) {$z = \zeta(x,y,t) $};
	\end{tikzpicture}
\end{center}

Incompressibility and inviscid together imply that 

\[ \nabla^{2} \phi = 0 \]

Have no flow through the bottom or sides, so must have

\[ \frac{\partial \phi }{\partial z} = 0 \]

when $ z = -h $,

\[ \frac{\partial \phi }{\partial x} = 0 \]

when $ x = 0, a $ and similarly 

\[ \frac{\partial \phi }{\partial y} = 0 \]

when $ y = 0,a $.

At the free surface, we have

\[ \frac{\partial \phi }{\partial z} = \frac{D \zeta}{D t} = \frac{\partial \zeta }{\partial t} + u \frac{\partial \zeta }{\partial x} + v \frac{\partial \zeta }{\partial y}  \]

We then have the dynamic boundary condition that the pressure at the surface is the atmospheric pressure, i.e.\ at $z = \zeta$, we have
\[
p = p_0 = \text{constant}.
\]
We need to relate this to the flow. So we apply the time-dependent Bernoulli equation
\[
\rho \frac{\partial \phi}{\partial t} + \frac{1}{2} \rho|\nabla \phi|^2 + g\rho h + p_0 = f(t)\text{ on }z = h.
\]
The equation is not hard, but the boundary conditions are. Apart from them being non-linear, there is this surface $h$ that we know nothing about.

It is impossible to solve these equations just as they are. So we want to make some approximations. We assume that the waves amplitudes are small, i.e.\ that
\[
\zeta \ll h.
\]
Moreover, we assume that the waves are relatively flat, so that
\[
\frac{\partial \zeta}{\partial x},\frac{\partial \zeta}{\partial y} \ll 1,
\]
We then ignore quadratic terms in small quantities. For example, since the waves are small, the velocities $u$ and $v$ also are. So we ignore $u \frac{\partial \zeta}{\partial x}$ and $v\frac{\partial \zeta}{\partial y}$. Similarly, we ignore the whole of $|\nabla \phi|^2$ in Bernoulli's equations since it is small.

Next, we use Taylor series to write
\[
\left.\frac{\partial \phi}{\partial z}\right|_{z = \zeta} = \left.\frac{\partial \phi}{\partial z}\right|_{z = 0} + \zeta \left.\frac{\partial^2 \phi}{\partial z^2}\right|_{z = 0} + \cdots.
\]
Again, we ignore all quadratic terms. So we just approximate
\[
\left.\frac{\partial \phi}{\partial z}\right|_{z = \zeta} = \left.\frac{\partial \phi}{\partial z}\right|_{z = 0}.
\]
We are then left with linear water waves. The equations are then
\begin{align*}
\nabla^2 \phi &= 0 & -h < z &\leq 0\\
\frac{\partial \phi}{\partial z} &= 0 & z &= -h\\
\frac{\partial \phi}{\partial x} &= 0 & x &= 0,a\\
\frac{\partial \phi}{\partial y} &= 0 & y &= 0,a\\
\frac{\partial \phi}{\partial z} &= \frac{\partial \zeta}{\partial t} & z &= 0\\
\frac{\partial \phi}{\partial t} + g\zeta &= f(t) & z &= \zeta.
\end{align*}
Note that the last equation is just Bernoulli equations, after removing the small terms and throwing our constants and factors in to the function $f$.

Don't have time to solve to get double infinite family of solutions but think I can do it fine. 


\section{QUESTION 6}

Not sure about this question.

\section{QUESTION 7}

\begin{center}
	\begin{tikzpicture}
	\draw (0, 3) -- (0, 0) -- (3, 0) -- (3, 3);
	\draw [dashed] (0, 2.5) -- (3, 2.5);
	
	\fill [mblue, opacity=0.5] (0, 3) -- (0,0) -- (3,0) -- (3,3) arc(0:-180:1.5 and 1);
	
%	\draw [<->] (0.8, 1.7) -- (0.8, 2.5) node [pos=0.5, right] {$ - \frac{A_{2}}{A_{1}}\zeta$};
%	\draw [<->] (2.2, 2.5) -- (2.2, 3.3) node [pos=0.5, left] {$\zeta$};
%	
%	\node [left] at (0, 2) {$A_{1}$};
	\node [right] at (3, 2.5) {$z = h(r)$};
	\end{tikzpicture}
\end{center}

Assume the fluid is inviscid, so the Navier-Stokes equations become the Euler momentum equation
	\[
	\rho \frac{\D \mathbf{u}}{\D t} = - \nabla p + \mathbf{f}.
	\]
	
Note the advective term $ \mathbf{u} \cdot \nabla \mathbf{u} $ becomes:

\[ u_{x} \frac{\partial \mathbf{u} }{\partial x} + u_{y} \frac{\partial \mathbf{u} }{\partial y} + u_{z} \frac{\partial \mathbf{u} }{\partial z} \]

After a long time we have $ \mathbf{u} = (- \Omega y, \Omega x, 0) $ for constant $ \Omega $ (in Cartesians). Then we get

\begin{align*}
- \rho \Omega^{2} x & =  - \frac{\partial p }{\partial x} \\
- \rho \Omega^{2} y & = - \frac{\partial p }{\partial y} \\
0 & = - \frac{\partial p }{\partial z} - \rho g 
\end{align*} 

From the last equation, plus the boundary condition $ p = p_{0} $ on $ z = h $, we know

\[ p = p_{0} + g \rho (h - z) \]

We can put this expression into the horizontal components to get


\begin{align*}
 \Omega^{2} x & =  g \frac{\partial h }{\partial x} \\
 \Omega^{2} y & = g \frac{\partial h }{\partial y} \\
\end{align*} 

which gives

\[ h(x,y) = \frac{\Omega^{2}x^{2}}{2g} + \frac{\Omega^{2}x^{2}}{2g} + A  \]

for some constant $ A $, ie.

\[ h(r) = \frac{\Omega^{2}r^{2}}{2g} + A \]

The volume of the water in this situation is given by

\[ V = \pi a^{2} h(0) + \int_{h(0)}^{h(a)} \pi x^{2} \d y \]

where we have taken a volume of revolution along the curve $ y = \frac{\Omega^{2} x^{2}}{2g} + A $. Note that $ h(0) = A $ and $ h(a) = \frac{\Omega^{2}a^{2}}{2g} + A $. Thus

\[ V = A \pi a^{2} + \frac{2 \pi g}{\Omega^{2}} \int_{A}^{A + \frac{\Omega^{2}a^{2}}{2g}} y - A \; \d y \]

the integral evaluates to $ \frac{\Omega^{4} a^{4}}{8 g^{2}} $, hence

\[ V = A \pi a^{2} + \frac{\pi \Omega^{2} a^{4}}{4 g} \]

Thus

\[ h(r) = \frac{\Omega^{2}r^{2}}{2g} + \frac{1}{a^{2}} \left( V  -  \frac{\pi \Omega^{2} a^{4}}{4 g} \right)  \]

For the fluid to cover the whole base, want $ h(0) > 0 $, hence

\[ V > \frac{\pi \Omega^{2} a^{4}}{4 g}  \]

Thus the largest largest value of $ \Omega $ is given by 


\[ \Omega = \sqrt{\frac{4gV}{\pi a^{4}}}   \]















\section{QUESTION 8}

Not sure about this question.
\section{QUESTION 9}


Suppose we have a shallow layer of depth $z = h(x, y)$ with $p = p_0$ on $z = h$.
\begin{center}
	\begin{tikzpicture}
	\fill [gray, path fading=south] (-3, 0) rectangle (3, -1);
	
	\draw [dashed] (-3, 1) -- (3, 1);
	
	\fill [mblue, opacity=0.5] (-3, 0) -- (-3, 1) sin (-2, 1.2) cos (-1, 1) sin (0, 0.8) cos (1, 1) sin (2, 1.2) cos (3, 1) -- (3, 0) -- cycle;;
	
	\draw (-3, 1) sin (-2, 1.2) cos (-1, 1) sin (0, 0.8) cos (1, 1) sin (2, 1.2) cos (3, 1);
	
	\draw (-1.5, 2.3) node [right] {$h(x, y, t)$} edge [out=180, in=90, -latex] (-2, 1.2);
	
	\draw [->] (4, 1.5) -- +(0, -1) node [below] {$\mathbf{g}$};
	
	\draw [->] (4.5, 0.5) -- +(0, 1) node [right] {$\frac{1}{2} \mathbf{f}$};
	
	\draw [-latex'](4.35, 1) arc(135:405:0.2 and 0.1);
	\end{tikzpicture}
\end{center}
We consider motions with horizontal scales $L$ much greater than vertical scales $H$.

We use the fact that the fluid is incompressible, i.e.\ $\nabla \cdot \mathbf{u} = 0$. Writing $\mathbf{u} = (u, v, w)$, we get
\[
\frac{\partial w}{\partial z} = -\frac{\partial u}{\partial x} - \frac{\partial v}{\partial y}.
\]
The scales of the terms are $W/H$, $U/L$ and $V/L$ respectively. Since $H \ll L$, we know $W \ll U, V$, i.e.\ most of the movement is horizontal, which makes sense, since there isn't much vertical space to move around.

We consider only horizontal velocities, and write
\[
\mathbf{u} = (u, v, 0),
\]
and
\[
\mathbf{f} = (0, 0, f).
\]
Then from Euler's equations, we get
\begin{align*}
\frac{\partial u}{\partial t} - fv &= -\frac{1}{\rho} \frac{\partial p}{\partial x},\\
\frac{\partial v}{\partial t} + fu &= -\frac{1}{\rho} \frac{\partial p}{\partial y},\\
0 &= -\frac{1}{\rho}\frac{\partial p}{\partial z} -g.
\end{align*}
From the last equation, plus the boundary conditions, we know
\[
p = p_0 = g\rho(h - z).
\]
This is just the hydrostatic balance. We now put this expression into the horizontal components to get
\begin{align*}
\frac{\partial u}{\partial t} - fv &= -g\frac{\partial h}{\partial x},\\
\frac{\partial v}{\partial t} + fu &= -g\frac{\partial h}{\partial y}.
\end{align*}


To simplify the situation, we suppose we have small oscillations, so we have $h = h_0 + \eta(x, y, t)$, where $\eta \ll h_0$

Hence have

\begin{align*}
\frac{\partial u}{\partial t} - fv &= -g \frac{\partial \eta}{\partial x},\\
\frac{\partial v}{\partial t} + fu &= -g\frac{\partial \eta}{\partial y}.
\end{align*}

Not sure about rest


\end{document}