\documentclass[a4paper]{article}
\usepackage{amsmath}
\def\npart {IB}
\def\nterm {Lent}
\def\nyear {2018}
\def\nlecturer {Prof. Haynes (P.H.Haynes@damtp.cam.ac.uk)}
\def\ncourse {Fluid Dynamics Example Sheet 1}

\input{header}

\newtheorem*{soln}{Solution}

\renewcommand{\thesection}{}
\renewcommand{\thesubsection}{\arabic{section}.\arabic{subsection}}
\makeatletter
\def\@seccntformat#1{\csname #1ignore\expandafter\endcsname\csname the#1\endcsname\quad}
\let\sectionignore\@gobbletwo
\let\latex@numberline\numberline
\def\numberline#1{\if\relax#1\relax\else\latex@numberline{#1}\fi}
\makeatother


\begin{document}
	
\maketitle

\section{QUESTION 1}
\section{QUESTION 2}
\section{QUESTION 3}
\section{QUESTION 4}
\section{QUESTION 5}



\section{QUESTION 6}

\begin{enumerate}
	\item At $ t = 1 $, the velocity makes an angle of $ 45^\circ $ angle with the horizontal , and the streamlines are slanted.
	
	\item For a particle released at $ (1,1) $ we get
	
	\[ \dot{x}(t) = \frac{1}{1+t}, \qquad \dot{y}(t) = 1  \]
	
	Hence we get 
	
	\[ x = \log(1+t) + 1, \qquad y = t + 1 \]
	
	Eliminating $ t $, we get that the path is given by
	
	\[ x = \log y + 1 \implies y = e^{x-1} \]
	
	
	
\end{enumerate}



\section{QUESTION 7}

Consider the fluid flow $ \mathbf{u} = (  1/(1+t),1,0 ) $ for $ t > 0 $. Supposing our fluid is incompressible, $ \nabla \cdot \mathbf{u} = 0 $ and there exists some vector potential $ \mathbf{A} $ such that $ \mathbf{u} = \nabla  \times \mathbf{A} $. As $ \mathbf{u} $ is two dimensional, we know $ \mathbf{A} $ is of the from

\[ \mathbf{A} = (0,0,\psi(x,y,t)) \]

And taking the curl of this, 

\[ \mathbf{u} = ( \frac{\partial \psi }{\partial y}, - \frac{\partial \psi }{\partial x}, 0 ) \]

This $ \psi $ is our streamfunction.

We define streamlines to be contours of our stream function. These contours have normal $ \mathbf{n} = ( \psi_{x},\psi_{y},0 ) $, this normal is obviously perpendicular to the flow $ \mathbf{u} $ ($ \mathbf{u} \cdot \mathbf{n} = 0 $). ie. the flow is tangent to the contours of $ \psi$. 

\section{QUESTION 8}

\[ \nabla \cdot \mathbf{u} = 0 \iff \frac{\partial u }{\partial x} + \frac{\partial v}{\partial y} = 0  \]

By the quotient rule,

\begin{align*}
\frac{\partial u }{\partial x}& = \frac{-(y-b)[2(x-a)]}{( (x-a) + (y-b))^{2}} 
\end{align*}

\begin{align*}
\frac{\partial v }{\partial y} & = \frac{-(a-x)[2(y-b)] }{( (x-a)^{2} + (y-b)^{2}   )} \\
\end{align*}

So $  \frac{\partial u }{\partial x} = - \frac{\partial v }{\partial y} $ and $  \nabla \cdot \mathbf{u} = 0 $, our fluid is indeed incompressible. 



\section{QUESTION 9}

\[ \nabla \cdot \mathbf{u} = 0 \iff  \frac{1}{r} \frac{\partial }{\partial r}(ru_{r}) + \frac{1}{r} \frac{\partial }{\partial \theta}(u_{\theta}) = 0  \]

\begin{align*}
\frac{\partial }{\partial r}(ru_{r}) & = \frac{\partial }{\partial r} \left[  U \left(  r - \frac{a^{2}}{r} \right) \cos \theta \right]    \\
& = U \left(  1 + \frac{a^{2}}{r^{2}} \right) \cos \theta 
\end{align*}

\begin{align*}
\frac{\partial }{\partial \theta}(u_{\theta}) & = \frac{\partial }{\partial \theta} \left[ -U \left( 1 + \frac{a^{2}}{r^{2}} \right)  \sin \theta \right]    \\
& = U \left( 1 + \frac{a^{2}}{r^{2}} \right)  \cos \theta
\end{align*}

Hence 

\[  \frac{1}{r} \frac{\partial }{\partial r}(ru_{r}) = - \frac{1}{r} \frac{\partial }{\partial \theta}(u_{\theta}) \implies \nabla \cdot \mathbf{u} = 0  \]



\section{QUESTION 10}
\section{QUESTION 11}
\section{QUESTION 12}



\end{document}