\documentclass[a4paper]{article}
\usepackage{amsmath}
\def\npart {IB}
\def\nterm {Lent}
\def\nyear {2018}
\def\nlecturer {Prof. Haynes (P.H.Haynes@damtp.cam.ac.uk)}
\def\ncourse {Fluid Dynamics Example Sheet 1}

% Imports
\ifx \nauthor\undefined
  \def\nauthor{Christopher Turnbull}
\else
\fi

\author{Supervised by \nlecturer \\\small Solutions presented by \nauthor}
\date{\nterm\ \nyear}

\usepackage{alltt}
\usepackage{amsfonts}
\usepackage{amsmath}
\usepackage{amssymb}
\usepackage{amsthm}
\usepackage{booktabs}
\usepackage{caption}
\usepackage{enumitem}
\usepackage{fancyhdr}
\usepackage{graphicx}
\usepackage{mathdots}
\usepackage{mathtools}
\usepackage{microtype}
\usepackage{multirow}
\usepackage{pdflscape}
\usepackage{pgfplots}
\usepackage{siunitx}
\usepackage{slashed}
\usepackage{tabularx}
\usepackage{tikz}
\usepackage{tkz-euclide}
\usepackage[normalem]{ulem}
\usepackage[all]{xy}
\usepackage{imakeidx}

\makeindex[intoc, title=Index]
\indexsetup{othercode={\lhead{\emph{Index}}}}

\ifx \nextra \undefined
  \usepackage[pdftex,
    hidelinks,
    pdfauthor={Christopher Turnbull},
    pdfsubject={Cambridge Maths Notes: Part \npart\ - \ncourse},
    pdftitle={Part \npart\ - \ncourse},
  pdfkeywords={Cambridge Mathematics Maths Math \npart\ \nterm\ \nyear\ \ncourse}]{hyperref}
  \title{Part \npart\ --- \ncourse}
\else
  \usepackage[pdftex,
    hidelinks,
    pdfauthor={Christopher Turnbull},
    pdfsubject={Cambridge Maths Notes: Part \npart\ - \ncourse\ (\nextra)},
    pdftitle={Part \npart\ - \ncourse\ (\nextra)},
  pdfkeywords={Cambridge Mathematics Maths Math \npart\ \nterm\ \nyear\ \ncourse\ \nextra}]{hyperref}

  \title{Part \npart\ --- \ncourse \\ {\Large \nextra}}
  \renewcommand\printindex{}
\fi

\pgfplotsset{compat=1.12}

\pagestyle{fancyplain}
\lhead{\emph{\nouppercase{\leftmark}}}
\ifx \nextra \undefined
  \rhead{
    \ifnum\thepage=1
    \else
      \npart\ \ncourse
    \fi}
\else
  \rhead{
    \ifnum\thepage=1
    \else
      \npart\ \ncourse\ (\nextra)
    \fi}
\fi
\usetikzlibrary{arrows.meta}
\usetikzlibrary{decorations.markings}
\usetikzlibrary{decorations.pathmorphing}
\usetikzlibrary{positioning}
\usetikzlibrary{fadings}
\usetikzlibrary{intersections}
\usetikzlibrary{cd}

\newcommand*{\Cdot}{{\raisebox{-0.25ex}{\scalebox{1.5}{$\cdot$}}}}
\newcommand {\pd}[2][ ]{
  \ifx #1 { }
    \frac{\partial}{\partial #2}
  \else
    \frac{\partial^{#1}}{\partial #2^{#1}}
  \fi
}
\ifx \nhtml \undefined
\else
  \renewcommand\printindex{}
  \makeatletter
  \DisableLigatures[f]{family = *}
  \let\Contentsline\contentsline
  \renewcommand\contentsline[3]{\Contentsline{#1}{#2}{}}
  \renewcommand{\@dotsep}{10000}
  \newlength\currentparindent
  \setlength\currentparindent\parindent

  \newcommand\@minipagerestore{\setlength{\parindent}{\currentparindent}}
  \usepackage[active,tightpage,pdftex]{preview}
  \renewcommand{\PreviewBorder}{0.1cm}

  \newenvironment{stretchpage}%
  {\begin{preview}\begin{minipage}{\hsize}}%
    {\end{minipage}\end{preview}}
  \AtBeginDocument{\begin{stretchpage}}
  \AtEndDocument{\end{stretchpage}}

  \newcommand{\@@newpage}{\end{stretchpage}\begin{stretchpage}}

  \let\@real@section\section
  \renewcommand{\section}{\@@newpage\@real@section}
  \let\@real@subsection\subsection
  \renewcommand{\subsection}{\@@newpage\@real@subsection}
  \makeatother
\fi

% Theorems
\theoremstyle{definition}
\newtheorem*{aim}{Aim}
\newtheorem*{axiom}{Axiom}
\newtheorem*{claim}{Claim}
\newtheorem*{cor}{Corollary}
\newtheorem*{conjecture}{Conjecture}
\newtheorem*{defi}{Definition}
\newtheorem*{eg}{Example}
\newtheorem*{ex}{Exercise}
\newtheorem*{fact}{Fact}
\newtheorem*{law}{Law}
\newtheorem*{lemma}{Lemma}
\newtheorem*{notation}{Notation}
\newtheorem*{prop}{Proposition}
\newtheorem*{soln}{Solution}
\newtheorem*{thm}{Theorem}

\newtheorem*{remark}{Remark}
\newtheorem*{warning}{Warning}
\newtheorem*{exercise}{Exercise}

\newtheorem{nthm}{Theorem}[section]
\newtheorem{nlemma}[nthm]{Lemma}
\newtheorem{nprop}[nthm]{Proposition}
\newtheorem{ncor}[nthm]{Corollary}


\renewcommand{\labelitemi}{--}
\renewcommand{\labelitemii}{$\circ$}
\renewcommand{\labelenumi}{(\roman{*})}

\let\stdsection\section
\renewcommand\section{\newpage\stdsection}

% Strike through
\def\st{\bgroup \ULdepth=-.55ex \ULset}

% Maths symbols
\newcommand{\abs}[1]{\left\lvert #1\right\rvert}
\newcommand\ad{\mathrm{ad}}
\newcommand\AND{\mathsf{AND}}
\newcommand\Art{\mathrm{Art}}
\newcommand{\Bilin}{\mathrm{Bilin}}
\newcommand{\bket}[1]{\left\lvert #1\right\rangle}
\newcommand{\B}{\mathcal{B}}
\newcommand{\bolds}[1]{{\bfseries #1}}
\newcommand{\brak}[1]{\left\langle #1 \right\rvert}
\newcommand{\braket}[2]{\left\langle #1\middle\vert #2 \right\rangle}
\newcommand{\bra}{\langle}
\newcommand{\cat}[1]{\mathsf{#1}}
\newcommand{\C}{\mathbb{C}}
\newcommand{\CP}{\mathbb{CP}}
\newcommand{\cU}{\mathcal{U}}
\newcommand{\Der}{\mathrm{Der}}
\newcommand{\D}{\mathrm{D}}
\newcommand{\dR}{\mathrm{dR}}
\newcommand{\E}{\mathbb{E}}
\newcommand{\F}{\mathbb{F}}
\newcommand{\Frob}{\mathrm{Frob}}
\newcommand{\GG}{\mathbb{G}}
\newcommand{\gl}{\mathfrak{gl}}
\newcommand{\GL}{\mathrm{GL}}
\newcommand{\G}{\mathcal{G}}
\newcommand{\Gr}{\mathrm{Gr}}
\newcommand{\haut}{\mathrm{ht}}
\newcommand{\Id}{\mathrm{Id}}
\newcommand{\ket}{\rangle}
\newcommand{\lie}[1]{\mathfrak{#1}}
\newcommand{\Mat}{\mathrm{Mat}}
\newcommand{\N}{\mathbb{N}}
\newcommand{\norm}[1]{\left\lVert #1\right\rVert}
\newcommand{\normalorder}[1]{\mathop{:}\nolimits\!#1\!\mathop{:}\nolimits}
\newcommand\NOT{\mathsf{NOT}}
\newcommand{\Oc}{\mathcal{O}}
\newcommand{\Or}{\mathrm{O}}
\newcommand\OR{\mathsf{OR}}
\newcommand{\ort}{\mathfrak{o}}
\newcommand{\PGL}{\mathrm{PGL}}
\newcommand{\ph}{\,\cdot\,}
\newcommand{\pr}{\mathrm{pr}}
\newcommand{\Prob}{\mathbb{P}}
\newcommand{\PSL}{\mathrm{PSL}}
\newcommand{\Ps}{\mathcal{P}}
\newcommand{\PSU}{\mathrm{PSU}}
\newcommand{\pt}{\mathrm{pt}}
\newcommand{\qeq}{\mathrel{``{=}"}}
\newcommand{\Q}{\mathbb{Q}}
\newcommand{\R}{\mathbb{R}}
\newcommand{\RP}{\mathbb{RP}}
\newcommand{\Rs}{\mathcal{R}}
\newcommand{\SL}{\mathrm{SL}}
\newcommand{\so}{\mathfrak{so}}
\newcommand{\SO}{\mathrm{SO}}
\newcommand{\Spin}{\mathrm{Spin}}
\newcommand{\Sp}{\mathrm{Sp}}
\newcommand{\su}{\mathfrak{su}}
\newcommand{\SU}{\mathrm{SU}}
\newcommand{\term}[1]{\emph{#1}\index{#1}}
\newcommand{\T}{\mathbb{T}}
\newcommand{\tv}[1]{|#1|}
\newcommand{\U}{\mathrm{U}}
\newcommand{\uu}{\mathfrak{u}}
\newcommand{\Vect}{\mathrm{Vect}}
\newcommand{\wsto}{\stackrel{\mathrm{w}^*}{\to}}
\newcommand{\wt}{\mathrm{wt}}
\newcommand{\wto}{\stackrel{\mathrm{w}}{\to}}
\newcommand{\Z}{\mathbb{Z}}
\renewcommand{\d}{\mathrm{d}}
\renewcommand{\H}{\mathbb{H}}
\renewcommand{\P}{\mathbb{P}}
\renewcommand{\sl}{\mathfrak{sl}}
\renewcommand{\vec}[1]{\boldsymbol{\mathbf{#1}}}
%\renewcommand{\F}{\mathcal{F}}

\let\Im\relax
\let\Re\relax

\DeclareMathOperator{\adj}{adj}
\DeclareMathOperator{\Ann}{Ann}
\DeclareMathOperator{\area}{area}
\DeclareMathOperator{\Aut}{Aut}
\DeclareMathOperator{\Bernoulli}{Bernoulli}
\DeclareMathOperator{\betaD}{beta}
\DeclareMathOperator{\bias}{bias}
\DeclareMathOperator{\binomial}{binomial}
\DeclareMathOperator{\card}{card}
\DeclareMathOperator{\ccl}{ccl}
\DeclareMathOperator{\Char}{char}
\DeclareMathOperator{\ch}{ch}
\DeclareMathOperator{\cl}{cl}
\DeclareMathOperator{\cls}{\overline{\mathrm{span}}}
\DeclareMathOperator{\conv}{conv}
\DeclareMathOperator{\corr}{corr}
\DeclareMathOperator{\cosec}{cosec}
\DeclareMathOperator{\cosech}{cosech}
\DeclareMathOperator{\cov}{cov}
\DeclareMathOperator{\covol}{covol}
\DeclareMathOperator{\diag}{diag}
\DeclareMathOperator{\diam}{diam}
\DeclareMathOperator{\Diff}{Diff}
\DeclareMathOperator{\disc}{disc}
\DeclareMathOperator{\dom}{dom}
\DeclareMathOperator{\End}{End}
\DeclareMathOperator{\energy}{energy}
\DeclareMathOperator{\erfc}{erfc}
\DeclareMathOperator{\erf}{erf}
\DeclareMathOperator*{\esssup}{ess\,sup}
\DeclareMathOperator{\ev}{ev}
\DeclareMathOperator{\Ext}{Ext}
\DeclareMathOperator{\Fit}{Fit}
\DeclareMathOperator{\fix}{fix}
\DeclareMathOperator{\Frac}{Frac}
\DeclareMathOperator{\Gal}{Gal}
\DeclareMathOperator{\gammaD}{gamma}
\DeclareMathOperator{\gr}{gr}
\DeclareMathOperator{\hcf}{hcf}
\DeclareMathOperator{\Hom}{Hom}
\DeclareMathOperator{\id}{id}
\DeclareMathOperator{\image}{image}
\DeclareMathOperator{\im}{im}
\DeclareMathOperator{\Im}{Im}
\DeclareMathOperator{\Ind}{Ind}
\DeclareMathOperator{\Int}{Int}
\DeclareMathOperator{\Isom}{Isom}
\DeclareMathOperator{\lcm}{lcm}
\DeclareMathOperator{\length}{length}
\DeclareMathOperator{\Lie}{Lie}
\DeclareMathOperator{\like}{like}
\DeclareMathOperator{\Lk}{Lk}
\DeclareMathOperator{\mse}{mse}
\DeclareMathOperator{\multinomial}{multinomial}
\DeclareMathOperator{\orb}{orb}
\DeclareMathOperator{\ord}{ord}
\DeclareMathOperator{\otp}{otp}
\DeclareMathOperator{\Poisson}{Poisson}
\DeclareMathOperator{\poly}{poly}
\DeclareMathOperator{\rank}{rank}
\DeclareMathOperator{\rel}{rel}
\DeclareMathOperator{\Re}{Re}
\DeclareMathOperator*{\res}{res}
\DeclareMathOperator{\Res}{Res}
\DeclareMathOperator{\rk}{rk}
\DeclareMathOperator{\Root}{Root}
\DeclareMathOperator{\sech}{sech}
\DeclareMathOperator{\sgn}{sgn}
\DeclareMathOperator{\spn}{span}
\DeclareMathOperator{\stab}{stab}
\DeclareMathOperator{\St}{St}
\DeclareMathOperator{\supp}{supp}
\DeclareMathOperator{\Syl}{Syl}
\DeclareMathOperator{\Sym}{Sym}
\DeclareMathOperator{\tr}{tr}
\DeclareMathOperator{\Tr}{Tr}
\DeclareMathOperator{\var}{var}
\DeclareMathOperator{\vol}{vol}

\pgfarrowsdeclarecombine{twolatex'}{twolatex'}{latex'}{latex'}{latex'}{latex'}
\tikzset{->/.style = {decoration={markings,
                                  mark=at position 1 with {\arrow[scale=2]{latex'}}},
                      postaction={decorate}}}
\tikzset{<-/.style = {decoration={markings,
                                  mark=at position 0 with {\arrowreversed[scale=2]{latex'}}},
                      postaction={decorate}}}
\tikzset{<->/.style = {decoration={markings,
                                   mark=at position 0 with {\arrowreversed[scale=2]{latex'}},
                                   mark=at position 1 with {\arrow[scale=2]{latex'}}},
                       postaction={decorate}}}
\tikzset{->-/.style = {decoration={markings,
                                   mark=at position #1 with {\arrow[scale=2]{latex'}}},
                       postaction={decorate}}}
\tikzset{-<-/.style = {decoration={markings,
                                   mark=at position #1 with {\arrowreversed[scale=2]{latex'}}},
                       postaction={decorate}}}
\tikzset{->>/.style = {decoration={markings,
                                  mark=at position 1 with {\arrow[scale=2]{latex'}}},
                      postaction={decorate}}}
\tikzset{<<-/.style = {decoration={markings,
                                  mark=at position 0 with {\arrowreversed[scale=2]{twolatex'}}},
                      postaction={decorate}}}
\tikzset{<<->>/.style = {decoration={markings,
                                   mark=at position 0 with {\arrowreversed[scale=2]{twolatex'}},
                                   mark=at position 1 with {\arrow[scale=2]{twolatex'}}},
                       postaction={decorate}}}
\tikzset{->>-/.style = {decoration={markings,
                                   mark=at position #1 with {\arrow[scale=2]{twolatex'}}},
                       postaction={decorate}}}
\tikzset{-<<-/.style = {decoration={markings,
                                   mark=at position #1 with {\arrowreversed[scale=2]{twolatex'}}},
                       postaction={decorate}}}

\tikzset{circ/.style = {fill, circle, inner sep = 0, minimum size = 3}}
\tikzset{mstate/.style={circle, draw, blue, text=black, minimum width=0.7cm}}

\tikzset{commutative diagrams/.cd,cdmap/.style={/tikz/column 1/.append style={anchor=base east},/tikz/column 2/.append style={anchor=base west},row sep=tiny}}

\definecolor{mblue}{rgb}{0.2, 0.3, 0.8}
\definecolor{morange}{rgb}{1, 0.5, 0}
\definecolor{mgreen}{rgb}{0.1, 0.4, 0.2}
\definecolor{mred}{rgb}{0.5, 0, 0}

\def\drawcirculararc(#1,#2)(#3,#4)(#5,#6){%
    \pgfmathsetmacro\cA{(#1*#1+#2*#2-#3*#3-#4*#4)/2}%
    \pgfmathsetmacro\cB{(#1*#1+#2*#2-#5*#5-#6*#6)/2}%
    \pgfmathsetmacro\cy{(\cB*(#1-#3)-\cA*(#1-#5))/%
                        ((#2-#6)*(#1-#3)-(#2-#4)*(#1-#5))}%
    \pgfmathsetmacro\cx{(\cA-\cy*(#2-#4))/(#1-#3)}%
    \pgfmathsetmacro\cr{sqrt((#1-\cx)*(#1-\cx)+(#2-\cy)*(#2-\cy))}%
    \pgfmathsetmacro\cA{atan2(#2-\cy,#1-\cx)}%
    \pgfmathsetmacro\cB{atan2(#6-\cy,#5-\cx)}%
    \pgfmathparse{\cB<\cA}%
    \ifnum\pgfmathresult=1
        \pgfmathsetmacro\cB{\cB+360}%
    \fi
    \draw (#1,#2) arc (\cA:\cB:\cr);%
}
\newcommand\getCoord[3]{\newdimen{#1}\newdimen{#2}\pgfextractx{#1}{\pgfpointanchor{#3}{center}}\pgfextracty{#2}{\pgfpointanchor{#3}{center}}}

\def\Xint#1{\mathchoice
   {\XXint\displaystyle\textstyle{#1}}%
   {\XXint\textstyle\scriptstyle{#1}}%
   {\XXint\scriptstyle\scriptscriptstyle{#1}}%
   {\XXint\scriptscriptstyle\scriptscriptstyle{#1}}%
   \!\int}
\def\XXint#1#2#3{{\setbox0=\hbox{$#1{#2#3}{\int}$}
     \vcenter{\hbox{$#2#3$}}\kern-.5\wd0}}
\def\ddashint{\Xint=}
\def\dashint{\Xint-}

\newcommand\separator{{\centering\rule{2cm}{0.2pt}\vspace{2pt}\par}}

\newenvironment{own}{\color{gray!70!black}}{}

\newcommand\makecenter[1]{\raisebox{-0.5\height}{#1}}

\newtheorem*{soln}{Solution}

\renewcommand{\thesection}{}
\renewcommand{\thesubsection}{\arabic{section}.\arabic{subsection}}
\makeatletter
\def\@seccntformat#1{\csname #1ignore\expandafter\endcsname\csname the#1\endcsname\quad}
\let\sectionignore\@gobbletwo
\let\latex@numberline\numberline
\def\numberline#1{\if\relax#1\relax\else\latex@numberline{#1}\fi}
\makeatother


\begin{document}
	
\maketitle

\section{QUESTION 1}

In this example, $ \mathbf{u}(x) =  (\alpha x,- \alpha y) $

\begin{enumerate}
	\item Streamlines are curves $ \mathbf{X}(s; \mathbf{x}_{0}) $ with $ \mathbf{X} = \mathbf{x}_{0} $ at $ s = 0 $. 
	
	\[ \frac{\partial X }{\partial s} = \alpha X, \quad \frac{\partial Y }{\partial s} = - \alpha Y \]
	
	$ X = x_{0} e^{\alpha s} $ and $ Y = y_{0}e^{- \alpha s} $. Eliminating $ s $ to get the shape of the streamlines gives $ XY = x_{0}y_{0} $. These are hyperbola.
	
	\item For steady flow, pathlines and streamlines are coincident. For a particle released at $ \mathbf{x}_{0} = (x_{0},y_{0}) $, the particle pathline obeys
	
	\[ x = x_{0} e^{\alpha t}, \quad y = y_{0}e^{- \alpha t} \]
	
	as  As $ x_{0}^{2} + y_{0}^{2} = a^{2} $ we find that the curve evolves as
	
	\[ x^{2}e^{-2\alpha t} + y^{2}e^{2 \alpha t} = a^{2} \]
	
	ie; this is an ellipse,
	
	
	\[ \frac{x^{2}}{(ae^{at})^{2}} + \frac{y^{2}}{(ae^{-at})^{2}} = 1 \]
	
	\item 
	
	The area of this ellipse does not change in time; it is $ \pi \times ae^{at} \times ae^{-at} = \pi a^{2} $. As we would expect this is the area of the circle $ x^{2} + y^{2} = a^{2} $ at time $ t = 0 $. 
	
	Intuitively?
	
	
\end{enumerate}


\section{QUESTION 2}

In this example, $ \mathbf{u}(x) = (\gamma y, 0) $

\begin{enumerate}
	\item Streamlines are curves $ \mathbf{X}(s; \mathbf{x}_{0}) $ with $ \mathbf{X}_{0} = \mathbf{x}_{0} $ at $ s = 0 $. 
	
	\[ \frac{\partial X }{\partial s} = \gamma y, \quad \frac{\partial Y }{\partial s} = 0 \]
	
	$ X = x_{0} + \gamma y_{0} s $ and $ Y = y_{0}$. Note that we cannot eliminate $ s $ to get the shape of the streamlines; they are given by;
	
	\[ X = x_{0} + \gamma Y s \]
	
	\item For steady flow, pathlines and streamlines are coincident. For a particle released at $ \mathbf{x}_{0} = (x_{0},y_{0}) $, the particle pathline obeys
	
	\[ x = x_{0} + \gamma y_{0} t, \quad y = y_{0}  \]
	
	as  As $ x_{0}^{2} + y_{0}^{2} = a^{2} $ we find that the curve evolves as
	
	\[ (x - \gamma t y_{0})^{2} + y^{2} = a^{2}  \]
	
	ie. the circle is just shifted to the right, at a rate of $ \gamma y_{0} $ units per second. 
	
	\item The curve is just translated so the area remains fixed.

	
	
\end{enumerate}

At longer times, the ellipse in Question 1 is `stretched' faster, as this grows exponentially, yet the \st{stretch} translation in this circle is just linear with time. 
	


\section{QUESTION 3}

In this example, $ \mathbf{u}(x) = (\frac{\partial \psi }{\partial y}, - \frac{\partial \psi }{\partial x}) $.

\begin{enumerate}
	\item The contours $ \psi = c $ have normal
	
	\[ \mathbf{n} = \nabla  \psi = (\psi_{x},\psi_{y}) \]
	
	We see immediately that
	
	\[ \mathbf{u} \cdot \mathbf{n} = \psi_{x} \psi_{y} - \psi_{y} \psi_{x} = 0 \]
	
	So the flow is perpendicular to the normal, ie. tangent to the contours of $ \psi $. 
	
	\item $ \left| \mathbf{u} \right| = | (\psi_{y},-\psi_{x}) | =  | (\psi_{x},\psi_{y}) | = | \nabla \psi | $
	
	Hence flow is faster when streamlines are closer together. (Why does $ | \nabla  \psi | $ increasing mean that they are closer together?) Can also see this is true intuitively, as the fluid between any two streamlines must be between the streamlines. 
	
	
	\item The volume flux is
	\[
	q = \int_{\mathbf{x}_0}^{\mathbf{x}_1} \mathbf{u}\cdot \mathbf{n}\;\d s.
	\]
	We see that
	\[
	\mathbf{n}\;\d s = (-\d y, \d x).
	\]
	So we can write this as
	\[
	q = \int_{\mathbf{x}_0}^{\mathbf{x}_1} -\frac{\partial \psi}{\partial y}\;\d y - \frac{\partial \psi}{\partial x}\;\d x = \psi(\mathbf{x}_0) - \psi(\mathbf{x}_1).
	\]
	So the flux depends only the difference in the value of $\psi$. Hence, for closer streamlines, to maintain the same volume flux, we need a higher speed.
	
	\item Note that $\psi$ is constant on a stationary rigid boundary, i.e.\ the boundary is a streamline, since the flow is tangential at the boundary. This is a consequence of $\mathbf{u}\cdot \mathbf{n} = 0$. We often choose $\psi = 0$ as our boundary.
	
	
\end{enumerate}









\section{QUESTION 4}

\[ \nabla \cdot \mathbf{u} = 0 \iff \frac{\partial u }{\partial x} + \frac{\partial v}{\partial y} = 0  \]

By the quotient rule,

\begin{align*}
\frac{\partial u }{\partial x}& = \frac{-(y-b)[2(x-a)]}{( (x-a) + (y-b))^{2}} 
\end{align*}

\begin{align*}
\frac{\partial v }{\partial y} & = \frac{-(a-x)[2(y-b)] }{( (x-a)^{2} + (y-b)^{2}   )} \\
\end{align*}

So $  \frac{\partial u }{\partial x} = - \frac{\partial v }{\partial y} $ and $  \nabla \cdot \mathbf{u} = 0 $, our fluid is indeed incompressible.

By inspection, the required streamfunction is 

\[ \psi(x,y) = \frac{1}{2} \log \left[   (x-a)^{2} + (y-b)^{2}  \right]  \]

 


\section{QUESTION 5}

\[ \nabla \cdot \mathbf{u} = 0 \iff  \frac{1}{r} \frac{\partial }{\partial r}(ru_{r}) + \frac{1}{r} \frac{\partial }{\partial \theta}(u_{\theta}) = 0  \]

\begin{align*}
\frac{\partial }{\partial r}(ru_{r}) & = \frac{\partial }{\partial r} \left[  U \left(  r - \frac{a^{2}}{r} \right) \cos \theta \right]    \\
& = U \left(  1 + \frac{a^{2}}{r^{2}} \right) \cos \theta 
\end{align*}

\begin{align*}
\frac{\partial }{\partial \theta}(u_{\theta}) & = \frac{\partial }{\partial \theta} \left[ -U \left( 1 + \frac{a^{2}}{r^{2}} \right)  \sin \theta \right]    \\
& = U \left( 1 + \frac{a^{2}}{r^{2}} \right)  \cos \theta
\end{align*}

Hence 

\[  \frac{1}{r} \frac{\partial }{\partial r}(ru_{r}) = - \frac{1}{r} \frac{\partial }{\partial \theta}(u_{\theta}) \implies \nabla \cdot \mathbf{u} = 0  \]


By inspection the streamfunction is

\[ \psi(r,\theta) = U \left(  r - \frac{a^{2}}{r} \right) \sin \theta  \]




\section{QUESTION 6}

\[ \nabla \cdot \mathbf{u} = 0 \iff  \frac{1}{r} \frac{\partial }{\partial r}(ru_{r}) + \frac{\partial u_{z}}{\partial z} = 0  \]


\begin{align*}
\frac{\partial }{\partial r}(ru_{r}) & = \frac{\partial }{\partial r} \left[ - \frac{1}{2} \alpha r^{2} \right]     \\
& = - r \alpha
\end{align*}

\begin{align*}
\frac{\partial }{\partial z}(u_{z}) & = \frac{\partial }{\partial z} \left[ \alpha z \right]    \\
& = \alpha
\end{align*}

Hence 

\[  \frac{1}{r} \frac{\partial }{\partial r}(ru_{r}) = - \frac{\partial u_{z}}{\partial z} \implies \nabla \cdot \mathbf{u} = 0  \]


By inspection the streamfunction is

\[ \Psi(r,z) = \frac{1}{2} \alpha r^{2} z  \]






\section{QUESTION 7}


In this example, $ \mathbf{u}(x) = (  1/(1+t), 1  ) $

\begin{enumerate}
	\item Streamlines are curves $ \mathbf{X}(s; \mathbf{x}_{0}) $ with $ \mathbf{X}_{0} = \mathbf{x}_{0} $ at $ s = 0 $. 
	
	\[ \frac{\partial X }{\partial s} = \frac{1}{1+t}, \quad \frac{\partial Y }{\partial s} = 1 \]
	
	$ X = \frac{s}{1+t} + x_{0} $ and $ Y = s + y_{0} $. Eliminating $ s $ to get the shape of the streamlines gives $ X = \frac{Y - y_{0}}{1+t} + x_{0} $. At time $ t = 0 $, with $ \mathbf{x}_{0} = (1,1) $ this gives the streamline as
	
	\[ Y = X \]
	
	
	
	\item The pathline of the particle is given by
	
	\[ \frac{\partial X }{\partial t} = \frac{1}{1+t}, \quad \frac{\partial Y }{\partial t} = 1 \]
	
	with $ \mathbf{X} = \mathbf{x}_{0} $ at $ t = 0 $. Thus $ X = \log(1+t) + x_{0} $ and $ Y = t + y_{0} $. Eliminating $ t $ we get the shape of the pathline as
	
	\[ X = \log(1+ Y - y_{0}) + x_{0} \]
	
	At $ \mathbf{x}_{0} = (1,1) $ we have
	
	\[ X = \log Y + 1 \]
	
	or $ Y = e^{X - 1} $
	
	
	
	\item Streaklines are curves $ \mathbf{X}(s; \mathbf{x}_{0}) $ with $ \mathbf{X}_{0} = \mathbf{x}_{0} $ at $ t = s $.
	
	\[ \frac{\partial X }{\partial t} = \frac{1}{1+t}, \quad \frac{\partial Y }{\partial t} = 1 \]
	
	This gives $ X = \log \left(  \frac{1+t}{1+s} \right) + 1  $, $ Y = t + 1 - s   $. 
	
	Eliminating $ s $ to get the shape at time $ t > s $:
	
	\[ X = \log \left(  \frac{1+t}{1+t - Y} \right)  \]
	
	
	
	
\end{enumerate}





\section{QUESTION 8}


The integral from of the momentum equation:

\[ \frac{\d }{\d t}  \int_{V}  \rho \mathbf{u} \; \d V = \int_{V} \rho \mathbf{g} \; \d V - \int_{S} p \mathbf{n} + \rho \mathbf{u} (\mathbf{u} \cdot \mathbf{n}) \; \d S \]

For steady flow, the LHS is zero. Neglecting gravity, we have the total force on the wall as 

\begin{align*}
& = \int_{S} p \mathbf{n} \; \d S \\
& = - \int_{S} \rho \mathbf{u} (\mathbf{u} \cdot \mathbf{n}) \; \d S 
\end{align*}

So the force on the wall is equal to the surface integral, which gives $ - \rho A U^{2} \mathbf{n} $. Assuming $ \rho = 10^{3} \text{kg m}^{-3} $; $ \Rightarrow 0.6 $ N force. 



\section{QUESTION 9}

\begin{prop}[Euler momentum equation]
	\[
	\rho \frac{\D \mathbf{u}}{\D t} = - \nabla p + \mathbf{f}.
	\]
\end{prop}

For conservative forces, we can write $\mathbf{f} = -\nabla \chi$, where $\chi$ is a scalar potential.

We notice the vector identity
\[
\mathbf{u}\times (\nabla \times \mathbf{u}) = \nabla\left(\frac{1}{2}|\mathbf{u}|^2\right) - \mathbf{u}\cdot \nabla \mathbf{u}.
\]
We use this to rewrite the Euler momentum equation as
\[
\rho \frac{\partial \mathbf{u}}{\partial t} + \rho \nabla \left(\frac{1}{2}|\mathbf{u}|^2\right) - \rho \mathbf{u}\times (\nabla \times \mathbf{u}) = -\nabla p - \nabla \chi.
\]
Dotting with $\mathbf{u}$, the last term on the left vanishes, and we get
\begin{prop}[Bernoulli's equation]
	\[
	\frac{1}{2}\rho \frac{\partial|\mathbf{u}|^2}{\partial t} = -\mathbf{u}\cdot \nabla \left(\frac{1}{2} \rho |\mathbf{u}|^2 + p + \chi\right).
	\]
\end{prop}
Note also that this tells us that high velocity goes with low pressure; low pressure goes with high velocity.

In the case where we have a steady flow, we know
\[
H = \frac{1}{2}\rho |\mathbf{u}|^2 + p + \chi
\]
is constant along streamlines.

Even if the flow is not steady, we can still define the value $H$, and then we can integrate Bernoulli's equation over a volume $\mathcal{D}$ to obtain
\[
\frac{\d}{\d t}\int_{\mathcal{D}} \frac{1}{2}\rho |\mathbf{u}|^2 \;\d V + \int_{\partial \mathcal{D}} H \mathbf{u}\cdot \mathbf{n} \;\d S. = 0 
\]
So $H$ is the transportable energy of the flow.




\section{QUESTION 10}

Bernoulli says the quantity $ H $ is constant along streamlines:

\[ H = \frac{1}{2} | \mathbf{u} |^{2} + \frac{p}{\rho} + \chi \]

Applying Bernoulli along surface streamline gives

\[ \frac{p_{1}}{\rho} + \frac{1}{2} U_{1}^{2} = \frac{p_{2}}{\rho} + \frac{1}{2} U_{2}^{2} \]

The water has zero velocity when it hits the arm, and $ 1 $ before. Thus we have $ U_{1} =1  $ and $ U_{2} = 0 $. 


Then

\[ \frac{p_{1}}{\rho} + \frac{1}{2} = \frac{p_{2}}{\rho} \]

\[ \Rightarrow (p_{2} - p_{1} ) = \frac{\rho}{2} \]

Use $ p_{1} - p_{\text{atm}} = \rho g h_{1} $, $ p_{2} - p_{\text{atm}} = \rho g h_{2} $ 

\[ h_{2} - h_{1} = \frac{1}{2g} \]

ie, water can rise about $ 5 \text{ cm} $. 








\section{QUESTION 11}

Euler momentum equation states

\[\frac{\partial \mathbf{u} }{\partial t } + (\mathbf{u} \cdot \nabla ) \mathbf{u} = - \nabla  p + \rho \mathbf{g} \]

Have $  \rho \mathbf{g} = - \nabla  \chi  $, where $ \chi = g \rho z $ as $ \mathbf{g} = (0,0,-z) $.

Now if $ \rho = $ constant, and $ \mathbf{u} = (- \omega y, \omega x,0) $.

Note completely sure how to integrate this...





\section{QUESTION 12}

\[ \frac{p_{1}}{\rho} + \frac{1}{2} U_{1}^{2} + g h_{1} = \frac{p_{2}}{\rho} + \frac{1}{2} U_{2}^{2} + g h_{2} \qquad \text{(Bernoulli)} \]


\[ A_{1} U_{1} = A_{2} U_{2} = 10^{-4} \text{ m}^{3} \text{s}^{-1} \qquad \text{(mass conservation)} \]

\[ A_{2} = 4 \times 10^{-5} \text{ m}^{2} \Rightarrow U_{2} = 5\text{ ms}^{-1} \]

Use $ p_{1} - p_{\text{atm}} = \rho g h_{1} $, $ p_{2} - p_{\text{atm}} = \rho g h_{2} $ 

But not sure what $ U_{1}, A_{1} $ are?

\section{QUESTION 13}

Bernoulli:

\[ \frac{U^{2}}{2} + \frac{p}{\rho} + gh = \text{constant} \]

Apply at the opening, where $ h = 0 $, $ p = p_{\text{atm}} $. Apply again at the top of the vessel, where we can assume $ U = 0 $, $ h = h $, $ p = p_{atm} $. Comparing we find that 

\[ \frac{U^{2}}{2}  = gh  \]

\[ \Rightarrow U = \sqrt{2gh} \]

So the shape is...?


\end{document}