\documentclass[a4paper]{article}
\usepackage{amsmath}
\def\npart {IB}
\def\nterm {Lent}
\def\nyear {2018}
\def\nlecturer {Prof. Haynes (P.H.Haynes@damtp.cam.ac.uk)}
\def\ncourse {Fluid Dynamics Example Sheet 1}

\input{header}

\newtheorem*{soln}{Solution}

\renewcommand{\thesection}{}
\renewcommand{\thesubsection}{\arabic{section}.\arabic{subsection}}
\makeatletter
\def\@seccntformat#1{\csname #1ignore\expandafter\endcsname\csname the#1\endcsname\quad}
\let\sectionignore\@gobbletwo
\let\latex@numberline\numberline
\def\numberline#1{\if\relax#1\relax\else\latex@numberline{#1}\fi}
\makeatother


\begin{document}
	
\maketitle

\section{QUESTION 1}

In this example, $ \mathbf{u}(x) = \alpha(x,-y) $

\begin{enumerate}
	\item Streamlines are curves $ \mathbf{X}(s; \mathbf{x}_{0}) $ with $ \mathbf{X}_{0} = \mathbf{x}_{0} $ at $ s = 0 $. 
	
	\[ \frac{\partial X }{\partial s} = \alpha X, \quad \frac{\partial Y }{\partial s} = - \alpha Y \]
	
	$ X = x_{0} e^{\alpha s} $ and $ Y = y_{0}e^{- \alpha s} $. Eliminating $ s $ to get the shape of the streamlines gives $ XY = x_{0}y_{0} $. These are hyperbola.
	
	\item For steady flow, pathlines and streamlines are coincident. For a particle released at $ \mathbf{x}_{0} = (x_{0},y_{0}) $, the particle pathline obeys
	
	\[ x = x_{0} e^{\alpha t}, \quad y = y_{0}e^{- \alpha t} \]
	
	as  As $ x_{0}^{2} + y_{0}^{2} = a^{2} $ we find that the curve evolves as
	
	\[ x^{2}e^{-2\alpha t} + y^{2}e^{2 \alpha t} = a^{2} \]
	
	ie; this is an ellipse,
	
	
	\[ \frac{x^{2}}{(ae^{at})^{2}} + \frac{y^{2}}{(ae^{-at})^{2}} = 1 \]
	
	\item 
	
	The area of this ellipse does not change in time; it is $ \pi \times ae^{at} \times ae^{-at} = \pi a^{2} $. As we would expect this is the area of the circle $ x^{2} + y^{2} = a^{2} $ at time $ t = 0 $. 
	
	Intuitively?
	
	
\end{enumerate}


\section{QUESTION 2}

In this example, $ \mathbf{u}(x) = (\gamma y, 0) $

\begin{enumerate}
	\item Streamlines are curves $ \mathbf{X}(s; \mathbf{x}_{0}) $ with $ \mathbf{X}_{0} = \mathbf{x}_{0} $ at $ s = 0 $. 
	
	\[ \frac{\partial X }{\partial s} = \gamma y, \quad \frac{\partial Y }{\partial s} = 0 \]
	
	$ X = x_{0} + \gamma y_{0} s $ and $ Y = y_{0}$. Note that we cannot eliminate $ s $ to get the shape of the streamlines; they are given by;
	
	\[ X = x_{0} + \gamma Y s \]
	
	\item For steady flow, pathlines and streamlines are coincident. For a particle released at $ \mathbf{x}_{0} = (x_{0},y_{0}) $, the particle pathline obeys
	
	\[ x = x_{0} + \gamma y_{0} t, \quad y = y_{0}  \]
	
	as  As $ x_{0}^{2} + y_{0}^{2} = a^{2} $ we find that the curve evolves as
	
	\[ (x - \gamma t y_{0})^{2} + y^{2} = a^{2}  \]
	
	ie. it is just shifted to the right, at a rate of $ \gamma y_{0} $ units per second. 
	
	\item Clearly the curve is just translated so the area remains fixed.

	
	
\end{enumerate}

At longer times, the ellipse in Question 1 is stretched faster, as this grows exponentially, yet the stretch in this circle is just linear with time. 
	



\section{QUESTION 3}

\section{QUESTION 2}

In this example, $ \mathbf{u}(x) = (\frac{\partial \psi }{\partial y}, - \frac{\partial \psi }{\partial x}) $.

\begin{enumerate}
	\item The contours $ \psi = c $ have normal
	
	\[ \mathbf{n} = \nabla  \psi = (\psi_{x},\psi_{y}) \]
	
	We see immediately that
	
	\[ \mathbf{u} \cdot \mathbf{n} = \psi_{x} \psi_{y} - \psi_{y} \psi_{x} = 0 \]
	
	So the flow is perpendicular to the normal, ie. tangent to the contours of $ \psi $. 
	
	\item $ \left| \mathbf{u} \right| = | (\psi_{y},-\psi_{x}) | = \nabla^{2} \psi = | (\psi_{x},\psi_{y}) | = | \nabla \psi | $
	
	
	\item The volume flux is
	\[
	q = \int_{\mathbf{x}_0}^{\mathbf{x}_1} \mathbf{u}\cdot \mathbf{n}\;\d s.
	\]
	We see that
	\[
	\mathbf{n}\;\d s = (-\d y, \d x).
	\]
	So we can write this as
	\[
	q = \int_{\mathbf{x}_0}^{\mathbf{x}_1} -\frac{\partial \psi}{\partial y}\;\d y - \frac{\partial \psi}{\partial x}\;\d x = \psi(\mathbf{x}_0) - \psi(\mathbf{x}_1).
	\]
	So the flux depends only the difference in the value of $\psi$. Hence, for closer streamlines, to maintain the same volume flux, we need a higher speed.
	
	\item Note that $\psi$ is constant on a stationary rigid boundary, i.e.\ the boundary is a streamline, since the flow is tangential at the boundary. This is a consequence of $\mathbf{u}\cdot \mathbf{n} = 0$. We often choose $\psi = 0$ as our boundary.
	
	
\end{enumerate}









\section{QUESTION 4}

\[ \nabla \cdot \mathbf{u} = 0 \iff \frac{\partial u }{\partial x} + \frac{\partial v}{\partial y} = 0  \]

By the quotient rule,

\begin{align*}
\frac{\partial u }{\partial x}& = \frac{-(y-b)[2(x-a)]}{( (x-a) + (y-b))^{2}} 
\end{align*}

\begin{align*}
\frac{\partial v }{\partial y} & = \frac{-(a-x)[2(y-b)] }{( (x-a)^{2} + (y-b)^{2}   )} \\
\end{align*}

So $  \frac{\partial u }{\partial x} = - \frac{\partial v }{\partial y} $ and $  \nabla \cdot \mathbf{u} = 0 $, our fluid is indeed incompressible.

By inspection, the required streamfunction is 

\[ \psi(x,y) = \frac{1}{2} \log \left[   (x-a)^{2} + (y-b)^{2}  \right]  \]

 


\section{QUESTION 5}

\[ \nabla \cdot \mathbf{u} = 0 \iff  \frac{1}{r} \frac{\partial }{\partial r}(ru_{r}) + \frac{1}{r} \frac{\partial }{\partial \theta}(u_{\theta}) = 0  \]

\begin{align*}
\frac{\partial }{\partial r}(ru_{r}) & = \frac{\partial }{\partial r} \left[  U \left(  r - \frac{a^{2}}{r} \right) \cos \theta \right]    \\
& = U \left(  1 + \frac{a^{2}}{r^{2}} \right) \cos \theta 
\end{align*}

\begin{align*}
\frac{\partial }{\partial \theta}(u_{\theta}) & = \frac{\partial }{\partial \theta} \left[ -U \left( 1 + \frac{a^{2}}{r^{2}} \right)  \sin \theta \right]    \\
& = U \left( 1 + \frac{a^{2}}{r^{2}} \right)  \cos \theta
\end{align*}

Hence 

\[  \frac{1}{r} \frac{\partial }{\partial r}(ru_{r}) = - \frac{1}{r} \frac{\partial }{\partial \theta}(u_{\theta}) \implies \nabla \cdot \mathbf{u} = 0  \]


By inspection the streamfunction is

\[ \psi(r,\theta) = U \left(  r - \frac{a^{2}}{r} \right) \sin \theta  \]




\section{QUESTION 6}

\[ \nabla \cdot \mathbf{u} = 0 \iff  \frac{1}{r} \frac{\partial }{\partial r}(ru_{r}) + \frac{\partial u_{z}}{\partial z} = 0  \]


\begin{align*}
\frac{\partial }{\partial r}(ru_{r}) & = \frac{\partial }{\partial r} \left[ - \frac{1}{2} \alpha r^{2} \right]     \\
& = - r \alpha
\end{align*}

\begin{align*}
\frac{\partial }{\partial z}(u_{z}) & = \frac{\partial }{\partial z} \left[ \alpha z \right]    \\
& = \alpha
\end{align*}

Hence 

\[  \frac{1}{r} \frac{\partial }{\partial r}(ru_{r}) = - \frac{\partial u_{z}}{\partial z} \implies \nabla \cdot \mathbf{u} = 0  \]


By inspection the streamfunction is

\[ \Psi(r,z) = \frac{1}{2} \alpha r^{2} z  \]






\section{QUESTION 7}


In this example, $ \mathbf{u}(x) = (  1/(1+t), 1  ) $

\begin{enumerate}
	\item Streamlines are curves $ \mathbf{X}(s; \mathbf{x}_{0}) $ with $ \mathbf{X}_{0} = \mathbf{x}_{0} $ at $ s = 0 $. 
	
	\[ \frac{\partial X }{\partial s} = \frac{1}{1+t}, \quad \frac{\partial Y }{\partial s} = 1 \]
	
	$ X = \frac{s}{1+t} + x_{0} $ and $ Y = s + y_{0} $. Eliminating $ s $ to get the shape of the streamlines gives $ X = \frac{Y - y_{0}}{1+t} + x_{0} $. At time $ t = 0 $ this gives the streamline as
	
	\[ Y = X + y_{0} - x_{0} \]
	
	
	
	\item The pathline of the particle is given by
	
	\[ \frac{\partial X }{\partial t} = \frac{1}{1+t}, \quad \frac{\partial Y }{\partial t} = 1 \]
	
	with $ \mathbf{X} = \mathbf{x}_{0} $ at $ t = 0 $. Thus $ X = \log(1+t) + x_{0} $ and $ Y = t + y_{0} $. Eliminating $ t $ we get the shape of the pathline as
	
	\[ X = \log(1+ Y - y_{0}) + x_{0} \]
	
	
	
	\item 
	
	The area of this ellipse does not change in time; it is $ \pi \times ae^{at} \times ae^{-at} = \pi a^{2} $. As we would expect this is the area of the circle $ x^{2} + y^{2} = a^{2} $ at time $ t = 0 $. 
	
	Intuitively?
	
	
\end{enumerate}


















\begin{enumerate}
	\item At $ t = 1 $, the velocity makes an angle of $ 45^\circ $ angle with the horizontal , and the streamlines are slanted.
	
	\item For a particle released at $ (1,1) $ we get
	
	\[ \dot{x}(t) = \frac{1}{1+t}, \qquad \dot{y}(t) = 1  \]
	
	Hence we get 
	
	\[ x = \log(1+t) + 1, \qquad y = t + 1 \]
	
	Eliminating $ t $, we get that the path is given by
	
	\[ x = \log y + 1 \implies y = e^{x-1} \]
	
	
	
\end{enumerate}

Consider the fluid flow $ \mathbf{u} = (  1/(1+t),1,0 ) $ for $ t > 0 $. Supposing our fluid is incompressible, $ \nabla \cdot \mathbf{u} = 0 $ and there exists some vector potential $ \mathbf{A} $ such that $ \mathbf{u} = \nabla  \times \mathbf{A} $. As $ \mathbf{u} $ is two dimensional, we know $ \mathbf{A} $ is of the from

\[ \mathbf{A} = (0,0,\psi(x,y,t)) \]

And taking the curl of this, 

\[ \mathbf{u} = ( \frac{\partial \psi }{\partial y}, - \frac{\partial \psi }{\partial x}, 0 ) \]

This $ \psi $ is our streamfunction.

We define streamlines to be contours of our stream function. These contours have normal $ \mathbf{n} = ( \psi_{x},\psi_{y},0 ) $, this normal is obviously perpendicular to the flow $ \mathbf{u} $ ($ \mathbf{u} \cdot \mathbf{n} = 0 $). ie. the flow is tangent to the contours of $ \psi$. 

\section{QUESTION 8}


The integral from of the momentum equation:

\[ \frac{\d }{\d t}  \int_{V}  \rho \mathbf{u} \; \d V = \int_{V} \rho \mathbf{g} \; \d V - \int_{S} p \mathbf{n} + \rho \mathbf{u} (\mathbf{u} \cdot \mathbf{n}) \; \d S \]

For steady flow, the LHS is zero. Neglecting gravity, we have the total force on the wall as 

\begin{align*}
& = \int_{S} p \mathbf{n} \; \d S \\
& = - \int_{S} \rho \mathbf{u} (\mathbf{u} \cdot \mathbf{n}) \; \d S 
\end{align*}

So the force on the wall is equal to the surface integral, which gives $ - \rho A U^{2} \mathbf{n} $. Assuming $ \rho = 10^{3} \text{kg m}^{-3} $; $ \Rightarrow 0.6 $ N force. 



\section{QUESTION 9}

\begin{prop}[Euler momentum equation]
	\[
	\rho \frac{\D \mathbf{u}}{\D t} = - \nabla p + \mathbf{f}.
	\]
\end{prop}

For conservative forces, we can write $\mathbf{f} = -\nabla \chi$, where $\chi$ is a scalar potential.

We notice the vector identity
\[
\mathbf{u}\times (\nabla \times \mathbf{u}) = \nabla\left(\frac{1}{2}|\mathbf{u}|^2\right) - \mathbf{u}\cdot \nabla \mathbf{u}.
\]
We use this to rewrite the Euler momentum equation as
\[
\rho \frac{\partial \mathbf{u}}{\partial t} + \rho \nabla \left(\frac{1}{2}|\mathbf{u}|^2\right) - \rho \mathbf{u}\times (\nabla \times \mathbf{u}) = -\nabla p - \nabla \chi.
\]
Dotting with $\mathbf{u}$, the last term on the left vanishes, and we get
\begin{prop}[Bernoulli's equation]
	\[
	\frac{1}{2}\rho \frac{\partial|\mathbf{u}|^2}{\partial t} = -\mathbf{u}\cdot \nabla \left(\frac{1}{2} \rho |\mathbf{u}|^2 + p + \chi\right).
	\]
\end{prop}
Note also that this tells us that high velocity goes with low pressure; low pressure goes with high velocity.

In the case where we have a steady flow, we know
\[
H = \frac{1}{2}\rho |\mathbf{u}|^2 + p + \chi
\]
is constant along streamlines.

Even if the flow is not steady, we can still define the value $H$, and then we can integrate Bernoulli's equation over a volume $\mathcal{D}$ to obtain
\[
\frac{\d}{\d t}\int_{\mathcal{D}} \frac{1}{2}\rho |\mathbf{u}|^2 \;\d V + \int_{\partial \mathcal{D}} H \mathbf{u}\cdot \mathbf{n} \;\d S. = 0 
\]
So $H$ is the transportable energy of the flow.




\section{QUESTION 10}







\section{QUESTION 11}
\section{QUESTION 12}



\end{document}