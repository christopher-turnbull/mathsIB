	\documentclass[a4paper]{article}
\usepackage{amsmath}
\def\npart {IA}
\def\nterm {Michaelmas}
\def\nyear {2017}
\def\nlecturer {Mx Tsang (jmft2@cam.ac.uk)}
\def\ncourse {Variational Principles Example Sheet 2}

\input{header}

\newtheorem*{soln}{Solution}

\renewcommand{\thesection}{}
\renewcommand{\thesubsection}{\arabic{section}.\arabic{subsection}}
\makeatletter
\def\@seccntformat#1{\csname #1ignore\expandafter\endcsname\csname the#1\endcsname\quad}
\let\sectionignore\@gobbletwo
\let\latex@numberline\numberline
\def\numberline#1{\if\relax#1\relax\else\latex@numberline{#1}\fi}
\makeatother


\begin{document}
	
\maketitle

\section{QUESTION 1}

\begin{align*}
F[x + \delta x] - F[x] & = \int_{t_{1}}^{t_{2}} f(x + \delta x, \dot{x} + \delta \dot{x}, \ddot{x} + \delta \ddot x, t ) \; \d t - \int_{t_{1}}^{t_2} f(t,x,\dot{x},\ddot{x}) \; \d t   \\
& = \int_{t_{1}}^{t_{2}} \left\{  \delta x \frac{\partial f }{\partial x} + (\delta \dot{x}) \frac{\partial f }{\partial \dot{x}} + (\delta \ddot{x}) \frac{\partial f }{\partial \ddot{x}}  \right\} \; \d t + O(t^{2})
\end{align*}

Discarding the (small )terms of $ O(t^{2}) $, we call the first order variation $ \delta F[x] $ and integrating by parts (twice), we have

\begin{align*}
\delta F[x]  & = \int_{t_{1}}^{t_{2}} \left\{   \delta x \left[   \frac{\partial f }{\partial x }  - \frac{\d }{\d t} \left(  \frac{\partial f }{\partial \dot{x}} \right)    \right]  - (\delta \dot{x}) \frac{\d }{\d t} \left(  \frac{\partial f }{\partial \ddot{x}} \right)   \right\} \; \d t + \left[  \delta x \frac{\partial f }{\partial \dot{x}} + ( \delta \dot{x}) \frac{\partial f }{\partial \ddot{x}} \right]_{t_{1}}^{t_{2}}   \\
& = \int_{t_{1}}^{t_{2}} \left\{   \delta x \left[   \frac{\partial f }{\partial x }  - \frac{\d }{\d t} \left(  \frac{\partial f }{\partial \dot{x}} \right)  +  \frac{\d^{2} }{\d t^{2}} \left(  \frac{\partial f }{\partial \ddot{x}} \right)   \right] \right\} \; \d t + \left[  \delta x \left\{  \frac{\partial f }{\partial \dot{x}} - \frac{\d }{\d t} \left(  \frac{\partial f }{\partial \ddot{x}} \right)  \right\}  + ( \delta \dot{x}) \frac{\partial f }{\partial \ddot{x}}  \right]_{t_{1}}^{t_{2}}
\end{align*}

We have fixed end boundary conditions, so $ \delta x(t_{1}) = \delta x(t_{2}) = 0 $ and also $ \delta \dot{x} (t_{1}) = \delta \dot{x} (t_{2}) = 0  $. Thus the boundary term is zero and we can write $ \delta F[x] $ in the form


\[ \delta F[x] = \int_{t_{1}}^{t_{2}} \left\{ \delta x(t) \frac{ \delta F[x]}{\delta x(t)}  \right\}  \; \d t  \]


where the \emph{function derivative} $ \frac{ \delta F[x]}{\delta x(t)} $ is defined as 


\[ \frac{ \delta F[x]}{\delta x(t)} := \frac{\partial f }{\partial x} -  \frac{\d }{\d t} \left(  \frac{\partial f }{\partial \dot{x}} \right)  +  \frac{\d^{2} }{\d t^{2}} \left(  \frac{\partial f }{\partial \ddot{x}} \right)   \]

The functional $ F $ is stationary when its functional derivative is zero ( assuming that the b.c.s are such that this derivative is defined) and the condition for this to be true is the following Euler-Lagrange equation:

\[ \frac{\partial f }{\partial x} -  \frac{\d }{\d t} \left(  \frac{\partial f }{\partial \dot{x}} \right)  +  \frac{\d^{2} }{\d t^{2}} \left(  \frac{\partial f }{\partial \ddot{x}} \right) = 0, \qquad t_{1} < t < t_{2} \]


Given the functional

\[ L[x] = \int_{1}^{2} t^{4} |  \ddot{x}(t)  |^{2} \; \d t  \]

In this case, $ f =  t^{4} |  \ddot{x}(t)  |^{2}  $, so $   \frac{\partial f }{\partial x} = \frac{\partial f }{\partial \dot{x}} = 0 $ and the EL equation can be immediately twice integrated to give 

\[ \frac{\partial }{\partial \ddot{x}} \left[ t^{4} |  \ddot{x}(t)  |^{2}  \right] = At + B   \]

for some constants $ A $ and $ B $.


\section{QUESTION 2}


Our aim is to maximise $ A[x,y] $ subject to the constraint $ P[x,y] = L $, where $ L $ is the fixed length and $ P[x,y] = \int_{0}^{2\pi} \sqrt{  (x')^{2}  + (y')^{2} } \; \d \theta  $

Using a Lagrange multiplier $ \lambda $ to impose this, we seek to maximize the functional

\begin{align*}
\phi_{\lambda}[x,y]  & = A[x,y] - \lambda ( P[y] - L )  \\
& = \int_{0}^{2 \pi} \underbrace{\frac{1}{2} ( xy' - yx' ) - \lambda \sqrt{  (x')^{2}  + (y')^{2} }}_{=f_{\lambda}(\mathbf{x},\mathbf{x}')} \; \d \theta + \lambda L
\end{align*}

$ f_{\lambda}(\mathbf{x},\mathbf{x}') $ has no explicit $ \theta $ dependence, setting $ \frac{\d \phi_{\lambda}[x,y]}{\d y} = 0 $ the E-L equations imply (under the assumption of appropriate boundary conditions)


\[ f_{\lambda}(y,y') - y' \frac{\partial f_{\lambda} }{\partial y'} = \text{ constant}  \]

\begin{align*}
& \Rightarrow f - y' \left(  \frac{1}{2} x - \frac{\lambda y'}{\sqrt{  (x')^{2}  + (y')^{2} }} \right) = \text{constant} \\
& \Rightarrow  -\frac{1}{2} yx' - \lambda \frac{ ( (x')^{2}  + (y')^{2} )}{\sqrt{  (x')^{2}  + (y')^{2} }} + \lambda \frac{(y')^{2}}{\sqrt{  (x')^{2}  + (y')^{2} }} = \text{constant} \\
& \Rightarrow - \frac{1}{2} y x' - \lambda \frac{(x')^{2}}{\sqrt{  (x')^{2}  + (y')^{2} }} = \text{constant}
\end{align*}

Similarly considering $ f_{\lambda}(x,x') $ we have

\[ - \frac{1}{2} x y' - \lambda \frac{(y')^{2}}{\sqrt{  (x')^{2}  + (y')^{2} }} = \text{constant} \]

Adding,

\[ -\frac{1}{2}(xy' - yx' ) - \lambda \sqrt{  (x')^{2}  + (y')^{2} } = \text{ constant} \]




\section{QUESTION 3}

Using Lagrange multiplier $ \lambda $, wish to minimize

\[ I[\psi]_{\lambda} = \int_{-\infty}^{\infty} \underbrace{(\psi')^{2} + (x^{2} - \lambda \psi^{2} )}_{f_{\lambda}(\psi,\psi' ; x)} \; \d x + \lambda   \]


Euler-Lagrange equations imply:

$  $





\section{QUESTION 4}
\section{QUESTION 5}
\section{QUESTION 6}
\section{QUESTION 7}
\section{QUESTION 8}
\section{QUESTION 9}
\section{QUESTION 10}
\section{QUESTION 11}









\end{document}