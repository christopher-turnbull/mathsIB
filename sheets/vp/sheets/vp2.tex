	\documentclass[a4paper]{article}
\usepackage{amsmath}
\def\npart {IA}
\def\nterm {Michaelmas}
\def\nyear {2017}
\def\nlecturer {Mx Tsang (jmft2@cam.ac.uk)}
\def\ncourse {Variational Principles Example Sheet 2}

% Imports
\ifx \nauthor\undefined
  \def\nauthor{Christopher Turnbull}
\else
\fi

\author{Supervised by \nlecturer \\\small Solutions presented by \nauthor}
\date{\nterm\ \nyear}

\usepackage{alltt}
\usepackage{amsfonts}
\usepackage{amsmath}
\usepackage{amssymb}
\usepackage{amsthm}
\usepackage{booktabs}
\usepackage{caption}
\usepackage{enumitem}
\usepackage{fancyhdr}
\usepackage{graphicx}
\usepackage{mathdots}
\usepackage{mathtools}
\usepackage{microtype}
\usepackage{multirow}
\usepackage{pdflscape}
\usepackage{pgfplots}
\usepackage{siunitx}
\usepackage{slashed}
\usepackage{tabularx}
\usepackage{tikz}
\usepackage{tkz-euclide}
\usepackage[normalem]{ulem}
\usepackage[all]{xy}
\usepackage{imakeidx}

\makeindex[intoc, title=Index]
\indexsetup{othercode={\lhead{\emph{Index}}}}

\ifx \nextra \undefined
  \usepackage[pdftex,
    hidelinks,
    pdfauthor={Christopher Turnbull},
    pdfsubject={Cambridge Maths Notes: Part \npart\ - \ncourse},
    pdftitle={Part \npart\ - \ncourse},
  pdfkeywords={Cambridge Mathematics Maths Math \npart\ \nterm\ \nyear\ \ncourse}]{hyperref}
  \title{Part \npart\ --- \ncourse}
\else
  \usepackage[pdftex,
    hidelinks,
    pdfauthor={Christopher Turnbull},
    pdfsubject={Cambridge Maths Notes: Part \npart\ - \ncourse\ (\nextra)},
    pdftitle={Part \npart\ - \ncourse\ (\nextra)},
  pdfkeywords={Cambridge Mathematics Maths Math \npart\ \nterm\ \nyear\ \ncourse\ \nextra}]{hyperref}

  \title{Part \npart\ --- \ncourse \\ {\Large \nextra}}
  \renewcommand\printindex{}
\fi

\pgfplotsset{compat=1.12}

\pagestyle{fancyplain}
\lhead{\emph{\nouppercase{\leftmark}}}
\ifx \nextra \undefined
  \rhead{
    \ifnum\thepage=1
    \else
      \npart\ \ncourse
    \fi}
\else
  \rhead{
    \ifnum\thepage=1
    \else
      \npart\ \ncourse\ (\nextra)
    \fi}
\fi
\usetikzlibrary{arrows.meta}
\usetikzlibrary{decorations.markings}
\usetikzlibrary{decorations.pathmorphing}
\usetikzlibrary{positioning}
\usetikzlibrary{fadings}
\usetikzlibrary{intersections}
\usetikzlibrary{cd}

\newcommand*{\Cdot}{{\raisebox{-0.25ex}{\scalebox{1.5}{$\cdot$}}}}
\newcommand {\pd}[2][ ]{
  \ifx #1 { }
    \frac{\partial}{\partial #2}
  \else
    \frac{\partial^{#1}}{\partial #2^{#1}}
  \fi
}
\ifx \nhtml \undefined
\else
  \renewcommand\printindex{}
  \makeatletter
  \DisableLigatures[f]{family = *}
  \let\Contentsline\contentsline
  \renewcommand\contentsline[3]{\Contentsline{#1}{#2}{}}
  \renewcommand{\@dotsep}{10000}
  \newlength\currentparindent
  \setlength\currentparindent\parindent

  \newcommand\@minipagerestore{\setlength{\parindent}{\currentparindent}}
  \usepackage[active,tightpage,pdftex]{preview}
  \renewcommand{\PreviewBorder}{0.1cm}

  \newenvironment{stretchpage}%
  {\begin{preview}\begin{minipage}{\hsize}}%
    {\end{minipage}\end{preview}}
  \AtBeginDocument{\begin{stretchpage}}
  \AtEndDocument{\end{stretchpage}}

  \newcommand{\@@newpage}{\end{stretchpage}\begin{stretchpage}}

  \let\@real@section\section
  \renewcommand{\section}{\@@newpage\@real@section}
  \let\@real@subsection\subsection
  \renewcommand{\subsection}{\@@newpage\@real@subsection}
  \makeatother
\fi

% Theorems
\theoremstyle{definition}
\newtheorem*{aim}{Aim}
\newtheorem*{axiom}{Axiom}
\newtheorem*{claim}{Claim}
\newtheorem*{cor}{Corollary}
\newtheorem*{conjecture}{Conjecture}
\newtheorem*{defi}{Definition}
\newtheorem*{eg}{Example}
\newtheorem*{ex}{Exercise}
\newtheorem*{fact}{Fact}
\newtheorem*{law}{Law}
\newtheorem*{lemma}{Lemma}
\newtheorem*{notation}{Notation}
\newtheorem*{prop}{Proposition}
\newtheorem*{soln}{Solution}
\newtheorem*{thm}{Theorem}

\newtheorem*{remark}{Remark}
\newtheorem*{warning}{Warning}
\newtheorem*{exercise}{Exercise}

\newtheorem{nthm}{Theorem}[section]
\newtheorem{nlemma}[nthm]{Lemma}
\newtheorem{nprop}[nthm]{Proposition}
\newtheorem{ncor}[nthm]{Corollary}


\renewcommand{\labelitemi}{--}
\renewcommand{\labelitemii}{$\circ$}
\renewcommand{\labelenumi}{(\roman{*})}

\let\stdsection\section
\renewcommand\section{\newpage\stdsection}

% Strike through
\def\st{\bgroup \ULdepth=-.55ex \ULset}

% Maths symbols
\newcommand{\abs}[1]{\left\lvert #1\right\rvert}
\newcommand\ad{\mathrm{ad}}
\newcommand\AND{\mathsf{AND}}
\newcommand\Art{\mathrm{Art}}
\newcommand{\Bilin}{\mathrm{Bilin}}
\newcommand{\bket}[1]{\left\lvert #1\right\rangle}
\newcommand{\B}{\mathcal{B}}
\newcommand{\bolds}[1]{{\bfseries #1}}
\newcommand{\brak}[1]{\left\langle #1 \right\rvert}
\newcommand{\braket}[2]{\left\langle #1\middle\vert #2 \right\rangle}
\newcommand{\bra}{\langle}
\newcommand{\cat}[1]{\mathsf{#1}}
\newcommand{\C}{\mathbb{C}}
\newcommand{\CP}{\mathbb{CP}}
\newcommand{\cU}{\mathcal{U}}
\newcommand{\Der}{\mathrm{Der}}
\newcommand{\D}{\mathrm{D}}
\newcommand{\dR}{\mathrm{dR}}
\newcommand{\E}{\mathbb{E}}
\newcommand{\F}{\mathbb{F}}
\newcommand{\Frob}{\mathrm{Frob}}
\newcommand{\GG}{\mathbb{G}}
\newcommand{\gl}{\mathfrak{gl}}
\newcommand{\GL}{\mathrm{GL}}
\newcommand{\G}{\mathcal{G}}
\newcommand{\Gr}{\mathrm{Gr}}
\newcommand{\haut}{\mathrm{ht}}
\newcommand{\Id}{\mathrm{Id}}
\newcommand{\ket}{\rangle}
\newcommand{\lie}[1]{\mathfrak{#1}}
\newcommand{\Mat}{\mathrm{Mat}}
\newcommand{\N}{\mathbb{N}}
\newcommand{\norm}[1]{\left\lVert #1\right\rVert}
\newcommand{\normalorder}[1]{\mathop{:}\nolimits\!#1\!\mathop{:}\nolimits}
\newcommand\NOT{\mathsf{NOT}}
\newcommand{\Oc}{\mathcal{O}}
\newcommand{\Or}{\mathrm{O}}
\newcommand\OR{\mathsf{OR}}
\newcommand{\ort}{\mathfrak{o}}
\newcommand{\PGL}{\mathrm{PGL}}
\newcommand{\ph}{\,\cdot\,}
\newcommand{\pr}{\mathrm{pr}}
\newcommand{\Prob}{\mathbb{P}}
\newcommand{\PSL}{\mathrm{PSL}}
\newcommand{\Ps}{\mathcal{P}}
\newcommand{\PSU}{\mathrm{PSU}}
\newcommand{\pt}{\mathrm{pt}}
\newcommand{\qeq}{\mathrel{``{=}"}}
\newcommand{\Q}{\mathbb{Q}}
\newcommand{\R}{\mathbb{R}}
\newcommand{\RP}{\mathbb{RP}}
\newcommand{\Rs}{\mathcal{R}}
\newcommand{\SL}{\mathrm{SL}}
\newcommand{\so}{\mathfrak{so}}
\newcommand{\SO}{\mathrm{SO}}
\newcommand{\Spin}{\mathrm{Spin}}
\newcommand{\Sp}{\mathrm{Sp}}
\newcommand{\su}{\mathfrak{su}}
\newcommand{\SU}{\mathrm{SU}}
\newcommand{\term}[1]{\emph{#1}\index{#1}}
\newcommand{\T}{\mathbb{T}}
\newcommand{\tv}[1]{|#1|}
\newcommand{\U}{\mathrm{U}}
\newcommand{\uu}{\mathfrak{u}}
\newcommand{\Vect}{\mathrm{Vect}}
\newcommand{\wsto}{\stackrel{\mathrm{w}^*}{\to}}
\newcommand{\wt}{\mathrm{wt}}
\newcommand{\wto}{\stackrel{\mathrm{w}}{\to}}
\newcommand{\Z}{\mathbb{Z}}
\renewcommand{\d}{\mathrm{d}}
\renewcommand{\H}{\mathbb{H}}
\renewcommand{\P}{\mathbb{P}}
\renewcommand{\sl}{\mathfrak{sl}}
\renewcommand{\vec}[1]{\boldsymbol{\mathbf{#1}}}
%\renewcommand{\F}{\mathcal{F}}

\let\Im\relax
\let\Re\relax

\DeclareMathOperator{\adj}{adj}
\DeclareMathOperator{\Ann}{Ann}
\DeclareMathOperator{\area}{area}
\DeclareMathOperator{\Aut}{Aut}
\DeclareMathOperator{\Bernoulli}{Bernoulli}
\DeclareMathOperator{\betaD}{beta}
\DeclareMathOperator{\bias}{bias}
\DeclareMathOperator{\binomial}{binomial}
\DeclareMathOperator{\card}{card}
\DeclareMathOperator{\ccl}{ccl}
\DeclareMathOperator{\Char}{char}
\DeclareMathOperator{\ch}{ch}
\DeclareMathOperator{\cl}{cl}
\DeclareMathOperator{\cls}{\overline{\mathrm{span}}}
\DeclareMathOperator{\conv}{conv}
\DeclareMathOperator{\corr}{corr}
\DeclareMathOperator{\cosec}{cosec}
\DeclareMathOperator{\cosech}{cosech}
\DeclareMathOperator{\cov}{cov}
\DeclareMathOperator{\covol}{covol}
\DeclareMathOperator{\diag}{diag}
\DeclareMathOperator{\diam}{diam}
\DeclareMathOperator{\Diff}{Diff}
\DeclareMathOperator{\disc}{disc}
\DeclareMathOperator{\dom}{dom}
\DeclareMathOperator{\End}{End}
\DeclareMathOperator{\energy}{energy}
\DeclareMathOperator{\erfc}{erfc}
\DeclareMathOperator{\erf}{erf}
\DeclareMathOperator*{\esssup}{ess\,sup}
\DeclareMathOperator{\ev}{ev}
\DeclareMathOperator{\Ext}{Ext}
\DeclareMathOperator{\Fit}{Fit}
\DeclareMathOperator{\fix}{fix}
\DeclareMathOperator{\Frac}{Frac}
\DeclareMathOperator{\Gal}{Gal}
\DeclareMathOperator{\gammaD}{gamma}
\DeclareMathOperator{\gr}{gr}
\DeclareMathOperator{\hcf}{hcf}
\DeclareMathOperator{\Hom}{Hom}
\DeclareMathOperator{\id}{id}
\DeclareMathOperator{\image}{image}
\DeclareMathOperator{\im}{im}
\DeclareMathOperator{\Im}{Im}
\DeclareMathOperator{\Ind}{Ind}
\DeclareMathOperator{\Int}{Int}
\DeclareMathOperator{\Isom}{Isom}
\DeclareMathOperator{\lcm}{lcm}
\DeclareMathOperator{\length}{length}
\DeclareMathOperator{\Lie}{Lie}
\DeclareMathOperator{\like}{like}
\DeclareMathOperator{\Lk}{Lk}
\DeclareMathOperator{\mse}{mse}
\DeclareMathOperator{\multinomial}{multinomial}
\DeclareMathOperator{\orb}{orb}
\DeclareMathOperator{\ord}{ord}
\DeclareMathOperator{\otp}{otp}
\DeclareMathOperator{\Poisson}{Poisson}
\DeclareMathOperator{\poly}{poly}
\DeclareMathOperator{\rank}{rank}
\DeclareMathOperator{\rel}{rel}
\DeclareMathOperator{\Re}{Re}
\DeclareMathOperator*{\res}{res}
\DeclareMathOperator{\Res}{Res}
\DeclareMathOperator{\rk}{rk}
\DeclareMathOperator{\Root}{Root}
\DeclareMathOperator{\sech}{sech}
\DeclareMathOperator{\sgn}{sgn}
\DeclareMathOperator{\spn}{span}
\DeclareMathOperator{\stab}{stab}
\DeclareMathOperator{\St}{St}
\DeclareMathOperator{\supp}{supp}
\DeclareMathOperator{\Syl}{Syl}
\DeclareMathOperator{\Sym}{Sym}
\DeclareMathOperator{\tr}{tr}
\DeclareMathOperator{\Tr}{Tr}
\DeclareMathOperator{\var}{var}
\DeclareMathOperator{\vol}{vol}

\pgfarrowsdeclarecombine{twolatex'}{twolatex'}{latex'}{latex'}{latex'}{latex'}
\tikzset{->/.style = {decoration={markings,
                                  mark=at position 1 with {\arrow[scale=2]{latex'}}},
                      postaction={decorate}}}
\tikzset{<-/.style = {decoration={markings,
                                  mark=at position 0 with {\arrowreversed[scale=2]{latex'}}},
                      postaction={decorate}}}
\tikzset{<->/.style = {decoration={markings,
                                   mark=at position 0 with {\arrowreversed[scale=2]{latex'}},
                                   mark=at position 1 with {\arrow[scale=2]{latex'}}},
                       postaction={decorate}}}
\tikzset{->-/.style = {decoration={markings,
                                   mark=at position #1 with {\arrow[scale=2]{latex'}}},
                       postaction={decorate}}}
\tikzset{-<-/.style = {decoration={markings,
                                   mark=at position #1 with {\arrowreversed[scale=2]{latex'}}},
                       postaction={decorate}}}
\tikzset{->>/.style = {decoration={markings,
                                  mark=at position 1 with {\arrow[scale=2]{latex'}}},
                      postaction={decorate}}}
\tikzset{<<-/.style = {decoration={markings,
                                  mark=at position 0 with {\arrowreversed[scale=2]{twolatex'}}},
                      postaction={decorate}}}
\tikzset{<<->>/.style = {decoration={markings,
                                   mark=at position 0 with {\arrowreversed[scale=2]{twolatex'}},
                                   mark=at position 1 with {\arrow[scale=2]{twolatex'}}},
                       postaction={decorate}}}
\tikzset{->>-/.style = {decoration={markings,
                                   mark=at position #1 with {\arrow[scale=2]{twolatex'}}},
                       postaction={decorate}}}
\tikzset{-<<-/.style = {decoration={markings,
                                   mark=at position #1 with {\arrowreversed[scale=2]{twolatex'}}},
                       postaction={decorate}}}

\tikzset{circ/.style = {fill, circle, inner sep = 0, minimum size = 3}}
\tikzset{mstate/.style={circle, draw, blue, text=black, minimum width=0.7cm}}

\tikzset{commutative diagrams/.cd,cdmap/.style={/tikz/column 1/.append style={anchor=base east},/tikz/column 2/.append style={anchor=base west},row sep=tiny}}

\definecolor{mblue}{rgb}{0.2, 0.3, 0.8}
\definecolor{morange}{rgb}{1, 0.5, 0}
\definecolor{mgreen}{rgb}{0.1, 0.4, 0.2}
\definecolor{mred}{rgb}{0.5, 0, 0}

\def\drawcirculararc(#1,#2)(#3,#4)(#5,#6){%
    \pgfmathsetmacro\cA{(#1*#1+#2*#2-#3*#3-#4*#4)/2}%
    \pgfmathsetmacro\cB{(#1*#1+#2*#2-#5*#5-#6*#6)/2}%
    \pgfmathsetmacro\cy{(\cB*(#1-#3)-\cA*(#1-#5))/%
                        ((#2-#6)*(#1-#3)-(#2-#4)*(#1-#5))}%
    \pgfmathsetmacro\cx{(\cA-\cy*(#2-#4))/(#1-#3)}%
    \pgfmathsetmacro\cr{sqrt((#1-\cx)*(#1-\cx)+(#2-\cy)*(#2-\cy))}%
    \pgfmathsetmacro\cA{atan2(#2-\cy,#1-\cx)}%
    \pgfmathsetmacro\cB{atan2(#6-\cy,#5-\cx)}%
    \pgfmathparse{\cB<\cA}%
    \ifnum\pgfmathresult=1
        \pgfmathsetmacro\cB{\cB+360}%
    \fi
    \draw (#1,#2) arc (\cA:\cB:\cr);%
}
\newcommand\getCoord[3]{\newdimen{#1}\newdimen{#2}\pgfextractx{#1}{\pgfpointanchor{#3}{center}}\pgfextracty{#2}{\pgfpointanchor{#3}{center}}}

\def\Xint#1{\mathchoice
   {\XXint\displaystyle\textstyle{#1}}%
   {\XXint\textstyle\scriptstyle{#1}}%
   {\XXint\scriptstyle\scriptscriptstyle{#1}}%
   {\XXint\scriptscriptstyle\scriptscriptstyle{#1}}%
   \!\int}
\def\XXint#1#2#3{{\setbox0=\hbox{$#1{#2#3}{\int}$}
     \vcenter{\hbox{$#2#3$}}\kern-.5\wd0}}
\def\ddashint{\Xint=}
\def\dashint{\Xint-}

\newcommand\separator{{\centering\rule{2cm}{0.2pt}\vspace{2pt}\par}}

\newenvironment{own}{\color{gray!70!black}}{}

\newcommand\makecenter[1]{\raisebox{-0.5\height}{#1}}

\newtheorem*{soln}{Solution}

\renewcommand{\thesection}{}
\renewcommand{\thesubsection}{\arabic{section}.\arabic{subsection}}
\makeatletter
\def\@seccntformat#1{\csname #1ignore\expandafter\endcsname\csname the#1\endcsname\quad}
\let\sectionignore\@gobbletwo
\let\latex@numberline\numberline
\def\numberline#1{\if\relax#1\relax\else\latex@numberline{#1}\fi}
\makeatother


\begin{document}
	
\maketitle

\section{QUESTION 1}

\begin{align*}
F[x + \delta x] - F[x] & = \int_{t_{1}}^{t_{2}} f(x + \delta x, \dot{x} + \delta \dot{x}, \ddot{x} + \delta \ddot x, t ) \; \d t - \int_{t_{1}}^{t_2} f(t,x,\dot{x},\ddot{x}) \; \d t   \\
& = \int_{t_{1}}^{t_{2}} \left\{  \delta x \frac{\partial f }{\partial x} + (\delta \dot{x}) \frac{\partial f }{\partial \dot{x}} + (\delta \ddot{x}) \frac{\partial f }{\partial \ddot{x}}  \right\} \; \d t + O(t^{2})
\end{align*}

Discarding the (small )terms of $ O(t^{2}) $, we call the first order variation $ \delta F[x] $ and integrating by parts (twice), we have

\begin{align*}
\delta F[x]  & = \int_{t_{1}}^{t_{2}} \left\{   \delta x \left[   \frac{\partial f }{\partial x }  - \frac{\d }{\d t} \left(  \frac{\partial f }{\partial \dot{x}} \right)    \right]  - (\delta \dot{x}) \frac{\d }{\d t} \left(  \frac{\partial f }{\partial \ddot{x}} \right)   \right\} \; \d t + \left[  \delta x \frac{\partial f }{\partial \dot{x}} + ( \delta \dot{x}) \frac{\partial f }{\partial \ddot{x}} \right]_{t_{1}}^{t_{2}}   \\
& = \int_{t_{1}}^{t_{2}} \left\{   \delta x \left[   \frac{\partial f }{\partial x }  - \frac{\d }{\d t} \left(  \frac{\partial f }{\partial \dot{x}} \right)  +  \frac{\d^{2} }{\d t^{2}} \left(  \frac{\partial f }{\partial \ddot{x}} \right)   \right] \right\} \; \d t\\
& \quad + \left[  \delta x \left\{  \frac{\partial f }{\partial \dot{x}} - \frac{\d }{\d t} \left(  \frac{\partial f }{\partial \ddot{x}} \right)  \right\}  + ( \delta \dot{x}) \frac{\partial f }{\partial \ddot{x}}  \right]_{t_{1}}^{t_{2}}
\end{align*}

We have fixed end boundary conditions, so $ \delta x(t_{1}) = \delta x(t_{2}) = 0 $ and also $ \delta \dot{x} (t_{1}) = \delta \dot{x} (t_{2}) = 0  $. Thus the boundary term is zero and we can write $ \delta F[x] $ in the form


\[ \delta F[x] = \int_{t_{1}}^{t_{2}} \left\{ \delta x(t) \frac{ \delta F[x]}{\delta x(t)}  \right\}  \; \d t  \]


where the \emph{functional derivative} $ \frac{ \delta F[x]}{\delta x(t)} $ is defined as 


\[ \frac{ \delta F[x]}{\delta x(t)} := \frac{\partial f }{\partial x} -  \frac{\d }{\d t} \left(  \frac{\partial f }{\partial \dot{x}} \right)  +  \frac{\d^{2} }{\d t^{2}} \left(  \frac{\partial f }{\partial \ddot{x}} \right)   \]

The functional $ F $ is stationary when its functional derivative is zero ( assuming that this derivative is defined on $ (t_{1},t_{2}) $) and the condition for this to be true is the following Euler-Lagrange equation:

\[ \frac{\partial f }{\partial x} -  \frac{\d }{\d t} \left(  \frac{\partial f }{\partial \dot{x}} \right)  +  \frac{\d^{2} }{\d t^{2}} \left(  \frac{\partial f }{\partial \ddot{x}} \right) = 0, \qquad t_{1} < t < t_{2} \]


Given the functional

\[ L[x] = \int_{1}^{2} t^{4}  [\ddot{x}(t)]^{2} \; \d t  \]

In this case, $ f =  t^{4} [\ddot{x}(t)]^{2}  $, so $   \frac{\partial f }{\partial x} = \frac{\partial f }{\partial \dot{x}} = 0 $ and the EL equation can be immediately twice integrated to give 

\[ \frac{\partial }{\partial \ddot{x}} \left[ t^{4} [\ddot{x}(t)]^{2} \right] = At + B   \]

for some constants $ A $ and $ B $.

ie.

\[  2 t^{4}  \ddot{(x)}(t) = At + B \]
\[ \Rightarrow \ddot{x}(t) = At^{-3} + B t^{-4} \qquad t \neq 0 \]

\[ \Rightarrow \dot{x}(t) = A' t^{-2} + B' t^{-3} + C \]
Using b.c.s we have

\[ -2 = A' + B' + C \]
\[ -\frac{1}{4} = \frac{1}{4} A' + \frac{1}{8} B' + C \Rightarrow -2 = 2 A' + B' + 8C \]

Subtracting immediately yields $ A' = 7 C $:

\[ -2 = 8 C + B' \]
\[ -2 = 22C + B' \]

Therefore $ C' = 0 $, $ B' = -2 $, and $ \dot{x}(n) = -2 t^{-3} $.

Integrating, we get

\[ x(t) = t^{-2} + D \]

b.c.s $ \Rightarrow D = 0 $, so $ x(t) = \frac{1}{t^{2}} $. 

This function is a global minimum? Hard to see. 

\section{QUESTION 2}


Our aim is to maximise $ A[x,y] $ subject to the constraint $ P[x,y] = L $, where $ L $ is the fixed length and $ P[x,y] = \int_{0}^{2\pi} \sqrt{  (x')^{2}  + (y')^{2} } \; \d \theta  $

Using a Lagrange multiplier $ \lambda $ to impose this, we seek to maximize the functional

\begin{align*}
\phi_{\lambda}[x,y]  & = A[x,y] - \lambda ( P[y] - L )  \\
& = \int_{0}^{2 \pi} \underbrace{\frac{1}{2} ( xy' - yx' ) - \lambda \sqrt{  (x')^{2}  + (y')^{2} }}_{=f_{\lambda}(\mathbf{x},\mathbf{x}')} \; \d \theta + \lambda L
\end{align*}

where $ \mathbf{x} = (x,y) $. The boundary conditions fix $ x, x',y,y' $ at the endpoints, so the functional is stationary for solutions of the E-L equation.  $ f_{\lambda}(\mathbf{x},\mathbf{x}') $ has no explicit $ \theta $ dependence, so considering the $ y $ E-L equation we have that:

\[ f_{\lambda}(y,y') - y' \frac{\partial f_{\lambda} }{\partial y'} = \text{ constant}  \]

\begin{align*}
& \Rightarrow f - y' \left(  \frac{1}{2} x - \frac{\lambda y'}{\sqrt{  (x')^{2}  + (y')^{2} }} \right) = \text{constant} \\
& \Rightarrow  -\frac{1}{2} yx' - \lambda \frac{ ( (x')^{2}  + (y')^{2} )}{\sqrt{  (x')^{2}  + (y')^{2} }} + \lambda \frac{(y')^{2}}{\sqrt{  (x')^{2}  + (y')^{2} }} = \text{constant} \\
& \Rightarrow - \frac{1}{2} y x' - \lambda \frac{(x')^{2}}{\sqrt{  (x')^{2}  + (y')^{2} }} = C
\end{align*}

for some constant $ C $. Similarly considering $ f_{\lambda}(x,x') $ we have

\[ - \frac{1}{2} x y' - \lambda \frac{(y')^{2}}{\sqrt{  (x')^{2}  + (y')^{2} }} = D \]

for some constant $ D $. Squaring and adding,

\[ (C + \frac{1}{2} y x')^{2} + (D + \frac{1}{2} x y')^{2} = \lambda^{2} \]

Feel like the $ x' $ and $ y' $ shouldn't be here, and it would be a circle?





\section{QUESTION 3}

Using Lagrange multiplier $ \lambda $, wish to minimize


\begin{align*}
\Phi_{\lambda}[\psi] & = I[\psi] - \lambda  \left[    \int_{-\infty}^{\infty}  \psi^{2} \; \d x = 1  \right] \\
& = \int_{-\infty}^{\infty} \underbrace{(\psi')^{2} + (x^{2} - \lambda  ) \psi^{2} }_{f_{\lambda}(\psi,\psi' ; x)} \; \d x + \lambda
\end{align*}

Normalisation condition, can assume $ \phi = 0 $ at endpoints, so the functional is stationary for solutions of the E-L equation.
Euler-Lagrange equations imply:

\[ 2(x^{2} - \lambda) \psi - \frac{\d }{\d x} \left[  2 \psi' \right] = 0  \]

\[ \Rightarrow \psi'' + (x^{2} - \lambda) \psi = 0 \]

Note that

\begin{align*}
I[\psi]  & =  \int_{-\infty}^{\infty}  (\psi' + x^{2} \psi^{2} )^{2} - 2 x \psi \psi' \; \d x    \\
& =  \int_{-\infty}^{\infty}  (\psi' + x^{2} \psi^{2} )^{2} \; \d x  - \int_{-\infty}^{\infty}  x \frac{\d }{\d x}[\psi^{2}] \; \d x \\
& =  \int_{-\infty}^{\infty}  (\psi' + x^{2} \psi^{2} )^{2} \; \d x  - \left[     x \psi^{2} \right]_{-\infty}^{\infty} + \int_{-\infty}^{\infty} \psi^{2} \; \d x  \quad \text{by parts} \\
& =   \int_{-\infty}^{\infty}  (\psi' + x^{2} \psi^{2} )^{2} \; \d x + 1   \end{align*}

As $ (psi' + x^{2} \psi^{2}) $ is real valued, its square gives a positive function, thus the integral is positive and $ I[\psi] \geq 1 $. Equality holds for 

\begin{align*}
& \quad \; \; (\psi' + x \psi) = 0 \\
& \Rightarrow \psi'+ x \psi = 0 \\
& \Rightarrow \frac{\d }{\d x}  \left( \psi e^{x^{2}/2} \right) = 0 \\
& \Rightarrow \psi = C e^{- x^{2} / 2 }  
\end{align*}

for some constant $ C $ (which we recognise as the Gaussian Wave Function).
The normalisation condition implies that $ C = \left(  \frac{1}{\pi} \right)^{1/4}  $. 

Showing it satisfies E-L: $ \psi'' = -C e^{-x^{2}/2}  + x^{2} C e^{-x^{2}/2} $





\section{QUESTION 4}
                                                 
Using Lagrange multiplier $ \lambda $ with constraint $ | \mathbf{x} | = 1 $ wish to minimize
                                                 
\begin{align*}
\Phi_{\lambda}[\mathbf{x}] & = I[\mathbf{x}] - \lambda(  | \mathbf{x} | - 1 ) \\
& = \int_{t_{1}}^{t_{2}} \underbrace{| \dot{\mathbf{x}} |^{2} - \lambda | \mathbf{x} |}_{f_{\lambda}(\mathbf{x},\dot{\mathbf{x}} ; t )  } \; \d t + \lambda
\end{align*}   

The E-L equation for $ x_{i} $ component is $ \frac{\partial f }{\partial x_{i}} - \frac{\d }{\d t} \left(  \frac{\partial f}{\partial \dot{x}_{i}} \right) = 0   $, giving

\[ - \frac{\lambda x_{i}}{| \mathbf{x} |}  - \frac{\d }{\d t} (  2 \dot{x}_{i} ) = 0 \]     

\[ 2 \ddot{x_{i}} + \lambda x_{i} = 0  \qquad \text{as } | \mathbf{x} | = 1 \]        

\[ \Rightarrow \ddot{\mathbf{x}}  + \frac{\lambda}{2} \mathbf{x} = 0 \]    

Also, there is no $ t $ dependence in $ f_{\lambda} $, so E-L equations imply that

\[ f - \dot{x}_{i} \frac{\partial f }{\partial \dot{x}_{i}} = \text{ constant} \]    

\[ f - 2  \dot{x}_{i}^{2} = \text{ constant} \]                         




\section{QUESTION 5}


Have the Lagrangian

\[ L = \underbrace{\frac{1}{2} m a^{2} \dot{\theta}^{2}  + \frac{1}{2} m a^{2} \dot{\theta}^{2}  }_{T} + \underbrace{mga \cos \theta}_{- V} \]

Note that $ \frac{\partial L}{\partial \phi} = 0 $ so we have the first integral

\[ \text{const. } = \frac{\partial L }{\partial \dot{\phi}} = m a^{2} \sin^{2} \theta  \]

Note too that $ \frac{\partial L }{\partial t} = 0 $ (no t dependence) so we have another first integral

\begin{align*}
\text{const. } & = L - \dot{\phi} \frac{\partial L }{\partial \dot{\phi}} - \dot{\theta} \frac{\partial L }{\partial \dot{\theta}} \\
& = - T - V
\end{align*}

from which we deduce that

\[ \frac{1}{2} m a^{2} \dot{\theta}^{2}  + \frac{1}{2} m a^{2} \dot{\theta}^{2}  - mga \cos \theta = E \]

for the total energy.

The Hamiltonian is defined as the Legendre transform of the Lagrangian with respect to velocity $ \mathbf{v} = \dot{\mathbf{x}} $:

\[ H(\mathbf{x},\mathbf{p};t)  = [ \mathbf{p} \cdot \mathbf{v}  - L(\mathbf{x},\mathbf{v}) ]_{\mathbf{v} = \mathbf{v(\mathbf{p})}} \]

where $ \mathbf{v}(\mathbf{p}) $ is the solution to $ \frac{\partial L }{\partial \mathbf{v}} = \mathbf{p} $.

The momentum $ p_{\theta} $ is given by

\begin{align*}
p_{\theta}  & = \frac{\partial  L}{\partial \dot{\theta}} \\
& = m a^{2} \dot{\theta}
\end{align*}

Similarly,

\begin{align*}
p_{\phi} & = \frac{\partial  L}{\partial \dot{\phi}}  \\
& = m a^{2} (\sin^{2} \theta ) \dot{\phi}
\end{align*}

Thus $ \mathbf{v} = (\dot{\theta},\dot{\phi}) = \left(  \frac{p_{\theta}}{ma^{2}},  \frac{p_{\phi}}{ma^{2}\sin^{2}\theta   } \right)   $ and

\begin{align*}
H & = \frac{p_{\theta}^{2}}{ma^{2}} +  \frac{p_{\phi}^{2}}{ma^{2}\sin^{2}\theta } - \left(  \frac{1}{2} m a^{2} \left(  \frac{p_{\theta}}{ma^{2}}\right)^{2}   + \frac{1}{2} m a^{2} \left( \frac{p_{\phi}}{ma^{2}\sin^{2}\theta } \right)^{2}  + mga \cos \theta  \right)    \\
& =  \frac{1}{2} \frac{p_{\theta}^{2}}{ma^{2}} + \frac{1}{2} \frac{p_{\phi}^{2}}{ma^{2}\sin^{2}\theta } -mga \cos \theta
\end{align*}

Hamilton's equations are given by

\[ \dot{\mathbf{x}} = \frac{\partial H }{\partial \mathbf{p}}, \qquad \dot{\mathbf{p}} = - \frac{\partial H }{\partial \mathbf{x}}   \]

ie.

\[ \dot{\theta}  = \frac{p_{\theta}}{ma^{2}} \]
\[ \dot{\phi}  = \frac{p_{\phi}}{ma^{2} \sin^{2} \theta } \]

\[ \dot{p_{\theta}} = \frac{-p_{\phi}^{2} \sin(2\theta) }{2 ma^{2} \sin^{4} \theta}  - mga \cos \theta \]
\[ \dot{p_{\phi}} = 0 \]

\section{QUESTION 6}

\begin{enumerate}
	\item Consider the variation directly, $ u_{t} \to u_{t} + (\delta u)_{t} $ and $ u_{x} \to u_{x} + (\delta u)_{x} $. Then
	
	\begin{align*}
	\delta I[u] & = I[u + \delta u] - I[u] \\
	& = \int \left[ \frac{1}{2}\left(  u_{t} + (\delta u)_{t} \right)^{2} - F(u_{x} + (\delta u)_{x}) \right]  \; \d x \; \d t - \int \left[  \frac{1}{2} u_{t}^{2} - F(u_{x}) \; \d x \; \d t    \right]\\
	& = \int u_{t} (\delta u)_{t} - (\delta u)_{x} \frac{\partial F(u_{x}) }{\partial x} \; \d x \; \d t +   O(t^{2}) \\
	& = \int u_{tt} (\delta u) - (\delta u) \frac{\partial F(u_{x})  }{\partial x^{2}}\; \d x \; \d t + \text{boundary terms} \qquad \text{by parts}
	\end{align*}
	
	ignoring boundary terms
	
	So the Euler-Lagrange equation is 
	
	\[ \frac{\partial^{2} u}{\partial t^{2}} - \frac{\partial F(u_{x}) }{\partial x^{2}} = 0  \]
	
	\item 
	
	Discarding second order terms throughout:
	
	\begin{align*}
	\delta I[u] & = \int \left[ (u_{x} + (\delta u)_{x})^{2} + (  u_{y} + (\delta u)_{y} )^{2} + e^{2(u + \delta u)}  \right] \; \d x \; \d y - \int u_{x}^{2} + u_{y}^{2} + e^{2u} \; \d x \; \d y   \\
	& = \int 2 u_{x} (\delta u)_{x} + 2 u_{y} ( \delta u)_{y} + \left(  e^{2(u + \delta u)} - e^{2u} \right) \; \d x \; \d y + O(u^{2}) \\
	& = \int 2 u_{x} (\delta u)_{x} + 2 u_{y} ( \delta u)_{y} + \left( 2 \delta u  \right) \; \d x \; \d y  \qquad \text{ using } e^{x} \approx 1 + x\\
	& = 2 \int (\delta u) \{ u_{xx} + u_{yy}  + 1 \} \; \d x \; \d y
	\end{align*}
	
	So E-L equation is 
	
	\[ u_{xx} + u_{yy} = -1  \]
	
	
	
	
	
\end{enumerate}


\section{QUESTION 7}

The E-L equations are of the form $ \frac{\partial L }{\partial x_{i} } - \frac{\d }{\d t} \left(  \frac{\partial L}{\partial \dot{x}_{i}} \right) = 0 $

We have
\[ \frac{\partial L }{\partial x_{i}} = - q \nabla_{i} \phi + q \frac{\partial }{\partial x_{i}} (\mathbf{v} \cdot \mathbf{A}) \]

and

\[ \frac{\partial L }{\partial \dot{x}_{i}} = m \gamma v_{i} + q A_{i} \]


E-L equations imply

\begin{align*}
& \quad \quad - q \nabla_{i} \phi + q \frac{\partial }{\partial x_{i}} (\mathbf{v} \cdot \mathbf{A}) - \frac{\d }{\d t}  \left(  m \gamma v_{i} + q A_{i} \right) = 0   \\
& \Rightarrow  \frac{\d }{\d t}  \left(  m \gamma v_{i} \right) = - q \nabla_{i} \phi + q \frac{\partial }{\partial x_{i}} (\mathbf{v} \cdot \mathbf{A}) - q \frac{\d }{\d t}( A_{i})
 \end{align*}
 
 
Using the chain rule:
 
\begin{align*}
\frac{\d A_{i}}{\d t} & = \frac{\partial A_{i} }{\partial t} \frac{\d t}{\d t } + \frac{\partial A_{i} }{\partial x_{j}} \frac{\d x_{j}}{\d t} \\
& = \frac{\partial A_{i} }{\partial t} +  v_{j} \frac{\partial A_{i} }{\partial x_{j}} 
\end{align*}

and

\[ \frac{\partial }{\partial x} (\mathbf{v} \cdot \mathbf{A}) = v_{j} \frac{\partial A_{j} }{\partial x_{i}}  \]


\begin{align*}
\Rightarrow  \frac{\d }{\d t}  \left(  m \gamma v_{i} \right) & = - q \nabla_{i} \phi + q v_{j} \frac{\partial A_{j} }{\partial x_{i}} - q \left[  \frac{\partial A_{i} }{\partial t} +  v_{j} \frac{\partial A_{i} }{\partial x_{j}} \right] \\
 & = q \left( \underbrace{\nabla_{i} \phi - \frac{\partial A_{i} }{\partial t}}_{E_{i}}  + \left[ v_{j} \frac{\partial A_{j} }{\partial x_{i}}  -  v_{j} \frac{\partial A_{i} }{\partial x_{j}} \right]    \right)  
\end{align*}

and 


\section{QUESTION 8}



\section{QUESTION 9}


\section{QUESTION 10}
\section{QUESTION 11}









\end{document}