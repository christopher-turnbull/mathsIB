\documentclass[a4paper]{article}
\usepackage{amsmath}
\def\npart {IA}
\def\nterm {Michaelmas}
\def\nyear {2017}
\def\nlecturer {Mx Tsang}
\def\ncourse {Variational Principles Example Sheet 1}

% Imports
\ifx \nauthor\undefined
  \def\nauthor{Christopher Turnbull}
\else
\fi

\author{Supervised by \nlecturer \\\small Solutions presented by \nauthor}
\date{\nterm\ \nyear}

\usepackage{alltt}
\usepackage{amsfonts}
\usepackage{amsmath}
\usepackage{amssymb}
\usepackage{amsthm}
\usepackage{booktabs}
\usepackage{caption}
\usepackage{enumitem}
\usepackage{fancyhdr}
\usepackage{graphicx}
\usepackage{mathdots}
\usepackage{mathtools}
\usepackage{microtype}
\usepackage{multirow}
\usepackage{pdflscape}
\usepackage{pgfplots}
\usepackage{siunitx}
\usepackage{slashed}
\usepackage{tabularx}
\usepackage{tikz}
\usepackage{tkz-euclide}
\usepackage[normalem]{ulem}
\usepackage[all]{xy}
\usepackage{imakeidx}

\makeindex[intoc, title=Index]
\indexsetup{othercode={\lhead{\emph{Index}}}}

\ifx \nextra \undefined
  \usepackage[pdftex,
    hidelinks,
    pdfauthor={Christopher Turnbull},
    pdfsubject={Cambridge Maths Notes: Part \npart\ - \ncourse},
    pdftitle={Part \npart\ - \ncourse},
  pdfkeywords={Cambridge Mathematics Maths Math \npart\ \nterm\ \nyear\ \ncourse}]{hyperref}
  \title{Part \npart\ --- \ncourse}
\else
  \usepackage[pdftex,
    hidelinks,
    pdfauthor={Christopher Turnbull},
    pdfsubject={Cambridge Maths Notes: Part \npart\ - \ncourse\ (\nextra)},
    pdftitle={Part \npart\ - \ncourse\ (\nextra)},
  pdfkeywords={Cambridge Mathematics Maths Math \npart\ \nterm\ \nyear\ \ncourse\ \nextra}]{hyperref}

  \title{Part \npart\ --- \ncourse \\ {\Large \nextra}}
  \renewcommand\printindex{}
\fi

\pgfplotsset{compat=1.12}

\pagestyle{fancyplain}
\lhead{\emph{\nouppercase{\leftmark}}}
\ifx \nextra \undefined
  \rhead{
    \ifnum\thepage=1
    \else
      \npart\ \ncourse
    \fi}
\else
  \rhead{
    \ifnum\thepage=1
    \else
      \npart\ \ncourse\ (\nextra)
    \fi}
\fi
\usetikzlibrary{arrows.meta}
\usetikzlibrary{decorations.markings}
\usetikzlibrary{decorations.pathmorphing}
\usetikzlibrary{positioning}
\usetikzlibrary{fadings}
\usetikzlibrary{intersections}
\usetikzlibrary{cd}

\newcommand*{\Cdot}{{\raisebox{-0.25ex}{\scalebox{1.5}{$\cdot$}}}}
\newcommand {\pd}[2][ ]{
  \ifx #1 { }
    \frac{\partial}{\partial #2}
  \else
    \frac{\partial^{#1}}{\partial #2^{#1}}
  \fi
}
\ifx \nhtml \undefined
\else
  \renewcommand\printindex{}
  \makeatletter
  \DisableLigatures[f]{family = *}
  \let\Contentsline\contentsline
  \renewcommand\contentsline[3]{\Contentsline{#1}{#2}{}}
  \renewcommand{\@dotsep}{10000}
  \newlength\currentparindent
  \setlength\currentparindent\parindent

  \newcommand\@minipagerestore{\setlength{\parindent}{\currentparindent}}
  \usepackage[active,tightpage,pdftex]{preview}
  \renewcommand{\PreviewBorder}{0.1cm}

  \newenvironment{stretchpage}%
  {\begin{preview}\begin{minipage}{\hsize}}%
    {\end{minipage}\end{preview}}
  \AtBeginDocument{\begin{stretchpage}}
  \AtEndDocument{\end{stretchpage}}

  \newcommand{\@@newpage}{\end{stretchpage}\begin{stretchpage}}

  \let\@real@section\section
  \renewcommand{\section}{\@@newpage\@real@section}
  \let\@real@subsection\subsection
  \renewcommand{\subsection}{\@@newpage\@real@subsection}
  \makeatother
\fi

% Theorems
\theoremstyle{definition}
\newtheorem*{aim}{Aim}
\newtheorem*{axiom}{Axiom}
\newtheorem*{claim}{Claim}
\newtheorem*{cor}{Corollary}
\newtheorem*{conjecture}{Conjecture}
\newtheorem*{defi}{Definition}
\newtheorem*{eg}{Example}
\newtheorem*{ex}{Exercise}
\newtheorem*{fact}{Fact}
\newtheorem*{law}{Law}
\newtheorem*{lemma}{Lemma}
\newtheorem*{notation}{Notation}
\newtheorem*{prop}{Proposition}
\newtheorem*{soln}{Solution}
\newtheorem*{thm}{Theorem}

\newtheorem*{remark}{Remark}
\newtheorem*{warning}{Warning}
\newtheorem*{exercise}{Exercise}

\newtheorem{nthm}{Theorem}[section]
\newtheorem{nlemma}[nthm]{Lemma}
\newtheorem{nprop}[nthm]{Proposition}
\newtheorem{ncor}[nthm]{Corollary}


\renewcommand{\labelitemi}{--}
\renewcommand{\labelitemii}{$\circ$}
\renewcommand{\labelenumi}{(\roman{*})}

\let\stdsection\section
\renewcommand\section{\newpage\stdsection}

% Strike through
\def\st{\bgroup \ULdepth=-.55ex \ULset}

% Maths symbols
\newcommand{\abs}[1]{\left\lvert #1\right\rvert}
\newcommand\ad{\mathrm{ad}}
\newcommand\AND{\mathsf{AND}}
\newcommand\Art{\mathrm{Art}}
\newcommand{\Bilin}{\mathrm{Bilin}}
\newcommand{\bket}[1]{\left\lvert #1\right\rangle}
\newcommand{\B}{\mathcal{B}}
\newcommand{\bolds}[1]{{\bfseries #1}}
\newcommand{\brak}[1]{\left\langle #1 \right\rvert}
\newcommand{\braket}[2]{\left\langle #1\middle\vert #2 \right\rangle}
\newcommand{\bra}{\langle}
\newcommand{\cat}[1]{\mathsf{#1}}
\newcommand{\C}{\mathbb{C}}
\newcommand{\CP}{\mathbb{CP}}
\newcommand{\cU}{\mathcal{U}}
\newcommand{\Der}{\mathrm{Der}}
\newcommand{\D}{\mathrm{D}}
\newcommand{\dR}{\mathrm{dR}}
\newcommand{\E}{\mathbb{E}}
\newcommand{\F}{\mathbb{F}}
\newcommand{\Frob}{\mathrm{Frob}}
\newcommand{\GG}{\mathbb{G}}
\newcommand{\gl}{\mathfrak{gl}}
\newcommand{\GL}{\mathrm{GL}}
\newcommand{\G}{\mathcal{G}}
\newcommand{\Gr}{\mathrm{Gr}}
\newcommand{\haut}{\mathrm{ht}}
\newcommand{\Id}{\mathrm{Id}}
\newcommand{\ket}{\rangle}
\newcommand{\lie}[1]{\mathfrak{#1}}
\newcommand{\Mat}{\mathrm{Mat}}
\newcommand{\N}{\mathbb{N}}
\newcommand{\norm}[1]{\left\lVert #1\right\rVert}
\newcommand{\normalorder}[1]{\mathop{:}\nolimits\!#1\!\mathop{:}\nolimits}
\newcommand\NOT{\mathsf{NOT}}
\newcommand{\Oc}{\mathcal{O}}
\newcommand{\Or}{\mathrm{O}}
\newcommand\OR{\mathsf{OR}}
\newcommand{\ort}{\mathfrak{o}}
\newcommand{\PGL}{\mathrm{PGL}}
\newcommand{\ph}{\,\cdot\,}
\newcommand{\pr}{\mathrm{pr}}
\newcommand{\Prob}{\mathbb{P}}
\newcommand{\PSL}{\mathrm{PSL}}
\newcommand{\Ps}{\mathcal{P}}
\newcommand{\PSU}{\mathrm{PSU}}
\newcommand{\pt}{\mathrm{pt}}
\newcommand{\qeq}{\mathrel{``{=}"}}
\newcommand{\Q}{\mathbb{Q}}
\newcommand{\R}{\mathbb{R}}
\newcommand{\RP}{\mathbb{RP}}
\newcommand{\Rs}{\mathcal{R}}
\newcommand{\SL}{\mathrm{SL}}
\newcommand{\so}{\mathfrak{so}}
\newcommand{\SO}{\mathrm{SO}}
\newcommand{\Spin}{\mathrm{Spin}}
\newcommand{\Sp}{\mathrm{Sp}}
\newcommand{\su}{\mathfrak{su}}
\newcommand{\SU}{\mathrm{SU}}
\newcommand{\term}[1]{\emph{#1}\index{#1}}
\newcommand{\T}{\mathbb{T}}
\newcommand{\tv}[1]{|#1|}
\newcommand{\U}{\mathrm{U}}
\newcommand{\uu}{\mathfrak{u}}
\newcommand{\Vect}{\mathrm{Vect}}
\newcommand{\wsto}{\stackrel{\mathrm{w}^*}{\to}}
\newcommand{\wt}{\mathrm{wt}}
\newcommand{\wto}{\stackrel{\mathrm{w}}{\to}}
\newcommand{\Z}{\mathbb{Z}}
\renewcommand{\d}{\mathrm{d}}
\renewcommand{\H}{\mathbb{H}}
\renewcommand{\P}{\mathbb{P}}
\renewcommand{\sl}{\mathfrak{sl}}
\renewcommand{\vec}[1]{\boldsymbol{\mathbf{#1}}}
%\renewcommand{\F}{\mathcal{F}}

\let\Im\relax
\let\Re\relax

\DeclareMathOperator{\adj}{adj}
\DeclareMathOperator{\Ann}{Ann}
\DeclareMathOperator{\area}{area}
\DeclareMathOperator{\Aut}{Aut}
\DeclareMathOperator{\Bernoulli}{Bernoulli}
\DeclareMathOperator{\betaD}{beta}
\DeclareMathOperator{\bias}{bias}
\DeclareMathOperator{\binomial}{binomial}
\DeclareMathOperator{\card}{card}
\DeclareMathOperator{\ccl}{ccl}
\DeclareMathOperator{\Char}{char}
\DeclareMathOperator{\ch}{ch}
\DeclareMathOperator{\cl}{cl}
\DeclareMathOperator{\cls}{\overline{\mathrm{span}}}
\DeclareMathOperator{\conv}{conv}
\DeclareMathOperator{\corr}{corr}
\DeclareMathOperator{\cosec}{cosec}
\DeclareMathOperator{\cosech}{cosech}
\DeclareMathOperator{\cov}{cov}
\DeclareMathOperator{\covol}{covol}
\DeclareMathOperator{\diag}{diag}
\DeclareMathOperator{\diam}{diam}
\DeclareMathOperator{\Diff}{Diff}
\DeclareMathOperator{\disc}{disc}
\DeclareMathOperator{\dom}{dom}
\DeclareMathOperator{\End}{End}
\DeclareMathOperator{\energy}{energy}
\DeclareMathOperator{\erfc}{erfc}
\DeclareMathOperator{\erf}{erf}
\DeclareMathOperator*{\esssup}{ess\,sup}
\DeclareMathOperator{\ev}{ev}
\DeclareMathOperator{\Ext}{Ext}
\DeclareMathOperator{\Fit}{Fit}
\DeclareMathOperator{\fix}{fix}
\DeclareMathOperator{\Frac}{Frac}
\DeclareMathOperator{\Gal}{Gal}
\DeclareMathOperator{\gammaD}{gamma}
\DeclareMathOperator{\gr}{gr}
\DeclareMathOperator{\hcf}{hcf}
\DeclareMathOperator{\Hom}{Hom}
\DeclareMathOperator{\id}{id}
\DeclareMathOperator{\image}{image}
\DeclareMathOperator{\im}{im}
\DeclareMathOperator{\Im}{Im}
\DeclareMathOperator{\Ind}{Ind}
\DeclareMathOperator{\Int}{Int}
\DeclareMathOperator{\Isom}{Isom}
\DeclareMathOperator{\lcm}{lcm}
\DeclareMathOperator{\length}{length}
\DeclareMathOperator{\Lie}{Lie}
\DeclareMathOperator{\like}{like}
\DeclareMathOperator{\Lk}{Lk}
\DeclareMathOperator{\mse}{mse}
\DeclareMathOperator{\multinomial}{multinomial}
\DeclareMathOperator{\orb}{orb}
\DeclareMathOperator{\ord}{ord}
\DeclareMathOperator{\otp}{otp}
\DeclareMathOperator{\Poisson}{Poisson}
\DeclareMathOperator{\poly}{poly}
\DeclareMathOperator{\rank}{rank}
\DeclareMathOperator{\rel}{rel}
\DeclareMathOperator{\Re}{Re}
\DeclareMathOperator*{\res}{res}
\DeclareMathOperator{\Res}{Res}
\DeclareMathOperator{\rk}{rk}
\DeclareMathOperator{\Root}{Root}
\DeclareMathOperator{\sech}{sech}
\DeclareMathOperator{\sgn}{sgn}
\DeclareMathOperator{\spn}{span}
\DeclareMathOperator{\stab}{stab}
\DeclareMathOperator{\St}{St}
\DeclareMathOperator{\supp}{supp}
\DeclareMathOperator{\Syl}{Syl}
\DeclareMathOperator{\Sym}{Sym}
\DeclareMathOperator{\tr}{tr}
\DeclareMathOperator{\Tr}{Tr}
\DeclareMathOperator{\var}{var}
\DeclareMathOperator{\vol}{vol}

\pgfarrowsdeclarecombine{twolatex'}{twolatex'}{latex'}{latex'}{latex'}{latex'}
\tikzset{->/.style = {decoration={markings,
                                  mark=at position 1 with {\arrow[scale=2]{latex'}}},
                      postaction={decorate}}}
\tikzset{<-/.style = {decoration={markings,
                                  mark=at position 0 with {\arrowreversed[scale=2]{latex'}}},
                      postaction={decorate}}}
\tikzset{<->/.style = {decoration={markings,
                                   mark=at position 0 with {\arrowreversed[scale=2]{latex'}},
                                   mark=at position 1 with {\arrow[scale=2]{latex'}}},
                       postaction={decorate}}}
\tikzset{->-/.style = {decoration={markings,
                                   mark=at position #1 with {\arrow[scale=2]{latex'}}},
                       postaction={decorate}}}
\tikzset{-<-/.style = {decoration={markings,
                                   mark=at position #1 with {\arrowreversed[scale=2]{latex'}}},
                       postaction={decorate}}}
\tikzset{->>/.style = {decoration={markings,
                                  mark=at position 1 with {\arrow[scale=2]{latex'}}},
                      postaction={decorate}}}
\tikzset{<<-/.style = {decoration={markings,
                                  mark=at position 0 with {\arrowreversed[scale=2]{twolatex'}}},
                      postaction={decorate}}}
\tikzset{<<->>/.style = {decoration={markings,
                                   mark=at position 0 with {\arrowreversed[scale=2]{twolatex'}},
                                   mark=at position 1 with {\arrow[scale=2]{twolatex'}}},
                       postaction={decorate}}}
\tikzset{->>-/.style = {decoration={markings,
                                   mark=at position #1 with {\arrow[scale=2]{twolatex'}}},
                       postaction={decorate}}}
\tikzset{-<<-/.style = {decoration={markings,
                                   mark=at position #1 with {\arrowreversed[scale=2]{twolatex'}}},
                       postaction={decorate}}}

\tikzset{circ/.style = {fill, circle, inner sep = 0, minimum size = 3}}
\tikzset{mstate/.style={circle, draw, blue, text=black, minimum width=0.7cm}}

\tikzset{commutative diagrams/.cd,cdmap/.style={/tikz/column 1/.append style={anchor=base east},/tikz/column 2/.append style={anchor=base west},row sep=tiny}}

\definecolor{mblue}{rgb}{0.2, 0.3, 0.8}
\definecolor{morange}{rgb}{1, 0.5, 0}
\definecolor{mgreen}{rgb}{0.1, 0.4, 0.2}
\definecolor{mred}{rgb}{0.5, 0, 0}

\def\drawcirculararc(#1,#2)(#3,#4)(#5,#6){%
    \pgfmathsetmacro\cA{(#1*#1+#2*#2-#3*#3-#4*#4)/2}%
    \pgfmathsetmacro\cB{(#1*#1+#2*#2-#5*#5-#6*#6)/2}%
    \pgfmathsetmacro\cy{(\cB*(#1-#3)-\cA*(#1-#5))/%
                        ((#2-#6)*(#1-#3)-(#2-#4)*(#1-#5))}%
    \pgfmathsetmacro\cx{(\cA-\cy*(#2-#4))/(#1-#3)}%
    \pgfmathsetmacro\cr{sqrt((#1-\cx)*(#1-\cx)+(#2-\cy)*(#2-\cy))}%
    \pgfmathsetmacro\cA{atan2(#2-\cy,#1-\cx)}%
    \pgfmathsetmacro\cB{atan2(#6-\cy,#5-\cx)}%
    \pgfmathparse{\cB<\cA}%
    \ifnum\pgfmathresult=1
        \pgfmathsetmacro\cB{\cB+360}%
    \fi
    \draw (#1,#2) arc (\cA:\cB:\cr);%
}
\newcommand\getCoord[3]{\newdimen{#1}\newdimen{#2}\pgfextractx{#1}{\pgfpointanchor{#3}{center}}\pgfextracty{#2}{\pgfpointanchor{#3}{center}}}

\def\Xint#1{\mathchoice
   {\XXint\displaystyle\textstyle{#1}}%
   {\XXint\textstyle\scriptstyle{#1}}%
   {\XXint\scriptstyle\scriptscriptstyle{#1}}%
   {\XXint\scriptscriptstyle\scriptscriptstyle{#1}}%
   \!\int}
\def\XXint#1#2#3{{\setbox0=\hbox{$#1{#2#3}{\int}$}
     \vcenter{\hbox{$#2#3$}}\kern-.5\wd0}}
\def\ddashint{\Xint=}
\def\dashint{\Xint-}

\newcommand\separator{{\centering\rule{2cm}{0.2pt}\vspace{2pt}\par}}

\newenvironment{own}{\color{gray!70!black}}{}

\newcommand\makecenter[1]{\raisebox{-0.5\height}{#1}}

\newtheorem*{soln}{Solution}

\renewcommand{\thesection}{}
\renewcommand{\thesubsection}{\arabic{section}.\arabic{subsection}}
\makeatletter
\def\@seccntformat#1{\csname #1ignore\expandafter\endcsname\csname the#1\endcsname\quad}
\let\sectionignore\@gobbletwo
\let\latex@numberline\numberline
\def\numberline#1{\if\relax#1\relax\else\latex@numberline{#1}\fi}
\makeatother


\begin{document}
	
\maketitle

\section{QUESTION 1}

\begin{soln}
\[ \nabla \phi = (x_{1}^{3} - x_{2} - x_{3}, x_{2}^{3} - x_{3} - x_{1}, x_{3}^{3} - x_{1} - x_{2}) \]
\end{soln}

Setting $ \nabla \phi = 0 $ yields three equations satisfied by the coordinates $ \mathbf{x} = (x_{1},x_{2},x_{3}) $ of the stationary points of $ \phi $. Subtracting the first two of these gives

\[ (x_{1} - x_{2})(x_{1}^{2} + x_{1}x_{2} + x_{2}^{2}) = -(x_{1}-x_{2}) \]

\[ \Rightarrow (x_{1} - x_{2}) = 0 \quad \text{ or } \quad (x_{1}^{2} + x_{1}x_{2} + x_{2}^{2} + 1) = 0  \]

Treating the second equation as a quadratic in $ x_{1} $ gives the discriminant as $ -3x_{2}^{2} - 4  < 0 $ with no real solutions. Hence $ x_{1} = x_{2} $, and by symmetry, $ x_{1} = x_{2} = x_{3} $. Using  $ [\nabla \phi]_{1} = 0 $ gives $ x_{1}^{3} = 2 x_{1}  \Rightarrow x_{1} = \pm \sqrt{2}, 0 $. The stationary points of $ \phi $ are:

\[ (0,0,0) \qquad \text{and} \qquad (\pm \sqrt{2},\pm \sqrt{2},\pm \sqrt{2}) \]


For $ (\pm \sqrt{2},\pm \sqrt{2},\pm \sqrt{2}) $, the Hessian is 

\[ \mathbf{H} = \begin{pmatrix}
6 & -1 & -1 \\
-1 & 6 & -1 \\
-1 & -1 & 6 
\end{pmatrix} \]

with eigenvalues $ \lambda_{1} = \lambda_{2} = 7 $, $ \lambda_{3} = 4 $.
As all eigenvalues are positive, both of these points are minima. Hence $ \phi $ takes it's minimum value at both of these two points, and this minimum value is:

\begin{align*}
\phi (\pm \sqrt{2},\pm \sqrt{2},\pm \sqrt{2}) & = \frac{1}{4}( 3  (\sqrt{2})^{4} ) - 3(2)\\
& = -3
\end{align*}


For $ (0,0,0) $, the Hessian is 

\[ \mathbf{H} = \begin{pmatrix}
0 & -1 & -1 \\
-1 & 0 & -1 \\
-1 & -1 & 0 
\end{pmatrix} \]

with eigenvalues $ \lambda_{1} = \lambda_{2} = 1 $, $ \lambda_{3} = -1 $. As some are positive and the rest negative, $ (0,0,0) $ is a saddle point of $ \phi $.

Now let $ x_{i} = R_{ij} x_{j}' $ for some rotation matrix $ R $, which we can choose such that the matrix $ H' = R^{T} H R $ is diagonal. Neglecting terms of order $ x_{3} $ we have

\begin{align*}
\phi & = \frac{1}{2} x_{i} H_{ij} x_{j} + O(x^{3})\\
& = \frac{1}{2}\sum_{i = 1 }^{3} \lambda_{i} (x_{i}')^{2} \\
& = \frac{1}{2}\left( x_{1}' + x_{2}' - 2x_{3}' \right)  
\end{align*}

The surface here becomes $ x_{3}' = \frac{1}{2} (x_{1}' + x_{2}') $. Geometrically, we see this is a cone with semi-angle $ \arctan \sqrt{2} $



\section{QUESTION 2}

\begin{soln}
\begin{enumerate}
	\item The upper half plane is trivially a convex set, hence $ f(x,y) = x^{2} / y $ is convex on the upper half plane $ (x,y) : y > 0 $ if and only if for all $ (x,y), (x',y') $ in the upper half plane:
	
	\[ f \left[ (1 - t) \begin{pmatrix}
	x \\
	y
	\end{pmatrix}  + t \begin{pmatrix}
	x' \\
	y'
	\end{pmatrix} \right] \leq (1-t) f \begin{pmatrix}
	x \\
	y
	\end{pmatrix} + t f\begin{pmatrix}
	x' \\
	y'
	\end{pmatrix}, \qquad 0 < t < 1 \]
	
	This is true if and only if
	
	\[ \frac{[ (1 - t)x_ + t x' ]^{2}}{(1-t)y + y'} \leq (1-t) \frac{x^{2}}{y} + t \frac{(x')^{2}}{y'} \]
	
	\[ \iff [ (1 - t)x_ + t x' ]^{2}y y' = [(1-t) x^{2}y' + t(x')^{2}y][(1-t)y + y']  \]
	\[ \iff 2t(1-t) xx'yy' \leq t(1-t)\left[ x^{2}(y')^{2} +  (x')^{2}y^{2}\right]  \]
	\[ \iff \left[  xy' + x'y \right] \geq 0 \]
	
	where the last inequality is trivially true. Hence convexity follows.
	
	\item Given $ F(x,y) = y f(x/y) $, we are trying to show
	
	\[ F \left[  (1-t) \begin{pmatrix}
	x\\
	y
	\end{pmatrix} + t \begin{pmatrix}
	x' \\
	y'
	\end{pmatrix} \right] \leq (1-t) F \begin{pmatrix}
	x\\
	y
	\end{pmatrix} + t F \begin{pmatrix}
	x'\\
	y'
	\end{pmatrix} \]
	
	for some $0 < t < 1 $, for all $ (x,y), (x',y') $ in the upper half plane, given that $ f(x) $ is convex. This is true if and only if
	
	\[ \left[ (1-t)y + t y' \right] f \left( \frac{(1-t)x + t x'}{(1-t)y + t y'} \right)  \leq (1-t)f(x/y) + t y' f(x' / y') \]
	
	Using the fact that $ f(x/y) $ is convex, for some $ 0 < t < 1 $ we have
	
	\[ f \left[ (1-t)x/y + t x'/y' \right]  \leq (1-t)f(x/y) + t f(x'/y')  \]
	
	Upon replacing $ t  $  with $ s = \frac{ty'}{(1-t)y + ty'} $, the result follows immediately. 
\end{enumerate}
\end{soln}



\section{QUESTION 3}

	
\begin{soln}
The Legrende transform of $ f(x) = e^{x} $ is given by

\[ f^{*}(p) = \sup_{x} \left[ px - e^{x} \right]  \]

In this case $ p = e^{x} $, and hence $ x = \log p $ at the maximum of $ px - e^{x} $, which is then $ f^{*}(p) $. So

\[ f^{*}(p) = p \log p - p, \qquad p \in \R, p > 0 \]

Similarly, the Legrende transform of $ f(x) = a^{-1}x^{a} $, $ a > 1, x > 0 $ is given by

\[ f^{*}(p) = \sup_{x} \left[ px - a^{-1}x^{a} \right]  \]

In this case $ p = x^{a-1} $, and 

\[ f^{*}(p) = b^{-1}p^{b}, \qquad \text{ where } b = \frac{a}{a-1}  \]

\end{soln}	

	


\section{QUESTION 4}

\begin{soln}
	The Hemholtz free energy is defined by 
	
	\[ F(T,V) = \min_{S}\left[ U(S,V) - TS \right]  \]
	
Differentiating with respect to $ S $ gives 

\begin{align*}
T & = \frac{\partial U }{\partial S}  \\
& = T_{0} \left( \frac{V_{0}}{V} \right)^{1/\alpha} \exp\left(   \frac{S - S_{0}}{\alpha n R}\right) 
\end{align*}

Rearranging, $ S = S_{0} + \alpha n R \log \left[ \frac{T}{T_{0}}  \left( \frac{V_{0}}{V} \right)^{1/\alpha} \right]  $. Hence

\[  F(T,V) = U_{0} + \alpha n R (T - T_{0}) - T \left(  S_{0} + \alpha n R \log \left[ \frac{T}{T_{0}}  \left( \frac{V_{0}}{V} \right)^{1/\alpha} \right] \right)  \]
\end{soln}

\section{QUESTION 5}

\begin{soln}
	
	\begin{enumerate}
		\item For a triangle of given perimeter $ 2s $, the area
		
		\[ A = \sqrt{s(s-a)(s-b)(a+b-s)} \] is maximised when $ \frac{\partial A }{\partial a} = \frac{\partial A }{\partial b} = 0 $.
		
		\[ \frac{\partial A }{\partial a} = \frac{-\sqrt{s(s-b)(a+b-s)} }{2\sqrt{(s-a)}} + \frac{\sqrt{s(s-a)(s-b)}}{2\sqrt{(a+b-s)}}\]
		
		Setting $ \frac{\partial A }{\partial a} = 0 $ we recover $ a + b - s = s - a \Rightarrow 2s = 2a + b $. Similarly, $ \frac{\partial A }{\partial b} = 0 \Rightarrow 2s = a + 2b $. Hence $ a = b $, and by symmetry, $ a = b = c $ and the triangle is equilateral.
		
		
		\item For a right-angled triangle with sides $ a,b,c $ where $ c^{2} = a^{2} + b^{2} $, the perimeter $ P $ is given by
		
		\[ P = a^{2} + b^{2} + \sqrt{a^{2} + b^{2}} \]
		
		and the area $ A $ by
		
		\[ A = \frac{ab}{2} \]
		
		Rearranging the first expression for $ a $ we find that
		
		\[ a = \frac{1}{2} \left( \frac{2bP-P^{2}}{b - P} \right)  \]
		
		Substituting this in, we find
		
		\[ A = \frac{1}{4} \left(  \frac{2bP - P^{2}}{b - P} \right) b   \]
		
		So setting $ \frac{\partial A }{\partial b} = 0 $,
		
		\begin{align*}
		& \frac{1}{4} \left(  \frac{2bP - P^{2}}{b - P} \right) + \frac{1}{4} \left(  \frac{(b - P)2P - (2bP - P^{2})}{(b-P)^{2}}  \right) b = 0   \\
		\Rightarrow \; & (2bP - P^{2})(b - P) + (b - P)2P - (2bP - P^{2}) = 0
		\end{align*}
		
		
	\end{enumerate}
	
	

\end{soln}

\section{QUESTION 6}


\begin{soln}
The volume of the parallelepiped is $ 2x \times 2y \times 2z = 8xyz $
Using a Lagrange multiplier $ \lambda $ to impose the constraint, we have

\[ \Phi_{\lambda}[\mathbf{x}] = 8xyz - \lambda\left(  \frac{x^{2}}{a^{2}} + \frac{x^{2}}{x^{2}} + \frac{x^{2}}{c^{2}} - 1 \right)  \]
	
	
\[ 	\frac{\partial \Phi }{\partial x} = 0 \Rightarrow 8yz - \frac{2 x \lambda}{a^{2}} = 0 \Rightarrow \frac{1}{a^{2}} = \frac{4yz}{x\lambda} \]

Similarly

\[ 	\frac{\partial \Phi }{\partial y} = 0 \Rightarrow \frac{1}{b^{2}} = \frac{4zx}{y\lambda}, \quad 	\frac{\partial \Phi }{\partial z} = 0 \Rightarrow \frac{1}{c^{2}} = \frac{4xy}{z\lambda} \]

Whence, 

\begin{align*}
\frac{\partial \Phi }{\partial \lambda} = 0 & \Rightarrow \frac{x^{2}}{a^{2}} + \frac{x^{2}}{x^{2}} + \frac{x^{2}}{c^{2}} = 1 \\
& \Rightarrow \lambda = 12 xyz
\end{align*}

Substituting our value of $ \lambda $ into the $ \frac{\partial \Phi }{\partial x} = 0  $ equation gives

\[ \frac{1}{a^{2}} = \frac{4yz}{12x^{2}yz} \Rightarrow x = \frac{a}{\sqrt{3}} \]	

Similarly, $ y = \frac{b}{\sqrt{3}} $, $ z = \frac{c}{\sqrt{3}} $ and 

\begin{align*}
V & = 8xyz \\
& = \frac{8abc}{3\sqrt{3}}
\end{align*}
	
\end{soln}


\section{QUESTION 7}


\begin{soln}
	In spherical coordinates the distance functional is given by 
	
	\[ F[r,\theta,\phi] = \int_{t}^{t + \delta t} \sqrt{\dot{r}^{2} + r^{2} \dot{\theta}^{2} + r^{2}\sin^{2}\theta \dot{\phi}^{2}} \; \d t \]
	
	On the unit sphere, $ r = 1 $, and this becomes:
	
		\[ F[\theta,\phi] = \int_{t}^{t + \delta t} \sqrt{ \dot{\theta}^{2} + \sin^{2}\theta \dot{\phi}^{2}} \; \d t \]
	
	
	 Equivalently, if $ \theta $ is a good parameter for the curve, we can consider the functional obtained from a change of variables
	
		\[ F[\phi] = \int_{\theta}^{\theta + \delta \theta} \sqrt{1 +  \sin^{2}\theta(\phi')^{2}} \; \d \theta \]
		
		where the curve is now specified by the function $ \phi(\theta) $.
		
		The functional for the total path length between any two points on the unit sphere is given by:
		
			\[ L[\phi] = \int_{\theta_{A}}^{\theta_{B}} \sqrt{1 +  \sin^{2}\theta(\phi')^{2}} \; \d \theta \] 
			
		In this case with $ f = \sqrt{1 +  \sin^{2}\theta(\phi')^{2}} $, $ \frac{\partial f }{\partial \phi} = 0 $ and the Euler Lagrange equation can be immediately once integrated to give the first integral:
		
		\[ \frac{\phi'\sin^{2}\theta}{\sqrt{1 +  \sin^{2}\theta(\phi')^{2}}} = c \]
		
		for some constant $ c $. Hence 
		
		\[ \phi' = \frac{c}{\sin \theta \sqrt{\sin^{2}\theta - c^{2}}} \]
		
		Substitute $ u = \cot \theta $ so that $ \d u = - \cosec^{2} \theta \d \theta $. Then
		
		\begin{align*}
		\phi & = \int \frac{-c \; \d u}{\sqrt{1 - c^{2}\cosec^{2}\theta}}\\
		& =  \int \frac{-c \d u}{\sqrt{1 - c^{2}(1 + u^{2})}} \\
		& = \frac{- \d u}{\sqrt{a^{2} - u^{2}}} \qquad \text{ where } \; a = \sqrt{1 - c^{2}} / c \\
		& = \cos^{-1}(u/a) + \phi_{0}
		\end{align*}
		
		where $ \phi_{0} $ is a constant of integration. Hence, the path is given by
		
		\[ \cot \theta = a \cos (\phi - \phi_{0}) \]
		
		which is the path of a great circle. 
	
	

\end{soln}


\section{QUESTION 8}

\begin{soln}
	Assuming $ z $ is a good parameter for the curve, it can be written as $ r(z) $, and we are trying to maximize the functional for the surface area
	\[ S[r] = \int_{-b}^{b} 2 \pi r \sqrt{ 1 + (r')^{2}} \; \d z  \quad \Rightarrow \quad f = 2 \pi r \sqrt{ 1 + (r')^{2}} \]
	
	As $ f $ has no explicit $ z $-dependence, we have the first integral 
	
	\[ \text{constant } = f - r' \frac{\partial f }{\partial r'} = \frac{2 \pi r}{\sqrt{1 + (r')^{2}}} \quad \Rightarrow r = c \cosh (z / c)\]
	
	Using the boundary conditions $ r = a \Rightarrow z = \pm b $, we have $ a / c = \cosh(  b / c) $
\end{soln}


\section{QUESTION 9}
\begin{soln}
	Let Stratford be the origin of coordinates in the vertical plane with $ x $ being horizontal distance from the origin and $ y $ being the vertical distance below the origin. The train depart with zero velocity so conservation of energy implies that its speed $ v $at any later time is given by
	
	\[ \frac{1}{2}mv^{2} = mgy \; \Rightarrow \; v = \sqrt{2gy} \]
	
	We have to find the path that minimizes the travel time when the speed depends on position, ie. we have to minimize:
	
	\[ T = \int_{A}^{B} \frac{\d l}{v} = \frac{1}{\sqrt{2g}}\int_{A}^{B} \frac{\sqrt{\d x^{2} + \d y^{2}}}{\sqrt{y}} \]
	
	Assuming $ x $ is a good coordinate for the curve, 
	
	\[ T[y] \propto \int_{0}^{x_{B}} \sqrt{\frac{1 + (y')^{2}}{y}}  \; \d x, \quad \Rightarrow \quad f = \sqrt{\frac{1 + (y')^{2}}{y}}\]
	
	As $ f $ has no explicit $ x $-dependence, we have the first integral 
	
	\[ \text{constant } = f - y' \frac{\partial f }{\partial y'} = \frac{1}{y[1 + (y')^{2}]} \quad \Rightarrow \quad y [ 1 + (y')^{2}]  = 2c\]
	
	for positive constant $ c $. The solution of this first-order ODE with $ y(0) = 0 $ is given parametrically by
	
	\[ x = c(\theta - \sin \theta), \qquad y = c( 1 - \cos \theta) \]
	
	which is an inverted cycloid. The origin (Stratford) corresponds to $ \theta = 0 $. Requiring the cycloid passes through $ (l,0) $ (Acton) gives $ \theta = 2\pi $, $ c = \frac{l}{2 \pi} $. Hence, the time taken is given by
	
	\begin{align*}
	T & = \frac{1}{\sqrt{2g}}\int_{A}^{B} \frac{\sqrt{\d x^{2} + \d y^{2}}}{\sqrt{y}} \\
	& =  \frac{1}{\sqrt{2g}} \int_{0}^{2\pi} \sqrt{\frac{c^{2}(1 - \cos \theta)^{2} + \theta^{2} \sin^{2} \theta}{c(1 - \cos \theta)}} \; \d \theta \\
	& =  \sqrt{\frac{c}{2g}} \int_{0}^{2\pi} \sqrt{2} \; \d \theta\\
	& = \sqrt{\frac{2\pi l}{g}}
	\end{align*}
	
	
\end{soln}



\section{QUESTION 10}


\begin{soln}
	Fermat's principle states that light takes the path of least time. We wish to minimize
	
	\begin{align*}
	T[y] & = \int \frac{\d l}{v} \\
	& = \propto \int  \sqrt{(1 - ky)(1+(y')^{2}} \; \d x
	\end{align*}
	
	Notice that 
	
	\[ f =  \sqrt{(1 - ky)(1+(y')^{2}} \quad \Rightarrow \quad \frac{\partial f }{\partial x} = 0\]
	
	so we have the first integral
	
		\[ \text{constant } = f - y' \frac{\partial f }{\partial y'} = \sqrt{\frac{1-ky}{1+(y')^{2}}}\]
		
	Squaring, we deduce that 
	
	\[ (y')^{2} = (k / c^{2}) (y_{0} - y), \qquad y_{0} = \frac{1 - c^{2}}{k} \]
	
	Taking the square root we deduce that
	
	\[ \frac{\d }{\d x} \left[  \sqrt{y - y_{0}} \pm \frac{k}{2 \sqrt{c}} x \right] = 0 \Rightarrow y = y_{0} - \frac{k^{2}}{4 c^{2}} (x - x_{0})^{2}  \]
	
	where $ x_{0} $ is another integration constant. This is a parabola, with maximum height $ y = y_{0} $. If the ray enters the medium at $ (-x_{0},0) $ and leaves at $ (x_{0},0) $, maximum height is reached at $ x = 0 $. Substituting in $ y_{0} $ for $ c $ gives:
	
	\[ y = y_{0} - \frac{(kx_{0})^{2}}{4(1-k y_{0})^{2}} \]
	
\end{soln}
 
\section{QUESTION 11}



\section{QUESTION 12}
\begin{soln}
	
	Writing the area of the enclosed region as an integral over $ x $ of area elements of vertical strips, the total area is:
	
	\[ A[y] =  \int_{0}^{a} y(x) \; \d x \]
	
	We must maximize $ A $ subject to the condition that $ P[y] = L $, where 
	
	\[ P[y] = \int \sqrt{1 + (y')^{2}} \; \d x \]
	
	Using a Lagrange multiplier to impose the constraint, we have
	
	\[ \Phi_{\lambda}[y] = \int f_{\lambda} (y,y') \; \d x + \lambda L, \qquad f(y,y') = y - \lambda \sqrt{1 + (y')^{2}} \]
	
	$ f_{\lambda}(y,y') $ has no explicit $ x $-dependence, so the EL equations imply that
	
	\[ \text{constant } = f_{\lambda} - y' \frac{\partial f_{\lambda} }{\partial y'} = y - \frac{\lambda}{\sqrt{1 + (y')^{2}}}  \]
	
	This is equivalent to 
	
	\[ (y')^{2} = \frac{\lambda^{2}}{(y - y_{0})^{2}} - 1 \]
	
	for some constant $ y_{0} $. This ODE has the solution $ y = y_{0} \pm \sqrt{\lambda^{2} - (x - x_{0})^{2}} $ for some constant $ x_{0} $, so
	
	\[ (x - x_{0})^{2} + (y - y_{0})^{2} = \lambda^{2} \]
	
	
 	
\end{soln}

\section{QUESTION 13}


\begin{soln}
		Functional for the total length is 
		
		\[ P[y] = \int_{-a}^{a}  y\sqrt{1 + (y')^{2}} \; \d x \]
		
		Using a Lagrange multiplier to impose the constraint, we have
	
	\[ \Phi_{\lambda}[y] = \int f_{\lambda} (y,y') \; \d x + \lambda P, \qquad f(y,y') = (y- \lambda)\sqrt{1 + (y')^{2}} \]
	
		$ f_{\lambda}(y,y') $ has no explicit $ x $-dependence, so the EL equations imply that
	
	\[ \text{constant } = f_{\lambda} - y' \frac{\partial f_{\lambda} }{\partial y'} =  \]
	
	
	
	
\end{soln}


\end{document}