\documentclass[a4paper]{article}
\usepackage{amsmath}
\def\npart {IB}
\def\nterm {Michaelmas}
\def\nyear {2017}
\def\nlecturer {Mr Rawlinson ( jir25@cam.ac.uk )}
\def\ncourse {Linear Algebra Sheet 1}

% Imports
\ifx \nauthor\undefined
  \def\nauthor{Christopher Turnbull}
\else
\fi

\author{Supervised by \nlecturer \\\small Solutions presented by \nauthor}
\date{\nterm\ \nyear}

\usepackage{alltt}
\usepackage{amsfonts}
\usepackage{amsmath}
\usepackage{amssymb}
\usepackage{amsthm}
\usepackage{booktabs}
\usepackage{caption}
\usepackage{enumitem}
\usepackage{fancyhdr}
\usepackage{graphicx}
\usepackage{mathdots}
\usepackage{mathtools}
\usepackage{microtype}
\usepackage{multirow}
\usepackage{pdflscape}
\usepackage{pgfplots}
\usepackage{siunitx}
\usepackage{slashed}
\usepackage{tabularx}
\usepackage{tikz}
\usepackage{tkz-euclide}
\usepackage[normalem]{ulem}
\usepackage[all]{xy}
\usepackage{imakeidx}

\makeindex[intoc, title=Index]
\indexsetup{othercode={\lhead{\emph{Index}}}}

\ifx \nextra \undefined
  \usepackage[pdftex,
    hidelinks,
    pdfauthor={Christopher Turnbull},
    pdfsubject={Cambridge Maths Notes: Part \npart\ - \ncourse},
    pdftitle={Part \npart\ - \ncourse},
  pdfkeywords={Cambridge Mathematics Maths Math \npart\ \nterm\ \nyear\ \ncourse}]{hyperref}
  \title{Part \npart\ --- \ncourse}
\else
  \usepackage[pdftex,
    hidelinks,
    pdfauthor={Christopher Turnbull},
    pdfsubject={Cambridge Maths Notes: Part \npart\ - \ncourse\ (\nextra)},
    pdftitle={Part \npart\ - \ncourse\ (\nextra)},
  pdfkeywords={Cambridge Mathematics Maths Math \npart\ \nterm\ \nyear\ \ncourse\ \nextra}]{hyperref}

  \title{Part \npart\ --- \ncourse \\ {\Large \nextra}}
  \renewcommand\printindex{}
\fi

\pgfplotsset{compat=1.12}

\pagestyle{fancyplain}
\lhead{\emph{\nouppercase{\leftmark}}}
\ifx \nextra \undefined
  \rhead{
    \ifnum\thepage=1
    \else
      \npart\ \ncourse
    \fi}
\else
  \rhead{
    \ifnum\thepage=1
    \else
      \npart\ \ncourse\ (\nextra)
    \fi}
\fi
\usetikzlibrary{arrows.meta}
\usetikzlibrary{decorations.markings}
\usetikzlibrary{decorations.pathmorphing}
\usetikzlibrary{positioning}
\usetikzlibrary{fadings}
\usetikzlibrary{intersections}
\usetikzlibrary{cd}

\newcommand*{\Cdot}{{\raisebox{-0.25ex}{\scalebox{1.5}{$\cdot$}}}}
\newcommand {\pd}[2][ ]{
  \ifx #1 { }
    \frac{\partial}{\partial #2}
  \else
    \frac{\partial^{#1}}{\partial #2^{#1}}
  \fi
}
\ifx \nhtml \undefined
\else
  \renewcommand\printindex{}
  \makeatletter
  \DisableLigatures[f]{family = *}
  \let\Contentsline\contentsline
  \renewcommand\contentsline[3]{\Contentsline{#1}{#2}{}}
  \renewcommand{\@dotsep}{10000}
  \newlength\currentparindent
  \setlength\currentparindent\parindent

  \newcommand\@minipagerestore{\setlength{\parindent}{\currentparindent}}
  \usepackage[active,tightpage,pdftex]{preview}
  \renewcommand{\PreviewBorder}{0.1cm}

  \newenvironment{stretchpage}%
  {\begin{preview}\begin{minipage}{\hsize}}%
    {\end{minipage}\end{preview}}
  \AtBeginDocument{\begin{stretchpage}}
  \AtEndDocument{\end{stretchpage}}

  \newcommand{\@@newpage}{\end{stretchpage}\begin{stretchpage}}

  \let\@real@section\section
  \renewcommand{\section}{\@@newpage\@real@section}
  \let\@real@subsection\subsection
  \renewcommand{\subsection}{\@@newpage\@real@subsection}
  \makeatother
\fi

% Theorems
\theoremstyle{definition}
\newtheorem*{aim}{Aim}
\newtheorem*{axiom}{Axiom}
\newtheorem*{claim}{Claim}
\newtheorem*{cor}{Corollary}
\newtheorem*{conjecture}{Conjecture}
\newtheorem*{defi}{Definition}
\newtheorem*{eg}{Example}
\newtheorem*{ex}{Exercise}
\newtheorem*{fact}{Fact}
\newtheorem*{law}{Law}
\newtheorem*{lemma}{Lemma}
\newtheorem*{notation}{Notation}
\newtheorem*{prop}{Proposition}
\newtheorem*{soln}{Solution}
\newtheorem*{thm}{Theorem}

\newtheorem*{remark}{Remark}
\newtheorem*{warning}{Warning}
\newtheorem*{exercise}{Exercise}

\newtheorem{nthm}{Theorem}[section]
\newtheorem{nlemma}[nthm]{Lemma}
\newtheorem{nprop}[nthm]{Proposition}
\newtheorem{ncor}[nthm]{Corollary}


\renewcommand{\labelitemi}{--}
\renewcommand{\labelitemii}{$\circ$}
\renewcommand{\labelenumi}{(\roman{*})}

\let\stdsection\section
\renewcommand\section{\newpage\stdsection}

% Strike through
\def\st{\bgroup \ULdepth=-.55ex \ULset}

% Maths symbols
\newcommand{\abs}[1]{\left\lvert #1\right\rvert}
\newcommand\ad{\mathrm{ad}}
\newcommand\AND{\mathsf{AND}}
\newcommand\Art{\mathrm{Art}}
\newcommand{\Bilin}{\mathrm{Bilin}}
\newcommand{\bket}[1]{\left\lvert #1\right\rangle}
\newcommand{\B}{\mathcal{B}}
\newcommand{\bolds}[1]{{\bfseries #1}}
\newcommand{\brak}[1]{\left\langle #1 \right\rvert}
\newcommand{\braket}[2]{\left\langle #1\middle\vert #2 \right\rangle}
\newcommand{\bra}{\langle}
\newcommand{\cat}[1]{\mathsf{#1}}
\newcommand{\C}{\mathbb{C}}
\newcommand{\CP}{\mathbb{CP}}
\newcommand{\cU}{\mathcal{U}}
\newcommand{\Der}{\mathrm{Der}}
\newcommand{\D}{\mathrm{D}}
\newcommand{\dR}{\mathrm{dR}}
\newcommand{\E}{\mathbb{E}}
\newcommand{\F}{\mathbb{F}}
\newcommand{\Frob}{\mathrm{Frob}}
\newcommand{\GG}{\mathbb{G}}
\newcommand{\gl}{\mathfrak{gl}}
\newcommand{\GL}{\mathrm{GL}}
\newcommand{\G}{\mathcal{G}}
\newcommand{\Gr}{\mathrm{Gr}}
\newcommand{\haut}{\mathrm{ht}}
\newcommand{\Id}{\mathrm{Id}}
\newcommand{\ket}{\rangle}
\newcommand{\lie}[1]{\mathfrak{#1}}
\newcommand{\Mat}{\mathrm{Mat}}
\newcommand{\N}{\mathbb{N}}
\newcommand{\norm}[1]{\left\lVert #1\right\rVert}
\newcommand{\normalorder}[1]{\mathop{:}\nolimits\!#1\!\mathop{:}\nolimits}
\newcommand\NOT{\mathsf{NOT}}
\newcommand{\Oc}{\mathcal{O}}
\newcommand{\Or}{\mathrm{O}}
\newcommand\OR{\mathsf{OR}}
\newcommand{\ort}{\mathfrak{o}}
\newcommand{\PGL}{\mathrm{PGL}}
\newcommand{\ph}{\,\cdot\,}
\newcommand{\pr}{\mathrm{pr}}
\newcommand{\Prob}{\mathbb{P}}
\newcommand{\PSL}{\mathrm{PSL}}
\newcommand{\Ps}{\mathcal{P}}
\newcommand{\PSU}{\mathrm{PSU}}
\newcommand{\pt}{\mathrm{pt}}
\newcommand{\qeq}{\mathrel{``{=}"}}
\newcommand{\Q}{\mathbb{Q}}
\newcommand{\R}{\mathbb{R}}
\newcommand{\RP}{\mathbb{RP}}
\newcommand{\Rs}{\mathcal{R}}
\newcommand{\SL}{\mathrm{SL}}
\newcommand{\so}{\mathfrak{so}}
\newcommand{\SO}{\mathrm{SO}}
\newcommand{\Spin}{\mathrm{Spin}}
\newcommand{\Sp}{\mathrm{Sp}}
\newcommand{\su}{\mathfrak{su}}
\newcommand{\SU}{\mathrm{SU}}
\newcommand{\term}[1]{\emph{#1}\index{#1}}
\newcommand{\T}{\mathbb{T}}
\newcommand{\tv}[1]{|#1|}
\newcommand{\U}{\mathrm{U}}
\newcommand{\uu}{\mathfrak{u}}
\newcommand{\Vect}{\mathrm{Vect}}
\newcommand{\wsto}{\stackrel{\mathrm{w}^*}{\to}}
\newcommand{\wt}{\mathrm{wt}}
\newcommand{\wto}{\stackrel{\mathrm{w}}{\to}}
\newcommand{\Z}{\mathbb{Z}}
\renewcommand{\d}{\mathrm{d}}
\renewcommand{\H}{\mathbb{H}}
\renewcommand{\P}{\mathbb{P}}
\renewcommand{\sl}{\mathfrak{sl}}
\renewcommand{\vec}[1]{\boldsymbol{\mathbf{#1}}}
%\renewcommand{\F}{\mathcal{F}}

\let\Im\relax
\let\Re\relax

\DeclareMathOperator{\adj}{adj}
\DeclareMathOperator{\Ann}{Ann}
\DeclareMathOperator{\area}{area}
\DeclareMathOperator{\Aut}{Aut}
\DeclareMathOperator{\Bernoulli}{Bernoulli}
\DeclareMathOperator{\betaD}{beta}
\DeclareMathOperator{\bias}{bias}
\DeclareMathOperator{\binomial}{binomial}
\DeclareMathOperator{\card}{card}
\DeclareMathOperator{\ccl}{ccl}
\DeclareMathOperator{\Char}{char}
\DeclareMathOperator{\ch}{ch}
\DeclareMathOperator{\cl}{cl}
\DeclareMathOperator{\cls}{\overline{\mathrm{span}}}
\DeclareMathOperator{\conv}{conv}
\DeclareMathOperator{\corr}{corr}
\DeclareMathOperator{\cosec}{cosec}
\DeclareMathOperator{\cosech}{cosech}
\DeclareMathOperator{\cov}{cov}
\DeclareMathOperator{\covol}{covol}
\DeclareMathOperator{\diag}{diag}
\DeclareMathOperator{\diam}{diam}
\DeclareMathOperator{\Diff}{Diff}
\DeclareMathOperator{\disc}{disc}
\DeclareMathOperator{\dom}{dom}
\DeclareMathOperator{\End}{End}
\DeclareMathOperator{\energy}{energy}
\DeclareMathOperator{\erfc}{erfc}
\DeclareMathOperator{\erf}{erf}
\DeclareMathOperator*{\esssup}{ess\,sup}
\DeclareMathOperator{\ev}{ev}
\DeclareMathOperator{\Ext}{Ext}
\DeclareMathOperator{\Fit}{Fit}
\DeclareMathOperator{\fix}{fix}
\DeclareMathOperator{\Frac}{Frac}
\DeclareMathOperator{\Gal}{Gal}
\DeclareMathOperator{\gammaD}{gamma}
\DeclareMathOperator{\gr}{gr}
\DeclareMathOperator{\hcf}{hcf}
\DeclareMathOperator{\Hom}{Hom}
\DeclareMathOperator{\id}{id}
\DeclareMathOperator{\image}{image}
\DeclareMathOperator{\im}{im}
\DeclareMathOperator{\Im}{Im}
\DeclareMathOperator{\Ind}{Ind}
\DeclareMathOperator{\Int}{Int}
\DeclareMathOperator{\Isom}{Isom}
\DeclareMathOperator{\lcm}{lcm}
\DeclareMathOperator{\length}{length}
\DeclareMathOperator{\Lie}{Lie}
\DeclareMathOperator{\like}{like}
\DeclareMathOperator{\Lk}{Lk}
\DeclareMathOperator{\mse}{mse}
\DeclareMathOperator{\multinomial}{multinomial}
\DeclareMathOperator{\orb}{orb}
\DeclareMathOperator{\ord}{ord}
\DeclareMathOperator{\otp}{otp}
\DeclareMathOperator{\Poisson}{Poisson}
\DeclareMathOperator{\poly}{poly}
\DeclareMathOperator{\rank}{rank}
\DeclareMathOperator{\rel}{rel}
\DeclareMathOperator{\Re}{Re}
\DeclareMathOperator*{\res}{res}
\DeclareMathOperator{\Res}{Res}
\DeclareMathOperator{\rk}{rk}
\DeclareMathOperator{\Root}{Root}
\DeclareMathOperator{\sech}{sech}
\DeclareMathOperator{\sgn}{sgn}
\DeclareMathOperator{\spn}{span}
\DeclareMathOperator{\stab}{stab}
\DeclareMathOperator{\St}{St}
\DeclareMathOperator{\supp}{supp}
\DeclareMathOperator{\Syl}{Syl}
\DeclareMathOperator{\Sym}{Sym}
\DeclareMathOperator{\tr}{tr}
\DeclareMathOperator{\Tr}{Tr}
\DeclareMathOperator{\var}{var}
\DeclareMathOperator{\vol}{vol}

\pgfarrowsdeclarecombine{twolatex'}{twolatex'}{latex'}{latex'}{latex'}{latex'}
\tikzset{->/.style = {decoration={markings,
                                  mark=at position 1 with {\arrow[scale=2]{latex'}}},
                      postaction={decorate}}}
\tikzset{<-/.style = {decoration={markings,
                                  mark=at position 0 with {\arrowreversed[scale=2]{latex'}}},
                      postaction={decorate}}}
\tikzset{<->/.style = {decoration={markings,
                                   mark=at position 0 with {\arrowreversed[scale=2]{latex'}},
                                   mark=at position 1 with {\arrow[scale=2]{latex'}}},
                       postaction={decorate}}}
\tikzset{->-/.style = {decoration={markings,
                                   mark=at position #1 with {\arrow[scale=2]{latex'}}},
                       postaction={decorate}}}
\tikzset{-<-/.style = {decoration={markings,
                                   mark=at position #1 with {\arrowreversed[scale=2]{latex'}}},
                       postaction={decorate}}}
\tikzset{->>/.style = {decoration={markings,
                                  mark=at position 1 with {\arrow[scale=2]{latex'}}},
                      postaction={decorate}}}
\tikzset{<<-/.style = {decoration={markings,
                                  mark=at position 0 with {\arrowreversed[scale=2]{twolatex'}}},
                      postaction={decorate}}}
\tikzset{<<->>/.style = {decoration={markings,
                                   mark=at position 0 with {\arrowreversed[scale=2]{twolatex'}},
                                   mark=at position 1 with {\arrow[scale=2]{twolatex'}}},
                       postaction={decorate}}}
\tikzset{->>-/.style = {decoration={markings,
                                   mark=at position #1 with {\arrow[scale=2]{twolatex'}}},
                       postaction={decorate}}}
\tikzset{-<<-/.style = {decoration={markings,
                                   mark=at position #1 with {\arrowreversed[scale=2]{twolatex'}}},
                       postaction={decorate}}}

\tikzset{circ/.style = {fill, circle, inner sep = 0, minimum size = 3}}
\tikzset{mstate/.style={circle, draw, blue, text=black, minimum width=0.7cm}}

\tikzset{commutative diagrams/.cd,cdmap/.style={/tikz/column 1/.append style={anchor=base east},/tikz/column 2/.append style={anchor=base west},row sep=tiny}}

\definecolor{mblue}{rgb}{0.2, 0.3, 0.8}
\definecolor{morange}{rgb}{1, 0.5, 0}
\definecolor{mgreen}{rgb}{0.1, 0.4, 0.2}
\definecolor{mred}{rgb}{0.5, 0, 0}

\def\drawcirculararc(#1,#2)(#3,#4)(#5,#6){%
    \pgfmathsetmacro\cA{(#1*#1+#2*#2-#3*#3-#4*#4)/2}%
    \pgfmathsetmacro\cB{(#1*#1+#2*#2-#5*#5-#6*#6)/2}%
    \pgfmathsetmacro\cy{(\cB*(#1-#3)-\cA*(#1-#5))/%
                        ((#2-#6)*(#1-#3)-(#2-#4)*(#1-#5))}%
    \pgfmathsetmacro\cx{(\cA-\cy*(#2-#4))/(#1-#3)}%
    \pgfmathsetmacro\cr{sqrt((#1-\cx)*(#1-\cx)+(#2-\cy)*(#2-\cy))}%
    \pgfmathsetmacro\cA{atan2(#2-\cy,#1-\cx)}%
    \pgfmathsetmacro\cB{atan2(#6-\cy,#5-\cx)}%
    \pgfmathparse{\cB<\cA}%
    \ifnum\pgfmathresult=1
        \pgfmathsetmacro\cB{\cB+360}%
    \fi
    \draw (#1,#2) arc (\cA:\cB:\cr);%
}
\newcommand\getCoord[3]{\newdimen{#1}\newdimen{#2}\pgfextractx{#1}{\pgfpointanchor{#3}{center}}\pgfextracty{#2}{\pgfpointanchor{#3}{center}}}

\def\Xint#1{\mathchoice
   {\XXint\displaystyle\textstyle{#1}}%
   {\XXint\textstyle\scriptstyle{#1}}%
   {\XXint\scriptstyle\scriptscriptstyle{#1}}%
   {\XXint\scriptscriptstyle\scriptscriptstyle{#1}}%
   \!\int}
\def\XXint#1#2#3{{\setbox0=\hbox{$#1{#2#3}{\int}$}
     \vcenter{\hbox{$#2#3$}}\kern-.5\wd0}}
\def\ddashint{\Xint=}
\def\dashint{\Xint-}

\newcommand\separator{{\centering\rule{2cm}{0.2pt}\vspace{2pt}\par}}

\newenvironment{own}{\color{gray!70!black}}{}

\newcommand\makecenter[1]{\raisebox{-0.5\height}{#1}}
\newtheorem*{soln}{Solution}

\renewcommand{\thesection}{}
\renewcommand{\thesubsection}{\arabic{section}.\arabic{subsection}}
\makeatletter
\def\@seccntformat#1{\csname #1ignore\expandafter\endcsname\csname the#1\endcsname\quad}
\let\sectionignore\@gobbletwo
\let\latex@numberline\numberline
\def\numberline#1{\if\relax#1\relax\else\latex@numberline{#1}\fi}
\makeatother


\begin{document}
	
	\maketitle
	
	\section{QUESTION 1}
	
	As all of the following basis are of order $ n $, we need only check for linear independence (or spanning).
	
	\begin{enumerate}[label = (\alph*)]
		\item  \[ \alpha_{1} (\mathbf{e}_{1} + \mathbf{e}_{2}) + \alpha_{2} (\mathbf{e}_{2} + \mathbf{e}_{3}) + \cdots + \alpha_{n-1} ( \mathbf{e}_{n-1} + \mathbf{e}_{n}) + \alpha_{n} \mathbf{e}_{n} = \mathbf{0}  \]
		The first vector is the only one that contains $ \mathbf{e}_{1} $, so $ \alpha_{1} = 0 $. But then $ \alpha_{2} = 0, \cdots, \alpha_{n} = 0 $ so this set is linearly independent, and thus a basis.
		
		\item \[ \alpha_{1} (\mathbf{e}_{1} + \mathbf{e}_{2}) + \alpha_{2} (\mathbf{e}_{2} + \mathbf{e}_{3}) + \cdots + \alpha_{n-1} ( \mathbf{e}_{n-1} + \mathbf{e}_{n}) + \alpha_{n} (\mathbf{e}_{n} + \mathbf{e}_{1} ) = \mathbf{0}  \]
		
		Then $ \alpha_{2} = - \alpha_{1}, \alpha_{3} = \alpha_{1}, \cdots, \alpha_{n} = (-1)^{n+1}\alpha_{1} $. Thus for $ n $ even, it is possible to cancel out the $ \mathbf{e}_{1} $ and have linear dependence, but not when $ n $ is odd. Thus
		
		\[ \begin{cases} \text{basis }  & \text{ if } n \text{ odd } \\ \text{not a basis } &  \text{ if } n \text{ even} \end{cases} \]
		
		\item Vectors in this basis are of the form $ \mathbf{e}_{i} + (-1)^{i} \mathbf{e}_{n-i} $. If $ n $ is odd, say $ n = 2k + 1 $, setting 
		
		\begin{itemize}
			\item $ \alpha_{k+1} = 0 $ (middle coefficient), only vector containing $ \mathbf{e}_{k+1} $
			\item $ \alpha_{1} = - \alpha_{n}, \alpha_{2} = - \alpha_{n-1}, \cdots $ 
		\end{itemize}
		is enough to show linear dependence.
		
		If $ n $ is even, the first and last vector are $ \mathbf{e}_{1} - \mathbf{e}_{n} $ and $ \mathbf{e}_{1} + \mathbf{e}_{n} $, so these coefficients must both be set so zero. Likewise for $ \mathbf{e}_{2} - \mathbf{e}_{n-1} $ and $ \mathbf{e}_{2} + \mathbf{e}_{n-1} $,$ \cdots $ etc, all the coefficients are zero, thus linear independence, thus this set is a basis when $ n $ is even. 
		
	\end{enumerate}
	
	
	\section{QUESTION 2}
	
	\begin{enumerate}
	\item \begin{prop} 
		$ T \cup U $ is a subspace of $ V $ only if either $ T \leq U $ or $ U \leq T $
		
	\end{prop}

	\begin{proof}
			\begin{itemize}
			\item Choose $ v_{1} \in T \setminus U $, $ v_{2} \in U \setminus T  $
			\item As $ T \cup U $ is a subspace of $ V $.  $ v_{1},v_{2} \in T \cup U \Rightarrow v_{1} + v_{2} \in T \cup U $
			\item $ \Rightarrow v_{1} + v_{2} \in T$ or $ U $ 
			\item If $ v_{1} + v_{2} \in T $, then $ v_{2} \in T $. But we said $ v_{2} \in U \setminus T $. Contradiction.
			\item Hence $ U \setminus T $ is empty and $ U \leq T $. 
			\item Similarly, $ v_{1} + v_{2} \in U $ then $ T \leq U $
		\end{itemize} 
	\end{proof}

	\item 
	\begin{enumerate}[label = (\alph*)]
		\item Choose 
		
		\[ T = \left\{  \begin{pmatrix}
		x\\
		x\\
		\end{pmatrix} \subset \R^{2} \; | \; x \in \R \right\}, U = \left\{  \begin{pmatrix}
		x\\
		2x\\
		\end{pmatrix} \subset \R^{2} \; | \; x \in \R \right\} \]
		
		and 
		
		\[ W = \left\{  \begin{pmatrix}
		x\\
		3x\\
		\end{pmatrix} \subset \R^{2} \; | \; x \in \R \right\} \]
		
		
		Then LHS $ =  T + (U \cap W) = T + \mathbf{0} = T$, and RHS $ = (\R^{2}) \cap (\R^{2})  = \R^{2} $ 
		
		\item Choosing $ T, U $ and $ W $ as before,
		LHS $ = (\R^{2}) \cap W = W $, and RHS $ = \mathbf{0} + \mathbf{0} = \mathbf{0} $
		
	\end{enumerate}

	\item The counter examples suggest which way the inclusions are:
	
	\begin{prop} 
		$ T + (U \cap W) \subset (T + U) \cap ( T + W ) $
	\end{prop} 

\begin{proof}
	\begin{itemize}
		\item Let $ a + b \in T + (U \cap W) $
		\item $ a \in T $, $ b \in U \cap W $
		\item Then $ b \in U $ and $ b \in W $
		\item $ a \in T $, $ b \in U  \Rightarrow a + b \in (T + U) $
		\item $ a \in T, b \in W \Rightarrow a + b \in (T + W) $
		\item Thus $ a + b \in (T + U) \cap (T + W) $
		\end{itemize}
\end{proof}

\begin{prop} 
	$ (T + U) \cap W \supset (T \cap W) + (U \cap W) $
\end{prop}

\begin{proof}
	\begin{itemize}
		\item Similarly, let $ a + b \in $ RHS
		\item so $ a \in (T \cap W), b \in (U \cap W) $
		\item In particular, $ a \in T $, $ b \in U  \Rightarrow a + b \in (T + U)$
		\item And $ a \in W $, $ b \in W \Rightarrow a + b \in W + W = W $
		\item Thus $ a + b \in (T + U) \cap W $
	\end{itemize}
\end{proof}
		
	\end{enumerate}
	
	\section{QUESTION 3}
	
	Hint to show isomorphism: Guess an explicit inverse, compose both with right and left to get the identity
	
	\begin{enumerate}[label = (\alph*)]
		\item Let $ T : V \to W $ be defined by
		
		\[ T \begin{pmatrix}
		v_{1}\\
		v_{2}\\
		v_{3}\\
		v_{4}
		\end{pmatrix}  = \begin{pmatrix}
		v_{1}\\
		v_{2}\\
		v_{3}\\
		v_{4} \\
		-v_{1} - v_{2} - v_{3} - v_{4}
		\end{pmatrix}\]
		
		It is straightforward to see that $ T(\mathbf{x} + \mathbf{y}) = T(\mathbf{x}) + T(\mathbf{y}) $ and $ T(\alpha \mathbf{x} ) = \alpha T(x) $, thus $ T $ is linear.	
		
		To show it is one-to-one, consider the map $ T': W \to V $ defined by
		
		\[ T' \begin{pmatrix}
		w_{1}\\
		w_{2}\\
		w_{3}\\
		w_{4}\\
		w_{5}
		\end{pmatrix}  = \begin{pmatrix}
		w_{1}\\
		w_{2}\\
		w_{3}\\
		w_{4} \\
		\end{pmatrix} \]	
		
		Then $ T \circ T' = T' \circ T = \id $.
		
		\item Note that $ \{1,x,x^{2},x^{3},x^{4},x^{5}\} $ is a spanning set for $ W $. It is also linearly independent; suppose that
		
		\[ a_{0} + a_{1} x + a_{2} x^{2} + a_{3}x^{3} + a_{4}x^{4} + a_{5}x^{5} = \theta(x) \]
		
		where $ \theta(x) $ is the zero polynomial. If this holds for all values of $ x $, then (since $ \theta'(x) = \theta(x) $) we can differentiate both sides to obtain 
		
		\[ a_{1} + 2a_{2} x + 3 a_{3} x^{2} + 4 a_{4} x^{3} + 5 a_{5} x^{4} = \theta(x) \]
		
		Continuing differentiation in this fashion we arrive at
		
		\[ 5! a_{5} = \theta(x) \]
		
		And we must have $ a_{5} = 0 $. Going one diffentiation step back the previous equation insist $ a_{4} = 0 $, and so we have $ a_{i} = 0 $ for all $ i $, and thus  $ \{1,x,x^{2},x^{3},x^{4},x^{5}\} $ is linearly independent in $ W $. 
		
		Hence we have found a basis for $ W $ and conclude $ \dim W = 6 $. But $ \dim V = 5$, and therefore there can be no such isomorphism.
		
		
		\item Define $ T : W \to V $ as $ T(f(x)) \mapsto f(2x + 1) $:
		
		\begin{itemize}
			\item Linear: 
			
			\begin{align*}
			T(\lambda f_{1}(x) + \mu f_{2}(x)& = (\lambda f_{1} + \mu f_{2})(2x + 1) \\
			& = \lambda f_{1} (2x + 1) + \mu f_{2} (2x + 1)\\
			& = \lambda T(f_{1}(x)) + \mu T(f_{2}(x))
			\end{align*}
			
			\item Bijective: Define $ T':W \to V $ as $ T'(f(x)) = f(\frac{x-1}{2}) $ 
			
			Show that $ T \circ T' = T' \circ T = \id $
			
		\end{itemize}
		
		\item Define $ T : V \to W $ as $ T(f(x)) \mapsto \int^{x} f(t) \; \d t $
		
		\item A natural basis for $ W $ is $ \{ A,B \} $ where solutions are of the form $ A \cos t + B \sin t $. Hence define $ T : V \to W $ as $ T(v_{1},v_{2}) = v_{1} \cos t + v_{2} \sin t $.
		
		\item Suppose $ \varphi : R^{4} \to C[0,1] $ is an isomorphism. Let $ e_{1},e_{2},e_{3},e_{4} $ be a basis for $ \R^{4} $. Then 
		
		\[ \{ \varphi(e_{1}),\varphi(e_{2}),\varphi(e_{3}),\varphi(e_{4}) \} \]
		
		is a basis for $ C[0,1] $.
		
		In particular, we have a spanning set of size 4. But, eg. $ \{1,x,x^{2},x^{3},x^{4},x^{5}\} $ is a linearly independent set of size 5. This is a contradiction (by Steinitz)
		
		\item Suppose $ \phi : \mathcal{P} \to \R^{\N} $ is an isomorphism, with $ \phi $ having the natural basis $ \{  1,x,x^{2},\cdots,x^{N}\} $. Then

		\[ \underbrace{\{ \phi(1),\phi(x),\cdots,\phi(x^{N}) \}} \]
		
		is a countable basis for $ \R^{\N} $. But, $ \R^{\N} $ has no countable basis, no $ \phi $ cannot be an isomorphism. 
		
		
	\end{enumerate}
		
	\section{QUESTION 4}
	
\begin{enumerate}
	\item 	Let $ \alpha,\beta $ be linear maps from $ U $ to $ V $. Then
	
	\begin{align*}
	(\alpha + \beta)(v_{1} + v_{2})& = \alpha(v_{1}+v_{2}) + \beta(v_{1} + v_{2}) \\
	& = \alpha(v_{1}) + \alpha(v_{2}) + \beta(v_{1}) + \beta(v_{2})\\
	& = (\alpha + \beta)(v_{1}) + (\alpha + \beta)(v_{2})
 	\end{align*}
 	
 	and 
 	
 	\begin{align*}
 	(\alpha + \beta)(\lambda v)& = \alpha(\lambda v) + \beta(\lambda v) \\
 	& = \lambda \alpha(v) + \lambda \beta(v) \\
 	& = \lambda(\alpha + \beta)(v)
 	\end{align*}
 	
 	Thus $ \alpha + \beta $ is also a linear map
 	
	 	\begin{enumerate}[label = (\alph*)]
	 		\item Let $ \alpha, \beta: V \to V $ st. $ \alpha = \id $, $ \beta = - \alpha $.
	 		
	 		Then $ \Im(\alpha + \beta) = 0 $, $ \Im(\alpha) = V $, $ \Im(\beta) = V $.
	 		
	 		\[ \Im (\alpha + \beta) \neq \Im \alpha + \Im \beta  \]
	 		
	 		\item Using the same maps, $ \ker (\alpha + \beta) = V $, $ \ker \alpha = \mathbf{0}  $ and $ \ker \beta = \mathbf{0} $, hence 
	 		
	 		\[ \ker( \alpha + \beta) \neq \ker \alpha \cap \ker \beta \]
	 		
	 		\end{enumerate}
 		
	 		\begin{prop} 
	 			\[ \Im (\alpha + \beta) \subset \Im \alpha + \Im \beta  \]
	 		\end{prop}
 		
 		\begin{proof}
 			
 			Suppose $ v \in  $ LHS, that is
 			
 			\begin{align*}
 			 v & \in \{  v \in V \; | \; v  = (\alpha + \beta)(u), \text{ some } u \in U      \} \\
 			&  = \{  v \in V \; | \; v = \alpha(u) + \beta(u), \text{ some } u \in U      \}\\
 			&  \subset \{  v \in V \; | \; v = \alpha(u), \text{ some } u \in U  \} + \{  v \in V \; | \; v = \beta(u), \text{ some } u \in U      \} \\
 			& = \Im \alpha + \Im \beta
 			\end{align*}
 			
 			Hence $ v \in  $ RHS
 			
 		\end{proof}
 	
 	\begin{prop} 
 	\[ 	\ker( \alpha + \beta) \supset \ker \alpha \cap \ker \beta \]
 	\end{prop}
 	
 	\begin{proof}
 		Suppose $ u \in $ RHS 
 	\end{proof}
 	
 	\begin{proof}
 		Let $ u \in $ RHS, ie
 		
 		\begin{align*}
 		u & \in \{ u \in U \; | \; \alpha(u) = \mathbf{0} \} \cap \{ u \in U \; | \; \beta(u) = \mathbf{0} \}  \\
 		& = \{ u \in U \; | \; \alpha(u) = \beta(u) = \mathbf{0} \}\\
 		& \subset \{ u \in U \; | \; \alpha(u) + \beta(u) = \mathbf{0} \}\\
 		& = \ker (\alpha + \beta)
 		\end{align*}
 	\end{proof}
 
 
 \item (Might be helpful to think of $ \alpha $ geometrically as a projection).
 We want to prove that if $ \alpha^{2} = \alpha $, then
 
 \begin{itemize}
 	\item $ \Im \alpha  \cap \ker \alpha = \{ \mathbf{0} \}$
 	\item $ \Im \alpha + \ker \alpha = V $
 \end{itemize}

\begin{proof}
	\begin{itemize}
		\item Given $ v \in \Im \alpha  \cap \ker \alpha $, there exists some $ w $ st. $ v = \alpha(w) $. So
		
		\begin{align*}
		v & = \alpha(w) \\
		& = \alpha^{2}(w) \\
		& = \alpha(\alpha(w)) \\
		& = \alpha(v) \qquad \in \ker \alpha\\
		& = \mathbf{0}
		\end{align*}
		
		\item Given $ v \in V $, then
		
		\[ v = \underbrace{\alpha(v)}_{\in \Im \alpha} + \underbrace{(v - \alpha(v))}_{\in \ker \alpha}  \]
		
		since 
		
		\begin{align*}
		\alpha(v-\alpha(v)) & = \alpha(v) - \alpha^{2}(v) \\
		& = \alpha(v) - \alpha(v) \\
		& = \mathbf{0}
		\end{align*}
		
		So $ V = \ker \oplus \im \alpha $
		
	\end{itemize}
\end{proof}
	
\end{enumerate}
 	
 	
	 
	\section{QUESTION 5}
	
	$ U \cap W = \{ \mathbf{x} \in \R^{5} : x_{1} + 2x_{2} =0, x_{2} = x_{3} = x_{4}, x_{1} + x_{5} = 0 \} $ by combining the conditions on $ U $ and $ W $. Vectors in $ U $, $ W $ and $ U \cap W $ respectively have the form:
	
	\[ \begin{pmatrix}
	x_{1} \\
	x_{2} \\
	x_{3} \\
	- x_{1} - x_{3} \\
	- \frac{1}{2} (x_{1} + x_{2})
	\end{pmatrix}, \qquad \begin{pmatrix}
	x_{1} \\
	x_{2} \\
	x_{2} \\
	x_{2} \\
	- x_{1} \\
	\end{pmatrix} 
	\quad \text{and}  \quad 
	\begin{pmatrix}
	-2x \\
	x \\
	x \\
	x \\
	2x
	\end{pmatrix} \]
	
	Thus a natural basis for $ U \cap W $ is 
	
	\[ \left\{  \begin{pmatrix}
	-2 \\
	1 \\
	1 \\
	1 \\
	2
	\end{pmatrix} \right\} \]
	
	Basis for $ U $, $ W $:
	
	\[ \left\{  
	\begin{pmatrix}
	1\\
	0\\
	0\\
	-1\\
	-\frac{1}{2}
	\end{pmatrix}, 
	\begin{pmatrix}
	0\\
	2\\
	0\\
	0\\
	-\frac{1}{2}
	\end{pmatrix},\\
	\begin{pmatrix}
	0\\
	0\\
	1\\
	-1\\
	0
	\end{pmatrix}\right\} \qquad 	
	 \left\{  \begin{pmatrix}
	0 \\
	1 \\
	1 \\
	1 \\
	0
	\end{pmatrix}, \begin{pmatrix}
	1 \\
	0 \\
	0 \\
	0 \\
	-1
	\end{pmatrix}\right\} \]
	
	
	Now add the vector to each of these basis and perform Gaussian elimination. Or, note that we can switch it for the first vector in $ U $ and the second vector in $ W $, as the first component is non-zero. Thus
	the required basis for $ U $, $ W $ are:
	
	\[ \left\{  
	 \begin{pmatrix}
	-2 \\
	1 \\
	1 \\
	1 \\
	2
	\end{pmatrix} , 
	\begin{pmatrix}
	0\\
	2\\
	0\\
	0\\
	-\frac{1}{2}
	\end{pmatrix},\\
	\begin{pmatrix}
	0\\
	0\\
	1\\
	-1\\
	0
	\end{pmatrix}\right\} \qquad 	
	\left\{  \begin{pmatrix}
	0 \\
	1 \\
	1 \\
	1 \\
	0
	\end{pmatrix},  \begin{pmatrix}
	-2 \\
	1 \\
	1 \\
	1 \\
	2
	\end{pmatrix} \right\}  \]
	
	Now the basis for $ U + W $ is just $ \text{basis for }U \cup \text{basis for }V $, provided the basis for $ U \cap W $ is a subset of both. 
	
	So a basis for $ U + W $ is 
	
	\[  \left\{  
	\begin{pmatrix}
	-2 \\
	1 \\
	1 \\
	1 \\
	2
	\end{pmatrix} , 
	\begin{pmatrix}
	0\\
	2\\
	0\\
	0\\
	-\frac{1}{2}
	\end{pmatrix},\\
	\begin{pmatrix}
	0\\
	0\\
	1\\
	-1\\
	0
	\end{pmatrix},
	\begin{pmatrix}
	0 \\
	1 \\
	1 \\
	1 \\
	0
	\end{pmatrix} \right\}  \]
	
	
	\section{QUESTION 6}
	
	
Let $ \alpha : V \to V $ linear, and let $ v_{1} = \alpha(u_{1}) $, $ v_{2} = \alpha(v_{2}) $

From the first isomorphism theorem we have $ \Im(\alpha) \leq V $,
$ \ker	(\alpha) \leq V $

\begin{prop} 
	$ \ker (\alpha) \leq V $
\end{prop}

\begin{proof}
	\begin{itemize}
		\item $ \mathbf{0} \in \ker \alpha $
		\item Let $ v_{1},v_{2} \in \ker \alpha $. Then
		
		\begin{align*}
		\alpha( \lambda v_{1} + \mu v_{2}) & = \lambda \alpha(v_{1}) + \mu \alpha (v_{2}) \\
		& = \mathbf{0} + \mathbf{0} = \mathbf{0} \qquad \text{ as } v_{1},v_{2} \in \ker \alpha
		\end{align*}
		
		Hence $ \lambda v_{1} + \mu v_{2} \in \ker \alpha $
	\end{itemize}
\end{proof}

\begin{align*}
\Im (\alpha^{k+1}) & = \{  v \in V \; | \; \alpha^{k+1}(u) \in V, \text{some } u \in V \} \\
& = \{  v \in V \; | \; \alpha^{k}(\alpha(u)) \in V, \text{some } u \in V \}\\
& \subseteq \{  v \in V \; | \; \alpha^{k}(v) \in V, \text{some } v \in V \}  \qquad \text{ as $ \Im(\alpha) \leq V $ } \\
& = \Im(\alpha^{k})
\end{align*}

Hence

\[ V \geq \Im (\alpha) \geq \Im (\alpha^{2}) \geq \cdots \]

Next, $ \alpha(\mathbf{0}) = \mathbf{0} $, so trivially $ \{ 0 \} \leq \ker(\alpha) $, and

\begin{align*}
\ker (\alpha^{k+1}) & = \{  v \in V \; | \; \alpha^{k+1}(v) =0\} \\
& = \{  v \in V \; | \; \alpha^{k}(\alpha(v)) =0 \}\\
& \subseteq \{  v \in V \; | \; \alpha^{k}(v) =0 \}  \qquad \text{ as $ \ker(\alpha) \leq V $ } \\
& = \ker(\alpha^{k})
\end{align*}

It now follows that

\[ \{ \mathbf{0} \} \leq \ker \alpha \leq \ker \alpha^{2} \leq \cdots \]

Next, taking dim of the first inequality gives

\[ \dim V \geq r_{1} \geq r_{2} \geq \cdots  \]
	
Thus $ r_{k} \geq r_{k+1} $. Similarly for $ n_{k} = n(\alpha^{k}) $, we have $ n_{k} \leq n_{k+1} $.

Let $ \widetilde{\alpha}_{k} : \Im \alpha_{k} \to V $ be defined by $ v \mapsto \alpha(v) $. Note that $ \Im(\widetilde{\alpha}_{k}) = \Im(\alpha^{k+1}) $

Applying R-N to $ \widetilde{\alpha}_{k} $,

\[ \dim (\Im(\alpha^{k})) = r(\widetilde{\alpha}_{k}) + n(\widetilde{\alpha}_{k}) \]
So
\[ n_{k+1} = r_{k} - r_{k+1} \]




\section{QUESTION 7}

With respect to the standard basis, $ \alpha $ is represented by the matrix $ A $, where

\[ A =  \begin{pmatrix}
2 & 1 & 0\\
0 & 2 & 1\\
0 & 0 & 2
\end{pmatrix} \]

Change of basis matrix $ P $ and it's inverse are given by

\[ P = \begin{pmatrix}
1 & 1 & 1 \\
1 & 1 & 0\\
1 & 0 & 0
\end{pmatrix}, \quad P^{-1} = \begin{pmatrix}
0 & 0 & 1 \\
0 & 1 & -1 \\
1 & -1 & 0
\end{pmatrix} \]

So the matrix $ \tilde{A} $ representing the linear map with respect to the new basis is given by

\begin{align*}
\tilde{A}  & =  P^{-1} A P \\
& = \begin{pmatrix}
2 & 0 & 0\\
1 & 2 & 0 \\
0 & 1 & 2
\end{pmatrix}
\end{align*}

EASIER:
Given the basis 

\[ \{  \begin{pmatrix}
1\\
1\\
1
\end{pmatrix},
\begin{pmatrix}
1\\
1\\
0
\end{pmatrix},
\begin{pmatrix}
1\\
0\\
0
\end{pmatrix} \} \]

for the domain, and the same one for the range, $ \alpha $ maps

\[ \alpha \begin{pmatrix}
1\\
1\\
1
\end{pmatrix} = \begin{pmatrix}
3\\
3\\
2
\end{pmatrix} = 2 \begin{pmatrix}
1\\
1\\
1
\end{pmatrix} 
+ 1 \begin{pmatrix}
1\\
1\\
0
\end{pmatrix}
+ 
\begin{pmatrix}
1\\
0\\
0
\end{pmatrix} \]

So the first column of $ A $ is $ \begin{pmatrix}
2\\
1\\
0
\end{pmatrix} $, etc. 



		
\section{QUESTION 8}

(i) $ \Rightarrow $ (iii) Let $ B  = B_{1} \cup \cdots \cup B_{n}$. Basis for $ \sum_{i}  U_{i } $, write $ u_{i} $ in terms of $ B_{i} $, Then $ u_{1} +\cdots +  u_{n} $ is a lin comb. of elements of $ B $.

$ B $ indep? \[ \sum_{v \in B} \lambda_{v} v = \mathbf{0} = \mathbf{0}_{U_{1}} + \cdots + \mathbf{0}_{U_{n}} \]

\[ \underbrace{\sum_{v \in B_{1}} \lambda_{v} v}_{\in U_{1}} + \cdots + \sum_{v \in B_{n}} \lambda_{v} v \]

%under, in U, W

By uniqueness of expressions, 

\[ \sum_{v \in B_{1}} \lambda_{v} v = \mathbf{0}_{U_{1}} \cdots \sum_{v \in B_{n}} \lambda_{v} v = \mathbf{0}_{U_{n}} \]

As $ B_{1}, \cdots, B_{n} $ are basis, all of the $ \lambda_{v} $ are zero. \\


(iii) $ \Rightarrow $ (ii). 

Given $ v \in U_{j} \cap \sum_{i\neq j} U_{i} $

Since $ v \in U_{j} $, can write

\[ v = \sum_{b_{i} \in B_{i}} \lambda_{i} b_{i} \]

Since $ v \in \sum_{i\neq j} U_{i} $, can write

\[ v = \sum_{b_{i} \in \cup_{k \neq j}B_{k}} \mu_{i} b_{i} \]

No $ b_{i} $'s in common because of the pairwise disjointness of the $ B_{i} $.

But $ \cup_{k} B_{k} $ is a basis, so by uniqueness of expression, 

$ \lambda_{i} = \mu_{i} = 0 $ for all $ i $.

So $ v = \mathbf{0} $.

(ii) $ \Rightarrow $ (i)

Suppose  that

\[ \sum u_{i} = \sum u_{i}' \]

Then for each $ j $,

\[ u_{j} - u_{j}'  = \sum_{i \neq j} (u_{i}' - u_{i} )  \in U_{j} \cap \sum_{i \neq j} U_{i} = \mathbf{0}  \]

So $ u_{j} = u_{j}' $, for all $ j $. Thus uniqueness of expression

	
	
	
	
	
	
\end{document}