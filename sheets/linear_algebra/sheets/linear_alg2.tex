\documentclass[a4paper]{article}
\usepackage{amsmath}
\def\npart {IB}
\def\nterm {Michaelmas}
\def\nyear {2017}
\def\nlecturer {Mr Rawlinson ( jir25@cam.ac.uk )}
\def\ncourse {Linear Algebra Sheet 2}

% Imports
\ifx \nauthor\undefined
  \def\nauthor{Christopher Turnbull}
\else
\fi

\author{Supervised by \nlecturer \\\small Solutions presented by \nauthor}
\date{\nterm\ \nyear}

\usepackage{alltt}
\usepackage{amsfonts}
\usepackage{amsmath}
\usepackage{amssymb}
\usepackage{amsthm}
\usepackage{booktabs}
\usepackage{caption}
\usepackage{enumitem}
\usepackage{fancyhdr}
\usepackage{graphicx}
\usepackage{mathdots}
\usepackage{mathtools}
\usepackage{microtype}
\usepackage{multirow}
\usepackage{pdflscape}
\usepackage{pgfplots}
\usepackage{siunitx}
\usepackage{slashed}
\usepackage{tabularx}
\usepackage{tikz}
\usepackage{tkz-euclide}
\usepackage[normalem]{ulem}
\usepackage[all]{xy}
\usepackage{imakeidx}

\makeindex[intoc, title=Index]
\indexsetup{othercode={\lhead{\emph{Index}}}}

\ifx \nextra \undefined
  \usepackage[pdftex,
    hidelinks,
    pdfauthor={Christopher Turnbull},
    pdfsubject={Cambridge Maths Notes: Part \npart\ - \ncourse},
    pdftitle={Part \npart\ - \ncourse},
  pdfkeywords={Cambridge Mathematics Maths Math \npart\ \nterm\ \nyear\ \ncourse}]{hyperref}
  \title{Part \npart\ --- \ncourse}
\else
  \usepackage[pdftex,
    hidelinks,
    pdfauthor={Christopher Turnbull},
    pdfsubject={Cambridge Maths Notes: Part \npart\ - \ncourse\ (\nextra)},
    pdftitle={Part \npart\ - \ncourse\ (\nextra)},
  pdfkeywords={Cambridge Mathematics Maths Math \npart\ \nterm\ \nyear\ \ncourse\ \nextra}]{hyperref}

  \title{Part \npart\ --- \ncourse \\ {\Large \nextra}}
  \renewcommand\printindex{}
\fi

\pgfplotsset{compat=1.12}

\pagestyle{fancyplain}
\lhead{\emph{\nouppercase{\leftmark}}}
\ifx \nextra \undefined
  \rhead{
    \ifnum\thepage=1
    \else
      \npart\ \ncourse
    \fi}
\else
  \rhead{
    \ifnum\thepage=1
    \else
      \npart\ \ncourse\ (\nextra)
    \fi}
\fi
\usetikzlibrary{arrows.meta}
\usetikzlibrary{decorations.markings}
\usetikzlibrary{decorations.pathmorphing}
\usetikzlibrary{positioning}
\usetikzlibrary{fadings}
\usetikzlibrary{intersections}
\usetikzlibrary{cd}

\newcommand*{\Cdot}{{\raisebox{-0.25ex}{\scalebox{1.5}{$\cdot$}}}}
\newcommand {\pd}[2][ ]{
  \ifx #1 { }
    \frac{\partial}{\partial #2}
  \else
    \frac{\partial^{#1}}{\partial #2^{#1}}
  \fi
}
\ifx \nhtml \undefined
\else
  \renewcommand\printindex{}
  \makeatletter
  \DisableLigatures[f]{family = *}
  \let\Contentsline\contentsline
  \renewcommand\contentsline[3]{\Contentsline{#1}{#2}{}}
  \renewcommand{\@dotsep}{10000}
  \newlength\currentparindent
  \setlength\currentparindent\parindent

  \newcommand\@minipagerestore{\setlength{\parindent}{\currentparindent}}
  \usepackage[active,tightpage,pdftex]{preview}
  \renewcommand{\PreviewBorder}{0.1cm}

  \newenvironment{stretchpage}%
  {\begin{preview}\begin{minipage}{\hsize}}%
    {\end{minipage}\end{preview}}
  \AtBeginDocument{\begin{stretchpage}}
  \AtEndDocument{\end{stretchpage}}

  \newcommand{\@@newpage}{\end{stretchpage}\begin{stretchpage}}

  \let\@real@section\section
  \renewcommand{\section}{\@@newpage\@real@section}
  \let\@real@subsection\subsection
  \renewcommand{\subsection}{\@@newpage\@real@subsection}
  \makeatother
\fi

% Theorems
\theoremstyle{definition}
\newtheorem*{aim}{Aim}
\newtheorem*{axiom}{Axiom}
\newtheorem*{claim}{Claim}
\newtheorem*{cor}{Corollary}
\newtheorem*{conjecture}{Conjecture}
\newtheorem*{defi}{Definition}
\newtheorem*{eg}{Example}
\newtheorem*{ex}{Exercise}
\newtheorem*{fact}{Fact}
\newtheorem*{law}{Law}
\newtheorem*{lemma}{Lemma}
\newtheorem*{notation}{Notation}
\newtheorem*{prop}{Proposition}
\newtheorem*{soln}{Solution}
\newtheorem*{thm}{Theorem}

\newtheorem*{remark}{Remark}
\newtheorem*{warning}{Warning}
\newtheorem*{exercise}{Exercise}

\newtheorem{nthm}{Theorem}[section]
\newtheorem{nlemma}[nthm]{Lemma}
\newtheorem{nprop}[nthm]{Proposition}
\newtheorem{ncor}[nthm]{Corollary}


\renewcommand{\labelitemi}{--}
\renewcommand{\labelitemii}{$\circ$}
\renewcommand{\labelenumi}{(\roman{*})}

\let\stdsection\section
\renewcommand\section{\newpage\stdsection}

% Strike through
\def\st{\bgroup \ULdepth=-.55ex \ULset}

% Maths symbols
\newcommand{\abs}[1]{\left\lvert #1\right\rvert}
\newcommand\ad{\mathrm{ad}}
\newcommand\AND{\mathsf{AND}}
\newcommand\Art{\mathrm{Art}}
\newcommand{\Bilin}{\mathrm{Bilin}}
\newcommand{\bket}[1]{\left\lvert #1\right\rangle}
\newcommand{\B}{\mathcal{B}}
\newcommand{\bolds}[1]{{\bfseries #1}}
\newcommand{\brak}[1]{\left\langle #1 \right\rvert}
\newcommand{\braket}[2]{\left\langle #1\middle\vert #2 \right\rangle}
\newcommand{\bra}{\langle}
\newcommand{\cat}[1]{\mathsf{#1}}
\newcommand{\C}{\mathbb{C}}
\newcommand{\CP}{\mathbb{CP}}
\newcommand{\cU}{\mathcal{U}}
\newcommand{\Der}{\mathrm{Der}}
\newcommand{\D}{\mathrm{D}}
\newcommand{\dR}{\mathrm{dR}}
\newcommand{\E}{\mathbb{E}}
\newcommand{\F}{\mathbb{F}}
\newcommand{\Frob}{\mathrm{Frob}}
\newcommand{\GG}{\mathbb{G}}
\newcommand{\gl}{\mathfrak{gl}}
\newcommand{\GL}{\mathrm{GL}}
\newcommand{\G}{\mathcal{G}}
\newcommand{\Gr}{\mathrm{Gr}}
\newcommand{\haut}{\mathrm{ht}}
\newcommand{\Id}{\mathrm{Id}}
\newcommand{\ket}{\rangle}
\newcommand{\lie}[1]{\mathfrak{#1}}
\newcommand{\Mat}{\mathrm{Mat}}
\newcommand{\N}{\mathbb{N}}
\newcommand{\norm}[1]{\left\lVert #1\right\rVert}
\newcommand{\normalorder}[1]{\mathop{:}\nolimits\!#1\!\mathop{:}\nolimits}
\newcommand\NOT{\mathsf{NOT}}
\newcommand{\Oc}{\mathcal{O}}
\newcommand{\Or}{\mathrm{O}}
\newcommand\OR{\mathsf{OR}}
\newcommand{\ort}{\mathfrak{o}}
\newcommand{\PGL}{\mathrm{PGL}}
\newcommand{\ph}{\,\cdot\,}
\newcommand{\pr}{\mathrm{pr}}
\newcommand{\Prob}{\mathbb{P}}
\newcommand{\PSL}{\mathrm{PSL}}
\newcommand{\Ps}{\mathcal{P}}
\newcommand{\PSU}{\mathrm{PSU}}
\newcommand{\pt}{\mathrm{pt}}
\newcommand{\qeq}{\mathrel{``{=}"}}
\newcommand{\Q}{\mathbb{Q}}
\newcommand{\R}{\mathbb{R}}
\newcommand{\RP}{\mathbb{RP}}
\newcommand{\Rs}{\mathcal{R}}
\newcommand{\SL}{\mathrm{SL}}
\newcommand{\so}{\mathfrak{so}}
\newcommand{\SO}{\mathrm{SO}}
\newcommand{\Spin}{\mathrm{Spin}}
\newcommand{\Sp}{\mathrm{Sp}}
\newcommand{\su}{\mathfrak{su}}
\newcommand{\SU}{\mathrm{SU}}
\newcommand{\term}[1]{\emph{#1}\index{#1}}
\newcommand{\T}{\mathbb{T}}
\newcommand{\tv}[1]{|#1|}
\newcommand{\U}{\mathrm{U}}
\newcommand{\uu}{\mathfrak{u}}
\newcommand{\Vect}{\mathrm{Vect}}
\newcommand{\wsto}{\stackrel{\mathrm{w}^*}{\to}}
\newcommand{\wt}{\mathrm{wt}}
\newcommand{\wto}{\stackrel{\mathrm{w}}{\to}}
\newcommand{\Z}{\mathbb{Z}}
\renewcommand{\d}{\mathrm{d}}
\renewcommand{\H}{\mathbb{H}}
\renewcommand{\P}{\mathbb{P}}
\renewcommand{\sl}{\mathfrak{sl}}
\renewcommand{\vec}[1]{\boldsymbol{\mathbf{#1}}}
%\renewcommand{\F}{\mathcal{F}}

\let\Im\relax
\let\Re\relax

\DeclareMathOperator{\adj}{adj}
\DeclareMathOperator{\Ann}{Ann}
\DeclareMathOperator{\area}{area}
\DeclareMathOperator{\Aut}{Aut}
\DeclareMathOperator{\Bernoulli}{Bernoulli}
\DeclareMathOperator{\betaD}{beta}
\DeclareMathOperator{\bias}{bias}
\DeclareMathOperator{\binomial}{binomial}
\DeclareMathOperator{\card}{card}
\DeclareMathOperator{\ccl}{ccl}
\DeclareMathOperator{\Char}{char}
\DeclareMathOperator{\ch}{ch}
\DeclareMathOperator{\cl}{cl}
\DeclareMathOperator{\cls}{\overline{\mathrm{span}}}
\DeclareMathOperator{\conv}{conv}
\DeclareMathOperator{\corr}{corr}
\DeclareMathOperator{\cosec}{cosec}
\DeclareMathOperator{\cosech}{cosech}
\DeclareMathOperator{\cov}{cov}
\DeclareMathOperator{\covol}{covol}
\DeclareMathOperator{\diag}{diag}
\DeclareMathOperator{\diam}{diam}
\DeclareMathOperator{\Diff}{Diff}
\DeclareMathOperator{\disc}{disc}
\DeclareMathOperator{\dom}{dom}
\DeclareMathOperator{\End}{End}
\DeclareMathOperator{\energy}{energy}
\DeclareMathOperator{\erfc}{erfc}
\DeclareMathOperator{\erf}{erf}
\DeclareMathOperator*{\esssup}{ess\,sup}
\DeclareMathOperator{\ev}{ev}
\DeclareMathOperator{\Ext}{Ext}
\DeclareMathOperator{\Fit}{Fit}
\DeclareMathOperator{\fix}{fix}
\DeclareMathOperator{\Frac}{Frac}
\DeclareMathOperator{\Gal}{Gal}
\DeclareMathOperator{\gammaD}{gamma}
\DeclareMathOperator{\gr}{gr}
\DeclareMathOperator{\hcf}{hcf}
\DeclareMathOperator{\Hom}{Hom}
\DeclareMathOperator{\id}{id}
\DeclareMathOperator{\image}{image}
\DeclareMathOperator{\im}{im}
\DeclareMathOperator{\Im}{Im}
\DeclareMathOperator{\Ind}{Ind}
\DeclareMathOperator{\Int}{Int}
\DeclareMathOperator{\Isom}{Isom}
\DeclareMathOperator{\lcm}{lcm}
\DeclareMathOperator{\length}{length}
\DeclareMathOperator{\Lie}{Lie}
\DeclareMathOperator{\like}{like}
\DeclareMathOperator{\Lk}{Lk}
\DeclareMathOperator{\mse}{mse}
\DeclareMathOperator{\multinomial}{multinomial}
\DeclareMathOperator{\orb}{orb}
\DeclareMathOperator{\ord}{ord}
\DeclareMathOperator{\otp}{otp}
\DeclareMathOperator{\Poisson}{Poisson}
\DeclareMathOperator{\poly}{poly}
\DeclareMathOperator{\rank}{rank}
\DeclareMathOperator{\rel}{rel}
\DeclareMathOperator{\Re}{Re}
\DeclareMathOperator*{\res}{res}
\DeclareMathOperator{\Res}{Res}
\DeclareMathOperator{\rk}{rk}
\DeclareMathOperator{\Root}{Root}
\DeclareMathOperator{\sech}{sech}
\DeclareMathOperator{\sgn}{sgn}
\DeclareMathOperator{\spn}{span}
\DeclareMathOperator{\stab}{stab}
\DeclareMathOperator{\St}{St}
\DeclareMathOperator{\supp}{supp}
\DeclareMathOperator{\Syl}{Syl}
\DeclareMathOperator{\Sym}{Sym}
\DeclareMathOperator{\tr}{tr}
\DeclareMathOperator{\Tr}{Tr}
\DeclareMathOperator{\var}{var}
\DeclareMathOperator{\vol}{vol}

\pgfarrowsdeclarecombine{twolatex'}{twolatex'}{latex'}{latex'}{latex'}{latex'}
\tikzset{->/.style = {decoration={markings,
                                  mark=at position 1 with {\arrow[scale=2]{latex'}}},
                      postaction={decorate}}}
\tikzset{<-/.style = {decoration={markings,
                                  mark=at position 0 with {\arrowreversed[scale=2]{latex'}}},
                      postaction={decorate}}}
\tikzset{<->/.style = {decoration={markings,
                                   mark=at position 0 with {\arrowreversed[scale=2]{latex'}},
                                   mark=at position 1 with {\arrow[scale=2]{latex'}}},
                       postaction={decorate}}}
\tikzset{->-/.style = {decoration={markings,
                                   mark=at position #1 with {\arrow[scale=2]{latex'}}},
                       postaction={decorate}}}
\tikzset{-<-/.style = {decoration={markings,
                                   mark=at position #1 with {\arrowreversed[scale=2]{latex'}}},
                       postaction={decorate}}}
\tikzset{->>/.style = {decoration={markings,
                                  mark=at position 1 with {\arrow[scale=2]{latex'}}},
                      postaction={decorate}}}
\tikzset{<<-/.style = {decoration={markings,
                                  mark=at position 0 with {\arrowreversed[scale=2]{twolatex'}}},
                      postaction={decorate}}}
\tikzset{<<->>/.style = {decoration={markings,
                                   mark=at position 0 with {\arrowreversed[scale=2]{twolatex'}},
                                   mark=at position 1 with {\arrow[scale=2]{twolatex'}}},
                       postaction={decorate}}}
\tikzset{->>-/.style = {decoration={markings,
                                   mark=at position #1 with {\arrow[scale=2]{twolatex'}}},
                       postaction={decorate}}}
\tikzset{-<<-/.style = {decoration={markings,
                                   mark=at position #1 with {\arrowreversed[scale=2]{twolatex'}}},
                       postaction={decorate}}}

\tikzset{circ/.style = {fill, circle, inner sep = 0, minimum size = 3}}
\tikzset{mstate/.style={circle, draw, blue, text=black, minimum width=0.7cm}}

\tikzset{commutative diagrams/.cd,cdmap/.style={/tikz/column 1/.append style={anchor=base east},/tikz/column 2/.append style={anchor=base west},row sep=tiny}}

\definecolor{mblue}{rgb}{0.2, 0.3, 0.8}
\definecolor{morange}{rgb}{1, 0.5, 0}
\definecolor{mgreen}{rgb}{0.1, 0.4, 0.2}
\definecolor{mred}{rgb}{0.5, 0, 0}

\def\drawcirculararc(#1,#2)(#3,#4)(#5,#6){%
    \pgfmathsetmacro\cA{(#1*#1+#2*#2-#3*#3-#4*#4)/2}%
    \pgfmathsetmacro\cB{(#1*#1+#2*#2-#5*#5-#6*#6)/2}%
    \pgfmathsetmacro\cy{(\cB*(#1-#3)-\cA*(#1-#5))/%
                        ((#2-#6)*(#1-#3)-(#2-#4)*(#1-#5))}%
    \pgfmathsetmacro\cx{(\cA-\cy*(#2-#4))/(#1-#3)}%
    \pgfmathsetmacro\cr{sqrt((#1-\cx)*(#1-\cx)+(#2-\cy)*(#2-\cy))}%
    \pgfmathsetmacro\cA{atan2(#2-\cy,#1-\cx)}%
    \pgfmathsetmacro\cB{atan2(#6-\cy,#5-\cx)}%
    \pgfmathparse{\cB<\cA}%
    \ifnum\pgfmathresult=1
        \pgfmathsetmacro\cB{\cB+360}%
    \fi
    \draw (#1,#2) arc (\cA:\cB:\cr);%
}
\newcommand\getCoord[3]{\newdimen{#1}\newdimen{#2}\pgfextractx{#1}{\pgfpointanchor{#3}{center}}\pgfextracty{#2}{\pgfpointanchor{#3}{center}}}

\def\Xint#1{\mathchoice
   {\XXint\displaystyle\textstyle{#1}}%
   {\XXint\textstyle\scriptstyle{#1}}%
   {\XXint\scriptstyle\scriptscriptstyle{#1}}%
   {\XXint\scriptscriptstyle\scriptscriptstyle{#1}}%
   \!\int}
\def\XXint#1#2#3{{\setbox0=\hbox{$#1{#2#3}{\int}$}
     \vcenter{\hbox{$#2#3$}}\kern-.5\wd0}}
\def\ddashint{\Xint=}
\def\dashint{\Xint-}

\newcommand\separator{{\centering\rule{2cm}{0.2pt}\vspace{2pt}\par}}

\newenvironment{own}{\color{gray!70!black}}{}

\newcommand\makecenter[1]{\raisebox{-0.5\height}{#1}}
\newtheorem*{soln}{Solution}

\renewcommand{\thesection}{}
\renewcommand{\thesubsection}{\arabic{section}.\arabic{subsection}}
\makeatletter
\def\@seccntformat#1{\csname #1ignore\expandafter\endcsname\csname the#1\endcsname\quad}
\let\sectionignore\@gobbletwo
\let\latex@numberline\numberline
\def\numberline#1{\if\relax#1\relax\else\latex@numberline{#1}\fi}
\makeatother


\begin{document}
	
	\maketitle
	
\section{QUESTION 1}

The three types of elementary matrices are:

\begin{enumerate}
	\item $ \begin{pmatrix}
	1\\
	& \ddots\\
	& & 1\\
	& & & 0 & & & & 1\\
	& & & & 1\\
	& & & & & \ddots\\
	& & & & & & 1\\
	& & & 1 & & & & 0\\
	& & & & & & & & 1\\
	& & & & & & & & & \ddots\\
	& & & & & & & & & 1
	\end{pmatrix} $
	
	The zeros appear in row $ i $, row $ j $.
	This swaps column $ i $ and column $ j $, and is self-inverse.
	
	\item $ \begin{pmatrix}
	1 & & & & & &\\
	& \ddots & & & & & \\
	& & 1 & & & & \\
	& & & \lambda & & & \\
	& & & & 1 & & \\
	& & & & & \ddots & \\
	& & & & & & 1
	\end{pmatrix} $
	
	with $ \lambda $ in the $ i^{\text{th}} $ row. (Multiplies column $ i $ by $ \lambda $) This has inverse
	
	$ \begin{pmatrix}
	1 & & & & & &\\
	& \ddots & & & & & \\
	& & 1 & & & & \\
	& & & \frac{1}{\lambda} & & & \\
	& & & & 1 & & \\
	& & & & & \ddots & \\
	& & & & & & 1
	\end{pmatrix} $
	
	\item $ I_{n} + \lambda E_{ij}  $, where $ E_{ij} $ is defined as $ 1 $ in the $ (i,j) $ position and $ 0 $ everywhere else. ($ i \neq j $). This has inverse $ I_{n} + \lambda E_{ij}  $.
	
	
	
\end{enumerate}

To find inverse of this matrix, we

\begin{itemize}
	\item add column 1 to column 2
	\item swap rows 2 and 3
	\item add row 3 to row 2
	\item multiply row 2 by $ \frac{1}{3} $.
\end{itemize}

\[ \begin{pmatrix}
1 & -1 & 0\\
0 & 0 & 1 \\
0 & 3 & -1
\end{pmatrix}
\begin{pmatrix}
1 & -1 & 0\\
0 & 0 & 1 \\
0 & 3 & -1
\end{pmatrix}
\]


\section{QUESTION 2}



\begin{itemize}
	\item \emph{Minimality of} $ r $: 
	
	Suppose we have 
	
	\[ \underbrace{A}_{m \times n} = \underbrace{B}_{m \times k} \underbrace{C}_{k \times n} \]
	
	wrt. standard basis for $ \R^{n}, \R^{k}, \R^{n} $ the matrices $ A,B,C $ correspond to lin. maps $ \alpha, \beta, \gamma $ st. $ \alpha = \beta \circ \gamma $
	
	\begin{center}
		\begin{tikzcd}
			\R^{m} \ar[r, "\gamma"] \ar[rr, bend right, "\alpha"] & \R^{k} \ar[r, "\beta"] & \R^{n}
		\end{tikzcd}
	\end{center}

for some $ k $, $ r \leq k \leq n  $, and 

\[ \Im \alpha \leq \Im \beta \]

since if $ v \in \Im \alpha $, then $ v = \alpha(\omega) $ for some $ \omega $, then $ v = \beta(\gamma \omega) $ so $ v \in \Im \beta $

Taking dimensions,

\[ r \leq r(\beta) \leq k \]

where the last inequality follows from Rank Nullity. 

\item $ r $\emph{ is possible}: $ \alpha $ has rank $ r $ so we seek a map such that

\begin{center}
	\begin{tikzcd}
		\R^{m} \ar[r, "\gamma"] \ar[rr, bend right, "\alpha"] & \im \alpha \ar[r, "\beta"] & \R^{n}
	\end{tikzcd}
\end{center}

Define $ \gamma: \R^{m} \to \Im \alpha $ by $ \gamma(v) = \alpha(v) $. Define $ \beta : \Im(\alpha) \to \R^{n} $ by $ \beta(\omega) = \omega $.

Then, picking bases for $ \R $, $ \Im (\alpha) $, $ \R^{m} $, we get corresponding matrices $ A,B,C $ st.

\[ A = \underbrace{B}_{m \times r} \underbrace{C}_{r \times n} \]

	
\end{itemize}

For the last part, define $ r' = \text{ col rank of } A^{T} $

We know that $ A = BC $ with $ B $ $ m \times r $ ...
So 

\[ A^{T} = C^{T} B^{T} \text{ with } C^{T} \; n \times r \]

so $ r' \leq r $ by previous work

Applying this argument to $ A^{T}, (A^{T})^{T} (= A) $, we also see that $ \gamma \leq r' $. So $ r = r' $.

 

\section{QUESTION 3}

If $ V $ is the vector space with finite basis $\mathcal{B} =  \{ x_{1},x_{2},x_{3},x_{4} \} $ then there is a basis for $ V^{*} $, given by $ \mathcal{B}^{*} = \{  \xi_{1},\xi_{2},\xi_{3},\xi_{4} \} $ where

\[ \xi_{j}  \underbrace{\left(  \sum_{i=1}^{4} a_{i}x_{i} \right)}_{\in V} = a_{j} \quad 1 \leq j \leq 4 \qquad (*) \]

\begin{enumerate}[label=(\alph*)]
	\item By (*), the dual basis is
	
	\[ \{  \xi_{2}, \xi_{1},\xi_{4},\xi_{3} \} \]
	
	\item we have $ \xi_{2}  \left(  \sum_{i=1}^{4} a_{i}x_{i} \right)  = a_{2} \Rightarrow \xi_{2} (  a_{2}x_{2} ) = a_{2} $. Hence clear to see dual basis is  
	
	\[ \{  \xi_{1}, \frac{1}{2} \xi_{2}, 2 \xi_{3},\xi_{4} \} \]
	
	\item Call the new dual basis $ \{  \eta_{1},\eta_{2},\eta_{3},\eta_{4} \} $. It is clear that $ \eta_{1} = \xi_{1} $. To find $ \eta_{2} $, we aim to solve the system of linear equations
	
	
\begin{align*}
\eta_{2}(x_{1} +x_{2})  & = 0  \\
\eta_{2}(x_{2}+x_{3}) & = 1 \\
\eta_{2}(x_{3} + x_{4}) & = 0 \\
\eta_{2} (x_{4}) & = 0
\end{align*} 


and we deduce that $ \eta_{2} = \xi_{2} - \xi_{1} $. Similarly, $ \eta_{3} = \xi_{3} - \xi_{2} $, $ \eta_{4} = \xi_{4} - \xi_{3} $.

\[ \{  \xi_{1}, \xi_{2} - \xi_{1}, \xi_{3} - \xi_{2}, \xi_{4} - \xi_{3}  \} \]

\item Similar method to (c), the dual basis is:

\[ \{  \xi_{1} + \xi_{2}, \xi_{2} + \xi_{3}, \xi_{3} + \xi_{4}, \xi_{4}  \} \]

\end{enumerate}





\section{QUESTION 4}

We have that $ \tau_{A}(B) = \sum_{i} \sum_{j} a_{ij} b_{ji}    $, so linearity follows immediately by the definition of the sum. 

Next, want to show that 

	\begin{center}
	\begin{tikzcd}
		\text{Mat}_{m,n(\F)} \ar[r, "\phi"] & \text{Mat}_{m,n(\F)}^{*}
	\end{tikzcd}
\end{center}

defined by

\[ A \mapsto \tau_{A} \]

 defines an iso. Have already show linearity. Easy to see this is well defined. 
\begin{itemize}
	\item Injective: Suppose $ \phi(A) = 0 $. Then $ \tau_{A}(B) = 0 \; \forall \; B $, ie. $ \text{tr}(AB) = 0 \; \forall \; B $.
	
	In particular, for each $ i,j $, we have that
	
	\[ \text{tr}(AE_{ij}) = 0 \]
	
	where $ E_{ij} $ is the matrix with 1 in the $ i,j $ position and zeroes everywhere else. 
	
	Hence by definition of trace, $ \sum_{k,l} A_{k,l}(e_{ij})_{k,l} = A_{ji} $. So $ A = 0 $.
	
	\item  $ \dim(\Mat_{m,n}(\F)^{*}) = \dim(\Mat_{m,n}(\F)) = mn $, and $ \dim(\Mat_{m,n}(\F)) = mn $. So isomorphism. 
	

	
\end{itemize}


\section{QUESTION 5}

\begin{enumerate} [label = (\alph*)]
	\item Suppose two such endomorphisms exists, with matrices $ A $, $  $ respectively. Take the trace of both sides of the equation. As $ \text{tr}(AB) = \text{tr}(BA)  $, clearly the LHS is zero, but the RHS is $ \dim V $. Contradiction. 
	
	\item Define
	
	\[ \alpha : V \to V \qquad \beta : V \to V \]
	\[ f(x) \mapsto xf(x) \qquad f(x) \mapsto f'(x) \]
	
	Then
	
	\begin{align*}
	(\alpha\beta - \beta \alpha)(f) & = (xf)'  - xf'  \\
	& = f
	\end{align*}
	
	That is, $ \alpha \beta- \beta \alpha = \id_{V} $
	
\end{enumerate}





\section{QUESTION 6}

Let $ \psi: U \times V \to \F $ represent our bilinear form. Pick any bases, 

\[ e_{1}', \cdots, e_{m}'  \text{ for }  U  \]
\[ f_{1}',\cdots,f_{n}'  \text{ for }  V  \]


If $ \psi(e_{i}',f_{j}') = 0 \; \forall \; i,j $ then $ \psi = 0 $ and we're done. Otherwise pick some 

\[ e_{i}',f_{j}' \text {s.t. } \psi(e_{i}',f_{j}') \neq 0 \]

(After rescaling, assume $ \psi(e_{i}',f_{j}') = 1 $).

Set $ e_{1} = e_{i}' $, $ f_{1} = f_{j}' $. Pick a basis $ \{ f_{2},\cdots,f_{n} \} $ for $ \ker(\psi_{L}(e_{1})) $.

Pick a basis $ \{ e_{2},\cdots,e_{m} \} $ for $ \ker(\psi_{R})(f_{1}) $. Then, wrt. $ \{ e_{1},\cdots,e_{m}\}, \{  f_{1},\cdots,f_{n} \} $, have matrix 

\[ \begin{pmatrix}
1 & 0 & \cdots & 0\\
0 & & & \\
\vdots & & \text{something}& \\
0 & & &
\end{pmatrix} \]

Continue inductively, end up with

\[ \begin{pmatrix}
I_{r} & 0 \\
0 & 0
\end{pmatrix} \]

Hence


\begin{align*}
\psi( \sum_{i=1}^{m}  x_{i}e_{i}  , \sum_{j=1}^{n}  y_{j} f_{j} ) & = \sum_{i=1}^{m} \sum_{j=1}^{n}  x_{i} y_{j} \psi(e_{i},f_{j}) \text{ by linearity of } \psi \\
& = \sum_{k=1}^{r} x_{k}y_{k} 
\end{align*}

The dimensions of the left and right kernels are $ m - r $ and $ n - r $ respectively, by R-N.


\section{QUESTION 7}

\begin{enumerate}[label = (\alph*)]
	\item We show the rows are linearly independent: suppose
	
	\[ \lambda_{0} \begin{pmatrix}
	1 \\
	1 \\
	\vdots \\
	1
	\end{pmatrix} + 
	\lambda_{1} \begin{pmatrix}
	a_{0}\\
	a_{1} \\
	\vdots\\
	a_{n}
	\end{pmatrix} + \cdots + 
	\lambda_{n} \begin{pmatrix}
	a_{0}^{n}\\
	a_{1}^{n} \\
	\vdots\\
	a_{n}^{n}
	\end{pmatrix} = 0 \]
	
	This says that the polynomial 
	
	\[ f(x) = \lambda_{0} + \lambda_{1} x + \cdots + \lambda_{n} x^{n} \]
	
	has roots $ a_{0},a_{1},\cdots,a_{n} $. $ f $ of degree $ n $ has $ n+1 $ distinct roots; but this can only be the case if $ f $ is the zero polynomial. So $ \lambda_{1} = \lambda_{2} = \cdots = \lambda_{n} = 0 $.
	
	
	Rows are linearly independent, so the matrix is of full rank. Thus $ n(A) = 0 $ by rank nulity, and $ \det A \neq 0 $
	
	\item $ e_{x} \in P_{n}^{*} $ with $ e_{x}(p) = p(x) $. Want to show that with respect to the standard basis, $ \{ e_{0},\cdots,e_{n} \} $ is linearly independent.
	
	\begin{proof}
		Suppose $ \lambda_{0} e_{0} + \cdots + \lambda_{n} e_{n} = 0 $.
		
		Then
		
		\begin{align*}
		\lambda_{0} + \cdots + \lambda_{n} e_{n} & = 0 \\
		0 \cdot \lambda_{0} + \cdots + n \cdot \lambda_{n} & = 0 \\
		\vdots \\
		0^{n} \cdot \lambda_{0} + \cdots + n^{n} \cdot \lambda_{n} & = 0
		 \end{align*}
		 
		 
		 ie. 
		 
		 \[ \lambda_{0} \begin{pmatrix}
		 1 \\
		 0^{1} \\
		 \vdots \\
		 0^{n}
		 \end{pmatrix} + 
		  \cdots + 
		 \lambda_{n} \begin{pmatrix}
		 1\\
		 n^{1} \\
		 \vdots\\
		 n^{n}
		 \end{pmatrix} = 0 \]
		 
		 So by (a), $ \lambda_{1} = \lambda_{2} = \cdots = \lambda_{n} = 0 $.
		 
		 
	\end{proof}

	Now, $ \dim P_{n}^{*} = n+1 $, and $ \{ e_{0},\cdots,e_{n+1} \} $ is a linearly independent set. Hence it is a basis for $ P_{n}^{*} $
	
	\item For the basis of $ P_{n} $ for which $ (e_{0},\cdots,e_{n}) $ is dual, we want $ \{ p_{0},\cdots,p_{n} \} $ with $ e_{i}(p_{j}) = \delta_{ij} $.
	So this polynomial is zero for all $ i \in \{ 0,1,\cdots,i-1,i+1,n \} $. Hence $ p_{i} \propto (x-1)(x-2)\cdots(x-(i-1))(x-(i+1))\cdots(x-n) $
	
	As $ p_{i}(i) = 1 $,
	
	\[ p_{i} = \frac{(x-1)(x-2)\cdots(x-(i-1))(x-(i+1))\cdots(x-n)}{(i-1)(i-2)\cdots(i-(i-1))(i-(i+1))\cdots(i-n)  } \]
	
	
	
	
\end{enumerate}


\section{QUESTION 8}

\begin{enumerate}
	\item \begin{align*}
	\text{adj }(AB)  & =  \det(AB) (AB)^{-1}  \\
	& = \det(A)\det(B)B^{-1}A^{-1} \\
	& = \det(B) B^{-1} \det(A) A^{-1} \\
	& = \text{adj }(B) \text{adj }(A)
	\end{align*}
	
	\item \begin{align*}
	\det (\text{adj } A)& = \\
	& = \det ( \det (A) A^{-1} ) \\
	& = \det ( \det(A) I ) \det (A^{-1})
	& = (\det A)^{n} (\det A)^{-1} \\
	& = (\det A)^{n-1}
	\end{align*}	
	
	\item \begin{align*}
	\text{adj } (\text{adj } A)& = \text{adj } (  \det(A) A^{-1} ) \\
	& = \det ( \det (A) A^{-1} ) ( \det (A) A^{-1} )^{-1} \\
	& = (\det A)^{n-1} A (\det A)^{-1} \\
	& = (\det A)^{n-2} A 
	\end{align*}
\end{enumerate}


\begin{itemize}
	\item If $ r(A) = n $, then
	\[ \text{adj } A = \underbrace{\det A}_{\neq 0} \underbrace{A^{-1}}_{\text{invertible}} \]
	
	So $ \text{adj} A $ invertible so $ r(\text{adj } A) = n $
	
	\item If $ r(A) = n-1  $, recall that
	
	\[ (\text{adj } A) A = 0 \text{ in this case}  \]
	
	So $ \text{adj} A $ maps $ \Im A $ to $ 0 $
	
	ie. $ \ker(\text{adj } A) $ has dimension at least $ \dim(\Im A) = n-1 $. So $ r(\text{adj } A)  = 0 $ or $ 1 $.
	
	However, $ \text{adj }A \neq 0 $: We can remove some column $ i $ st. the remaining cols are L.I $ (r(A) = n -1) $. This gives us an $ n \times (n-1) $ matrix with row rank $ n - 1 $ (since row rank $ = $ column rank.) So we can remove a row $ j $ st. remaining rows are LI.
	
	
	
	Then $ A_{\hat{ij}} \neq 0 $ ($ (n-1) \times (n-1) $ matrix left over has full rank). So $ \text{adj } A \neq 0 $. So $ r(\text{adj } A) = 1 $.
	
	\item If $ r(A) = n - 2 $
	
	If we remove a col, the remaining cols are L.D (still). This does not change if we further remove a row. So $ \text{adj } A = 0 $.
	
\end{itemize}

Singular case: 

\begin{enumerate}
	\item Define $ f(\lambda) = \text{adj}((A + \lambda I) B) - \text{adj}(B)\text{adj}(A + \lambda I) $. We know that $ f(\lambda) $ is zero whenever $ \lambda $ is st. 
	
	\[ \det(A + \lambda I) \neq 0 \]
	
	ie. $ f(\lambda) $ is a poly with infinitely many roots.
	
	(since $ \det(A + \lambda I) $ is a non-zero poly so has at most $ n $ roots). So $ f $ is the zero polynomial, and (i) holds in general. 	
\end{enumerate}


\section{QUESTION 9}

Let $ \alpha $ be the map $ \alpha : P^{*} \to \R^{\N}  $ defined by 

\[ [\xi : P \to \R] \mapsto (\xi(1),\xi(t),\xi(t^{2}),\cdots)  \]

\begin{align*}
\alpha (\lambda \xi_{1} + \mu \xi_{2}) & = ( (\lambda \xi_{1} + \mu \xi_{2})(1),(\lambda \xi_{1} + \mu \xi_{2})(t),(\lambda \xi_{1} + \mu \xi_{2})(t^{2}),\cdots)  \\
& = (\lambda \xi_{1}(1)  + \mu \xi_{2}(1), \lambda \xi_{1}(t)  + \mu \xi_{2}(t), \lambda \xi_{1}(t^{2})  + \mu \xi_{2}(t^{2}), \cdots  ) \\
& = \lambda(  \xi_{1}(1), \xi_{1}(t), \xi_{1}(t^{2}), \cdots ) + \mu ( \xi_{2}(1),\xi_{2}(t),\xi_{2}(t^{2}),\cdots )    
\end{align*}

Hence $ \alpha $ is linear. 

Let the map $ B : \R^{\N} \to P^{*} $ be defined as

\[ (a_{0},a_{1},a_{2},\cdots) \mapsto [ \xi : P \to \R \; | \; \xi(t^{n}) = a_{n} ] \]

As this defines $ \xi $ on the basis of $ P $, it fully defines $ \xi $ so this is well defined. This is an inverse to $ \alpha $ so $ \alpha $ is a bijective map. Hence $ \alpha $ is an isomorphism and $ P^{*} \simeq \R^{\N} $


Let the sequence $ (a_{0},a_{1},a_{2},\cdots) $ correspond to the linear map $ \xi : P  \to \R $ with $ \xi(t^{n}) = a_{n} $. For $ \alpha \in L(P,P) $

$ \alpha^{*} $ dual to $ \alpha $ is defined by $ \alpha^{*} : P^{*} \to P^{*} $, $ \varepsilon \mapsto  \varepsilon \circ \alpha $.

\begin{enumerate}[label = (\alph*)]
	\item $ D^{*} (\xi) \mapsto \xi \circ D $ where $ \xi \circ D : P \to \R $ with $ \xi \circ D (t^{n}) = \xi(n t^{n-1})  = n a_{n-1}$ 
	
	\item $ S^{*}(\xi) \mapsto \xi \circ S $ where $ \xi \circ S : P \to \R $ with $ \xi \circ S(t^{n}) = \xi(t^{2n}) = a_{2n} $
	
	\item $ (DS)^{*} (\xi) \mapsto \xi \circ D S $ where $ \xi \circ DS : P \to \R $ with $ \xi \circ DS (t^{n}) = \xi(2n t^{2n-1})  = 2n a_{2n-1}$ 
		
	\item $ (SD)^{*} (\xi) \mapsto \xi \circ SD $ where $ \xi \circ SD : P \to \R $ with $ \xi \circ SD (t^{n}) = \xi(n t^{2(n-1)})  = n a_{2(n-1)}$ 
\end{enumerate}

\section{QUESTION 10}
\section{QUESTION 11}

	
	
	
\end{document}	