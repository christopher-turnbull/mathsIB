\documentclass[a4paper]{article}
\usepackage{amsmath}
\def\npart {IB}
\def\nterm {Michaelmas}
\def\nyear {2017}
\def\nlecturer {Mr Rawlinson ( jir25@cam.ac.uk )}
\def\ncourse {Linear Algebra Sheet 3}

\input{header}
\newtheorem*{soln}{Solution}

\renewcommand{\thesection}{}
\renewcommand{\thesubsection}{\arabic{section}.\arabic{subsection}}
\makeatletter
\def\@seccntformat#1{\csname #1ignore\expandafter\endcsname\csname the#1\endcsname\quad}
\let\sectionignore\@gobbletwo
\let\latex@numberline\numberline
\def\numberline#1{\if\relax#1\relax\else\latex@numberline{#1}\fi}
\makeatother


\begin{document}
	
\maketitle
	
\section{QUESTION 1}


\[ A_{1} = \begin{pmatrix}
1 & 1 & 0 \\
0 & 3 & -2\\
0 & 1 & 0
\end{pmatrix}, A_{2} = \begin{pmatrix}
1 & 1 & 1 \\
0 & 3 & -2\\
0 & 1 & 0
\end{pmatrix}, A_{3} = \begin{pmatrix}
1 & 1 & 0 \\
0 & 3 & -2\\
0 & 1 & 0
\end{pmatrix} \]

This matrix has characteristic polynomial

\[ \chi_{A_{1}}(\lambda) = (\lambda - 1)^{2} (\lambda - 2) \]


For $ \lambda = 2 $ eigenvectors satisfy

\[ \begin{pmatrix}
-1 & 1 & 0 \\
0 & 1 & -2\\
0 & 1 & -2
\end{pmatrix} \begin{pmatrix}
v_{1} \\
v_{2} \\
v_{3}
\end{pmatrix} = \mathbf{0} \]

So $ \mathbf{v} = (2,2,1) $, and we take this as a basis for the $ \lambda = 2 $ eigenspace. Similarly for $ \lambda= 1 $ we have

\[ \begin{pmatrix}
0 & 1 & 0 \\
0 & 2 & -2\\
0 & 1 & -1
\end{pmatrix} \begin{pmatrix}
v_{1} \\
v_{2} \\
v_{3}
\end{pmatrix} = \mathbf{0} \]

This implies $ v_{2} = v_{3} = 0 $, so eigenvector must be of the form $ (1,0,0) $, again a basis with one element.


$ A_{2} $:  Next, we note that $  \chi_{A_{1}}(\lambda) = \chi_{A_{2}}(\lambda)  $ as the determinant calculation expanding down the first column will remain unchanged, so same eigenvalues. We can see that for $ \lambda = 2 $, $ \mathbf{v} = (1,2,1) $ is a basis. For $ \lambda = 1 $ we have $ v_{2} = v_{3}  $, so an eigenvector basis is given by

\[ \left\{ \begin{pmatrix}
1\\
0\\
0
\end{pmatrix}, \begin{pmatrix}
0\\
1\\
-1
\end{pmatrix} \right\} \]

Next, $ \chi_{A_{3}}(\lambda) = (\lambda - 1) (\lambda - 2)^{2} $. For $ \lambda = 1 $ the eigenspace basis is $ \{ (1,1,1) \} $, for $ \lambda = 2 $ it is $ \{  (1,-2,1) \} $.



\section{QUESTION 2}

Consider $ \det(A - \kappa \iota) $.


Add all the columns to the first column; it becomes a column where all entries are equal to $ \lambda - \mu + (n-1) $. Now subtract first row from all others. We are left with $ \det(A - \kappa \iota) = (\lambda - \mu + (n-1)) \det (M) $ where $ M $ is an $ n-1 \times n-1 $ lower triangular matrix with $ \lambda - \mu - 1 $ in every diagonal entry.

Hence 

\begin{align*}
\det(A - \kappa \iota) & =  (\lambda - \mu + (n-1))(\lambda - \mu - 1)^{n-1}  
\end{align*}

So $ n  $ eigenvalues are given as

\[ \mu = \lambda + n-1, \underbrace{\lambda - 1,\cdots,\lambda - 1}_{n-1 \text{ times}}  \]


\section{QUESTION 3}

Define $ \pi_{j} = q_{j}(\alpha) : V \to V $ by

\[ q_{j}(\alpha) = \prod_{i \neq j}^{k}  \frac{\alpha - \lambda_{i}}{\lambda_{j} - \lambda_{i}} \]

\section{QUESTION 4}
\section{QUESTION 5}
\section{QUESTION 6}
\section{QUESTION 7}
\section{QUESTION 8}
\section{QUESTION 9}
\section{QUESTION 10}
\section{QUESTION 11}
\section{QUESTION 12}

	
	
	
\end{document}	