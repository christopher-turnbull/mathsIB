\documentclass[a4paper]{article}
\usepackage{amsmath}
\def\npart {IA}
\def\nterm {Michaelmas}
\def\nyear {2017}
\def\nlecturer {Prof Weber (rrw1@cam.ac.uk)}
\def\ncourse {Markov Chains Example Sheet 2}

% Imports
\ifx \nauthor\undefined
  \def\nauthor{Christopher Turnbull}
\else
\fi

\author{Supervised by \nlecturer \\\small Solutions presented by \nauthor}
\date{\nterm\ \nyear}

\usepackage{alltt}
\usepackage{amsfonts}
\usepackage{amsmath}
\usepackage{amssymb}
\usepackage{amsthm}
\usepackage{booktabs}
\usepackage{caption}
\usepackage{enumitem}
\usepackage{fancyhdr}
\usepackage{graphicx}
\usepackage{mathdots}
\usepackage{mathtools}
\usepackage{microtype}
\usepackage{multirow}
\usepackage{pdflscape}
\usepackage{pgfplots}
\usepackage{siunitx}
\usepackage{slashed}
\usepackage{tabularx}
\usepackage{tikz}
\usepackage{tkz-euclide}
\usepackage[normalem]{ulem}
\usepackage[all]{xy}
\usepackage{imakeidx}

\makeindex[intoc, title=Index]
\indexsetup{othercode={\lhead{\emph{Index}}}}

\ifx \nextra \undefined
  \usepackage[pdftex,
    hidelinks,
    pdfauthor={Christopher Turnbull},
    pdfsubject={Cambridge Maths Notes: Part \npart\ - \ncourse},
    pdftitle={Part \npart\ - \ncourse},
  pdfkeywords={Cambridge Mathematics Maths Math \npart\ \nterm\ \nyear\ \ncourse}]{hyperref}
  \title{Part \npart\ --- \ncourse}
\else
  \usepackage[pdftex,
    hidelinks,
    pdfauthor={Christopher Turnbull},
    pdfsubject={Cambridge Maths Notes: Part \npart\ - \ncourse\ (\nextra)},
    pdftitle={Part \npart\ - \ncourse\ (\nextra)},
  pdfkeywords={Cambridge Mathematics Maths Math \npart\ \nterm\ \nyear\ \ncourse\ \nextra}]{hyperref}

  \title{Part \npart\ --- \ncourse \\ {\Large \nextra}}
  \renewcommand\printindex{}
\fi

\pgfplotsset{compat=1.12}

\pagestyle{fancyplain}
\lhead{\emph{\nouppercase{\leftmark}}}
\ifx \nextra \undefined
  \rhead{
    \ifnum\thepage=1
    \else
      \npart\ \ncourse
    \fi}
\else
  \rhead{
    \ifnum\thepage=1
    \else
      \npart\ \ncourse\ (\nextra)
    \fi}
\fi
\usetikzlibrary{arrows.meta}
\usetikzlibrary{decorations.markings}
\usetikzlibrary{decorations.pathmorphing}
\usetikzlibrary{positioning}
\usetikzlibrary{fadings}
\usetikzlibrary{intersections}
\usetikzlibrary{cd}

\newcommand*{\Cdot}{{\raisebox{-0.25ex}{\scalebox{1.5}{$\cdot$}}}}
\newcommand {\pd}[2][ ]{
  \ifx #1 { }
    \frac{\partial}{\partial #2}
  \else
    \frac{\partial^{#1}}{\partial #2^{#1}}
  \fi
}
\ifx \nhtml \undefined
\else
  \renewcommand\printindex{}
  \makeatletter
  \DisableLigatures[f]{family = *}
  \let\Contentsline\contentsline
  \renewcommand\contentsline[3]{\Contentsline{#1}{#2}{}}
  \renewcommand{\@dotsep}{10000}
  \newlength\currentparindent
  \setlength\currentparindent\parindent

  \newcommand\@minipagerestore{\setlength{\parindent}{\currentparindent}}
  \usepackage[active,tightpage,pdftex]{preview}
  \renewcommand{\PreviewBorder}{0.1cm}

  \newenvironment{stretchpage}%
  {\begin{preview}\begin{minipage}{\hsize}}%
    {\end{minipage}\end{preview}}
  \AtBeginDocument{\begin{stretchpage}}
  \AtEndDocument{\end{stretchpage}}

  \newcommand{\@@newpage}{\end{stretchpage}\begin{stretchpage}}

  \let\@real@section\section
  \renewcommand{\section}{\@@newpage\@real@section}
  \let\@real@subsection\subsection
  \renewcommand{\subsection}{\@@newpage\@real@subsection}
  \makeatother
\fi

% Theorems
\theoremstyle{definition}
\newtheorem*{aim}{Aim}
\newtheorem*{axiom}{Axiom}
\newtheorem*{claim}{Claim}
\newtheorem*{cor}{Corollary}
\newtheorem*{conjecture}{Conjecture}
\newtheorem*{defi}{Definition}
\newtheorem*{eg}{Example}
\newtheorem*{ex}{Exercise}
\newtheorem*{fact}{Fact}
\newtheorem*{law}{Law}
\newtheorem*{lemma}{Lemma}
\newtheorem*{notation}{Notation}
\newtheorem*{prop}{Proposition}
\newtheorem*{soln}{Solution}
\newtheorem*{thm}{Theorem}

\newtheorem*{remark}{Remark}
\newtheorem*{warning}{Warning}
\newtheorem*{exercise}{Exercise}

\newtheorem{nthm}{Theorem}[section]
\newtheorem{nlemma}[nthm]{Lemma}
\newtheorem{nprop}[nthm]{Proposition}
\newtheorem{ncor}[nthm]{Corollary}


\renewcommand{\labelitemi}{--}
\renewcommand{\labelitemii}{$\circ$}
\renewcommand{\labelenumi}{(\roman{*})}

\let\stdsection\section
\renewcommand\section{\newpage\stdsection}

% Strike through
\def\st{\bgroup \ULdepth=-.55ex \ULset}

% Maths symbols
\newcommand{\abs}[1]{\left\lvert #1\right\rvert}
\newcommand\ad{\mathrm{ad}}
\newcommand\AND{\mathsf{AND}}
\newcommand\Art{\mathrm{Art}}
\newcommand{\Bilin}{\mathrm{Bilin}}
\newcommand{\bket}[1]{\left\lvert #1\right\rangle}
\newcommand{\B}{\mathcal{B}}
\newcommand{\bolds}[1]{{\bfseries #1}}
\newcommand{\brak}[1]{\left\langle #1 \right\rvert}
\newcommand{\braket}[2]{\left\langle #1\middle\vert #2 \right\rangle}
\newcommand{\bra}{\langle}
\newcommand{\cat}[1]{\mathsf{#1}}
\newcommand{\C}{\mathbb{C}}
\newcommand{\CP}{\mathbb{CP}}
\newcommand{\cU}{\mathcal{U}}
\newcommand{\Der}{\mathrm{Der}}
\newcommand{\D}{\mathrm{D}}
\newcommand{\dR}{\mathrm{dR}}
\newcommand{\E}{\mathbb{E}}
\newcommand{\F}{\mathbb{F}}
\newcommand{\Frob}{\mathrm{Frob}}
\newcommand{\GG}{\mathbb{G}}
\newcommand{\gl}{\mathfrak{gl}}
\newcommand{\GL}{\mathrm{GL}}
\newcommand{\G}{\mathcal{G}}
\newcommand{\Gr}{\mathrm{Gr}}
\newcommand{\haut}{\mathrm{ht}}
\newcommand{\Id}{\mathrm{Id}}
\newcommand{\ket}{\rangle}
\newcommand{\lie}[1]{\mathfrak{#1}}
\newcommand{\Mat}{\mathrm{Mat}}
\newcommand{\N}{\mathbb{N}}
\newcommand{\norm}[1]{\left\lVert #1\right\rVert}
\newcommand{\normalorder}[1]{\mathop{:}\nolimits\!#1\!\mathop{:}\nolimits}
\newcommand\NOT{\mathsf{NOT}}
\newcommand{\Oc}{\mathcal{O}}
\newcommand{\Or}{\mathrm{O}}
\newcommand\OR{\mathsf{OR}}
\newcommand{\ort}{\mathfrak{o}}
\newcommand{\PGL}{\mathrm{PGL}}
\newcommand{\ph}{\,\cdot\,}
\newcommand{\pr}{\mathrm{pr}}
\newcommand{\Prob}{\mathbb{P}}
\newcommand{\PSL}{\mathrm{PSL}}
\newcommand{\Ps}{\mathcal{P}}
\newcommand{\PSU}{\mathrm{PSU}}
\newcommand{\pt}{\mathrm{pt}}
\newcommand{\qeq}{\mathrel{``{=}"}}
\newcommand{\Q}{\mathbb{Q}}
\newcommand{\R}{\mathbb{R}}
\newcommand{\RP}{\mathbb{RP}}
\newcommand{\Rs}{\mathcal{R}}
\newcommand{\SL}{\mathrm{SL}}
\newcommand{\so}{\mathfrak{so}}
\newcommand{\SO}{\mathrm{SO}}
\newcommand{\Spin}{\mathrm{Spin}}
\newcommand{\Sp}{\mathrm{Sp}}
\newcommand{\su}{\mathfrak{su}}
\newcommand{\SU}{\mathrm{SU}}
\newcommand{\term}[1]{\emph{#1}\index{#1}}
\newcommand{\T}{\mathbb{T}}
\newcommand{\tv}[1]{|#1|}
\newcommand{\U}{\mathrm{U}}
\newcommand{\uu}{\mathfrak{u}}
\newcommand{\Vect}{\mathrm{Vect}}
\newcommand{\wsto}{\stackrel{\mathrm{w}^*}{\to}}
\newcommand{\wt}{\mathrm{wt}}
\newcommand{\wto}{\stackrel{\mathrm{w}}{\to}}
\newcommand{\Z}{\mathbb{Z}}
\renewcommand{\d}{\mathrm{d}}
\renewcommand{\H}{\mathbb{H}}
\renewcommand{\P}{\mathbb{P}}
\renewcommand{\sl}{\mathfrak{sl}}
\renewcommand{\vec}[1]{\boldsymbol{\mathbf{#1}}}
%\renewcommand{\F}{\mathcal{F}}

\let\Im\relax
\let\Re\relax

\DeclareMathOperator{\adj}{adj}
\DeclareMathOperator{\Ann}{Ann}
\DeclareMathOperator{\area}{area}
\DeclareMathOperator{\Aut}{Aut}
\DeclareMathOperator{\Bernoulli}{Bernoulli}
\DeclareMathOperator{\betaD}{beta}
\DeclareMathOperator{\bias}{bias}
\DeclareMathOperator{\binomial}{binomial}
\DeclareMathOperator{\card}{card}
\DeclareMathOperator{\ccl}{ccl}
\DeclareMathOperator{\Char}{char}
\DeclareMathOperator{\ch}{ch}
\DeclareMathOperator{\cl}{cl}
\DeclareMathOperator{\cls}{\overline{\mathrm{span}}}
\DeclareMathOperator{\conv}{conv}
\DeclareMathOperator{\corr}{corr}
\DeclareMathOperator{\cosec}{cosec}
\DeclareMathOperator{\cosech}{cosech}
\DeclareMathOperator{\cov}{cov}
\DeclareMathOperator{\covol}{covol}
\DeclareMathOperator{\diag}{diag}
\DeclareMathOperator{\diam}{diam}
\DeclareMathOperator{\Diff}{Diff}
\DeclareMathOperator{\disc}{disc}
\DeclareMathOperator{\dom}{dom}
\DeclareMathOperator{\End}{End}
\DeclareMathOperator{\energy}{energy}
\DeclareMathOperator{\erfc}{erfc}
\DeclareMathOperator{\erf}{erf}
\DeclareMathOperator*{\esssup}{ess\,sup}
\DeclareMathOperator{\ev}{ev}
\DeclareMathOperator{\Ext}{Ext}
\DeclareMathOperator{\Fit}{Fit}
\DeclareMathOperator{\fix}{fix}
\DeclareMathOperator{\Frac}{Frac}
\DeclareMathOperator{\Gal}{Gal}
\DeclareMathOperator{\gammaD}{gamma}
\DeclareMathOperator{\gr}{gr}
\DeclareMathOperator{\hcf}{hcf}
\DeclareMathOperator{\Hom}{Hom}
\DeclareMathOperator{\id}{id}
\DeclareMathOperator{\image}{image}
\DeclareMathOperator{\im}{im}
\DeclareMathOperator{\Im}{Im}
\DeclareMathOperator{\Ind}{Ind}
\DeclareMathOperator{\Int}{Int}
\DeclareMathOperator{\Isom}{Isom}
\DeclareMathOperator{\lcm}{lcm}
\DeclareMathOperator{\length}{length}
\DeclareMathOperator{\Lie}{Lie}
\DeclareMathOperator{\like}{like}
\DeclareMathOperator{\Lk}{Lk}
\DeclareMathOperator{\mse}{mse}
\DeclareMathOperator{\multinomial}{multinomial}
\DeclareMathOperator{\orb}{orb}
\DeclareMathOperator{\ord}{ord}
\DeclareMathOperator{\otp}{otp}
\DeclareMathOperator{\Poisson}{Poisson}
\DeclareMathOperator{\poly}{poly}
\DeclareMathOperator{\rank}{rank}
\DeclareMathOperator{\rel}{rel}
\DeclareMathOperator{\Re}{Re}
\DeclareMathOperator*{\res}{res}
\DeclareMathOperator{\Res}{Res}
\DeclareMathOperator{\rk}{rk}
\DeclareMathOperator{\Root}{Root}
\DeclareMathOperator{\sech}{sech}
\DeclareMathOperator{\sgn}{sgn}
\DeclareMathOperator{\spn}{span}
\DeclareMathOperator{\stab}{stab}
\DeclareMathOperator{\St}{St}
\DeclareMathOperator{\supp}{supp}
\DeclareMathOperator{\Syl}{Syl}
\DeclareMathOperator{\Sym}{Sym}
\DeclareMathOperator{\tr}{tr}
\DeclareMathOperator{\Tr}{Tr}
\DeclareMathOperator{\var}{var}
\DeclareMathOperator{\vol}{vol}

\pgfarrowsdeclarecombine{twolatex'}{twolatex'}{latex'}{latex'}{latex'}{latex'}
\tikzset{->/.style = {decoration={markings,
                                  mark=at position 1 with {\arrow[scale=2]{latex'}}},
                      postaction={decorate}}}
\tikzset{<-/.style = {decoration={markings,
                                  mark=at position 0 with {\arrowreversed[scale=2]{latex'}}},
                      postaction={decorate}}}
\tikzset{<->/.style = {decoration={markings,
                                   mark=at position 0 with {\arrowreversed[scale=2]{latex'}},
                                   mark=at position 1 with {\arrow[scale=2]{latex'}}},
                       postaction={decorate}}}
\tikzset{->-/.style = {decoration={markings,
                                   mark=at position #1 with {\arrow[scale=2]{latex'}}},
                       postaction={decorate}}}
\tikzset{-<-/.style = {decoration={markings,
                                   mark=at position #1 with {\arrowreversed[scale=2]{latex'}}},
                       postaction={decorate}}}
\tikzset{->>/.style = {decoration={markings,
                                  mark=at position 1 with {\arrow[scale=2]{latex'}}},
                      postaction={decorate}}}
\tikzset{<<-/.style = {decoration={markings,
                                  mark=at position 0 with {\arrowreversed[scale=2]{twolatex'}}},
                      postaction={decorate}}}
\tikzset{<<->>/.style = {decoration={markings,
                                   mark=at position 0 with {\arrowreversed[scale=2]{twolatex'}},
                                   mark=at position 1 with {\arrow[scale=2]{twolatex'}}},
                       postaction={decorate}}}
\tikzset{->>-/.style = {decoration={markings,
                                   mark=at position #1 with {\arrow[scale=2]{twolatex'}}},
                       postaction={decorate}}}
\tikzset{-<<-/.style = {decoration={markings,
                                   mark=at position #1 with {\arrowreversed[scale=2]{twolatex'}}},
                       postaction={decorate}}}

\tikzset{circ/.style = {fill, circle, inner sep = 0, minimum size = 3}}
\tikzset{mstate/.style={circle, draw, blue, text=black, minimum width=0.7cm}}

\tikzset{commutative diagrams/.cd,cdmap/.style={/tikz/column 1/.append style={anchor=base east},/tikz/column 2/.append style={anchor=base west},row sep=tiny}}

\definecolor{mblue}{rgb}{0.2, 0.3, 0.8}
\definecolor{morange}{rgb}{1, 0.5, 0}
\definecolor{mgreen}{rgb}{0.1, 0.4, 0.2}
\definecolor{mred}{rgb}{0.5, 0, 0}

\def\drawcirculararc(#1,#2)(#3,#4)(#5,#6){%
    \pgfmathsetmacro\cA{(#1*#1+#2*#2-#3*#3-#4*#4)/2}%
    \pgfmathsetmacro\cB{(#1*#1+#2*#2-#5*#5-#6*#6)/2}%
    \pgfmathsetmacro\cy{(\cB*(#1-#3)-\cA*(#1-#5))/%
                        ((#2-#6)*(#1-#3)-(#2-#4)*(#1-#5))}%
    \pgfmathsetmacro\cx{(\cA-\cy*(#2-#4))/(#1-#3)}%
    \pgfmathsetmacro\cr{sqrt((#1-\cx)*(#1-\cx)+(#2-\cy)*(#2-\cy))}%
    \pgfmathsetmacro\cA{atan2(#2-\cy,#1-\cx)}%
    \pgfmathsetmacro\cB{atan2(#6-\cy,#5-\cx)}%
    \pgfmathparse{\cB<\cA}%
    \ifnum\pgfmathresult=1
        \pgfmathsetmacro\cB{\cB+360}%
    \fi
    \draw (#1,#2) arc (\cA:\cB:\cr);%
}
\newcommand\getCoord[3]{\newdimen{#1}\newdimen{#2}\pgfextractx{#1}{\pgfpointanchor{#3}{center}}\pgfextracty{#2}{\pgfpointanchor{#3}{center}}}

\def\Xint#1{\mathchoice
   {\XXint\displaystyle\textstyle{#1}}%
   {\XXint\textstyle\scriptstyle{#1}}%
   {\XXint\scriptstyle\scriptscriptstyle{#1}}%
   {\XXint\scriptscriptstyle\scriptscriptstyle{#1}}%
   \!\int}
\def\XXint#1#2#3{{\setbox0=\hbox{$#1{#2#3}{\int}$}
     \vcenter{\hbox{$#2#3$}}\kern-.5\wd0}}
\def\ddashint{\Xint=}
\def\dashint{\Xint-}

\newcommand\separator{{\centering\rule{2cm}{0.2pt}\vspace{2pt}\par}}

\newenvironment{own}{\color{gray!70!black}}{}

\newcommand\makecenter[1]{\raisebox{-0.5\height}{#1}}

\newtheorem*{soln}{Solution}

\renewcommand{\thesection}{}
\renewcommand{\thesubsection}{\arabic{section}.\arabic{subsection}}
\makeatletter
\def\@seccntformat#1{\csname #1ignore\expandafter\endcsname\csname the#1\endcsname\quad}
\let\sectionignore\@gobbletwo
\let\latex@numberline\numberline
\def\numberline#1{\if\relax#1\relax\else\latex@numberline{#1}\fi}
\makeatother


\begin{document}
	
\maketitle

\section{QUESTION 1}

Suppose there exists some state $ i \in S, i \neq s $, st. $ i $ is recurrent. Then

 \[ \P_{i}(T_{i} < \infty ) = 1 \text{ where } T_{i} := \min \{  n \geq  1 : X_{n} = i \}  \]
 
 But $ i \to s $, so by definition there exists some $ r > 0 $ st. $ p_{i,s}(r) > 0 $, and as $ s $ is absorbing, $ p_{s,i}(n) = 0 \; \forall \;  n \geq 0$. Hence there is a non-zero probability of being `trapped' in $ s $, ie. a non-zero probability that $ T_{i} = \infty $. 
 Hence, for all $ i \in S $, $ \P_{i}(T_{i} < \infty) < 1$, ie. $ i $ is transient. 



\section{QUESTION 2}

Let $ P $ denote the transition matrix in question. The characteristic equation $ \det(P - \kappa \iota ) = 0 $ gives roots $ \kappa_{1} = 1 $, $ \kappa_{2} = 1 - 2p $, $ \kappa_{3} = 1 - 4p $. 

Thus 

\[ P^{n} = U^{-1} \begin{pmatrix}
1 & 0 & 0 \\
0 & (1-2p)^{n} & 0 \\
0 & 0 & (1 - 4p)^{n}
\end{pmatrix} U  \]

for some invertible matrix $ U $, and so

\[ p_{1,1}(n) = A + B(1-2p)^{n} + C(1-4p)^{n} \]

Using $ p_{1,1}(0) = 1, p_{1,1}(1) = 1 - 2p, p_{1,1}(2) = (1-2p)^{2} + 2p^{2} $ gives

\begin{align*}
A + B + C & = 1 \\
A + B(1-2p) + C(1-4p) & = 1 - 2p \\
A + B(1-2p)^{2} + C(1-4p)^{2} & = 1 - 4p + 6p^{2} 
\end{align*}

Setting $ p = \frac{1}{2} $ in the second equation, $ p = \frac{1}{4} $ in the third yields

\begin{align*}
A + B + C & = 1 \\
A  - C & = 0\\
A + B/4 & = 3/8 	
\end{align*}

which gives $ A = C = 1/4 $, $ B = 1/2 $, thus

\[ p_{1,1}(n) = \frac{1}{4}  + \frac{1}{2}(1-2p)^{n} + \frac{1}{4}(1-4p)^{n} \]

We have $ 1 \leftrightarrow 2 \leftrightarrow 3 $ so the chain is irreducible (all states recurrent of all states transient).
Thus as $ n \to \infty $, $ p_{1,1} \to 1/4 $, so state $ 1 $ is recurrent.  hence all states are recurrent.




\section{QUESTION 3}

Using the symmetry of the problem, the invariant probabilities are $ \pi_{i} = 1/2^{3} $ for all vertices $ x $. For a finite irreducible MC, the mean recurrent time $ \mu_{i} $ to state $ i $ is 

\[ \mu_{i} = \frac{1}{\pi_{i}} \]

Thus the expected number of steps until the particle first returns to $ v $ is $ 2^{3} = 8 $.

Should be a way to do the next two using invariant distributions but I can't see it. Let $ e_{i} $ be the expected number of steps to reach $ w $ given we are $ i $ steps away from $ w $. Hence

\begin{align*}
e_{0} & = 0 \\
e_{1} & = 1 + \frac{1}{4} e_{1} + \frac{1}{2} e_{2} \\
e_{2} & = 1 + \frac{1}{2} e_{1} + \frac{1}{4} e_{2} + \frac{1}{4} e_{3} \\
e_{3} & = 1 + \frac{3}{4} e_{2} + \frac{1}{4} e_{3}  
\end{align*}

Solving gives $ e_{3} = 40/3 $, the expected number of steps to reach $ v $ starting in $ v $.

For the last part: as the chain is irreducible and positive recurrent, the mean number of visits to state $ w $ between two consecutive visits to state $ v $ equals $ \pi_{w} / \pi_{v} = 8/8 = 1$.



\section{QUESTION 4}

Suppose after one step the particle is in state $ j $, $ a_{j} > 0 $. Now the particle can only travel to state $ j $ or $ j-1 $; once it is in state $ j-1 $, it can only travel to state $ j-1 $ or $ j-2 $, and so on... eventually it returns to state $ 0 $, where it is then sent to some other state $ k $ st. $ a_{k} > 0 $. In other words, we can only travel to a higher state by doing so from the origin. Motivated by this, we define 

\[ J = \sup\{ j \; | \; a_{j} > 0 \}  \]

It is now easy to see that all states $ \geq J $ are transient; after we leave them (which we will do eventually), we can never return there.

Not sure how to find the mean recurrent times. 

\section{QUESTION 5}

Possible transitions of the chain are illustrated below:
\begin{center}
	\begin{tikzpicture}
	\node [mstate] (1) at (0, 0) {$1$};
	\node [mstate] (5) at (2, 0) {$5$};
	\node [mstate] (3) at (3, -1.4) {$3$};
	\node [mstate] (4) at (4, 0) {$4$};
	\node [mstate] (2) at (6, 0) {$2$};
	\draw (1) edge [loop below, ->] (1);
	\draw (2) edge [bend left, ->] (4);
	\draw (4) edge [bend left, ->] (2);
	\draw (4) edge [loop above, ->] (4);
	\draw (5) edge [loop below, ->] (6);
	\draw (1) edge [bend left, ->] (5);
	\draw (5) edge [bend left, ->] (1);
	\draw (4) edge [->] (5);
	\draw (4) edge [->] (3);
	\draw (3) edge [loop below, ->] (3);
	\end{tikzpicture}
\end{center}

The communicating classes are $ C_{1} = \{ 1,5 \}, C_{2} = \{ 3 \} $ and $ C_{3} = \{ 2,4 \} $. The classes $ C_{1} $ and $ C_{3} $ are not closed, but $ C_{2} $ is closed.
We know that inside communicating classes, every state is recurrent or every state is transient. It is easy to see from the diagram that

\[ C_{1} \text{ recurrent }, C_{2} \text{ recurrent}, C_{3} \text{ transient} \]


\section{QUESTION 6}


\begin{itemize}
	\item There is a non zero probability that a MC beginning in a transient state will never return to that state
	\item There is a guarantee that a process beginning in a recurrent state will return to that state. 
\end{itemize}

\section{QUESTION 7}

Collapsing the tree into a random walk on $ \Z^{+} $, where the root of the tree $ R $ is represented by $ 0 $, with two branches extending to $ 1 $, four extending to $ 2 $ etc. 

The walker moves rightwards with probability $ 2/3 $ and leftwards with probability $ 1/3 $, at at $ R $ moves rightwards with probability $ 1 $.

It is now intuitively obvious that this random walk is transient. For a concrete proof, we will argue that the state $ 0 $ is transient, and as the MC is irreducible, all states are transient.
By the usual arguments, the probability of return in $ 2n $ steps is given by

\[ p_{0}(2n) = \binom{2n}{n} \left( \frac{2}{3} \right)^{n} \left( \frac{1}{3} \right)^{n}     \]

Using Stirling's formula: $ n! \sim (n/e)^{n} \sqrt{2 \pi n} $ as $ n \to \infty $, we have

\begin{align*}
p_{0}(2n) & = \frac{(2n)!}{(n!)^{2}} \left( \frac{2}{3} \right)^{n} \left( \frac{1}{3} \right)^{n} \\
& \sim \frac{2^{3n}}{3^{2n} \sqrt{n \pi} } \\
& = \left( \frac{8}{9} \right)^{n}  \frac{1}{\sqrt{n \pi}}
\end{align*}

We can conclude $ \sum_{n=0}^{\infty} p_{0}(2n) < \infty $ as this series can be shown to converge using the ratio test. 



 
\section{QUESTION 8}

The walk is at the origin $ \mathbf{0} = (0,0,0,0) $ at time $ 2n $ if and only if it has taken equal number of steps negative and positive in each dimension (ie. in 1D, equal number right and left). Therefore,

\begin{align*}
p_{\mathbf{0},\mathbf{0}}(2n) & = \left(  \frac{1}{8} \right)^{2n} \sum_{i_{1},\cdots,i_{4}} \frac{(2n)!}{(i_{1}!i_{2}!i_{3}!i_{4}!)^{2}}   \\
\end{align*}

where $ i_{1} + i_{2} + i_{3} + i_{4} = n $

Thus

\begin{align*}
p_{\mathbf{0},\mathbf{0}}(2n)& = \left( \frac{1}{2} \right)^{2n} \binom{2n}{n} \sum_{i_{1},\cdots,i_{4}} \left( \frac{n!}{4^{n}i_{1}!i_{2}!i_{3}!i_{4}!}   \right)^{2}   \\
& \leq \left( \frac{1}{2} \right)^{2n} \binom{2n}{n} M \sum_{i_{1},\cdots,i_{4}} \frac{n!}{4^{n}i_{1}!i_{2}!i_{3}!i_{4}!}  \qquad (*) \\
\end{align*}

where

\[ M = \max \left\{  \frac{n!}{4^{n}i_{1}!i_{2}!i_{3}!i_{4}!} \; : \; i_{1}\cdots,i_{4}  \geq 0, \sum_{r=1}^{4} i_{r} = n   \right\}  \]

It is not difficult to see that the maximum is attained when $ i,j $ and $ k $ are all closest to $ \frac{1}{3}n $, so that

\[ M \leq \frac{n!}{4^{n}( \lfloor \frac{1}{4} n \rfloor ! )^{4}   } \]

Furthermore, the final summation in (*) equals 1, since the summand is the probability that, in allocating $ n $ balls randomly to four urns, the urns contain $ i_{1},\cdots,i_{4} $ balls respectively. It follows that

\[ p_{\mathbf{0},\mathbf{0}}(2n) \leq \frac{(2n)!}{16^{n}n! ( \lfloor \frac{1}{4}n \rfloor ! )^{4}   } \]

which, by Stirling's formula, is strictly no bigger than $ C n^{-2} $, for some constant $ C $. Therefore:

\[ \sum_{n=0}^{\infty} p_{\mathbf{0},\mathbf{0}}(2n) < \infty  \]

implying that the origin $ \mathbf{0} $ is transient.

Shorter way: project onto $ \Z^{3} $ by discarding all coordinates except the first 3. Now we have a new possibility of the random walk $ X_{n}^{\text{proj}} $ staying where it is with probability $ \frac{4-3}{4} = \frac{1}{4} $ (when the original walk jumps in one of the discarded directions), and when it jumps,

\[ \P(X_{n}^{\text{proj}} = \mathbf{i} + \mathbf{e}_{\alpha}  \; | \; X_{n}^{\text{proj}} = \mathbf{i} ) = \frac{1 / 8}{1 - 1/4} = \frac{1}{6} \quad \alpha =1,2,3  \]

where $ \mathbf{e}_{1} = (1,0,0) $, $ \mathbf{e}_{2} = (0,1,0) $, $ \mathbf{e}_{3} = (0,0,1) $.  

Clearly, if the original walk is recurrent then the projected walk is too. But we see when the chain jumps, it behaves like the simple symmetric random walk on $ \Z^{3} $, which we know is transient. Hence the projected walk is transient, and so must the original chain be.




\section{QUESTION 9}

Setting $ \pi = \pi P $ reveals

\begin{align*}
\frac{1}{2} \pi_{1} + \frac{1}{2} \pi_{5} & = \pi_{1} \\
\frac{1}{2} \pi_{2} + \frac{1}{4} \pi_{4}  & = \pi_{2} \\
\pi_{3} + \frac{1}{4} \pi_{4} & = \pi_{3} \\
\frac{1}{2} \pi_{2} + \frac{1}{4} \pi_{4} & =  \pi_{4} \\
\frac{1}{2} \pi_{1} + \frac{1}{4} \pi_{4} + \frac{1}{2} \pi_{5} & = \pi_{5} 
\end{align*}

which gives immediately $ \pi_{4} = 0 = \pi_{2} $, $ \pi_{1} = \pi_{5} $. Thus any vector of the form $ \pi = (\pi_{1},0,\pi_{3},0,\pi_{1}) $ st. $ 2 \pi_{1} + \pi_{3} = 1 $ is an invariant distribution 

\section{QUESTION 10}


Let $ X_{n} $ be the number of molecules in $ A $ after $ n $ epoch of time. $ X = \{  X_{n} \; | \; n \geq 0 \}  $ as a Markov chain, owing to the independence of the choice for the passing molecule from previous events. We have $ p_{i,i+1} = (N - i)/N $, $ p_{i,i-1} = i/N $. 

The detailed balance equations

\[ \pi_{i}p_{i,i+1} = \pi_{i+1}p_{i+1,i} \]

are solved by $ \pi_{i+1} = \pi_{i} (N - i)/(i+1) $, thus

\[ \pi_{N} = \frac{1 \times 2 \times \cdots \times (N-1) \times N}{N \times (N-1) \times \cdots \times 2 \times 1} \pi_{0} = \frac{N!}{N!} \pi_{0} = \pi_{0} \]

So for some $ 0 < M < N $,

\begin{align*}
\pi_{M} = \frac{N - (M - 1)}{M} \pi_{M-1} & = \frac{(N-(M-1)) \times (N - (M - 2)) \times \cdots \times  N  }{M \times (M-1) \times \cdots \times 1} \pi_{0} \\
& = \frac{N!}{(N-M)!M!} \pi_{0} = \binom{N}{M} \pi_{0}
\end{align*}

As $ \pi_{i} $ is a distribution, must have $ \sum_{i} \pi_{i} = 1 $, ie.

\[ \pi_{0} \sum_{i=0}^{N} \binom{N}{i} = 1   \]

Thus $ \pi_{0} = 2^{-N} $, and 

\[ \pi_{i} = 2^{-N} \binom{N}{i}, \qquad i = 0,1,\cdots,N \]

ie $ \pi_{i} \sim \text{Bin }(N, 1/2) $



\section{QUESTION 11}
\section{QUESTION 12}

Equations yield

\begin{align*}
\pi_{1} & = p \pi_{3} \\
\pi_{2} & = \pi_{1} + \frac{2}{3} \pi_{2} + (1-p) \pi_{3} \\
\pi_{3} & = \frac{1}{3} \pi_{2} \\ 
\end{align*}

Hence the invariant distribution is of the form $ C(p/3,1,1/3) $ for different normalisation constants $ C $ depending on the value of $ p $. For $ p = 1/16 $, we obtain

\[ \pi = \left( \frac{1}{65}, \frac{48}{65}, \frac{16}{65} \right) \]

For $ p = 1/6 $,

\[ \pi = \left( \frac{1}{25}, \frac{18}{25}, \frac{6}{25} \right) \]

and for $ p = 1/12 $,

\[ \pi = \left( \frac{1}{49}, \frac{36}{49}, \frac{12}{49} \right) \]

We can see that the first entry of the invariant distribution in each case is exactly what we would obtain letting $ n \to \infty $ in the calculation of of $ p_{1,1}(n) $ in the previous example sheet.






\section{QUESTION 13}
\section{QUESTION 14}

\begin{enumerate}[label = (\alph*)]
	\item Detailed balance equations are
	
	\begin{align*}
	\pi_{1}(1-p)  & = \pi_{2} q \\
	\end{align*}
	
	Hence reversible.
	

	
	\item Detailed balance equations are
	
	\begin{align*}
	\pi_{1} p & = \pi_{2}(1-p)  \\
	\pi_{2} p & = \pi_{3}(1-p) \\
	\pi_{3} p & = \pi_{1}(1-p) 
	\end{align*}
	
	Hence reversible, except when $ p = \frac{1}{2} $
	
	\item Have
	
	\[ \pi_{i} = \pi_{i+1}(1-p) \Rightarrow \pi_{i} = \frac{p^{i+1}}{(1-p)^{i}}\pi_{0} \]
	
	Hence reversible.
	
	\item Detailed balance equations give $ \pi_{i} = \frac{1}{n} $ for each $ i $. Thus reversible. 
	
	
	
	
\end{enumerate}


\section{QUESTION 15}

	Detailed balance equations give
	
	\[ \pi_{i}\lambda_{i} = \pi_{i+1}\mu_{i+1} \Rightarrow \pi_{i} = \frac{\lambda_{i-1}\cdots\lambda_{0}}{\mu_{i}\cdots\mu_{1}} \pi_{0} \]
	
	Hence reversible.


\end{document}

