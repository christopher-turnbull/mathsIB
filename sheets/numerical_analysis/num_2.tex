\documentclass[a4paper]{article}
\usepackage{amsmath}
\def\npart {IB}
\def\nterm {Lent}
\def\nyear {2018}
\def\nlecturer {Dr. Saxton}
\def\ncourse {Numerical Analysis Example Sheet 2}

% Imports
\ifx \nauthor\undefined
  \def\nauthor{Christopher Turnbull}
\else
\fi

\author{Supervised by \nlecturer \\\small Solutions presented by \nauthor}
\date{\nterm\ \nyear}

\usepackage{alltt}
\usepackage{amsfonts}
\usepackage{amsmath}
\usepackage{amssymb}
\usepackage{amsthm}
\usepackage{booktabs}
\usepackage{caption}
\usepackage{enumitem}
\usepackage{fancyhdr}
\usepackage{graphicx}
\usepackage{mathdots}
\usepackage{mathtools}
\usepackage{microtype}
\usepackage{multirow}
\usepackage{pdflscape}
\usepackage{pgfplots}
\usepackage{siunitx}
\usepackage{slashed}
\usepackage{tabularx}
\usepackage{tikz}
\usepackage{tkz-euclide}
\usepackage[normalem]{ulem}
\usepackage[all]{xy}
\usepackage{imakeidx}

\makeindex[intoc, title=Index]
\indexsetup{othercode={\lhead{\emph{Index}}}}

\ifx \nextra \undefined
  \usepackage[pdftex,
    hidelinks,
    pdfauthor={Christopher Turnbull},
    pdfsubject={Cambridge Maths Notes: Part \npart\ - \ncourse},
    pdftitle={Part \npart\ - \ncourse},
  pdfkeywords={Cambridge Mathematics Maths Math \npart\ \nterm\ \nyear\ \ncourse}]{hyperref}
  \title{Part \npart\ --- \ncourse}
\else
  \usepackage[pdftex,
    hidelinks,
    pdfauthor={Christopher Turnbull},
    pdfsubject={Cambridge Maths Notes: Part \npart\ - \ncourse\ (\nextra)},
    pdftitle={Part \npart\ - \ncourse\ (\nextra)},
  pdfkeywords={Cambridge Mathematics Maths Math \npart\ \nterm\ \nyear\ \ncourse\ \nextra}]{hyperref}

  \title{Part \npart\ --- \ncourse \\ {\Large \nextra}}
  \renewcommand\printindex{}
\fi

\pgfplotsset{compat=1.12}

\pagestyle{fancyplain}
\lhead{\emph{\nouppercase{\leftmark}}}
\ifx \nextra \undefined
  \rhead{
    \ifnum\thepage=1
    \else
      \npart\ \ncourse
    \fi}
\else
  \rhead{
    \ifnum\thepage=1
    \else
      \npart\ \ncourse\ (\nextra)
    \fi}
\fi
\usetikzlibrary{arrows.meta}
\usetikzlibrary{decorations.markings}
\usetikzlibrary{decorations.pathmorphing}
\usetikzlibrary{positioning}
\usetikzlibrary{fadings}
\usetikzlibrary{intersections}
\usetikzlibrary{cd}

\newcommand*{\Cdot}{{\raisebox{-0.25ex}{\scalebox{1.5}{$\cdot$}}}}
\newcommand {\pd}[2][ ]{
  \ifx #1 { }
    \frac{\partial}{\partial #2}
  \else
    \frac{\partial^{#1}}{\partial #2^{#1}}
  \fi
}
\ifx \nhtml \undefined
\else
  \renewcommand\printindex{}
  \makeatletter
  \DisableLigatures[f]{family = *}
  \let\Contentsline\contentsline
  \renewcommand\contentsline[3]{\Contentsline{#1}{#2}{}}
  \renewcommand{\@dotsep}{10000}
  \newlength\currentparindent
  \setlength\currentparindent\parindent

  \newcommand\@minipagerestore{\setlength{\parindent}{\currentparindent}}
  \usepackage[active,tightpage,pdftex]{preview}
  \renewcommand{\PreviewBorder}{0.1cm}

  \newenvironment{stretchpage}%
  {\begin{preview}\begin{minipage}{\hsize}}%
    {\end{minipage}\end{preview}}
  \AtBeginDocument{\begin{stretchpage}}
  \AtEndDocument{\end{stretchpage}}

  \newcommand{\@@newpage}{\end{stretchpage}\begin{stretchpage}}

  \let\@real@section\section
  \renewcommand{\section}{\@@newpage\@real@section}
  \let\@real@subsection\subsection
  \renewcommand{\subsection}{\@@newpage\@real@subsection}
  \makeatother
\fi

% Theorems
\theoremstyle{definition}
\newtheorem*{aim}{Aim}
\newtheorem*{axiom}{Axiom}
\newtheorem*{claim}{Claim}
\newtheorem*{cor}{Corollary}
\newtheorem*{conjecture}{Conjecture}
\newtheorem*{defi}{Definition}
\newtheorem*{eg}{Example}
\newtheorem*{ex}{Exercise}
\newtheorem*{fact}{Fact}
\newtheorem*{law}{Law}
\newtheorem*{lemma}{Lemma}
\newtheorem*{notation}{Notation}
\newtheorem*{prop}{Proposition}
\newtheorem*{soln}{Solution}
\newtheorem*{thm}{Theorem}

\newtheorem*{remark}{Remark}
\newtheorem*{warning}{Warning}
\newtheorem*{exercise}{Exercise}

\newtheorem{nthm}{Theorem}[section]
\newtheorem{nlemma}[nthm]{Lemma}
\newtheorem{nprop}[nthm]{Proposition}
\newtheorem{ncor}[nthm]{Corollary}


\renewcommand{\labelitemi}{--}
\renewcommand{\labelitemii}{$\circ$}
\renewcommand{\labelenumi}{(\roman{*})}

\let\stdsection\section
\renewcommand\section{\newpage\stdsection}

% Strike through
\def\st{\bgroup \ULdepth=-.55ex \ULset}

% Maths symbols
\newcommand{\abs}[1]{\left\lvert #1\right\rvert}
\newcommand\ad{\mathrm{ad}}
\newcommand\AND{\mathsf{AND}}
\newcommand\Art{\mathrm{Art}}
\newcommand{\Bilin}{\mathrm{Bilin}}
\newcommand{\bket}[1]{\left\lvert #1\right\rangle}
\newcommand{\B}{\mathcal{B}}
\newcommand{\bolds}[1]{{\bfseries #1}}
\newcommand{\brak}[1]{\left\langle #1 \right\rvert}
\newcommand{\braket}[2]{\left\langle #1\middle\vert #2 \right\rangle}
\newcommand{\bra}{\langle}
\newcommand{\cat}[1]{\mathsf{#1}}
\newcommand{\C}{\mathbb{C}}
\newcommand{\CP}{\mathbb{CP}}
\newcommand{\cU}{\mathcal{U}}
\newcommand{\Der}{\mathrm{Der}}
\newcommand{\D}{\mathrm{D}}
\newcommand{\dR}{\mathrm{dR}}
\newcommand{\E}{\mathbb{E}}
\newcommand{\F}{\mathbb{F}}
\newcommand{\Frob}{\mathrm{Frob}}
\newcommand{\GG}{\mathbb{G}}
\newcommand{\gl}{\mathfrak{gl}}
\newcommand{\GL}{\mathrm{GL}}
\newcommand{\G}{\mathcal{G}}
\newcommand{\Gr}{\mathrm{Gr}}
\newcommand{\haut}{\mathrm{ht}}
\newcommand{\Id}{\mathrm{Id}}
\newcommand{\ket}{\rangle}
\newcommand{\lie}[1]{\mathfrak{#1}}
\newcommand{\Mat}{\mathrm{Mat}}
\newcommand{\N}{\mathbb{N}}
\newcommand{\norm}[1]{\left\lVert #1\right\rVert}
\newcommand{\normalorder}[1]{\mathop{:}\nolimits\!#1\!\mathop{:}\nolimits}
\newcommand\NOT{\mathsf{NOT}}
\newcommand{\Oc}{\mathcal{O}}
\newcommand{\Or}{\mathrm{O}}
\newcommand\OR{\mathsf{OR}}
\newcommand{\ort}{\mathfrak{o}}
\newcommand{\PGL}{\mathrm{PGL}}
\newcommand{\ph}{\,\cdot\,}
\newcommand{\pr}{\mathrm{pr}}
\newcommand{\Prob}{\mathbb{P}}
\newcommand{\PSL}{\mathrm{PSL}}
\newcommand{\Ps}{\mathcal{P}}
\newcommand{\PSU}{\mathrm{PSU}}
\newcommand{\pt}{\mathrm{pt}}
\newcommand{\qeq}{\mathrel{``{=}"}}
\newcommand{\Q}{\mathbb{Q}}
\newcommand{\R}{\mathbb{R}}
\newcommand{\RP}{\mathbb{RP}}
\newcommand{\Rs}{\mathcal{R}}
\newcommand{\SL}{\mathrm{SL}}
\newcommand{\so}{\mathfrak{so}}
\newcommand{\SO}{\mathrm{SO}}
\newcommand{\Spin}{\mathrm{Spin}}
\newcommand{\Sp}{\mathrm{Sp}}
\newcommand{\su}{\mathfrak{su}}
\newcommand{\SU}{\mathrm{SU}}
\newcommand{\term}[1]{\emph{#1}\index{#1}}
\newcommand{\T}{\mathbb{T}}
\newcommand{\tv}[1]{|#1|}
\newcommand{\U}{\mathrm{U}}
\newcommand{\uu}{\mathfrak{u}}
\newcommand{\Vect}{\mathrm{Vect}}
\newcommand{\wsto}{\stackrel{\mathrm{w}^*}{\to}}
\newcommand{\wt}{\mathrm{wt}}
\newcommand{\wto}{\stackrel{\mathrm{w}}{\to}}
\newcommand{\Z}{\mathbb{Z}}
\renewcommand{\d}{\mathrm{d}}
\renewcommand{\H}{\mathbb{H}}
\renewcommand{\P}{\mathbb{P}}
\renewcommand{\sl}{\mathfrak{sl}}
\renewcommand{\vec}[1]{\boldsymbol{\mathbf{#1}}}
%\renewcommand{\F}{\mathcal{F}}

\let\Im\relax
\let\Re\relax

\DeclareMathOperator{\adj}{adj}
\DeclareMathOperator{\Ann}{Ann}
\DeclareMathOperator{\area}{area}
\DeclareMathOperator{\Aut}{Aut}
\DeclareMathOperator{\Bernoulli}{Bernoulli}
\DeclareMathOperator{\betaD}{beta}
\DeclareMathOperator{\bias}{bias}
\DeclareMathOperator{\binomial}{binomial}
\DeclareMathOperator{\card}{card}
\DeclareMathOperator{\ccl}{ccl}
\DeclareMathOperator{\Char}{char}
\DeclareMathOperator{\ch}{ch}
\DeclareMathOperator{\cl}{cl}
\DeclareMathOperator{\cls}{\overline{\mathrm{span}}}
\DeclareMathOperator{\conv}{conv}
\DeclareMathOperator{\corr}{corr}
\DeclareMathOperator{\cosec}{cosec}
\DeclareMathOperator{\cosech}{cosech}
\DeclareMathOperator{\cov}{cov}
\DeclareMathOperator{\covol}{covol}
\DeclareMathOperator{\diag}{diag}
\DeclareMathOperator{\diam}{diam}
\DeclareMathOperator{\Diff}{Diff}
\DeclareMathOperator{\disc}{disc}
\DeclareMathOperator{\dom}{dom}
\DeclareMathOperator{\End}{End}
\DeclareMathOperator{\energy}{energy}
\DeclareMathOperator{\erfc}{erfc}
\DeclareMathOperator{\erf}{erf}
\DeclareMathOperator*{\esssup}{ess\,sup}
\DeclareMathOperator{\ev}{ev}
\DeclareMathOperator{\Ext}{Ext}
\DeclareMathOperator{\Fit}{Fit}
\DeclareMathOperator{\fix}{fix}
\DeclareMathOperator{\Frac}{Frac}
\DeclareMathOperator{\Gal}{Gal}
\DeclareMathOperator{\gammaD}{gamma}
\DeclareMathOperator{\gr}{gr}
\DeclareMathOperator{\hcf}{hcf}
\DeclareMathOperator{\Hom}{Hom}
\DeclareMathOperator{\id}{id}
\DeclareMathOperator{\image}{image}
\DeclareMathOperator{\im}{im}
\DeclareMathOperator{\Im}{Im}
\DeclareMathOperator{\Ind}{Ind}
\DeclareMathOperator{\Int}{Int}
\DeclareMathOperator{\Isom}{Isom}
\DeclareMathOperator{\lcm}{lcm}
\DeclareMathOperator{\length}{length}
\DeclareMathOperator{\Lie}{Lie}
\DeclareMathOperator{\like}{like}
\DeclareMathOperator{\Lk}{Lk}
\DeclareMathOperator{\mse}{mse}
\DeclareMathOperator{\multinomial}{multinomial}
\DeclareMathOperator{\orb}{orb}
\DeclareMathOperator{\ord}{ord}
\DeclareMathOperator{\otp}{otp}
\DeclareMathOperator{\Poisson}{Poisson}
\DeclareMathOperator{\poly}{poly}
\DeclareMathOperator{\rank}{rank}
\DeclareMathOperator{\rel}{rel}
\DeclareMathOperator{\Re}{Re}
\DeclareMathOperator*{\res}{res}
\DeclareMathOperator{\Res}{Res}
\DeclareMathOperator{\rk}{rk}
\DeclareMathOperator{\Root}{Root}
\DeclareMathOperator{\sech}{sech}
\DeclareMathOperator{\sgn}{sgn}
\DeclareMathOperator{\spn}{span}
\DeclareMathOperator{\stab}{stab}
\DeclareMathOperator{\St}{St}
\DeclareMathOperator{\supp}{supp}
\DeclareMathOperator{\Syl}{Syl}
\DeclareMathOperator{\Sym}{Sym}
\DeclareMathOperator{\tr}{tr}
\DeclareMathOperator{\Tr}{Tr}
\DeclareMathOperator{\var}{var}
\DeclareMathOperator{\vol}{vol}

\pgfarrowsdeclarecombine{twolatex'}{twolatex'}{latex'}{latex'}{latex'}{latex'}
\tikzset{->/.style = {decoration={markings,
                                  mark=at position 1 with {\arrow[scale=2]{latex'}}},
                      postaction={decorate}}}
\tikzset{<-/.style = {decoration={markings,
                                  mark=at position 0 with {\arrowreversed[scale=2]{latex'}}},
                      postaction={decorate}}}
\tikzset{<->/.style = {decoration={markings,
                                   mark=at position 0 with {\arrowreversed[scale=2]{latex'}},
                                   mark=at position 1 with {\arrow[scale=2]{latex'}}},
                       postaction={decorate}}}
\tikzset{->-/.style = {decoration={markings,
                                   mark=at position #1 with {\arrow[scale=2]{latex'}}},
                       postaction={decorate}}}
\tikzset{-<-/.style = {decoration={markings,
                                   mark=at position #1 with {\arrowreversed[scale=2]{latex'}}},
                       postaction={decorate}}}
\tikzset{->>/.style = {decoration={markings,
                                  mark=at position 1 with {\arrow[scale=2]{latex'}}},
                      postaction={decorate}}}
\tikzset{<<-/.style = {decoration={markings,
                                  mark=at position 0 with {\arrowreversed[scale=2]{twolatex'}}},
                      postaction={decorate}}}
\tikzset{<<->>/.style = {decoration={markings,
                                   mark=at position 0 with {\arrowreversed[scale=2]{twolatex'}},
                                   mark=at position 1 with {\arrow[scale=2]{twolatex'}}},
                       postaction={decorate}}}
\tikzset{->>-/.style = {decoration={markings,
                                   mark=at position #1 with {\arrow[scale=2]{twolatex'}}},
                       postaction={decorate}}}
\tikzset{-<<-/.style = {decoration={markings,
                                   mark=at position #1 with {\arrowreversed[scale=2]{twolatex'}}},
                       postaction={decorate}}}

\tikzset{circ/.style = {fill, circle, inner sep = 0, minimum size = 3}}
\tikzset{mstate/.style={circle, draw, blue, text=black, minimum width=0.7cm}}

\tikzset{commutative diagrams/.cd,cdmap/.style={/tikz/column 1/.append style={anchor=base east},/tikz/column 2/.append style={anchor=base west},row sep=tiny}}

\definecolor{mblue}{rgb}{0.2, 0.3, 0.8}
\definecolor{morange}{rgb}{1, 0.5, 0}
\definecolor{mgreen}{rgb}{0.1, 0.4, 0.2}
\definecolor{mred}{rgb}{0.5, 0, 0}

\def\drawcirculararc(#1,#2)(#3,#4)(#5,#6){%
    \pgfmathsetmacro\cA{(#1*#1+#2*#2-#3*#3-#4*#4)/2}%
    \pgfmathsetmacro\cB{(#1*#1+#2*#2-#5*#5-#6*#6)/2}%
    \pgfmathsetmacro\cy{(\cB*(#1-#3)-\cA*(#1-#5))/%
                        ((#2-#6)*(#1-#3)-(#2-#4)*(#1-#5))}%
    \pgfmathsetmacro\cx{(\cA-\cy*(#2-#4))/(#1-#3)}%
    \pgfmathsetmacro\cr{sqrt((#1-\cx)*(#1-\cx)+(#2-\cy)*(#2-\cy))}%
    \pgfmathsetmacro\cA{atan2(#2-\cy,#1-\cx)}%
    \pgfmathsetmacro\cB{atan2(#6-\cy,#5-\cx)}%
    \pgfmathparse{\cB<\cA}%
    \ifnum\pgfmathresult=1
        \pgfmathsetmacro\cB{\cB+360}%
    \fi
    \draw (#1,#2) arc (\cA:\cB:\cr);%
}
\newcommand\getCoord[3]{\newdimen{#1}\newdimen{#2}\pgfextractx{#1}{\pgfpointanchor{#3}{center}}\pgfextracty{#2}{\pgfpointanchor{#3}{center}}}

\def\Xint#1{\mathchoice
   {\XXint\displaystyle\textstyle{#1}}%
   {\XXint\textstyle\scriptstyle{#1}}%
   {\XXint\scriptstyle\scriptscriptstyle{#1}}%
   {\XXint\scriptscriptstyle\scriptscriptstyle{#1}}%
   \!\int}
\def\XXint#1#2#3{{\setbox0=\hbox{$#1{#2#3}{\int}$}
     \vcenter{\hbox{$#2#3$}}\kern-.5\wd0}}
\def\ddashint{\Xint=}
\def\dashint{\Xint-}

\newcommand\separator{{\centering\rule{2cm}{0.2pt}\vspace{2pt}\par}}

\newenvironment{own}{\color{gray!70!black}}{}

\newcommand\makecenter[1]{\raisebox{-0.5\height}{#1}}

\newtheorem*{soln}{Solution}

\renewcommand{\thesection}{}
\renewcommand{\thesubsection}{\arabic{section}.\arabic{subsection}}
\makeatletter
\def\@seccntformat#1{\csname #1ignore\expandafter\endcsname\csname the#1\endcsname\quad}
\let\sectionignore\@gobbletwo
\let\latex@numberline\numberline
\def\numberline#1{\if\relax#1\relax\else\latex@numberline{#1}\fi}
\makeatother


\begin{document}
	
\maketitle

\section{QUESTION 1}

The differential equations, with initial condition $ y(0)= 1 $ have exact solutions given by

\[ y = \frac{1}{1+t} \quad \text{ and } \quad y = (1 + t)^{2}, \quad 0 \leq t \leq 1\]

respectively.

Using the Euler method for the first ODE we have $ f(t,y) = -\frac{y}{1+t} $.

Here, $ y_{0} = 1, t_{m} = mh$. For $ n \geq 1 $,

\begin{align*}
y_{n} & = y_{n-1} + h f(t_{n-1},y_{n-1}) \\
& = y_{n-1} \left(  1 - \frac{h}{1+(n-1)h} \right)  \\
& = y_{n-1} \cdot \frac{1 + (n-2)h}{1 + (n-1)h} 
\end{align*}

Have that $ y_{1} = 1 - h $, thus

\begin{align*}
y_{n} & =  1 \cdot (1-h) \left( \frac{1}{1+h} \right)\left( \frac{1+h}{1+2h} \right) \cdots \left(  \frac{1 + (n-3)h}{1 + (n-2)h}  \right)   \left( \frac{1 + (n-2)h}{1 + (n-1)h}  \right)  \\
& = \frac{1-h}{1 + (n-1)h}
\end{align*}

As $  h \to 0 $, $ n \to \infty $ in such a way that $ nh \to t $. So we deduce $ y = 1/(1+t) $ as required.

Moreover, the error is 


\begin{align*}
y_{n} - y(nh) & = \frac{1-h}{1 + (n-1)h} - \frac{1}{1+nh} \\
\end{align*}

which is clearly $ O(h) $.




For the second ODE we have $ f(t,y) = \frac{2y}{1+t} $.

Calculating the first few terms we find that

\begin{align*}
y_{1} & = y_{0} \left(  1 + \frac{2h}{1+t_{0}} \right) \qquad t_{0} = 0 \\
& = (1+2h)
\end{align*}

\begin{align*}
y_{2} & = y_{1} \left(  1 + \frac{2h}{1+t_{1}} \right) \qquad t_{1} = h \\
& = (1+2h) \cdot \left(  \frac{1+3h}{1+h}  \right) 
\end{align*}


\begin{align*}
y_{3} & = y_{2} \left(  1 + \frac{2h}{1+t_{1}} \right) \qquad t_{2} = 2h \\
& = (1+2h) \cdot \left(  \frac{1+3h}{1+h}  \right) \cdot \left(  \frac{1+4h}{1+2h} \right) 
\end{align*}

and so

\begin{align*}
y_{n} & = (1+2h) \cdot \left(  \frac{1+3h}{1+h}  \right) \cdot \left(  \frac{1+4h}{1+2h} \right) \cdots \left(  \frac{1 + (n+1)h}{1 + (n-1)h} \right)  \\
& = \frac{(1+nh)(1+(n+1)h)}{1+h} 
\end{align*}

again $ nh \to t $, so we have the result as required. 

Here, the error is 


\begin{align*}
y_{n} - y(nh) & = \frac{(1+nh)(1+(n+1)h)}{1+h} - (1 + nh)^{2} \\
& =  \frac{(1+nh)^{2} + h(1 + nh) -(1+h)(1 + nh)^{2}  }{1+h} 
\end{align*}

which is clearly $ O(h) $. 








\section{QUESTION 2}

The trapezoidal rule states that the numerical solution to the differential equation

\[ \mathbf{y}' = \mathbf{f}(t,\mathbf{y}_{n}) \qquad (2.1) \]

is given by

\[ \mathbf{y}_{n+1} = \mathbf{y}_{n} + \frac{1}{2} h [ \mathbf{f}(t_{n},\mathbf{y}_{n}) + \mathbf{f}(t_{n+1},\mathbf{y}_{n+1})  ] \qquad (2.2) \]

Assuming that $ \mathbf{f} $ satisfies the Lipschitz condition: there exists $ \lambda  \geq 0 $ such that

\[ | | \mathbf{f}(t,\mathbf{v}) - \mathbf{f}(t,\mathbf{w}) | | \leq \lambda | |  \mathbf{v} - \mathbf{w} | |, \quad t \in [0,t^{*}], \quad \mathbf{v},\mathbf{w} \in \R^{N} \]

we will prove that the trapezoidal rule converges; ie. 

\[ \lim\limits_{h \to 0} \max_{n = 0,\cdots,\lfloor t^{*} / h \rfloor } | |  \mathbf{y}_{n}(h) - \mathbf{y}(nh)  | | = 0 \]

where $ \mathbf{y}(nh) $ is the evaluation at time $ t = nh $ of the exact solution of (2.1).




\begin{proof}
	Let $ \mathbf{e}_{n} = \mathbf{y}_{n} - \mathbf{y}(t_{n}) $, the error at step $ n $, where $ 0 \leq n \leq t^{*} / h $, $ t_{n} := n h $. Thus,
	
	\[ \mathbf{e}_{n+1} = \mathbf{y}_{n+1} - \mathbf{y}(t_{n+1}) = [  \mathbf{y}_{n} + \frac{1}{2} h [ \mathbf{f}(t_{n},\mathbf{y}_{n}) + \mathbf{f}(t_{n+1},\mathbf{y}_{n+1})  ] ] - [  \mathbf{y}(t_{n}) + h \mathbf{y}'(t_{n}) + O(h^{2}) ] \]
	
	By the Taylor theorem, the $ O(h^{2}) $ term can be bounded uniformly for all $ [0,t^{*}] $ by $ ch^{2} $, where $ c > 0 $. Thus, using (2.1) and the triangle inequality, 
	
	\begin{align*}
	| | \mathbf{e}_{n+1} | |  & \leq | | \mathbf{y}_{n} - \mathbf{y}(t_{n}) | | + h | | \frac{1}{2} \{ \mathbf{f}(t_{n},\mathbf{y}_{n}) + \mathbf{f}(t_{n+1},\mathbf{y}_{n+1}) \} - \mathbf{f}(t_{n},\mathbf{y}(t_{n})) | | + ch^{2} \\
	& = 
	\end{align*}
	
	Want this in terms of $ | | \mathbf{e}_{n} | | $, but how?
\end{proof}

\section{QUESTION 3}
The $ s $-step Adams-Bashforth method is of order $ s $ and has the form

\[ \mathbf{y}_{n+s} - \mathbf{y}_{n+s-1} = h \sum_{j=0}^{s-1} \sigma_{j} \mathbf{f}(t_{n+j},\mathbf{y}_{n+j}) \]

For $ s =3 $ we have  $ \rho(w) = w^{2}(w - 1) $.  To maximize order, we let $ \sigma $  be the $ 2 $ degree polynomial ($ \sigma_{3} = 0 $) arising from the truncation of the Taylor expanison of

\[ \frac{\rho(w)}{\log w} \]


Letting $ \xi = w - 1 $ and expanding,

\begin{align*}
\frac{w^{2}(w-1)}{\log w} & = \frac{(\xi + 1)^{2}\xi}{\log (1 + \xi)} = \frac{\xi + 2 \xi^{2} + \xi^{3}}{\xi - \frac{1}{2} \xi^{2} + \frac{1}{3} \xi^{3} - \cdots } \\
& = \frac{1 + 2 \xi + \xi^{2}}{1 - \frac{1}{2} \xi + \frac{1}{3} \xi^{2} - \cdots} \\
& = [1 + 2 \xi + \xi^{2}][ 1 + (\frac{1}{2} \xi - \frac{1}{3} \xi^{2}) + (\frac{1}{2} \xi - \frac{1}{3} \xi^{2})^{2} + O(\xi^{3})     ] \\
& = 1 + \frac{5}{2} \xi + \frac{5}{3} \xi^{2} + O(\xi^{3}) \\
& = 1 + \frac{5}{2} (w - 1) + \frac{5}{3} (w - 1)^{2} + O(| w-1 |^{3}) \\
& = \frac{1}{6} - \frac{5}{3} w + \frac{5}{3} w^{2} +  O(| w-1 |^{3})
\end{align*}

Therefore $ \sigma_{0} = \frac{1}{6} , \sigma_{1} = - \frac{5}{3}  , \sigma_{2} = \frac{5}{3}  , \sigma_{3} = 0 $


\section{QUESTION 4}

Applying the \emph{explicit midpoint rule}

\[ \mathbf{y}_{n+2} = \mathbf{y}_{n} + 2 h \mathbf{f}(t_{n+1},\mathbf{y}_{n+1}) \]

to the ODE $ y' = - y $, we have


\[ y_{n+2} = y_{n} - 2hy_{n+1} \]

Making the ansatz $ y_{n} = k^{n} $ gives

\[ k^{2} + 2hk - 1 = 0 \]

and hence

\[ k = -h  \pm \sqrt{h^{2} - 1} \]

giving 

\[ y_{n} = A \left(   -h  - \sqrt{h^{2} - 1} \right)^{n} + B  \left(   -h  + \sqrt{h^{2} - 1} \right)^{n}   \]

Now $ y_{0} = 1 \Rightarrow A + B = 1 $, and $ y_{1} = 1 - h \Rightarrow 1 = (B - A) \sqrt{h^{2} - 1} $, thus 

\[ A = \frac{1}{2 \sqrt{h^{2} - 1}} + \frac{1}{2}  \]
\[ B = \frac{1}{2 \sqrt{h^{2} - 1}} - \frac{1}{2}  \]

Now as $ n \to \infty $, we wish to show that $ y_{n} $ diverges, ie. one of the terms blow up, and we want to show this happens for all $ h > 0 $. Can see that if $ h > 1 $, the $ A \left(   -h  - \sqrt{h^{2} - 1} \right)^{n} $ explodes as $ -h  - \sqrt{h^{2} - 1} < - 1 $. If $ 0 < h < 1 $, I don't know...







\section{QUESTION 5}

The multistep method

\[ \sum_{j=0}^{3} \rho_{j} \mathbf{y}_{n+j} = h \sum_{j=0}^{2} \sigma_{j} \mathbf{f}(t_{n+j}, \mathbf{y}_{n+j} ), \quad \rho_{3} = 1 \]

is of order $ 4 $ iff

\[ \rho(e^{z}) - z \sigma(e^{z}) = O(z^{5}), \quad z \to 0 \]

Expanding into Taylor series,


\[ e^{z} = 1 + z + \frac{1}{2} z^{2} + \frac{1}{6} z^{3} + \frac{1}{24} z^{4} + O(z^{5}) \]

\[ e^{2z} = 1 + 2z + 2z^{2} + \frac{4}{3} z^{3} + \frac{5}{8} z^{4} + O(z^{5})  \]

\[ e^{3z} = 1 + 3z + \frac{9}{2} z^{2} + \frac{9}{2} z^{3} + \frac{10}{3} z^{4} + O(z^{5})    \]

\begin{align*}
\rho(e^{z}) - z \sigma(e^{z}) & = [1 + 3z + \frac{9}{2} z^{2} + \frac{9}{2} z^{3} + \frac{10}{3} z^{4}] + \rho_{2} [1 + 2z + 2z^{2} + \frac{4}{3} z^{3} + \frac{5}{8} z^{4}] \\
& \quad + \rho_{1} [1 + z + \frac{1}{2} z^{2} + \frac{1}{6} z^{3} + \frac{1}{24} z^{4}] + \rho_{0} - z \sigma_{2} \left[  1 + 2z + 2z^{2} + \frac{4}{3} z^{3} + \frac{5}{8} z^{4} \right]    \\
& \quad - z \sigma_{1} \left[  1 + z + \frac{1}{2} z^{2} + \frac{1}{6} z^{3} + \frac{1}{24} z^{4} \right]  - z \sigma_{0} \\
\end{align*}

For this expression to be $ O(z^{5}) $, looking at first order terms we deduce that $ \rho_{1} + \rho_{2} + \rho_{3} = - 1 $. 

So we have $ \rho(w) = w^{3} + \rho_{2} w^{2} - 9 w + \rho_{0} $, $ \rho_{0} + \rho_{2} = 8 $ for this to satisfy the root condition we must have all zeros residing in $ | w | \leq 1 $, and all zeros of unit modulus simple. 






\section{QUESTION 6}

starred


\section{QUESTION 7}

Consider the ODE $ y' = y $ with $ y(0) = 1 $ whose solution is $ y(t) = e^{t} $. For this ODE we can write the local error explicitly: indeed we have 

\begin{align*}
k_{1} & = f(t_{n},y(t_{n})) = e^{t_{n}} \\
k_{2} & = y(t_{n}) + \frac{1}{3} h k_{1} = e^{t_{n}} (1 + \frac{1}{3} h) \\
k_{3} & = y(t_{n}) - \frac{1}{3} h k_{1} + h k_{2} = e^{t_{n}}\left(  1 + \frac{2}{3}h + \frac{1}{3} h^{2}  \right)    \\
k_{4} & =  y(t_{n}) + h k_{1} - h k_{2} + h k_{3} = e^{t_{n}}\left(  1 + h + \frac{1}{3} h^{2} + \frac{1}{3} h^{3} \right) 
\end{align*} 

Then the local error is 

\begin{align*}
y(t_{n+1}) -  ( y(t_{n}) + \frac{1}{8} h k_{1} + \frac{3}{8} h k_{2} + \frac{3}{8} h k_{3} + \frac{1}{8} h k_{4}   ) & =  e^{t_{n+1}} - e^{t_{n}} - e^{t_{n}}  \\
& = 
\end{align*}




Let us now show that the method has order at least 3. To do this we restrict our attention to scalar, autonomous equations of the form $ y' = f(y) $. 

\section{QUESTION 8}

Consider the linear scalar system 

\[ \begin{cases} y' = \lambda y  & \\ y(0) = 1&  \end{cases} \]

where $ \lambda < 0 $. The solution is $ y(t) = e^{\lambda t} $ which decays to $ 0 $ as $ t \to \infty $. 

\begin{enumerate}
	\item For the explicit Euler method we get $ y_{n+1} = y_{n} + h \lambda y_{n} $ whose solution is $ y_{n} = (1+ h \lambda)^{n} $, so $ y_{n} \to 0 $ iff $ | 1 + h \lambda | < 1 $, therefore $ \mathcal{D} = \{ z \in \C \; ; \; | 1 + z | < 1  \} $, and $ \mathcal{D} \cap \R =  $
	
	\item Considering now the trapezoidal rule we get $ y_{n+1} = [  (1+\frac{1}{2} h \lambda) / (1 - \frac{1}{2} h \lambda) ] y_{n} $, and thus by induction, $ y_{n} = [  (1+\frac{1}{2} h \lambda) / (1 - \frac{1}{2} h \lambda) ]^{n} y_{0} $. Therefore
	
	\[ z \in \mathcal{D} \iff \left| \frac{1+\frac{1}{2}z}{1 - \frac{1}{2} z} \right| < 1 \iff \text{Re } z < 0  \]
	
	and we deduce that $ \mathcal{D} = \C^{-} $. Hence the method is A-stable. 
	
	\item 
	
	\item 
	
	\item Applying the RK method to $ y' = \lambda y $ we have
	
	\begin{align*}
	h k_{1} & = h \lambda y_{n} \\
	h k_{2} & = h \lambda ( y_{n} + h k_{1}) 
	\end{align*}
	
	therefore
	
	\[ y_{n+1} = y_{n} + \frac{1}{2} h k_{1} + \frac{1}{2} h k_{2} = (1+ h \lambda  + \frac{1}{2} h^{2} \lambda^{2} ) y_{n}    \]
	
	Let 
	
	\[ r(z) = 1 + z + \frac{1}{2} z^{2}  \]
	
	Then $ y_{n+1} = r(h \lambda) y_{n} $, therefore, by induction, $ y_{n} = [r(h \lambda)]^{n} y_{0} $ and we deduce that 
	
	\[ \mathcal{D} = \{  z \in \C \; ; \; | r(z) | < 1 \} \]
	
	$ r $ is analytic in $ \mathcal{V} = \{  z \in \C \; ; \; \text{Re } z < \leq 0 \} $. Therefore it attains its maximum on $ \partial \mathcal{V} = i \R $. 
	
\end{enumerate}

\section{QUESTION 9}

Consider the two-step BDF method: $ \mathbf{y}_{n+2} - \frac{4}{3} \mathbf{y}_{n+1} + \frac{1}{3} \mathbf{y}_{n} = \frac{2}{3} hf(t_{n+2},\mathbf{y}_{n+2}) $. Applied to $ y' = \lambda y $ we get 

\[ y_{n+2} - \frac{4}{3} y_{n+1} + \frac{1}{3} y_{n} = \frac{2}{3} h \lambda y_{n+2} \]

\[ (3 - 2h \lambda)y_{n+2} - 4 y_{n+1} + y_{n} = 0 \]

We try $ y_{n} = k^{n} $ and obtain 

\[ (3 - 2h \lambda)k^{2} - 4 k + 1 = 0 \]

So

\begin{align*}
k & = \frac{4 \pm \sqrt{16 - 4(3 - 2h \lambda)}}{(6 - 4h \lambda)} \\
& = \frac{2 \pm \sqrt{1 + 2 h \lambda}}{3 - 2 h \lambda} 
\end{align*}

Hence 

\[ y_{n} = A \left( \frac{2 + \sqrt{1 + 2 h \lambda}}{3 - 2 h \lambda} \right)^{n}  + B \left( \frac{2 - \sqrt{1 + 2 h \lambda}}{3 - 2 h \lambda} \right)^{n}  \]

Unsure what to deduce.


\section{QUESTION 10}

Given that $ | y_{n} - y(t_{n}) |  \leq  10^{-6} $, with Euler's method, setting $ h = 2 \times 10^{-4} $, we have

\begin{align*}
| y_{n+1} - y(t_{n+1}) |  & = | y_{n} + h_{n}f(t_{n},y_{n})   - t_{n}^{-1} |  \\
& = | y_{n} + 2 \times 10^{-4} (   -10^{4}  (y_{n} - t_{n}^{-1} ) - t_{n}^{-2} )    - t_{n+1}^{-1} | \\
& = |  (1 + 2 \times 10^{-8}) y_{n} - (2 \times 10^{-8}) t_{n}^{-1} - 2 \times 10^{-4}) t_{n}^{-2} - t_{n+1}^{-1}   |
\end{align*}

No idea what I'm doing here. 


\section{QUESTION 11}

First consider the predictor; substituting the true solution

\[ \mathbf{y}(t_{n+3}) - \{  - \frac{1}{2} \mathbf{y}(t_{n}) + 3 \mathbf{y}(t_{n+1})  - \frac{3}{2} \mathbf{y}(t_{n+2}) + 3h\mathbf{y}'(t_{n+2})  \} \quad (*) \]

Performing Taylor expansions:

\[ \mathbf{y}(t_{n+3}) = \mathbf{y}(t_{n}) + 3h \mathbf{y}'(t_{n}) + \frac{9}{2} h^{2} \mathbf{y}''(t_{n}) + \frac{9}{2} h^{3}  \mathbf{y}'''(t_{n}) + \frac{27}{8} h^{4} \mathbf{y}''''(t_{n})+ O(h^{5}) \]


\[ \mathbf{y}(t_{n+1}) = \mathbf{y}(t_{n}) + h \mathbf{y}'(t_{n}) + \frac{1}{2} h^{2} \mathbf{y}''(t_{n}) + \frac{1}{6} h^{3}  \mathbf{y}'''(t_{n}) + \frac{1}{24} h^{4} \mathbf{y}''''(t_{n})+ O(h^{5}) \]



\[ \mathbf{y}(t_{n+2}) = \mathbf{y}(t_{n}) + 2h \mathbf{y}'(t_{n}) + 2 h^{2} \mathbf{y}''(t_{n}) + \frac{4}{3} h^{3}  \mathbf{y}'''(t_{n}) + \frac{2}{3} h^{4} \mathbf{y}''''(t_{n})+ O(h^{5}) \]

\[ h\mathbf{y}'(t_{n+2}) = h \mathbf{y}'(t_{n}) + 2h^{2} \mathbf{y}''(t_{n}) + 2 h^{3} \mathbf{y}'''(t_{n}) + \frac{4}{3} h^{4}  \mathbf{y}''''(t_{n}) + O(h^{5}) \]

Substituting these into $ (*) $ it is clear that the predictor method is third order; moreover we deduce that 

\[ \mathbf{y}(t_{n+3}) - \{  - \frac{1}{2} \mathbf{y}(t_{n}) + 3 \mathbf{y}(t_{n+1})  - \frac{3}{2} \mathbf{y}(t_{n+2}) + 3h\mathbf{y}'(t_{n+2})  \} =  \frac{1}{4} h^{4} \mathbf{y}''''(t_{n}) + O(h^{5})   \]

and thus

\begin{align*}
\mathbf{y}_{n+3}^{P} & \approx \mathbf{y}(t_{n+3}) - \frac{1}{4} h^{4} \mathbf{y}''''(t_{n}) \\
\end{align*}

Similarly for the corrector; substituting the true solution

\[ \mathbf{y}(t_{n+3}) - \frac{1}{11} \{ 2 \mathbf{y}(t_{n}) - 9 \mathbf{y}(t_{n+1})  + 18 \mathbf{y}(t_{n+2}) + 6h\mathbf{y}'(t_{n+3})  \} \quad (**) \]

Noting that

\[ h\mathbf{y}'(t_{n+3}) = h \mathbf{y}'(t_{n}) + 3h^{2} \mathbf{y}''(t_{n}) + \frac{9}{2} h^{3} \mathbf{y}'''(t_{n}) + 9 h^{4}  \mathbf{y}''''(t_{n}) + O(h^{5}) \]

We again see this method is third order, and that 

\[ \mathbf{y}(t_{n+3}) - \frac{1}{11} \{ 2 \mathbf{y}(t_{n}) - 9 \mathbf{y}(t_{n+1})  + 18 \mathbf{y}(t_{n+2}) + 6h\mathbf{y}'(t_{n+3})  \} =  \]




\section{QUESTION 12}

starred



\end{document}