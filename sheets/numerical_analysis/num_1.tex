\documentclass[a4paper]{article}
\usepackage{amsmath}
\def\npart {IB}
\def\nterm {Lent}
\def\nyear {2018}
\def\nlecturer {Dr. Saxton}
\def\ncourse {Numerical Analysis Example Sheet 1}

% Imports
\ifx \nauthor\undefined
  \def\nauthor{Christopher Turnbull}
\else
\fi

\author{Supervised by \nlecturer \\\small Solutions presented by \nauthor}
\date{\nterm\ \nyear}

\usepackage{alltt}
\usepackage{amsfonts}
\usepackage{amsmath}
\usepackage{amssymb}
\usepackage{amsthm}
\usepackage{booktabs}
\usepackage{caption}
\usepackage{enumitem}
\usepackage{fancyhdr}
\usepackage{graphicx}
\usepackage{mathdots}
\usepackage{mathtools}
\usepackage{microtype}
\usepackage{multirow}
\usepackage{pdflscape}
\usepackage{pgfplots}
\usepackage{siunitx}
\usepackage{slashed}
\usepackage{tabularx}
\usepackage{tikz}
\usepackage{tkz-euclide}
\usepackage[normalem]{ulem}
\usepackage[all]{xy}
\usepackage{imakeidx}

\makeindex[intoc, title=Index]
\indexsetup{othercode={\lhead{\emph{Index}}}}

\ifx \nextra \undefined
  \usepackage[pdftex,
    hidelinks,
    pdfauthor={Christopher Turnbull},
    pdfsubject={Cambridge Maths Notes: Part \npart\ - \ncourse},
    pdftitle={Part \npart\ - \ncourse},
  pdfkeywords={Cambridge Mathematics Maths Math \npart\ \nterm\ \nyear\ \ncourse}]{hyperref}
  \title{Part \npart\ --- \ncourse}
\else
  \usepackage[pdftex,
    hidelinks,
    pdfauthor={Christopher Turnbull},
    pdfsubject={Cambridge Maths Notes: Part \npart\ - \ncourse\ (\nextra)},
    pdftitle={Part \npart\ - \ncourse\ (\nextra)},
  pdfkeywords={Cambridge Mathematics Maths Math \npart\ \nterm\ \nyear\ \ncourse\ \nextra}]{hyperref}

  \title{Part \npart\ --- \ncourse \\ {\Large \nextra}}
  \renewcommand\printindex{}
\fi

\pgfplotsset{compat=1.12}

\pagestyle{fancyplain}
\lhead{\emph{\nouppercase{\leftmark}}}
\ifx \nextra \undefined
  \rhead{
    \ifnum\thepage=1
    \else
      \npart\ \ncourse
    \fi}
\else
  \rhead{
    \ifnum\thepage=1
    \else
      \npart\ \ncourse\ (\nextra)
    \fi}
\fi
\usetikzlibrary{arrows.meta}
\usetikzlibrary{decorations.markings}
\usetikzlibrary{decorations.pathmorphing}
\usetikzlibrary{positioning}
\usetikzlibrary{fadings}
\usetikzlibrary{intersections}
\usetikzlibrary{cd}

\newcommand*{\Cdot}{{\raisebox{-0.25ex}{\scalebox{1.5}{$\cdot$}}}}
\newcommand {\pd}[2][ ]{
  \ifx #1 { }
    \frac{\partial}{\partial #2}
  \else
    \frac{\partial^{#1}}{\partial #2^{#1}}
  \fi
}
\ifx \nhtml \undefined
\else
  \renewcommand\printindex{}
  \makeatletter
  \DisableLigatures[f]{family = *}
  \let\Contentsline\contentsline
  \renewcommand\contentsline[3]{\Contentsline{#1}{#2}{}}
  \renewcommand{\@dotsep}{10000}
  \newlength\currentparindent
  \setlength\currentparindent\parindent

  \newcommand\@minipagerestore{\setlength{\parindent}{\currentparindent}}
  \usepackage[active,tightpage,pdftex]{preview}
  \renewcommand{\PreviewBorder}{0.1cm}

  \newenvironment{stretchpage}%
  {\begin{preview}\begin{minipage}{\hsize}}%
    {\end{minipage}\end{preview}}
  \AtBeginDocument{\begin{stretchpage}}
  \AtEndDocument{\end{stretchpage}}

  \newcommand{\@@newpage}{\end{stretchpage}\begin{stretchpage}}

  \let\@real@section\section
  \renewcommand{\section}{\@@newpage\@real@section}
  \let\@real@subsection\subsection
  \renewcommand{\subsection}{\@@newpage\@real@subsection}
  \makeatother
\fi

% Theorems
\theoremstyle{definition}
\newtheorem*{aim}{Aim}
\newtheorem*{axiom}{Axiom}
\newtheorem*{claim}{Claim}
\newtheorem*{cor}{Corollary}
\newtheorem*{conjecture}{Conjecture}
\newtheorem*{defi}{Definition}
\newtheorem*{eg}{Example}
\newtheorem*{ex}{Exercise}
\newtheorem*{fact}{Fact}
\newtheorem*{law}{Law}
\newtheorem*{lemma}{Lemma}
\newtheorem*{notation}{Notation}
\newtheorem*{prop}{Proposition}
\newtheorem*{soln}{Solution}
\newtheorem*{thm}{Theorem}

\newtheorem*{remark}{Remark}
\newtheorem*{warning}{Warning}
\newtheorem*{exercise}{Exercise}

\newtheorem{nthm}{Theorem}[section]
\newtheorem{nlemma}[nthm]{Lemma}
\newtheorem{nprop}[nthm]{Proposition}
\newtheorem{ncor}[nthm]{Corollary}


\renewcommand{\labelitemi}{--}
\renewcommand{\labelitemii}{$\circ$}
\renewcommand{\labelenumi}{(\roman{*})}

\let\stdsection\section
\renewcommand\section{\newpage\stdsection}

% Strike through
\def\st{\bgroup \ULdepth=-.55ex \ULset}

% Maths symbols
\newcommand{\abs}[1]{\left\lvert #1\right\rvert}
\newcommand\ad{\mathrm{ad}}
\newcommand\AND{\mathsf{AND}}
\newcommand\Art{\mathrm{Art}}
\newcommand{\Bilin}{\mathrm{Bilin}}
\newcommand{\bket}[1]{\left\lvert #1\right\rangle}
\newcommand{\B}{\mathcal{B}}
\newcommand{\bolds}[1]{{\bfseries #1}}
\newcommand{\brak}[1]{\left\langle #1 \right\rvert}
\newcommand{\braket}[2]{\left\langle #1\middle\vert #2 \right\rangle}
\newcommand{\bra}{\langle}
\newcommand{\cat}[1]{\mathsf{#1}}
\newcommand{\C}{\mathbb{C}}
\newcommand{\CP}{\mathbb{CP}}
\newcommand{\cU}{\mathcal{U}}
\newcommand{\Der}{\mathrm{Der}}
\newcommand{\D}{\mathrm{D}}
\newcommand{\dR}{\mathrm{dR}}
\newcommand{\E}{\mathbb{E}}
\newcommand{\F}{\mathbb{F}}
\newcommand{\Frob}{\mathrm{Frob}}
\newcommand{\GG}{\mathbb{G}}
\newcommand{\gl}{\mathfrak{gl}}
\newcommand{\GL}{\mathrm{GL}}
\newcommand{\G}{\mathcal{G}}
\newcommand{\Gr}{\mathrm{Gr}}
\newcommand{\haut}{\mathrm{ht}}
\newcommand{\Id}{\mathrm{Id}}
\newcommand{\ket}{\rangle}
\newcommand{\lie}[1]{\mathfrak{#1}}
\newcommand{\Mat}{\mathrm{Mat}}
\newcommand{\N}{\mathbb{N}}
\newcommand{\norm}[1]{\left\lVert #1\right\rVert}
\newcommand{\normalorder}[1]{\mathop{:}\nolimits\!#1\!\mathop{:}\nolimits}
\newcommand\NOT{\mathsf{NOT}}
\newcommand{\Oc}{\mathcal{O}}
\newcommand{\Or}{\mathrm{O}}
\newcommand\OR{\mathsf{OR}}
\newcommand{\ort}{\mathfrak{o}}
\newcommand{\PGL}{\mathrm{PGL}}
\newcommand{\ph}{\,\cdot\,}
\newcommand{\pr}{\mathrm{pr}}
\newcommand{\Prob}{\mathbb{P}}
\newcommand{\PSL}{\mathrm{PSL}}
\newcommand{\Ps}{\mathcal{P}}
\newcommand{\PSU}{\mathrm{PSU}}
\newcommand{\pt}{\mathrm{pt}}
\newcommand{\qeq}{\mathrel{``{=}"}}
\newcommand{\Q}{\mathbb{Q}}
\newcommand{\R}{\mathbb{R}}
\newcommand{\RP}{\mathbb{RP}}
\newcommand{\Rs}{\mathcal{R}}
\newcommand{\SL}{\mathrm{SL}}
\newcommand{\so}{\mathfrak{so}}
\newcommand{\SO}{\mathrm{SO}}
\newcommand{\Spin}{\mathrm{Spin}}
\newcommand{\Sp}{\mathrm{Sp}}
\newcommand{\su}{\mathfrak{su}}
\newcommand{\SU}{\mathrm{SU}}
\newcommand{\term}[1]{\emph{#1}\index{#1}}
\newcommand{\T}{\mathbb{T}}
\newcommand{\tv}[1]{|#1|}
\newcommand{\U}{\mathrm{U}}
\newcommand{\uu}{\mathfrak{u}}
\newcommand{\Vect}{\mathrm{Vect}}
\newcommand{\wsto}{\stackrel{\mathrm{w}^*}{\to}}
\newcommand{\wt}{\mathrm{wt}}
\newcommand{\wto}{\stackrel{\mathrm{w}}{\to}}
\newcommand{\Z}{\mathbb{Z}}
\renewcommand{\d}{\mathrm{d}}
\renewcommand{\H}{\mathbb{H}}
\renewcommand{\P}{\mathbb{P}}
\renewcommand{\sl}{\mathfrak{sl}}
\renewcommand{\vec}[1]{\boldsymbol{\mathbf{#1}}}
%\renewcommand{\F}{\mathcal{F}}

\let\Im\relax
\let\Re\relax

\DeclareMathOperator{\adj}{adj}
\DeclareMathOperator{\Ann}{Ann}
\DeclareMathOperator{\area}{area}
\DeclareMathOperator{\Aut}{Aut}
\DeclareMathOperator{\Bernoulli}{Bernoulli}
\DeclareMathOperator{\betaD}{beta}
\DeclareMathOperator{\bias}{bias}
\DeclareMathOperator{\binomial}{binomial}
\DeclareMathOperator{\card}{card}
\DeclareMathOperator{\ccl}{ccl}
\DeclareMathOperator{\Char}{char}
\DeclareMathOperator{\ch}{ch}
\DeclareMathOperator{\cl}{cl}
\DeclareMathOperator{\cls}{\overline{\mathrm{span}}}
\DeclareMathOperator{\conv}{conv}
\DeclareMathOperator{\corr}{corr}
\DeclareMathOperator{\cosec}{cosec}
\DeclareMathOperator{\cosech}{cosech}
\DeclareMathOperator{\cov}{cov}
\DeclareMathOperator{\covol}{covol}
\DeclareMathOperator{\diag}{diag}
\DeclareMathOperator{\diam}{diam}
\DeclareMathOperator{\Diff}{Diff}
\DeclareMathOperator{\disc}{disc}
\DeclareMathOperator{\dom}{dom}
\DeclareMathOperator{\End}{End}
\DeclareMathOperator{\energy}{energy}
\DeclareMathOperator{\erfc}{erfc}
\DeclareMathOperator{\erf}{erf}
\DeclareMathOperator*{\esssup}{ess\,sup}
\DeclareMathOperator{\ev}{ev}
\DeclareMathOperator{\Ext}{Ext}
\DeclareMathOperator{\Fit}{Fit}
\DeclareMathOperator{\fix}{fix}
\DeclareMathOperator{\Frac}{Frac}
\DeclareMathOperator{\Gal}{Gal}
\DeclareMathOperator{\gammaD}{gamma}
\DeclareMathOperator{\gr}{gr}
\DeclareMathOperator{\hcf}{hcf}
\DeclareMathOperator{\Hom}{Hom}
\DeclareMathOperator{\id}{id}
\DeclareMathOperator{\image}{image}
\DeclareMathOperator{\im}{im}
\DeclareMathOperator{\Im}{Im}
\DeclareMathOperator{\Ind}{Ind}
\DeclareMathOperator{\Int}{Int}
\DeclareMathOperator{\Isom}{Isom}
\DeclareMathOperator{\lcm}{lcm}
\DeclareMathOperator{\length}{length}
\DeclareMathOperator{\Lie}{Lie}
\DeclareMathOperator{\like}{like}
\DeclareMathOperator{\Lk}{Lk}
\DeclareMathOperator{\mse}{mse}
\DeclareMathOperator{\multinomial}{multinomial}
\DeclareMathOperator{\orb}{orb}
\DeclareMathOperator{\ord}{ord}
\DeclareMathOperator{\otp}{otp}
\DeclareMathOperator{\Poisson}{Poisson}
\DeclareMathOperator{\poly}{poly}
\DeclareMathOperator{\rank}{rank}
\DeclareMathOperator{\rel}{rel}
\DeclareMathOperator{\Re}{Re}
\DeclareMathOperator*{\res}{res}
\DeclareMathOperator{\Res}{Res}
\DeclareMathOperator{\rk}{rk}
\DeclareMathOperator{\Root}{Root}
\DeclareMathOperator{\sech}{sech}
\DeclareMathOperator{\sgn}{sgn}
\DeclareMathOperator{\spn}{span}
\DeclareMathOperator{\stab}{stab}
\DeclareMathOperator{\St}{St}
\DeclareMathOperator{\supp}{supp}
\DeclareMathOperator{\Syl}{Syl}
\DeclareMathOperator{\Sym}{Sym}
\DeclareMathOperator{\tr}{tr}
\DeclareMathOperator{\Tr}{Tr}
\DeclareMathOperator{\var}{var}
\DeclareMathOperator{\vol}{vol}

\pgfarrowsdeclarecombine{twolatex'}{twolatex'}{latex'}{latex'}{latex'}{latex'}
\tikzset{->/.style = {decoration={markings,
                                  mark=at position 1 with {\arrow[scale=2]{latex'}}},
                      postaction={decorate}}}
\tikzset{<-/.style = {decoration={markings,
                                  mark=at position 0 with {\arrowreversed[scale=2]{latex'}}},
                      postaction={decorate}}}
\tikzset{<->/.style = {decoration={markings,
                                   mark=at position 0 with {\arrowreversed[scale=2]{latex'}},
                                   mark=at position 1 with {\arrow[scale=2]{latex'}}},
                       postaction={decorate}}}
\tikzset{->-/.style = {decoration={markings,
                                   mark=at position #1 with {\arrow[scale=2]{latex'}}},
                       postaction={decorate}}}
\tikzset{-<-/.style = {decoration={markings,
                                   mark=at position #1 with {\arrowreversed[scale=2]{latex'}}},
                       postaction={decorate}}}
\tikzset{->>/.style = {decoration={markings,
                                  mark=at position 1 with {\arrow[scale=2]{latex'}}},
                      postaction={decorate}}}
\tikzset{<<-/.style = {decoration={markings,
                                  mark=at position 0 with {\arrowreversed[scale=2]{twolatex'}}},
                      postaction={decorate}}}
\tikzset{<<->>/.style = {decoration={markings,
                                   mark=at position 0 with {\arrowreversed[scale=2]{twolatex'}},
                                   mark=at position 1 with {\arrow[scale=2]{twolatex'}}},
                       postaction={decorate}}}
\tikzset{->>-/.style = {decoration={markings,
                                   mark=at position #1 with {\arrow[scale=2]{twolatex'}}},
                       postaction={decorate}}}
\tikzset{-<<-/.style = {decoration={markings,
                                   mark=at position #1 with {\arrowreversed[scale=2]{twolatex'}}},
                       postaction={decorate}}}

\tikzset{circ/.style = {fill, circle, inner sep = 0, minimum size = 3}}
\tikzset{mstate/.style={circle, draw, blue, text=black, minimum width=0.7cm}}

\tikzset{commutative diagrams/.cd,cdmap/.style={/tikz/column 1/.append style={anchor=base east},/tikz/column 2/.append style={anchor=base west},row sep=tiny}}

\definecolor{mblue}{rgb}{0.2, 0.3, 0.8}
\definecolor{morange}{rgb}{1, 0.5, 0}
\definecolor{mgreen}{rgb}{0.1, 0.4, 0.2}
\definecolor{mred}{rgb}{0.5, 0, 0}

\def\drawcirculararc(#1,#2)(#3,#4)(#5,#6){%
    \pgfmathsetmacro\cA{(#1*#1+#2*#2-#3*#3-#4*#4)/2}%
    \pgfmathsetmacro\cB{(#1*#1+#2*#2-#5*#5-#6*#6)/2}%
    \pgfmathsetmacro\cy{(\cB*(#1-#3)-\cA*(#1-#5))/%
                        ((#2-#6)*(#1-#3)-(#2-#4)*(#1-#5))}%
    \pgfmathsetmacro\cx{(\cA-\cy*(#2-#4))/(#1-#3)}%
    \pgfmathsetmacro\cr{sqrt((#1-\cx)*(#1-\cx)+(#2-\cy)*(#2-\cy))}%
    \pgfmathsetmacro\cA{atan2(#2-\cy,#1-\cx)}%
    \pgfmathsetmacro\cB{atan2(#6-\cy,#5-\cx)}%
    \pgfmathparse{\cB<\cA}%
    \ifnum\pgfmathresult=1
        \pgfmathsetmacro\cB{\cB+360}%
    \fi
    \draw (#1,#2) arc (\cA:\cB:\cr);%
}
\newcommand\getCoord[3]{\newdimen{#1}\newdimen{#2}\pgfextractx{#1}{\pgfpointanchor{#3}{center}}\pgfextracty{#2}{\pgfpointanchor{#3}{center}}}

\def\Xint#1{\mathchoice
   {\XXint\displaystyle\textstyle{#1}}%
   {\XXint\textstyle\scriptstyle{#1}}%
   {\XXint\scriptstyle\scriptscriptstyle{#1}}%
   {\XXint\scriptscriptstyle\scriptscriptstyle{#1}}%
   \!\int}
\def\XXint#1#2#3{{\setbox0=\hbox{$#1{#2#3}{\int}$}
     \vcenter{\hbox{$#2#3$}}\kern-.5\wd0}}
\def\ddashint{\Xint=}
\def\dashint{\Xint-}

\newcommand\separator{{\centering\rule{2cm}{0.2pt}\vspace{2pt}\par}}

\newenvironment{own}{\color{gray!70!black}}{}

\newcommand\makecenter[1]{\raisebox{-0.5\height}{#1}}

\newtheorem*{soln}{Solution}

\renewcommand{\thesection}{}
\renewcommand{\thesubsection}{\arabic{section}.\arabic{subsection}}
\makeatletter
\def\@seccntformat#1{\csname #1ignore\expandafter\endcsname\csname the#1\endcsname\quad}
\let\sectionignore\@gobbletwo
\let\latex@numberline\numberline
\def\numberline#1{\if\relax#1\relax\else\latex@numberline{#1}\fi}
\makeatother


\begin{document}
	
\maketitle

\section{QUESTION 1}

We seek some polynomial interpolant $ p \in \mathbb{P}_{3}[x] $. Using the Lagrange formula we have that

\[ p(x) = \sum_{k=0}^{3} f(k) l_{k}  \]

where 

\[ l_{k} = \prod_{i=0, i \neq k}^{3} \frac{x - i}{k - i}  \] 

that is,

\[ p(x) = f(0) \frac{(x-1)(x-2)(x-3)}{-6} + f(1) \frac{x(x-2)(x-3)}{2} + f(2) \frac{x(x-1)(x-3)}{-2} + f(3) \frac{x(x-1)(x-2)}{6}  \]

\begin{enumerate}
	\item The approximant $ p(6) $
	
	We have
	
	\begin{align*}
	p(6) & = f(0) \frac{5 \cdot 4 \cdot 3}{-6} + f(1) \frac{6\cdot 4 \cdot 3}{2} + f(2) \frac{6 \cdot 5 \cdot 3}{-2} + f(3) \frac{6 \cdot 5 \cdot 4}{6} \\
	& = -10 f(0) + 36 f(1)+ -45 f(2) + 20 f(3) 
	\end{align*}
	
	\item The approximant $ p'(0) $
	
	Taking the derivative of each term individually, we then plug in $ x=0 $. We deduce that
	
	\begin{align*}
	p'(0) & = -\frac{11}{6} f(0) + 3f(1) - \frac{3}{2}  f(2) + \frac{1}{3} f(3) 
	\end{align*}
	
	\item The approximant $ \int_{0}^{3} p(x) \; \d x $
	
	Expanding each term and integrating (I can't see a shorter way) we have that

	\[ p(x) = f(0) \frac{x^{3} - 6x^{2} + 11 x - 6}{-6} + f(1) \frac{x^{3}  - 5 x^{2} + 6x}{2} + f(2) \frac{x^{3} - 4 x^{2} + 3x}{-2} + f(3) \frac{x^{3} - 3x^{2} + 2x}{6}  \]
	
	and thus
	
	\begin{align*}
	\int_{0}^{3} p(x) \; \d x & = \frac{3}{8} f(0) + \frac{9}{8} f(1) +  \frac{9}{8} f(2) + \frac{3}{8} f(3)  
	\end{align*}
	
	We can check this by supposing $ f(x) = x, $ so that $ f(k)  = k  $ for each $ k $. Indeed, $ p(6) = 6, p'(0) = 1 $, and $ \int_{0}^{3} p(x) \; \d x =  9/2 $, 
	
\end{enumerate}



\section{QUESTION 2}

The formula is true when $ x = 0,1 $ since both sides of the equation vanish. Let $ x \in (0,1) $ be any other point and define (for $ x $ fixed).

\[ \phi(t): = [f(t) - p(t)] \prod_{i=0}^{3} (x - x_{i}) - [f(x) - p(x)]\prod_{i=0}^{3} (t - x_{i}), \quad t \in (0,1)  \]

where $ x_{0} = x_{1} = 0, x_{2} = x_{3} = 1 $. Note $ \phi(0) = \phi(1) = 0 $, and also $ \phi(x) = 0 $. Hence, $ \phi $ has at least 3 zeroes. Applying Rolle's theorem, and using the condition that $ f'(0) = f'(1) = 0 $, we deduce that $ \phi'(t) $ has at least 4 zeroes: one at $ x_{0} = 0$, one at $ x_{2} = 1 $, and two more: one in the interval $ (0,x) $, and the other in $ (x,1) $.

Then $ \phi''(t) $ has at least 3 zeroes in $ (0,1) $, and... $ \phi^{(4)}(x) $ has at least one zero in $ (0,1) $; call it $ \xi $. Then

\[ 0 = \phi^{(4)}(\xi) = \left[  f^{(n+1)}(\xi) - p^{(n+1)}(\xi) \right] \prod_{i=0}^{3} (x - x_{i}) - \left[ f(x) - p(x) \right] \frac{\d^{4} }{\d t^{4}}\Big|_{t = \xi} \prod_{i=0}^{3} (t - x_{i})    \]

Since $ p^{(4)} \equiv 0 $, and $ \frac{\d^{4} }{\d t^{4}}\Big|_{t = \xi} \prod_{i=0}^{3} (t - x_{i}) = 4! $, we obtain 

\begin{align*}
f(x) - p(x) & = \frac{1}{4!} f^{(4)}(\xi) \prod_{i=0}^{3} (x - x_{i})  \\
& = \frac{1}{24} x^{2}(1-x)^{2} f^{(4)}(\xi) 
\end{align*}

\section{QUESTION 3}

%As $ a $ and $ b $ are arbitrary, we have two different cases to consider, $ a < b < c $ and $ a < c < b $. 
%
%\begin{enumerate}
%	\item $ a < b < c $:
%	
%	Here, we have
%	
%\end{enumerate}

%Nah fuck that, second attempt


%Using the Lagrange formula to interpolate the two points $ f(a),f(b) $, we have
%
%\[ p(x) = f(a)\left( \frac{x - b}{a - b} \right) + f(b)\left(   \frac{x - a}{b - a} \right)  \]
%
%We readily verify that $ p(a) = f(a)$, $ p(b) = f(b) $. Note that
%
%\[ p'(x) = \frac{f(a) - f(b)}{a - b} \]
%
%is a constant. 
%
%Next, consider the polynomial $ q(x) $ that interpolates $ f'(a),f'(b),f'(c) $, again found using Lagrange:
%
%\[ q(x) = f'(a) \left( \frac{x - b}{a - b} \right) \left( \frac{x - c}{a - c} \right) + f'(b) \left( \frac{x - a}{b - a} \right)\left( \frac{x - c}{b - c} \right)+ f'(c) \left( \frac{x - a}{c - a} \right)\left( \frac{x - b}{c - b} \right) \]
%
%Define 
%
%\[ r(x) = p(x) + \int_{x_{0}}^{x} q(t) \; \d t - x \frac{f(a) - f(b)}{a - b}  \]
%
%where $ x_{0} $ is chosen such that $ \int_{x_{0}}^{a} q(t) \; \d t = a  \frac{f(a) - f(b)}{a - b} $ and $ \int_{x_{0}}^{b} q(t) \; \d t = b \frac{f(a) - f(b)}{a - b}  $. Then $ r(a) = f(a), r(b) = f(b) $. We also have
%
%\[ r'(x) = q(x) \]
%
%So $ r'(a) = r'(b) = r'(c) $.
%
%Hence we have existence. 

Seeking a contradiction we suppose there exists some nonzero polynomial $ p \in \P_{4} [x] $ st. 

\[ p(a) = p(b) = p'(a) = p'(b) = p'(c) = 0 \qquad (*) \] 


Suppose that $ q_{1} \in \P_{4} [x] $ and $ q_{2} \in \P_{4} [x] $ both interpolate the data, then $ q_{1} - q_{2}  $ vanishes at these points. Hence, we have

\[ q_{1} = q_{2} + k p \]

for some $ k \in \R $, so the solution of this interpolation problem is not unique. 

To pick a value of $ c $ that satisfies $ (*) $, try

\[ p(x) = (x - a)(x - b) + (x - a)(x - b)(x - c) \]

Immediately we have $ p(a) = p(b) = 0 $. Now, 

\[ p'(x) =  (x- a) + (x - b)  (x - b)(x - c) +  (x - a)(x - c) +  (x - a)(x - b)   \]


\section{QUESTION 4}


%By definition, $ f[x_{0},x_{1},\cdots,x_{n},x] $ is the coefficient of $ x^{n+1} $ in $ q  \in \P_{n+1}[x] $ that interpolates the $ n+2 $ points $ [x_{0},x_{1},\cdots,x_{n},x] $ with $ f $.
%
%From the definition of the divided difference we have that
%
%\[ f[x_{0},x_{1},\cdots,x_{n},x] = \frac{f[x_{1},x_{2},\cdots,x_{n+1},x] - f[x_{0},x_{1},\cdots,x_{n}]  }{x - x_{0}} \]
%
%But also,
%
%\[ f[x_{1},\cdots,x_{n},x] = \frac{f[x_{2},\cdots,x_{n+1},x] - f[x_{1},\cdots,x_{n}]  }{x - x_{1}} \]
%
%When $ n = 0 $ the identity reads
%
%\begin{align*}
%f(x) - p(x) & = f[x_{0},x](x-x_{0}) \\
%& = f(x) - f(x_{0})
%\end{align*}
%
%When $ n = 1$ the identity reads
%
%\begin{align*}
%f(x) - p(x) & = f[x_{0},x_{1},x](x-x_{0})(x-x_{1}) \\
%& = \frac{f[x_{0},x_{1}] - f[x_{1},x]}{x - x_{0}}
%\end{align*}
%
%Nah fuck that, third attempt

The Newton interpolation formula states that a polynomial intepolating $ f $ at pairwise distinct points $ x_{0},\cdots,x_{n} $ is given by 

\[ p_{n}(x) : = f[x_{0}] + f[x_{0},x_{1}](x_{1} - x_{0}) + \cdots + f[x_{0},\cdots,x_{n}]\prod_{i=0}^{n-1} (x - x_{i})  \] 

In particular,

\[ p_{n+1}(x) - p_{n}(x) = f[x_{0},\cdots,x_{n+1}] \prod_{i=0}^{n} (x - x_{i}) \qquad (*)  \]



To deduce the identity in question 4, we think of $ x $ as a new interpolation point (the $ n+1^{\text{th}} $). As $ x \neq x_{i} $ for any $ i $, we can now apply (*), which gives

\[ p_{n+1}(t) - p_{n}(t) = f[x_{0},\cdots,x_{n},x] \prod_{i=0}^{n} (t - x_{i})  \]

for all $ t \in \R $. In particular, setting $ t = x $, we have $ p_{n+1}(x) = f(x) $, which is the identity as required. 


 
 




\section{QUESTION 5}

The Newton divided difference table for Question 5 is shown below, where arithmetic has been rounded to 4 decimal places at each step.

\begin{center}
	\begin{tabular}{ccccc}
		\toprule
		$x_i$ & $f_i$ & $f[*, *]$ & $f[*, *, *]$ & $f[*,*,*,*]$\\
		\midrule
		$0$ & $f[0] = 0$\\
		& & $f[0, 0.1] = 0.9980$\\
		$0.1$ & $f[0.1] = 0.9980$ & & $f[0, 0.1, 0.4] $ \\
		& & $f[0.1, 0.4] = 0.9653 $ & $ = -0.0817 $ & $f[0,0.1,0.4,0.7] $ \\
		$0.4$ & $f[0.4] = 0.3894$ & & $f[0.1, 0.4, 0.7] $ & $ = - 0.1680 $  \\
		& & $f[0.4, 0.7] = 0.8493$ & $ = -0.1993 $ \\
		$0.7$ & $f[0.7] = 0.6442$ & \\
		\bottomrule
	\end{tabular}
\end{center}

Using Newton's formula, the polynomial interpolating these points is given as

\begin{align*}
p(x) & = f[0] + f[0,0.1]x + f[0,0.1,0.4]x(x-0.1) + f[0,0.1,0.4,0.7]x(x-0.1)(x-0.4)  \\
& = 0.9980 x -0.0817(x^2 -0.1x) - 0.1680(x^{3} -0.5x^{2}  + 0.04x)\\
& = 0.9995 x + 0.0023 x^2 - 0.1680 x^{3}
\end{align*}

As we have rounded erroneously this is indeed different from $ \sin x $. 



\section{QUESTION 6}

The condition 

\[ \int_{0}^{1} [ f(x) - p(x)]^{2} \; \d x  < 10^{-4} \]

is equivalent to

\[ \frac{1}{3} - 2 \int_{0}^{1} f(x)p(x) \; \d x + \int_{0}^{1} p(x)^{2} \; \d x < 10^{-4} \]

Fourier series?






\section{QUESTION 7}

We will first prove that, under the substitution $ x = \cos \theta $, $ p_{n}(x) = \sin(n+1) \theta / \sin \theta $. We will use induction with two base cases;

For $ n = 0 $, we have $ p_{0}(x) = \sin \theta / \sin \theta = 1 $ as required, and for $ n = 1 $ we have  $ p_{0}(x) = \sin 2 \theta / \sin \theta = 2 \cos \theta = 2 x $, as required (as $ x = \cos \theta $). 

Now assuming true for $ p_{n-1}(x) $ and $ p_{n}(x) $ yields:

\begin{align*}
p_{n+1}(x) & = 2x p_{n}(x) -p_{n-1}(x) \\
& = (2 \cos \theta \sin(n+1)\theta  - \sin n \theta) / \sin \theta \qquad (x =  \cos \theta ) \\
\end{align*}

Fiddling around with the numerator;

\begin{align*}
2 \cos \theta \sin(n+1)\theta  - \sin n \theta & = ( \sin (n+1) \theta \cos \theta + \cos(n+1 )\theta \sin \theta ) \\
& \; \; + (\sin(n+1)\theta \cos \theta - \cos(n+1) \theta \sin \theta) - \sin n\theta  \\
& = \sin(n+2) \theta + \sin n \theta - \sin n\theta =  \sin(n+2) \theta
\end{align*}

Hence $ p_{n-1}(x) = \sin(n) \theta / \sin \theta $, $ p_{n}(x) = \sin(n+1) \theta / \sin \theta $ together imply that $ p_{n+1}(x) = \sin(n+2) \theta / \sin \theta $, which completes our inductive proof. 

Now we can show orthogonality: 

\begin{align*}
\langle p_{n},p_{m} \rangle & = \int_{-1}^{1} p_{n}(x) p_{m}(x) \sqrt{1-x^{2}}  \; d x \\
& = \int_{\pi}^{0} \frac{\sin(n+1)\theta}{\sin \theta} \frac{\sin(m+1)\theta}{\sin \theta} \sqrt{ 1 - \cos^{2} \theta} ( - \sin \theta) \; \d \theta \qquad (x = \cos \theta) \\
& = \int_{0}^{\pi} \sin(n+1) \theta \sin(m+1) \theta \; \d \theta \\
& = \frac{\pi}{2} \delta_{mn}
\end{align*}

Thus these polynomials are orthogonal with respect to the defined inner product, and $ \langle p_{n},p_{n} \rangle = \pi / 2 $


\section{QUESTION 8}

We apply Gaussian quadrature with Legendre polynomials, and pick $ c_{1} $ and $ c_{2} $ as the zeros of $ P_{2}(x) = \frac{3}{2} x^{2} - \frac{1}{2} $, so

\[ c_{1} = - \frac{\sqrt{3}}{3}, \qquad c_{2} =  \frac{\sqrt{3}}{3}  \]

Using the formula for the choice of weights given in lectures, we have

\[ b_{1} = \int_{0}^{1} \frac{x - c_{2}}{c_{1} - c_{2}} \; \d x, \quad b_{2} = \int_{0}^{1} \frac{x - c_{1}}{c_{2} - c_{1}} \; \d x \]

Thus $ b_{1} = \frac{1}{2} - \frac{\sqrt{3}}{4} $, $ b_{2} = \frac{1}{2} + \frac{\sqrt{3}}{4} $, and the exact approximate when $ f $ is cubic is: 

\[ \int_{0}^{1} f(x) \; \d x \approx  \left( \frac{1}{2} - \frac{\sqrt{3}}{4} \right) f \left(    -  \frac{\sqrt{3}}{3} \right) + \left( \frac{1}{2} + \frac{\sqrt{3}}{4} \right) f \left(   \frac{\sqrt{3}}{3} \right) \]

Note $ \int_{0}^{1} 1 \; \d x = 1  $, $ \int_{0}^{1} x \; \d x = \frac{1}{2}  $, $ \int_{0}^{1} x^{2} \; \d x = 1 / 3  $, $ \int_{0}^{1} x^{3} \; \d x = 1/4  $. 

 
\section{QUESTION 9}

We have that $ \frac{\d^{k} }{\d x^{k}} (e^{-x}) = (-1)^{k}e^{-x} $, thus using the Leibniz rule:

\begin{align*}
p_{n}(x) & = e^{x} \frac{\d^{n} }{\d x^{n}} (x^{n} e^{-x}) \\
& = e^{x} \sum_{r=0}^{n} \binom{n}{r} \frac{\d^{r} }{\d x^{r}} x^{n}  \frac{\d^{n-r} }{\d x^{n-r}} e^{-x} \\
& =  \sum_{r=0}^{n} \binom{n}{r} \frac{\d^{r} }{\d x^{r}} (x^{n}) (-1)^{n-r} \\
& =  \sum_{r=0}^{n} r! \binom{n}{r}^{2} (-x)^{n-r} (*)
\end{align*}

so $ p_{n}(x) $ indeed a polynomial. Next, with respect to the defined scalar product, we have 

\begin{align*}
\langle p_{n},p \rangle & = \int_{0}^{\infty} e^{-x} p_{n}(x) p(x) \; \d x  \\
& = \int_{0}^{\infty}  \frac{\d^{n} }{\d x^{n}} (x^{n} e^{-x}) p(x) \; \d x  \\
& = \left[  p(x) \frac{\d^{n-1} }{\d x^{n-1}}(x^{n}e^{-x})   \right]_{0}^{\infty} - \int_{0}^{\infty} \frac{\d }{\d x} p(x) \frac{\d^{n-1} }{\d x^{n-1}} (x^{n} e^{-x}) \; \d x \\
& = - \int_{0}^{\infty} \frac{\d^{n} }{\d x^{n}} p(x) (x^{n} e^{-x}) \; \d x \\
& = 0
\end{align*}

and in going to the final line we have used the fact that $ p(x) $ is a polynomial of degree $ n-1 $. 

The boundary terms vanish by inspection; each term in the expression $ \frac{\d^{r} }{\d x^{r}}(x^{l}e^{-x}) $ contains an $ e^{-x} $, and if $ r < l $, each term is a multiple of $ x $, thus the expression is zero at $ \infty $ and $ 0 $ respectively. 

To evaluate $ p_{3},p_{4} $ and $ p_{5} $ using the Rodrigues formula we use $ (*) $:

\[ p_{3}(x) = - x^{3} + 9x^{2} - 18 x + 6 \]
\[ p_{4}(x) = x^{4} -  16 x^{3} + 72 x^{2} - 96 x + 24 \]
\[ p_{5}(x) =  - x^{5} + 25 x^{4} - 200 x^{3} + 600 x^{2} - 600 x + 120 \]

If these are to obey the relation

\[ p_{5}(x) = (\gamma x - \alpha)p_{4}(x) - \beta p_{3}(x), \quad x \in \R \]

Then comparing coefficients of $ x^{5},x^{4}$ and $ x^{3} $ respectively give,  $ \gamma = -1 $, $ \alpha = -9, \beta = -200 $.


\section{QUESTION 10}

For the $ k = 0 $ case: want to choose the least $ c_{0}$ such that

\[ \left| f(\frac{1}{2}) - \frac{1}{2} \left(  f(0) + f(1)  \right)  \right|  \leq c_{0} | | f | |_{\infty}  \]


In the extreme case where $ f(\frac{1}{2}) \approx  | | f | |_{\infty}   $ and $ f(0) \approx f(1) \approx -  | | f | |_{\infty}  $, we see that $ c_{0} = 2 $.

I'm not sure how to do the $ p = 1 $ case without the Peano Kernel theorem. 






$ k = 2 $: Consider $ f(\frac{1}{2}) \approx \frac{1}{2} (   f(0) + f(1 )) $.




Let $ L(f) = f(\frac{1}{2}) - \frac{1}{2} (   f(0) + f(1 ))  $. $ L(f) = 0 $ for all $ f \in \mathcal{C}[0,1] $ since $ L(f) = 0 $ when $ f(x) = 1,x $, and using linearity. Peano Kernel theorem tells us that 

\[ L(f) = \int_{0}^{1} K(\theta) f''(\theta) \; \d \theta   \]


where $ K(\theta) = L(   x \mapsto (x-\theta)_{+} ) $. For fixed $ \theta $,  let $ g(x) : = (x-\theta)_{+} $. We have

\begin{align*}
K(\theta) & = L(g) = g(\frac{1}{2}) - \frac{1}{2}(g(0) + g(1) )\\
& = (1/2-\theta)_{+} - \frac{1}{2} \left(  (0-\theta)_{+} + (1-\theta)_{+} \right) \\
& = \begin{cases} -\frac{1}{2} (1-\theta) & \text{ if }  0 \leq \theta \leq 1/2 \\ - \frac{1}{2} \theta  & \text{ if } 1/2 \leq \theta \leq 1\end{cases}
\end{align*}

Now

\begin{align*}
\int_{0}^{1} | K(\theta) | \; \d \theta & = \int_{0}^{1/2} \frac{1}{2} (1 - \theta) \; \d \theta  + \int_{1/2}^{1} \frac{1}{2} \theta\; \d \theta \\
& = \left(  \frac{1}{4} - \frac{1}{16} \right) + \left(  \frac{1}{4} - \frac{1}{16} \right) = \frac{3}{8}
\end{align*}


This allows us to bound the approximation error, for any $ f \in \mathcal{C}^{2}[0,1] $ we get

\[ | L(f) | \leq \int_{0}^{1} |  K(\theta) f''(\theta) | \; \d \theta \leq | | f''| |_{\infty} \int_{0}^{1} | K(\theta) | \; \d \theta \leq \frac{3}{8 } | f''| |_{\infty} \]

We can try the Peano kernel theorem in the $ k=1 $ ($ n = 0 $) case, which tells us 

\[ L(f) = \int_{0}^{1} K(\theta) f'(\theta) \; \d \theta   \]


where $ K(\theta) = L(   x \mapsto (x-\theta)_{+}^{0} ) $. For fixed $ \theta $,  let $ g(x) : = (x-\theta)_{+}^{0} $. Similar to before, we have

\begin{align*}
K(\theta) & = L(g) = g(\frac{1}{2}) - \frac{1}{2}(g(0) + g(1) )\\
& = (1/2-\theta)_{+}^{0} - \frac{1}{2} \left(  (0-\theta)_{+}^{0} + (1-\theta)_{+}^{0} \right) \\
& = \begin{cases} -\frac{1}{2} & \text{ if }  0 \leq \theta \leq 1/2 \\ \frac{1}{2}  & \text{ if } 1/2 \leq \theta \leq 1\end{cases}
\end{align*}

We can verify that $ \int_{0}^{1} |  K(\theta) | \; \d \theta = \frac{1}{2} $. This allows us to bound the approximation error, for any $ f \in \mathcal{C}[0,1] $ we get

\[ | L(f) | \leq \int_{0}^{1} |  K(\theta) f'(\theta) | \; \d \theta \leq | | f'| |_{\infty} \int_{0}^{1} | K(\theta) | \; \d \theta \leq \frac{1}{2} | f'| |_{\infty} \]




\section{QUESTION 11}

Define $ L(f) :=  f[0,1,2,4] $. Easy to check that $ L(f) = 0 $ for $ f \in \P_{2}[x] $. Thus for $ f \in C^{3}[0,4] $, we have

\[ L(f) = \frac{1}{6} \int_{0}^{4} K(\theta)f'''(\theta) \; \d \theta   \]

with $ K(\theta) = L(x \mapsto (x - \theta)_{+}^{2} ) $. For fixed $ \theta $, let $ g(x) := (x - \theta)_{+}^{2} $. Then using the Lagrange formula, noting that $ g(0) = 0 $

\begin{align*}
K(\theta) & = L(g) = g[0,1,2,4] \\
& = g(1) \frac{1}{1 - 0} \frac{1}{1 - 2} \frac{1}{1 - 4} + g(2) \frac{1}{2 - 0} \frac{1}{2 - 1} \frac{1}{2-4} + g(4) \frac{1}{4 - 0} \frac{1}{4 - 1} \frac{1}{4 - 2} \\
& = \frac{1}{3} (1 - \theta)_{+}^{2} - \frac{1}{4} (2 - \theta)_{+}^{2} + \frac{1}{24} (4 - \theta)_{+}^{2} \\
& = \begin{cases}  \frac{1}{8} \theta^{2} & \text{ if } 0 \leq \theta \leq 1 \\ - \frac{1}{4} (2 - \theta)^{2} + \frac{1}{24} (4 - \theta)^{2}  & \text{ if } 1 \leq \theta \leq 2 \\  \frac{1}{24} (4 - \theta)^{2} & \text{ if } 2 \leq \theta \leq 4 \end{cases}
\end{align*}

\begin{center}
	\begin{tikzpicture}
	
	\draw[->] (0,0) -- (5,0) node[below]{$\theta$};
	\draw[->] (0,0) -- (0,4) node[left]{$K(\theta)$};
	\end{tikzpicture}
\end{center}






\section{QUESTION 12}


For $ f \in \P_{3}[x] $, consider the approximant

\[ f'''(\xi) \approx \alpha f(0) + \beta f(1) + \gamma f'(0) + \delta f'(1)  \qquad (*) \]

Note that $ f'''(\xi) = 3! \; \forall \; \xi \in \R $  . Requiring (*) is exact for $ f(x) = 1, x, x^{2} $ and $ x^{3} $ respectively yields

\begin{align*}
0 & = \alpha + \beta \\
0 & = \beta + \gamma + \delta \\
0 & = \beta + 2\delta \\
6 & = \beta + 3 \delta  
\end{align*}

which gives


\[ f'''(\xi) \approx 12 f(0) - 12 f(1) + 6 f'(0) + 6 f'(1)  \]

Let $ L(f) =  f'''(\xi) - [12 f(0) - 12 f(1) + 6 f'(0) + 6 f'(1) ] $. $ L(f) = 0 $ for all $ f \in C^{4}[0,1] $, so Peano Kernel theorem tells us that 

\[ L(f) = \frac{1}{3!}  \int_{0}^{1} K(\theta) f^{(4)}(\theta) \; \d \theta \]


where $ K(\theta) = L(x \mapsto (x - \theta)_{+}^{3} ) $. For fixed $ \theta $,  let $ g(x) : = (x - \theta)_{+}^{2} $. We have

\begin{align*}
K(\theta) & = L(g) = g'''(\xi) - [ 12g(0) - 12g(1) + 6g'(0) + 6g'(1)  ]   \\
& = 6 (\xi - \theta)_{+}^{0}   - [ 12 (0 - \theta)_{+}^{3} - 12 (1 - \theta)_{+}^{3} + 18 (0 - \theta)_{+}^{2} + 18 (1 - \theta)_{+}^{2}  ] \\
& = \begin{cases} 12(1-\theta)^{3} - 18(1-\theta)^{2}  & \text{ if } 0 \leq \theta \leq \xi \\ 6 + 12(1-\theta)^{3} - 18(1-\theta)^{2}  &   \text{ if }  \xi \leq \theta \leq 1 \end{cases}
\end{align*}

Consequently for any $ f \in C^{4}[0,1] $ we have



\[ | L(f) | \leq \frac{1}{3!} \int_{0}^{1} |  K(\theta) f^{(4)}(\theta) | \; \d \theta \leq \frac{1}{6} | | f^{(4)} | |_{\infty} \int_{0}^{1} | K(\theta) | \; \d \theta  \]

Now, 

\begin{align*}
\frac{1}{6} \int_{0}^{1} | K(\theta) | \; \d \theta & = \int_{0}^{\xi} 2(1-\theta)^{3} - 3(1-\theta)^{2} \; \d \theta + \int_{\xi}^{1} 1 + 2(1-\theta)^{3} - 3(1-\theta)^{2} \; \d \theta \\
& = \frac{1}{4} (1 - \xi)^{4} - (1 - \xi)^{3} + (1 - \xi) + 
\end{align*}










\end{document}