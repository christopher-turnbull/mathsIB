\documentclass[a4paper]{article}
\usepackage{amsmath}
\def\npart {IB}
\def\nterm {Lent}
\def\nyear {2018}
\def\nlecturer {Dr. Saxton}
\def\ncourse {Numerical Analysis Example Sheet 3}

\input{header}

\newtheorem*{soln}{Solution}

\renewcommand{\thesection}{}
\renewcommand{\thesubsection}{\arabic{section}.\arabic{subsection}}
\makeatletter
\def\@seccntformat#1{\csname #1ignore\expandafter\endcsname\csname the#1\endcsname\quad}
\let\sectionignore\@gobbletwo
\let\latex@numberline\numberline
\def\numberline#1{\if\relax#1\relax\else\latex@numberline{#1}\fi}
\makeatother


\begin{document}
	
\maketitle

\section{QUESTION 1}

(use col pivoting)

First, $ \mathbf{u}_{1}^{T} $ is just the first row of $ A $, ie $ (10,6,-2,1) $, and $ \mathbf{l}_{1} $ is the first column of $ A $ scaled so that $ L_{1,1} = 1 $, ie. $ (1,1,-\frac{1}{5},\frac{1}{10}) $. Calculating

\[ \mathbf{l}_{1} \mathbf{u}_{1}^{T} = \begin{pmatrix}
10 & 6 & -2 & 1 \\
10 & 6 & -2 & 1 \\
-2 & -\frac{6}{5} & \frac{2}{5} & - \frac{1}{5} \\
1 & \frac{3}{5} & - \frac{1}{5} & \frac{1}{10}  
\end{pmatrix} \]

Now

\begin{align*}
\mathbf{A}_{1} & := A - \mathbf{l}_{1} \mathbf{u}_{1}^{T} \\
& = \left(
\begin{array}{cccc}
0 & & & \\
& 4 & -3 & -1 \\
& \frac{16}{5} & -\frac{12}{5} &
\frac{6}{5} \\
& \frac{12}{5} & -\frac{9}{5} &
\frac{29}{10} \\
\end{array}
\right)
\end{align*}


And so $ \mathbf{u}_{2}^{T} = (0,4,-3,-1  ) $, $ \mathbf{l}_{2} = (0,1,\frac{4}{5},\frac{3}{5}) $.

Next,

\[ \mathbf{l}_{2} \mathbf{u}_{2}^{T} = \left(
\begin{array}{cccc}
0 & & & \\
 & 4 & -3 & -1 \\
& \frac{16}{5} & -\frac{12}{5} &
-\frac{4}{5} \\
& \frac{12}{5} & -\frac{9}{5} &
-\frac{3}{5} \\
\end{array}
\right) \]

Thus

\begin{align*}
\mathbf{A}_{2} & := A_{1} - \mathbf{l}_{2} \mathbf{u}_{2}^{T} \\
& = \left(
\begin{array}{cccc}
0 & & & \\
 & 0 & & \\
 & & 0 & 2 \\
& & & \frac{7}{2} \\
\end{array}
\right)
\end{align*}

Here $ (A_{2})_{3,3} = 0 $, so we have a problem. In fact, all the entries in the third column of $ A_{2} $ are zero. So we pick $ \mathbf{l}_{3} = (0,0,1,0) $, $ \mathbf{u}_{3}^{T} = (0,0,0,2) $. \st{(Is this how to proceed?)}  Hence

\begin{align*}
\mathbf{A}_{3} & := A_{2} - \mathbf{l}_{3} \mathbf{u}_{3}^{T} \\
& = \left(
\begin{array}{cccc}
0 & & & \\
& 0 & & \\
& & 0 & \\
& & & \frac{7}{2} \\
\end{array}
\right)
\end{align*}

Finally giving $ \mathbf{l}_{4} = (0,0,0,1) $, $ \mathbf{u}_{4}^{T} = (0,0,0,\frac{7}{2}) $. Hence one factorization gives


\[ L = \begin{pmatrix}
1 & & & \\
1 & 1  & & \\
-\frac{1}{5} & \frac{4}{5} & 1 & \\
\frac{1}{10} & \frac{3}{5} & 0 & 1
\end{pmatrix}, \quad U = \begin{pmatrix}
10 & 6 &-2 & 1 \\
0 & 4 & -3 & -1 \\
0 & 0 & 0 & 2\\
0 & 0 & 0 & \frac{7}{2}
\end{pmatrix} \]

Solving $ A \mathbf{x} = \mathbf{b} $ is the same as $ L(U \mathbf{x}) = \mathbf{b} $, which we can decompose into $ L \mathbf{y} = \mathbf{b} $, $ U \mathbf{x} = \mathbf{y} $ (Both of which are triangular and easily solved).

First consider $  L \mathbf{y} = \mathbf{b} $, with $ \mathbf{b}^{T} = (-2,0,2,1) $. This gives

\[ 1 y_{1} = -2  \Rightarrow y_{1} = -2 \]

\[ 1 y_{1} + 1 y_{2} = 0 \Rightarrow y_{2} = 2 \]

\[ - \frac{1}{5} y_{1} + \frac{4}{5} y_{2} + y_{3} = 2 \Rightarrow y_{3} =  0 \]

\[ \frac{1}{10} y_{1} + \frac{3}{5} y_{2} + y_{4} = 1 \Rightarrow y_{4} =  0 \]

Having found $ \mathbf{y} $, we solve in reverse order for $ U\mathbf{x} = \mathbf{y} $. 

\[ \frac{7}{2} x_{4} = 0 \Rightarrow x_{4} = 0 \]

\[ 4 x_{2} - 3 x_{3} = 2  \]

\[ 10 x_{1} - 6 x_{2} - 2 x_{3} = -2  \]

\[ \Rightarrow  \]









\section{QUESTION 2}

Consider the $ LU $ factorization of the real $ n \times n $ symmetric matrix $ A $. We wish to prove that the elements of $ U $ satisfy the condition 

\[ | u_{ij} | < 2^{i-1} \alpha, \quad j = 1,\cdots,n \]

where $ \alpha = \max( | A_{ij} |  ) $. 

We proceed by induction:

\begin{itemize}
	\item for $ i = 1 $, the condition is $ | u_{1j} | \leq \alpha  $. Now $ u_{1j} $ are elements of $ \mathbf{u}_{1}^{T} $, which by the LU algorithm is just the first row of $ A $, thus clearly all the elements satisfy $ | A_{1j} | \leq \alpha $, so base done. 
	
	
	\item for the inductive step we assume that $ | u_{rj} | \leq 2^{r-1} \alpha  $. The LU decomposition algorithm says that $ \mathbf{u}_{r+1}^{T} $ is the $ (r+1) $th row of $ A_{r} $, where 
	
	\[ A_{r} = A_{r-1} - \mathbf{l}_{r}\mathbf{u}_{r}^{T} \] 
	
	Want to show that $ \left| u_{(r+1),j} \right| < 2^{r} \alpha $. 
	
	
	
	
\end{itemize} 

We see the condition for a 2x2 matrix to have this is such that, in $ A_{1} : = A - \mathbf{l}_{1} \mathbf{u}_{1}^{T} $, we want the $ 2,2 $ entry in $ \mathbf{l}_{1} \mathbf{u}_{1}^{T} $ to be $ - A_{2,2} $. If $ A = \begin{pmatrix}
a & b \\
c & d
\end{pmatrix} $

then $ \mathbf{l}_{1} \mathbf{u}_{1}^{T} $ is formed by $ \begin{pmatrix}
1 \\
\frac{c}{a}
\end{pmatrix} \begin{pmatrix}
a & b
\end{pmatrix}  $

so if $ \max(A_{ij})  = d $, we want $ \frac{bc}{a} = -d $, ie. $ ad + bc = 0 $. So, \[ A = \begin{pmatrix}
3 & -6\\
2 & 4
\end{pmatrix} \]

will do. Similarly considering 

\[ A = \begin{pmatrix}
a & b & c \\
d & e & f \\
g & h & i
\end{pmatrix} \]

Our condition becomes 

\[ 3(ah - bg)(af - cd) = (ae - bd)(-ia) \]



\section{QUESTION 3}

As $ \mathbf{u}_{1}^{T} $ is just the first row of $ A $, this is trivially true for $ k = 1 $. 

For $ k = 2 $, we suppose the first 2 rows of $ A $ span some plane (and are therefore linearly independent). 

In our LU decomposition algorithm, we simply have $ A_{1} = A - \mathbf{l}_{1} \mathbf{u}_{1}^{T}  $, and the second row of $ U, \mathbf{u}_{2}^{T} $, is the second row of $ A_{1,1} $. 

(general linear  combination arguments)

(do general case). 





\section{QUESTION 4}

\[ A = \begin{pmatrix}
1 & 1 & & & & \\
1 & 2 & 1 & & & \\
& 1 & 3 & 1 & & \\
& & 1 & 4 & 1 & \\
& & & 1 & 5 & 1 \\
& & & & 1 & \lambda
\end{pmatrix} \]

Define $ D^{1/2} $ as the diagonal matrix whose $ (k,k) $ element is $ D_{k,k}^{1/2} $, hence $ D^{1/2} D^{1/2}  =  D $. Then, $ A $ being positive definite, we can write

\[ A = (LD^{1/2})(D^{1/2} L^{T}) = ( L D^{1/2}) (  L D^{1/2} )^{T} \]

In other words, letting $ \tilde{L} = L D^{1/2} $, we obtain the \emph{Cholesky factorisation} $ A = \tilde{L} \tilde{L}^{T} $. 

Proceeding with our usual LU algorithm on $ A $, we have $ \mathbf{l}_{1} = (1,1,0,0,0,0) $, $ D_{1,1} = 1 $, thus

\begin{align*}
A_{1} & = A - D_{1,1} \mathbf{l}_{1} \mathbf{l}_{1}^{T} \\
& = \begin{pmatrix}
0 &  & & & & \\
 & 1 & 1 & & & \\
& 1 & 3 & 1 & & \\
& & 1 & 4 & 1 & \\
& & & 1 & 5 & 1 \\
& & & & 1 & \lambda
\end{pmatrix}
\end{align*} 

Next $ \mathbf{l}_{2} = (0,1,1,0,0,0) $, $ D_{2,2} = 1 $, thus

\begin{align*}
A_{2} & = A_{1} - D_{2,2} \mathbf{l}_{2} \mathbf{l}_{2}^{T} \\
& = \begin{pmatrix}
0 & & & & & \\
& 0 &  & & & \\
& & 2 & 1 & & \\
& & 1 & 4 & 1 & \\
& & & 1 & 5 & 1 \\
& & & & 1 & \lambda
\end{pmatrix}
\end{align*} 


Next $ \mathbf{l}_{3} = (0,0,1,\frac{1}{2},0,0) $, $ D_{3,3} = 2 $, thus

\begin{align*}
A_{3} & = A_{2} - D_{3,3} \mathbf{l}_{3} \mathbf{l}_{3}^{T} \\
& = \begin{pmatrix}
0 & & & & & \\
& 0 &  & & & \\
& & 0 & & & \\
& & & \frac{7}{2} & 1 & \\
& & & 1 & 5 & 1 \\
& & & & 1 & \lambda
\end{pmatrix}
\end{align*}

Next $ \mathbf{l}_{4} = (0,0,0,1,\frac{2}{7},0) $, $ D_{4,4} = \frac{7}{2} $, thus

\begin{align*}
A_{4} & = A_{3} - D_{4,4} \mathbf{l}_{4} \mathbf{l}_{4}^{T} \\
& = \begin{pmatrix}
0 & & & & & \\
& 0 &  & & & \\
& & 0 & & & \\
& & & 0 & & \\
& & & & \frac{33}{7} & 1 \\
& & & & 1 & \lambda
\end{pmatrix}
\end{align*}

Next $ \mathbf{l}_{5} = (0,0,0,0,1,\frac{7}{33}) $, $ D_{5,5} = \frac{33}{7} $, thus

\begin{align*}
A_{5} & = A_{4} - D_{5,5} \mathbf{l}_{5} \mathbf{l}_{5}^{T} \\
& = \begin{pmatrix}
0 & & & & & \\
& 0 &  & & & \\
& & 0 & & & \\
& & & 0 & & \\
& & & & 0 & \\
& & & & & \lambda - \frac{7}{33}
\end{pmatrix}
\end{align*}

Finally giving  $ \mathbf{l}_{6} = (0,0,0,0,0,1) $, $ D_{6,6} = \frac{1}{\lambda  - 33/7} $.

Hence $ A $ has the Cholesky factorisation $ A = \tilde{L} \tilde{L}^{T} $,  $ \tilde{L} = L D^{1/2} $, where


\[ L =  \begin{pmatrix}
1 & & & & &\\
& 1 & & & & \\
& 1 & 1 & & & \\
 &  & \frac{1}{2} & 1 & & \\
 &  &  & \frac{2}{7} & 1 & \\
 &  &  &  & \frac{7}{33} & 1
\end{pmatrix}, \quad D^{1/2} = \begin{pmatrix}
1 & & & & &\\
& 1 & & & & \\
&  & \sqrt{2} & & & \\
&  &  & \sqrt{\frac{7}{2}} & & \\
&  &  &  & \sqrt{\frac{33}{7}} & \\
&  &  &  &  & \frac{1}{\sqrt{\lambda - 33/7}}
\end{pmatrix} \]

Thus if 

 


\section{QUESTION 5}

Draw up some different banded matrices for different $ n $ and $ r $, think about gram-schmidt for each of them, generalize in terms of $r $ and $ n $. (how many subtractions would you need to do to get each column to be upper triangular).

\section{QUESTION 6}

Using the Gram-Schmidt algorithm we can factorize $ A = QR $ 

\[ \mathbf{a}_{k} = \sum_{j=1}^{k} r_{jk} \mathbf{q}_{k}, k = 1,2,3 \qquad (*)  \]

Setting $ k = 1 $ in (*) tells us that we must have $ R_{1,1} = | | \mathbf{a}_{1} | | = 7 $, and

\[ \mathbf{q}_{1} = \mathbf{a}_{1} / R_{1,1} = \begin{pmatrix}
6/7\\
3/7\\
2/7
\end{pmatrix}  \] 

Next we form the vector $ \mathbf{b} = \mathbf{a}_{2} - \langle \mathbf{q}_{1},\mathbf{a}_{2} \rangle \mathbf{q}_{1}  $, which is orthogonal to $ \mathbf{q}_{1} $. Have

\begin{align*}
\mathbf{b} & = \begin{pmatrix}
6\\
6\\
1
\end{pmatrix} - 8 \begin{pmatrix}
6/7\\
3/7\\
2/7
\end{pmatrix} \\
& = \begin{pmatrix}
- 6/7 \\
18/7\\
-9/7
\end{pmatrix}
\end{align*}

with $ R_{1,2} = \langle \mathbf{q}_{1},\mathbf{a}_{2} \rangle = 8$, $ R_{2,2} = | | \mathbf{b} | | = \frac{21}{7} $, and 

\[ \mathbf{q}_{2} = \mathbf{b} / R_{2,2} = \begin{pmatrix}
-2/7\\
6/7\\
-3/7
\end{pmatrix}  \] 

Now we form the vector $ \mathbf{c} = \mathbf{a}_{3} - \langle \mathbf{q}_{1},\mathbf{a}_{3} \rangle \mathbf{q}_{1} - \langle \mathbf{q}_{2},\mathbf{a}_{3} \rangle \mathbf{q}_{2}  $, which is orthogonal to both $ \mathbf{q}_{1} $ and $ \mathbf{q}_{2} $. Have

\[ R_{1,3} = \langle \mathbf{q}_{1},\mathbf{a}_{3} \rangle = \frac{11}{7}, R_{2,3} = \langle \mathbf{q}_{2},\mathbf{a}_{3} \rangle = \frac{1}{7} \] 

Thus

\begin{align*}
\mathbf{c} & = \begin{pmatrix}
1\\
1\\
1
\end{pmatrix} - \frac{11}{7} \begin{pmatrix}
6/7\\
3/7\\
2/7
\end{pmatrix} - \frac{1}{7} \begin{pmatrix}
-2/7\\
6/7\\
-3/7
\end{pmatrix} \\
& = \begin{pmatrix}
- 6/7 \\
18/7\\
-9/7
\end{pmatrix}
\end{align*}




with $ R_{3,3} = | | \mathbf{c} | | = \frac{21}{7} $







\section{QUESTION 7}

First pick $ \Omega^{[1,2]} $ so that $ (\Omega^{[1,2]}A)_{2,1} = 0 $. The choice of $ \theta $ is given by

\[ \cos \theta = \frac{A_{1,1}}{\sqrt{A_{1,1}^{2} + A_{2,1}^{2}}}, \quad \sin \theta  = \frac{A_{1,1}}{\sqrt{A_{1,1}^{2} + A_{2,1}^{2}}}  \]

Hence $ \cos \theta = 6 / \sqrt{45} $, $ \sin \theta = 3 / \sqrt{45} $, and

\begin{align*}
\Omega^{[1,2]}A  & =  \begin{pmatrix}
6 / \sqrt{45} & 3 / \sqrt{45} & 0 \\
-3  / \sqrt{45} & 6 / \sqrt{45} & 0 \\
0 & 0 & 1
\end{pmatrix} \begin{pmatrix}
6 & 6 & 1 \\
3 & 6 & 1 \\
2 & 1 & 1
\end{pmatrix}  \\
& = \left(
\begin{array}{ccc}
3 \sqrt{5} & \frac{18}{\sqrt{5}} &
\frac{3}{\sqrt{5}} \\
0 & \frac{6}{\sqrt{5}} &
\frac{1}{\sqrt{5}} \\
2 & 1 & 1 \\
\end{array}
\right)
\end{align*}

Next, pick $ \Omega^{[1,3]} $ so that $ (\Omega^{[1,3]}\Omega^{[1,2]}A)_{3,1} = 0 $. Say $ B = \Omega^{[1,2]} A $, so the choice of $ \theta $ is given by

\[ \cos \theta = \frac{B_{1,1}}{\sqrt{B_{1,1}^{2} + B_{3,1}^{2}}}, \quad \sin \theta  = \frac{B_{3,1}}{\sqrt{B_{1,1}^{2} + B_{3,1}^{2}}}  \]

Hence $ \cos \theta = \frac{3}{7} \sqrt{5} $, $ \sin \theta = \frac{2}{7} \sqrt{5} $, and

\begin{align*}
\Omega^{[1,3]}\Omega^{[1,2]}A  & =  \begin{pmatrix}
\frac{3}{7} \sqrt{5} & 0 & \frac{2}{7} \\
0 & 1 & 0 \\
- \frac{2}{7} & 0 & \frac{3}{7} \sqrt{5}
\end{pmatrix} \left(
\begin{array}{ccc}
3 \sqrt{5} & \frac{18}{\sqrt{5}} &
\frac{3}{\sqrt{5}} \\
0 & \frac{6}{\sqrt{5}} &
\frac{1}{\sqrt{5}} \\
2 & 1 & 1 \\
\end{array}
\right)  \\
& = \left(
\begin{array}{ccc}
7 & 8 & \frac{11}{7} \\
0 & \frac{6}{\sqrt{5}} &
\frac{1}{\sqrt{5}} \\
0 & -\frac{36}{7 \sqrt{5}}+\frac{3
	\sqrt{5}}{7} & -\frac{6}{7
	\sqrt{5}}+\frac{3 \sqrt{5}}{7} \\
\end{array}
\right) 
\end{align*}


Finally, pick $ \Omega^{[2,3]} $ so that $ (\Omega^{[2,3]} \Omega^{[1,3]}\Omega^{[1,2]}A)_{3,2} = 0 $. Say $ C = \Omega^{[1,3]} \Omega^{[1,2]} A $, so the choice of $ \theta $ is given by

\[ \cos \theta = \frac{C_{2,2}}{\sqrt{C_{2,2}^{2} + C_{3,2}^{2}}}, \quad \sin \theta  = \frac{C_{3,2}}{\sqrt{C_{2,2}^{2} + C_{3,2}^{2}}}  \]

Hence $ \cos \theta = \frac{3}{7} \sqrt{5} $, $ \sin \theta = \frac{2}{7} \sqrt{5} $, and






\section{QUESTION 8}
\section{QUESTION 9}


\section{QUESTION 10}




\end{document}