\documentclass[a4paper]{article}
\usepackage{amsmath}
\def\npart {IB}
\def\nterm {Lent}
\def\nyear {2018}
\def\nlecturer {Dr. Saxton}
\def\ncourse {Numerical Analysis Example Sheet 3}

\input{header}

\newtheorem*{soln}{Solution}

\renewcommand{\thesection}{}
\renewcommand{\thesubsection}{\arabic{section}.\arabic{subsection}}
\makeatletter
\def\@seccntformat#1{\csname #1ignore\expandafter\endcsname\csname the#1\endcsname\quad}
\let\sectionignore\@gobbletwo
\let\latex@numberline\numberline
\def\numberline#1{\if\relax#1\relax\else\latex@numberline{#1}\fi}
\makeatother


\begin{document}
	
\maketitle

\section{QUESTION 1}

First, $ \mathbf{u}_{1}^{T} $ is just the first row of $ A $, ie $ (10,6,-2,1) $, and $ \mathbf{l}_{1} $ is the first column of $ A $ scaled so that $ L_{1,1} = 1 $, ie. $ (1,1,-\frac{1}{5},\frac{1}{10}) $. Calculating

\[ \mathbf{l}_{1} \mathbf{u}_{1}^{T} = \begin{pmatrix}
10 & 6 & -2 & 1 \\
10 & 6 & -2 & 1 \\
-2 & -\frac{6}{5} & \frac{2}{5} & - \frac{1}{5} \\
1 & \frac{6}{10} & - \frac{1}{5} & \frac{1}{10}  
\end{pmatrix} \]

Now

\begin{align*}
\mathbf{A}_{1} & := A - \mathbf{l}_{1} \mathbf{u}_{1}^{T} \\
& = \begin{pmatrix}
0 & 0 & 0 & 0 \\
0 & 4 & -3 & -1 \\
0 & \frac{16}{5} & - \frac{14}{5} & \frac{6}{5} \\
0 & \frac{24}{10} & - \frac{9}{10} & \frac{29}{10}
\end{pmatrix}
\end{align*}


And so $ \mathbf{u}_{2}^{T} = (0,4,-3,-1  ) $, $ \mathbf{l}_{2} = (0,1,\frac{4}{5},\frac{6}{10}) $.

Next,



\section{QUESTION 2}
\section{QUESTION 3}


\end{align*}


\section{QUESTION 4}
\section{QUESTION 5}
\section{QUESTION 6}
\section{QUESTION 7}
\section{QUESTION 8}
\section{QUESTION 9}


\section{QUESTION 10}
\section{QUESTION 11}
\section{QUESTION 12}



\end{document}