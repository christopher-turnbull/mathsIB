\documentclass[a4paper]{article}

\def\npart {IB}
\def\nterm {Michaelmas}
\def\nyear {2017}
\def\nlecturer {C.G.Caulfield }
\def\ncourse {Methods}

\include{header}

\begin{document}
\maketitle

\setcounter{section}{-1}
\section{Introduction}

I will never say anything that is untrue deliberately...

Self-adjoint ODEs

\section{Fourier Series}

\subsection{Peridoidic Functions}

\begin{defi}
	A function $ f(t) $ is \emph{periodic} with period $ T $ if $ f(t + T) = f(T) $
\end{defi}

Fig 1

\begin{eg}
	 \[ A \sin \omega t   \]
	
	$ A  $ is the \emph{amplitude}, $ \omega $ is the \emph{frequency}, $ 2 \pi / \omega $ is the \emph{period}.
\end{eg}

Sines and cosines are beautiful because they have an orthogonality property:

\[ \cos(A \pm B) = \cos A \cos B \mp \sin A \sin B \]


\[ \cos A \cos B = \frac{1}{2} \left[  \cos (A - B) + \cos(A + B) \right] \]
\[ \sin A \sin B = \frac{1}{2} \left[  \cos (A - B) - \cos(A + B) \right] \]

We want to consider $ \sin n\pi x / l $, $ \sin m \pi x / l $, where $ n, m $ are positive integers. These functions are periodic with period $ 2l $. \footnote{They have a common period of $ 2l $, not their smallest period!}



\begin{align*}
SS_{mn} & : = \int_{0}^{2l} \sin \left( \frac{m \pi x}{l} \right) \sin \left( \frac{n \pi x}{l} \right) \d x \\
& = \frac{1}{2} \int_{0}^{2l} \cos \left[  \frac{(m-n) \pi x}{l} \right]  \; \d x - \frac{1}{2} \cos \left[  \frac{(m+n) \pi x}{l} \right] \; \d x 
\end{align*}


if $ m \neq n $,

\[ SS_{mn} = \frac{l}{2 \pi} \left[  \frac{\sin (m-n)\pi x / l}{m - n} -  \frac{\sin (m+n)\pi x / l}{m + n} \right]_{0}^{2l} = 0  \]

if $ m = n $, then $  SS_{mn} = 1 $ (provided $ m \neq 0 $, $ n \neq 0 $). Hence
\[ SS_{mn} = \begin{cases} \delta_{mn}  & \text{ if }  m, n \neq 0 \\ 0 & \text{ if }   m \text{ or } n = 0\end{cases}  \]

Similarly, $ CC_{mn} = \int_{0}^{2l} \cos \left( \frac{m \pi x }{l} \right) \cos \left( \frac{n \pi x }{l} \right) \; \d x = l \delta_{mn} $
 $ \forall m,n \neq 0 $, and $ 2l $ if $ m = n = 0 $
 
Finally,

\begin{align*}
CS_{mn} & = \int_{0}^{2l} \cos \left( \frac{m \pi x }{l} \right) \sin \left( \frac{n \pi x }{l} \right) \; \d x  \\
& = \frac{1}{2} \int_{0}^{2l} \frac{\sin (m + n) \pi x }{l} \; \d x + \frac{1}{2} \int_{0}^{2l} \frac{\sin (m - n) \pi x }{l} \; \d x = 0
\end{align*}

By analogy with vectors [these integrals are indeed \emph{inner products}], $ \sin n \pi x / l $, $ \cos n \pi x / l $ are said to be orthogonal on the interval $ [0,2l] $.

They actually constitute an \emph{orthogonal basis}. ie. it is possible to represent an arbitrary (but sufficiently well behaved\footnote{to be definied}) function in terms of an infinite series (Fourier series) formed as a sum of sins and cosines. 


\subsection{Definition of a Fourier Series}

Any well behaved periodic function $ f(x) $ with periodic $ 2L $ can be written as a Fourier Series:

\[ \frac{f(x_{+}) + f(x_{-})}{2} = \frac{1}{2} a_{0} + \sum_{n=1}^{\infty} a_{n} \cos \left( \frac{n \pi x}{L} \right) + b_{n} \sin \left( \frac{m \pi x}{L} \right)   \] 

$ a_{n} $ and $ b_{n} $ are the Fourier Coefficients, $ f(x_{+}) $ and $ f(x_{-}) $ are the right limit approaching form above and the left limit approaching from below respectively 

If $ f(x) $ is continuous at $ x_{c} $, then the LHS is just $ f(x) $. If $ f(x) $ has a bounded discontinuity, at $ x_{d} $, ie. $ f(x_{d}^{-}) \neq f(x_{d}^{+})$, but $ (f(x_{d}^{-}) - f(x_{d}^{+})) $ is finite, then the FS tends to the mean value of the two limits. 

Coefficient construction:
Multiply rhs of (*) by $ \sin m \pi x / L $, integrate over $ 0 $ to $ 2L $, assume you can invert order or summation and integration.

\begin{align*}
\int_{0}^{2L} \left[  \frac{a_{0}}{2} + \sum_{n=1}^{\infty} a_{n} \cos \left( \frac{n \pi x}{L} \right) + b_{n} \sin \left( \frac{m \pi x}{L} \right) \right] \sin \frac{m \pi x}{2} \; \d x
\end{align*}

We see that


\[ \frac{a_{0}}{2}\int_{0}^{2L} \sin \frac{m \pi x}{L} \; \d x = 0 \]
\[ \sum_{n=1}^{\infty} \int_{0}^{2L} a_{n} \cos \left( \frac{n \pi x}{L} \right) \sin \left( \frac{m \pi x}{L} \right) \; d x = 0 \]
\[ \sum_{n=1}^{\infty}  \int_{0}^{2L} b_{n} \sin \left( \frac{n \pi x}{L} \right) \sin \left( \frac{m \pi x}{L} \right) \; d x = Lb_{n} \]

So 

\[ \text{LHS} = \int_{0}^{2L} f(x) \sin \left( \frac{m \pi x}{L} \right)  \; \d x \Rightarrow b_{m} = \frac{1}{L} \int_{0}^{2L} f(x) \sin \left( \frac{m \pi x}{L} \right) \; \d x \]


Multiply by $ \cos \frac{m \pi x}{l} $ and integrate from $ 0 $ to $ 2L $ (inc $ m = 0 $)


\[ \int_{0}^{2L} \left( \frac{1}{2} a_{0} + \sum_{n=1}^{\infty} a_{n} \cos \frac{n \pi x}{L}  + b_{n} \sin \frac{m \pi x}{L}\right) \cos \frac{m \pi x}{L} \; \d x \]

Non zero only when $ m = 0 $

Therefroe 

\[ \frac{a_{0}}{2} 2L = \int_{0}^{2L} f(x) \; \d x \Rightarrow \frac{a_{0}}{2} = \frac{1}{2L} \int_{0}^{2L} f(x) \; \d x \]

\[ a_{m} = \frac{1}{L} \int_{0}^{2L} f(x) \cos \left(  \frac{m \pi x}{L}  \right) \; \d x  \]

The range of integration is one period so its also permissibel to choose $ \int_{-L}^{L} $

a paricularly nice case is when $ L = \pi $.

\[ a_{m}  = \frac{1}{\pi} \int_{-\pi}^{\pi} f(x) \cos m x \; \d x \qquad m \geq 0 \]

\[ b_{m} = \frac{1}{\pi} \int_{\pi}^{\pi} f(x) \sin m x \; \d x \qquad m \geq 1 \]


\subsection{Dirichlet Conditions}

If $ f(x) $ is a periodic function with period $ 2l $ st.

\begin{enumerate}
	\item it is absolutely integrable \footnote{ie. $ \int_{0}^{2l} | f(x) | \; \d x $ is well defined}
	\item it has a finite number of extrema (ie maxs and mins) in $ [0,2l] $
	\item it has a finite number of bounded discontinuities in $ [0,2l] $
	
\end{enumerate}

then the FS representation converges to $ f(x) $ for all points where $ f(x) $ is cts, and at points $ x_{d} $ where $ f(x) $ is discontinuous, the series coverges to the avg value of the left and right limits, ie. to $ \frac{1}{2} \left( f(x_{d_{+}}) + f(x_{d_{-}}) \right)  $. These conditions are satisfied if the function is of `bounded variation'

\subsection{Smoothness and order of Fourier coefficients}

If the $ p^{\text{th}} $ derivative is the lowest derivative which is discontinuous somewhere (inc at the endpoints), then the F.C. are $ \Oc[n^{-(p+1)}] $ as $ n \to \infty $, eg. if a function has a bounded discontinuity, zeroth derivative is discontinuous: coefficients are of order $ \frac{1}{n} $ as $ n \to \infty $

\begin{eg}
	The sawtooth function, $ f(x) = x $ on $ -L \leq x \leq L $
	
	  \begin{center}
		\begin{tikzpicture}
		\draw (-3.2, 0) -- (3.2, 0);
		
		\draw [thick, mblue] (-3, -1) -- (-1, 1);
		\draw [dashed, mblue] (-1, 1) -- (-1, -1);
		\draw [thick, mblue] (-1, -1) -- (1, 1);
		\draw [dashed, mblue] (1, 1) -- (1, -1);
		\draw [thick, mblue] (1, -1) -- (3, 1);
		
		\node at (-3, 0) [below] {$-3\pi$};
		\node at (-1, 0) [below] {$-\pi$};
		\node at (1, 0) [below] {$\pi$};
		\node at (3, 0) [below] {$3\pi$};
		\end{tikzpicture}
	\end{center}
\end{eg}

Function is odd, so

\[ a_{m} = \frac{1}{L} \int_{L}^{-L} x cos\left( \frac{m \pi x}{L} \right) \\ \d x = 0 \],

\begin{align*}
b_{m} = \frac{1}{L} \int_{-L}^{L} x \sin \left( \frac{m \pi x}{L}\right) \; \d x & = \frac{1}{L} \left(  \left[  - \frac{xL}{m\pi} \cos\left( \frac{m \pi x}{L} \right)  \right]_{-L}^{L} - \int_{-L}^{L} \frac{-L}{m \pi} \cos\left( \frac{m \pi x}{L} \right) \; \d x  \right)   \\
& = \frac{1}{m\pi} \left( -2L cos(m \pi) + \left[  \sin \left( \frac{m \pi x}{L} \right) \frac{L}{m \pi}  \right]_{-L}^{L}  \right) \\
& = \frac{2L}{m\pi} (-1)^{m+1}
\end{align*}

So

\[ \frac{f(x_{+}) + f(x_{-})}{2} = \frac{2L}{\pi} \left[  \sin \left( \frac{\pi x}{L} \right) - \frac{1}{2} \sin  \left( \frac{2 \pi x}{L} \right) + \cdots  \right]  \]

\begin{enumerate}
	\item $ f_{N} (x) : = \sum_{n=1}^{N} b_{n} \sin\left( \frac{n \pi x}{L} \right) \to f(x)  $ almost everywhere, but the convergence is non-uniform. 
	\item Persistence overshoot @ $ x = L $: 'Gibbs phenomenon'
	\item $ f(L) = 0 $ average of right and left limits
	\item Coefficients are $ \Oc (\frac{1}{n}) $ as $ n \to \infty $
	
\end{enumerate}

\begin{eg}
	The integral of the sawtooth function, $ f(x) = \frac{1}{2} x^{2} $, $ - L \leq x \leq L $
\end{eg}
	
\begin{ex}
	\[ f(x) = L^{2}  \left[  \frac{1}{6} + 2 \sum_{n=1}^{\infty} \frac{(-1)^{n}}{(n \pi)^{2}} \cos \left( \frac{n \pi x}{L} \right)  \right]  \]
\end{ex}

Note at $ x = 0 $, 

\[ 0 = L^{2} \left[ \frac{1}{6}  + 2 \sum_{n=1}^{\infty} \frac{(-1)^{2}}{(n \pi)^{2}} \right] \Rightarrow \frac{\pi^{2}}{12} = \sum_{n=1}^{\infty} \frac{(-1)^{n+1}}{n^{2}} \]

\section{Properties of the Fourier Series}

\subsection{Integration and Differentiation}

\subsubsection{Integration: Always works!}

FS. can be integrated term by term:

$ f(x) $ periodic with period $ 2L $ and has a FS (so it satisfies Dirichlet conditions)\footnote{pay attention to the limits here}:

\[ \frac{f(x_{+}) + f(x_{-})}{2} = \frac{1}{2} a_{0} + \sum_{n=1}^{\infty} a_{n} \cos \left( \frac{n \pi x}{L} \right) + b_{n} \sin \left( \frac{m \pi x}{L} \right)   \] 

\begin{align*}
F(x) = \int_{-L}^{x} f(x') \; \d x' & = \frac{a_{0}(x + L)}{2} + \sum_{n=1}^{\infty} \frac{a_{n}L}{n \pi } \sin\left( \frac{n \pi x}{L} \right) \\
& \qquad  + \sum_{n = 1 }^{\infty} \frac{b_{n} L}{n \pi} \left[  (-1)^{n} - \cos \left( \frac{n \pi x}{L} \right)  \right] \\
& = \frac{a_{0}L}{2} + L \sum_{n=1}^{\infty} (-1)^{n} \frac{b_{n}}{n \pi} \\
& \qquad - L \sum_{n = 1 }^{\infty}\frac{b_{n}}{n \pi} \cos\left( \frac{n \pi x}{L} \right)  \\
& \qquad + L \sum_{n = 1 }^{\infty} \left( \frac{a_{n} - (-1)^{n}a_{0}}{n \pi} \right) \sin \left( \frac{n \pi x}{L} \right)  
 \end{align*}
 
If $ a_{n} $ and $ b_{n} $ are FC then the series involving $ \frac{a_{n}}{n} $ and $ \frac{b_{n}}{n}  $ (multipled by cos or sin) must also converge

The Fourier series of $ f(x) $ exists, so $ b_{n} $ is at least of $ \Oc (\frac{1}{n}) $ as $ n \to \infty \Rightarrow \frac{b_{n}}{n} $ is at least $ \Oc (\frac{1}{n^{2}}) $ as $ n \to \infty $, and so by the comparison test with $ \sum_{n=1}^{\infty} \frac{M}{n^{2}} $, the second term on the RHS converges $ \Rightarrow F(x) $ has a FS.

Note: integration smooths. Proof relies on discontinuity being bdd, ($ f(x) $ satisfies Dirichlet condition). 

\subsubsection{Differentiation: Doesn't always work!}

Let $ f(x) $ be a periodic function with period 2, st.

\[ f(x) = \begin{cases} 1  & \text{ if } 0 < x < 1 \\ - 1 & \text{ if } -1 < x < 0 \end{cases} \]

Odd function $ \Rightarrow a_{m} = 0 $.

\begin{align*}
	b_{m} & = - \int_{-1}^{0} \sin(m \pi x) \; \d x + \int_{0}^{1} \sin(m \pi x) \; \d x \\
	& = \left[  \frac{\cos(m \pi x)}{m \pi} \right]_{-1}^{0} - \left[  \frac{\cos(m \pi x)}{m \pi} \right]_{0}^{1}\\
	& = \frac{1}{m \pi} \left[  1 - (-1)^{m} - (-1)^{m} + 1 \right]   \\
	& = \frac{4}{\pi x} \text{ if } m  \text{ odd, or } 0 \text{ if } m \text{ even }
\end{align*}



\[ \frac{f(x_{+}) + f(x_{-})}{2}  = \frac{4}{\pi} \sum_{n=1}^{\infty} \frac{\sin\left(  (2n - 1)\pi x \right) }{2n - 1} \]

Apply diff rules:

\[ f'(x) = 4 \sum_{n=1}^{\infty} \cos \left( (2n - 1)\pi x \right)  \]

This is clearly divergent, even though $ f(x) = 0 $ for all $ x \neq 0 $.

 The extra factor of $ 2n - 1 $ is the problem. It's related to the discontinuity, $ f'(x) $ does not satisfy the Dirichlet condition
 
Differentiation can be done under certain circumstances. 

\begin{eg}
	Assume the function $ f(x) $ is continuous and is extended as a $ 2L $-periodic function, piece-wise continuously differentiable on $ (0,2L) $. Let $ g(x) = \frac{\d f}{\d x} $ such that $ g(x) $ satisfies D.C.\footnote{$ g(x) $ has at worst a finite number of bounded discontinuitites.} 
	
	\[ f(x) =  \frac{1}{2} a_{0} + \sum_{n=1}^{\infty} a_{n} \cos \left( \frac{n \pi x}{L} \right) + b_{n} \sin \left( \frac{m \pi x}{L} \right) \]
	
	\[ \frac{g(x_{+}) + g(x_{-})}{2} = \frac{A_{0}}{2} + \sum_{n=1}^{\infty} A_{n} \cos \left( \frac{n \pi x}{L} \right) + B_{n} \sin \left( \frac{n \pi x}{L} \right)   \]
	
	\[ A_{0} = \frac{1}{L} \int_{0}^{2L} g(x) \; \d x = \frac{f(2L)  - f(0)}{L} = 0 \qquad \text{ by periodicity} \]
	
	\begin{align*}
		A_{n} & = \frac{1}{L} \int_{0}^{2L} \frac{\d f}{\d x} \cos \left( \frac{n \pi x}{L} \right) \; \d x\\
		& = \frac{1}{L} \left[ f(x) \cos \left(\frac{n \pi x}{L} \right)  \right]_{0}^{2L} + \frac{n \pi}{L^{2}} \int_{0}^{2L} f(x) \sin \left( \frac{n \pi x}{L} \right) \; \d x \\
		&= 0 + \frac{n \pi b_{n}}{L}
	\end{align*}

	\begin{ex}
		$ B_{n} = \frac{- n \pi a_{n}}{L} $
	\end{ex}
\end{eg}

We see differentiation reduces to multiplying by $ \pm \frac{n \pi}{L} $


\subsection{Alternate representation: complex form}

Remember 

\[ \cos \left( \frac{n \pi x}{L} \right) = \frac{1}{2} \left(  e^{\frac{ i n \pi x}{L}} + e^{\frac{- i n \pi x}{L}}  \right)  \]

\[ \sin \left( \frac{n \pi x}{L} \right) = \frac{1}{2i} \left(  e^{\frac{ i n \pi x}{L}} - e^{\frac{- i n \pi x}{L}}  \right)  \]

\[ \frac{f(x_{+}) + f(x_{-})}{2} = \frac{a_{0}}{2} + \sum_{n=1}^{\infty} \frac{a_{n}}{2} \left(  e^{\frac{ i n \pi x}{L}} + e^{\frac{- i n \pi x}{L}}  \right) - \frac{b_{n i}}{2} \left(  e^{\frac{ i n \pi x}{L}} - e^{\frac{- i n \pi x}{L}}  \right)\]

\[ = \frac{a_{0}}{2} = \sum_{n=1}^{\infty} \left( \frac{a_{n} - i b_{n}}{2} \right) e^{\frac{i n \pi x}{L}} + \sum_{n=1}^{\infty} \left( \frac{a_{n} + i b_{n}}{2} \right) e^{\frac{- i n \pi x}{L}}  \]

\[ = \sum_{-\infty}^{\infty} c_{n} e^{\frac{i \pi n x}{L}} \quad c_{0} = \frac{a_{0}}{2} ; \underbrace{c_{n}}_{ (n > 0)} = \frac{a_{n} - i b_{n}}{2}; \underbrace{c_{n}}_{ (n < 0)} = \frac{a_{n} + i b_{n}}{2} \]

Note that 

\[ c_{m}^{*} = c_{-m} \]

complex exponentials are orthogonal

\[ \int_{0}^{2L} e^{\frac{i n \pi x}{L}} e ^{\frac{- i n \pi x}{L}} \; \d x = \int_{0}^{2L} \cos \left( \frac{(n - m) \pi x}{L} \right)  \; \d x  + i \int_{0}^{2L}  \underbrace{\sin(n-m)}_{0 \text{ by periodicity}}\frac{\pi x}{L} \; \d x = 2L \delta_{nm} \]


\[ c_{m} = \frac{1}{2L} \int_{0}^{2L} f(x) e^{\frac{-i m \pi x}{L}} \; \d x = \frac{1}{2L} \int_{0}^{2L} \sum_{-\infty}^{\infty} c_{n} e^{\frac{i n \pi x}{L}} e ^{\frac{- i n \pi x}{L}} \; \d x \]

Now assume $ g(x) = \frac{\d f}{\d x} = \sum_{n = -\infty}^{\infty} c_{n} e^{\frac{i n \pi x}{L}} $


\begin{align*}
	c_{n} & = \frac{1}{2L} \int_{0}^{2L} \frac{\d f}{\d x} e^{\frac{- i n \pi x}{L}} \; \d x \\
	& = \frac{1}{2L} \left[  f(x) e^{\frac{- i n \pi x }{L}} \right]_{0}^{2L} + \frac{i n \pi}{2L^{2}} \int_{0}^{2L} f(x) e^{\frac{- i n \pi x}{L}} \; \d x \\
	& = \frac{i n \pi}{L} c_{n} \quad \text{by periodicity}
\end{align*}

\subsection{Half-range series}

Consider a functon defined only on $ 0 \leq x \leq L $.

There are two possible ways to extend it to a $ 2L $-periodic function that can be represented as a FS. \footnote{It would be a dumbass thing to extend it as an odd function; you'll have a discontinuity!}

\subsubsection{Fourier sine series: odd function}

$ f(x) $ can be extended as an \emph{odd} function
$ f(x) = -f(-x) $ on $ -L \leq x \leq L $ and then extended as a $ 2L $-periodic function. In this case $ a_{n}= 0 $ and we can define the \emph{Fourier sine series}

\[ \frac{f(x_{+}) + f(x_{-})}{2} = \sum_{n=1}^{\infty} b_{n} \sin \left( \frac{n \pi x}{L} \right) \qquad b_{n} = \frac{2}{L} \int_{0}^{L} f(x) \sin \left(  \frac{n \pi x}{L} \right) \; \d x   \]

\begin{eg}
	Sawtooth function 
	
	
	 \begin{center}
		\begin{tikzpicture}
		\draw (-3.2, 0) -- (3.2, 0);
		
		\draw [thick, mblue] (-3, -1) -- (-1, 1);
		\draw [dashed, mblue] (-1, 1) -- (-1, -1);
		\draw [thick, red] (-1, -1) -- (0,0);
		\draw [thick, mgreen] (0, 0) -- (1, 1);
		\draw [dashed, mblue] (1, 1) -- (1, -1);
		\draw [thick, mblue] (1, -1) -- (3, 1);
		
		\node at (-3, 0) [below] {$-3\pi$};
		\node at (-1, 0) [below] {$-\pi$};
		\node at (1, 0) [below] {$\pi$};
		\node at (3, 0) [below] {$3\pi$};
		\end{tikzpicture}
	\end{center}


FS describes the $ 2L $ periodic fn. Sine seires still describes the fn on $ [0,L] $
\end{eg}


\subsubsection{Even functions: fourier cosine series}
$ f(x) $ can also be extended as an even fn on $ -L \leq x \leq L $ ie. $ f(x) = f(-x) $ and then extended as a $ 2L $-periodic fn: $ \Rightarrow b_{n} = 0 \; \forall n $.

Fourier cosine series:

\[ \frac{f(x_{+}) + f(x_{-})}{2} = \frac{a_{0}}{2} \sum_{n=1}^{\infty} a_{n} \cos \left( \frac{n \pi x}{L} \right) \qquad a_{n} = \frac{2}{L} \int_{0}^{L} f(x) \cos \left(  \frac{n \pi x}{L} \right) \; \d x   \]


\begin{center}
	\begin{tikzpicture}
	\draw (-3.2, 0) -- (3.2, 0);
	
	\draw [thick, mblue] (-3, 1) -- (-2, 0);
	\draw [thick, mblue] (-2, 0) -- (-1, 1);
	\draw [dashed, mblue] (-1, 1) -- (-1, -1);
	\draw [thick, red] (-1, 1) -- (0,0);
	\draw [thick, mgreen] (0, 0) -- (1, 1);
	\draw [dashed, mblue] (1, 1) -- (1, -1);
	\draw [thick, mblue] (1, 1) -- (2, 0);
	\draw [thick, mblue] (2, 0) -- (3, 1);
	
	\node at (-3, 0) [below] {$-3\pi$};
	\node at (-1, 0) [below] {$-\pi$};
	\node at (1, 0) [below] {$\pi$};
	\node at (3, 0) [below] {$3\pi$};
	\end{tikzpicture}
\end{center}


\subsection{Parserals Theorem}

'Energy' of a periodic signal is often of interest, ie

\[ E = \int_{0}^{2L} f^{2}(x) \; \d x  \]

Consider the general case 

\[ f(x) = \sum_{-\infty}^{\infty} c_{n} e^{\frac{i n \pi x}{L}} \]

where 

\[ g(x) = \sum_{m = - \infty}^{\infty} d_{m} e^{\frac{i m \pi x}{L}} \]


\[ \int_{0}^{2L} f(x) g(x) \; \d x = \sum_{n = - \infty}^{\infty} \sum_{m = -\infty}^{\infty} c_{n} d_{m} \int_{0}^{2L} \exp \left[  \frac{i \pi x}{L} (n + m) \right]  \; \d x \]

\[ =  \sum_{n = - \infty}^{\infty} \sum_{m = -\infty}^{\infty} c_{n} d_{m} \left( 2L \delta_{n[-m]} \right)  \]

\[ = \sum_{n = -\infty}^{\infty} c_{n} d_{-n} = 2L \sum_{n = -\infty}^{\infty} c_{n} d_{n}^{*} \]


So if $ g(x) = f(x) $

\[ \int_{0}^{2L} \left[  f(x) \right]^{2} \; \d x = 2L \sum_{n = -\infty}^{\infty} | c_{n} |^{2} = L\left[  \frac{a_{0}^{2}}{2} + \sum_{n=1}^{\infty}  \left(a_{n}^{2}  + b_{n}^{2} \right)  \right]  \]


\begin{eg}
	Remember $ f(x) = x $ for $ -L \leq x \leq L $
	\[ b_{n} = \frac{2L}{m \pi}(-1)^{m+1} \]
	
	\[ \int_{-L}^{L} x^{2} \; \d x = \frac{2L^{3}}{3} = L \sum_{m = 1}^{\infty}  \frac{4 L^{2}}{m^{2} \pi^{2}} \]
	
	\[ \Rightarrow \sum_{m=1}^{\infty} \frac{1}{m^{2}} = \frac{\pi^{2}}{6} \]
\end{eg}

\begin{ex}
	From the FS of $ \frac{x^{2}}{2} $ show that
	
	\[ \sum_{m=1}^{\infty} \frac{1}{m^{4}} = \frac{\pi^{4}}{90} \]
\end{ex}

\section{Strum-Liouville Theory Motivation}

\subsection{Second order ODEs}

\[ \mathcal{L} y(x) = \alpha(x) \frac{\d^{2} }{\d x^{2} } y + \beta(x) \frac{\d y }{\d x } + \gamma(x)  y = f(x) \]

$ \alpha, \beta, \gamma $ continuous, $ \alpha $ non zero except perhaps at a finite number of isolated points, 

$ f(x) $ is bounded, defined on $ a \leq x \leq b $ ( $ a $ or $ b $ may be $ \pm \infty $).

HOMOGENOUS: eg 
\[ \mathcal{L} y = 0  \]

has two linearly independent solutions $ y_{1},y_{2} $ complementary function $ y_{c} = Ay_{1} + By_{2} $

INHOMOGENEOUS OR FORCED EQUATION. 

\[ \mathcal{L} y = f(x)  \]

$ F $ is the \emph{forcing} and has a particular integral $ y_{p} (x) $. G.S. is $ y = y_{c}(x) + y_{p}(x) $ where $ A + B $ are determined in a `PROBLEM' by applying conditions





\end{document}