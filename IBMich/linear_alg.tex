\documentclass[a4paper]{article}

\def\npart {IB}
\def\nterm {Michaelmas}
\def\nyear {2017}
\def\nlecturer {Dr. Keating }
\def\ncourse {Linear Algebra}

% Imports
\ifx \nauthor\undefined
  \def\nauthor{Christopher Turnbull}
\else
\fi

\author{Supervised by \nlecturer \\\small Solutions presented by \nauthor}
\date{\nterm\ \nyear}

\usepackage{alltt}
\usepackage{amsfonts}
\usepackage{amsmath}
\usepackage{amssymb}
\usepackage{amsthm}
\usepackage{booktabs}
\usepackage{caption}
\usepackage{enumitem}
\usepackage{fancyhdr}
\usepackage{graphicx}
\usepackage{mathdots}
\usepackage{mathtools}
\usepackage{microtype}
\usepackage{multirow}
\usepackage{pdflscape}
\usepackage{pgfplots}
\usepackage{siunitx}
\usepackage{slashed}
\usepackage{tabularx}
\usepackage{tikz}
\usepackage{tkz-euclide}
\usepackage[normalem]{ulem}
\usepackage[all]{xy}
\usepackage{imakeidx}

\makeindex[intoc, title=Index]
\indexsetup{othercode={\lhead{\emph{Index}}}}

\ifx \nextra \undefined
  \usepackage[pdftex,
    hidelinks,
    pdfauthor={Christopher Turnbull},
    pdfsubject={Cambridge Maths Notes: Part \npart\ - \ncourse},
    pdftitle={Part \npart\ - \ncourse},
  pdfkeywords={Cambridge Mathematics Maths Math \npart\ \nterm\ \nyear\ \ncourse}]{hyperref}
  \title{Part \npart\ --- \ncourse}
\else
  \usepackage[pdftex,
    hidelinks,
    pdfauthor={Christopher Turnbull},
    pdfsubject={Cambridge Maths Notes: Part \npart\ - \ncourse\ (\nextra)},
    pdftitle={Part \npart\ - \ncourse\ (\nextra)},
  pdfkeywords={Cambridge Mathematics Maths Math \npart\ \nterm\ \nyear\ \ncourse\ \nextra}]{hyperref}

  \title{Part \npart\ --- \ncourse \\ {\Large \nextra}}
  \renewcommand\printindex{}
\fi

\pgfplotsset{compat=1.12}

\pagestyle{fancyplain}
\lhead{\emph{\nouppercase{\leftmark}}}
\ifx \nextra \undefined
  \rhead{
    \ifnum\thepage=1
    \else
      \npart\ \ncourse
    \fi}
\else
  \rhead{
    \ifnum\thepage=1
    \else
      \npart\ \ncourse\ (\nextra)
    \fi}
\fi
\usetikzlibrary{arrows.meta}
\usetikzlibrary{decorations.markings}
\usetikzlibrary{decorations.pathmorphing}
\usetikzlibrary{positioning}
\usetikzlibrary{fadings}
\usetikzlibrary{intersections}
\usetikzlibrary{cd}

\newcommand*{\Cdot}{{\raisebox{-0.25ex}{\scalebox{1.5}{$\cdot$}}}}
\newcommand {\pd}[2][ ]{
  \ifx #1 { }
    \frac{\partial}{\partial #2}
  \else
    \frac{\partial^{#1}}{\partial #2^{#1}}
  \fi
}
\ifx \nhtml \undefined
\else
  \renewcommand\printindex{}
  \makeatletter
  \DisableLigatures[f]{family = *}
  \let\Contentsline\contentsline
  \renewcommand\contentsline[3]{\Contentsline{#1}{#2}{}}
  \renewcommand{\@dotsep}{10000}
  \newlength\currentparindent
  \setlength\currentparindent\parindent

  \newcommand\@minipagerestore{\setlength{\parindent}{\currentparindent}}
  \usepackage[active,tightpage,pdftex]{preview}
  \renewcommand{\PreviewBorder}{0.1cm}

  \newenvironment{stretchpage}%
  {\begin{preview}\begin{minipage}{\hsize}}%
    {\end{minipage}\end{preview}}
  \AtBeginDocument{\begin{stretchpage}}
  \AtEndDocument{\end{stretchpage}}

  \newcommand{\@@newpage}{\end{stretchpage}\begin{stretchpage}}

  \let\@real@section\section
  \renewcommand{\section}{\@@newpage\@real@section}
  \let\@real@subsection\subsection
  \renewcommand{\subsection}{\@@newpage\@real@subsection}
  \makeatother
\fi

% Theorems
\theoremstyle{definition}
\newtheorem*{aim}{Aim}
\newtheorem*{axiom}{Axiom}
\newtheorem*{claim}{Claim}
\newtheorem*{cor}{Corollary}
\newtheorem*{conjecture}{Conjecture}
\newtheorem*{defi}{Definition}
\newtheorem*{eg}{Example}
\newtheorem*{ex}{Exercise}
\newtheorem*{fact}{Fact}
\newtheorem*{law}{Law}
\newtheorem*{lemma}{Lemma}
\newtheorem*{notation}{Notation}
\newtheorem*{prop}{Proposition}
\newtheorem*{soln}{Solution}
\newtheorem*{thm}{Theorem}

\newtheorem*{remark}{Remark}
\newtheorem*{warning}{Warning}
\newtheorem*{exercise}{Exercise}

\newtheorem{nthm}{Theorem}[section]
\newtheorem{nlemma}[nthm]{Lemma}
\newtheorem{nprop}[nthm]{Proposition}
\newtheorem{ncor}[nthm]{Corollary}


\renewcommand{\labelitemi}{--}
\renewcommand{\labelitemii}{$\circ$}
\renewcommand{\labelenumi}{(\roman{*})}

\let\stdsection\section
\renewcommand\section{\newpage\stdsection}

% Strike through
\def\st{\bgroup \ULdepth=-.55ex \ULset}

% Maths symbols
\newcommand{\abs}[1]{\left\lvert #1\right\rvert}
\newcommand\ad{\mathrm{ad}}
\newcommand\AND{\mathsf{AND}}
\newcommand\Art{\mathrm{Art}}
\newcommand{\Bilin}{\mathrm{Bilin}}
\newcommand{\bket}[1]{\left\lvert #1\right\rangle}
\newcommand{\B}{\mathcal{B}}
\newcommand{\bolds}[1]{{\bfseries #1}}
\newcommand{\brak}[1]{\left\langle #1 \right\rvert}
\newcommand{\braket}[2]{\left\langle #1\middle\vert #2 \right\rangle}
\newcommand{\bra}{\langle}
\newcommand{\cat}[1]{\mathsf{#1}}
\newcommand{\C}{\mathbb{C}}
\newcommand{\CP}{\mathbb{CP}}
\newcommand{\cU}{\mathcal{U}}
\newcommand{\Der}{\mathrm{Der}}
\newcommand{\D}{\mathrm{D}}
\newcommand{\dR}{\mathrm{dR}}
\newcommand{\E}{\mathbb{E}}
\newcommand{\F}{\mathbb{F}}
\newcommand{\Frob}{\mathrm{Frob}}
\newcommand{\GG}{\mathbb{G}}
\newcommand{\gl}{\mathfrak{gl}}
\newcommand{\GL}{\mathrm{GL}}
\newcommand{\G}{\mathcal{G}}
\newcommand{\Gr}{\mathrm{Gr}}
\newcommand{\haut}{\mathrm{ht}}
\newcommand{\Id}{\mathrm{Id}}
\newcommand{\ket}{\rangle}
\newcommand{\lie}[1]{\mathfrak{#1}}
\newcommand{\Mat}{\mathrm{Mat}}
\newcommand{\N}{\mathbb{N}}
\newcommand{\norm}[1]{\left\lVert #1\right\rVert}
\newcommand{\normalorder}[1]{\mathop{:}\nolimits\!#1\!\mathop{:}\nolimits}
\newcommand\NOT{\mathsf{NOT}}
\newcommand{\Oc}{\mathcal{O}}
\newcommand{\Or}{\mathrm{O}}
\newcommand\OR{\mathsf{OR}}
\newcommand{\ort}{\mathfrak{o}}
\newcommand{\PGL}{\mathrm{PGL}}
\newcommand{\ph}{\,\cdot\,}
\newcommand{\pr}{\mathrm{pr}}
\newcommand{\Prob}{\mathbb{P}}
\newcommand{\PSL}{\mathrm{PSL}}
\newcommand{\Ps}{\mathcal{P}}
\newcommand{\PSU}{\mathrm{PSU}}
\newcommand{\pt}{\mathrm{pt}}
\newcommand{\qeq}{\mathrel{``{=}"}}
\newcommand{\Q}{\mathbb{Q}}
\newcommand{\R}{\mathbb{R}}
\newcommand{\RP}{\mathbb{RP}}
\newcommand{\Rs}{\mathcal{R}}
\newcommand{\SL}{\mathrm{SL}}
\newcommand{\so}{\mathfrak{so}}
\newcommand{\SO}{\mathrm{SO}}
\newcommand{\Spin}{\mathrm{Spin}}
\newcommand{\Sp}{\mathrm{Sp}}
\newcommand{\su}{\mathfrak{su}}
\newcommand{\SU}{\mathrm{SU}}
\newcommand{\term}[1]{\emph{#1}\index{#1}}
\newcommand{\T}{\mathbb{T}}
\newcommand{\tv}[1]{|#1|}
\newcommand{\U}{\mathrm{U}}
\newcommand{\uu}{\mathfrak{u}}
\newcommand{\Vect}{\mathrm{Vect}}
\newcommand{\wsto}{\stackrel{\mathrm{w}^*}{\to}}
\newcommand{\wt}{\mathrm{wt}}
\newcommand{\wto}{\stackrel{\mathrm{w}}{\to}}
\newcommand{\Z}{\mathbb{Z}}
\renewcommand{\d}{\mathrm{d}}
\renewcommand{\H}{\mathbb{H}}
\renewcommand{\P}{\mathbb{P}}
\renewcommand{\sl}{\mathfrak{sl}}
\renewcommand{\vec}[1]{\boldsymbol{\mathbf{#1}}}
%\renewcommand{\F}{\mathcal{F}}

\let\Im\relax
\let\Re\relax

\DeclareMathOperator{\adj}{adj}
\DeclareMathOperator{\Ann}{Ann}
\DeclareMathOperator{\area}{area}
\DeclareMathOperator{\Aut}{Aut}
\DeclareMathOperator{\Bernoulli}{Bernoulli}
\DeclareMathOperator{\betaD}{beta}
\DeclareMathOperator{\bias}{bias}
\DeclareMathOperator{\binomial}{binomial}
\DeclareMathOperator{\card}{card}
\DeclareMathOperator{\ccl}{ccl}
\DeclareMathOperator{\Char}{char}
\DeclareMathOperator{\ch}{ch}
\DeclareMathOperator{\cl}{cl}
\DeclareMathOperator{\cls}{\overline{\mathrm{span}}}
\DeclareMathOperator{\conv}{conv}
\DeclareMathOperator{\corr}{corr}
\DeclareMathOperator{\cosec}{cosec}
\DeclareMathOperator{\cosech}{cosech}
\DeclareMathOperator{\cov}{cov}
\DeclareMathOperator{\covol}{covol}
\DeclareMathOperator{\diag}{diag}
\DeclareMathOperator{\diam}{diam}
\DeclareMathOperator{\Diff}{Diff}
\DeclareMathOperator{\disc}{disc}
\DeclareMathOperator{\dom}{dom}
\DeclareMathOperator{\End}{End}
\DeclareMathOperator{\energy}{energy}
\DeclareMathOperator{\erfc}{erfc}
\DeclareMathOperator{\erf}{erf}
\DeclareMathOperator*{\esssup}{ess\,sup}
\DeclareMathOperator{\ev}{ev}
\DeclareMathOperator{\Ext}{Ext}
\DeclareMathOperator{\Fit}{Fit}
\DeclareMathOperator{\fix}{fix}
\DeclareMathOperator{\Frac}{Frac}
\DeclareMathOperator{\Gal}{Gal}
\DeclareMathOperator{\gammaD}{gamma}
\DeclareMathOperator{\gr}{gr}
\DeclareMathOperator{\hcf}{hcf}
\DeclareMathOperator{\Hom}{Hom}
\DeclareMathOperator{\id}{id}
\DeclareMathOperator{\image}{image}
\DeclareMathOperator{\im}{im}
\DeclareMathOperator{\Im}{Im}
\DeclareMathOperator{\Ind}{Ind}
\DeclareMathOperator{\Int}{Int}
\DeclareMathOperator{\Isom}{Isom}
\DeclareMathOperator{\lcm}{lcm}
\DeclareMathOperator{\length}{length}
\DeclareMathOperator{\Lie}{Lie}
\DeclareMathOperator{\like}{like}
\DeclareMathOperator{\Lk}{Lk}
\DeclareMathOperator{\mse}{mse}
\DeclareMathOperator{\multinomial}{multinomial}
\DeclareMathOperator{\orb}{orb}
\DeclareMathOperator{\ord}{ord}
\DeclareMathOperator{\otp}{otp}
\DeclareMathOperator{\Poisson}{Poisson}
\DeclareMathOperator{\poly}{poly}
\DeclareMathOperator{\rank}{rank}
\DeclareMathOperator{\rel}{rel}
\DeclareMathOperator{\Re}{Re}
\DeclareMathOperator*{\res}{res}
\DeclareMathOperator{\Res}{Res}
\DeclareMathOperator{\rk}{rk}
\DeclareMathOperator{\Root}{Root}
\DeclareMathOperator{\sech}{sech}
\DeclareMathOperator{\sgn}{sgn}
\DeclareMathOperator{\spn}{span}
\DeclareMathOperator{\stab}{stab}
\DeclareMathOperator{\St}{St}
\DeclareMathOperator{\supp}{supp}
\DeclareMathOperator{\Syl}{Syl}
\DeclareMathOperator{\Sym}{Sym}
\DeclareMathOperator{\tr}{tr}
\DeclareMathOperator{\Tr}{Tr}
\DeclareMathOperator{\var}{var}
\DeclareMathOperator{\vol}{vol}

\pgfarrowsdeclarecombine{twolatex'}{twolatex'}{latex'}{latex'}{latex'}{latex'}
\tikzset{->/.style = {decoration={markings,
                                  mark=at position 1 with {\arrow[scale=2]{latex'}}},
                      postaction={decorate}}}
\tikzset{<-/.style = {decoration={markings,
                                  mark=at position 0 with {\arrowreversed[scale=2]{latex'}}},
                      postaction={decorate}}}
\tikzset{<->/.style = {decoration={markings,
                                   mark=at position 0 with {\arrowreversed[scale=2]{latex'}},
                                   mark=at position 1 with {\arrow[scale=2]{latex'}}},
                       postaction={decorate}}}
\tikzset{->-/.style = {decoration={markings,
                                   mark=at position #1 with {\arrow[scale=2]{latex'}}},
                       postaction={decorate}}}
\tikzset{-<-/.style = {decoration={markings,
                                   mark=at position #1 with {\arrowreversed[scale=2]{latex'}}},
                       postaction={decorate}}}
\tikzset{->>/.style = {decoration={markings,
                                  mark=at position 1 with {\arrow[scale=2]{latex'}}},
                      postaction={decorate}}}
\tikzset{<<-/.style = {decoration={markings,
                                  mark=at position 0 with {\arrowreversed[scale=2]{twolatex'}}},
                      postaction={decorate}}}
\tikzset{<<->>/.style = {decoration={markings,
                                   mark=at position 0 with {\arrowreversed[scale=2]{twolatex'}},
                                   mark=at position 1 with {\arrow[scale=2]{twolatex'}}},
                       postaction={decorate}}}
\tikzset{->>-/.style = {decoration={markings,
                                   mark=at position #1 with {\arrow[scale=2]{twolatex'}}},
                       postaction={decorate}}}
\tikzset{-<<-/.style = {decoration={markings,
                                   mark=at position #1 with {\arrowreversed[scale=2]{twolatex'}}},
                       postaction={decorate}}}

\tikzset{circ/.style = {fill, circle, inner sep = 0, minimum size = 3}}
\tikzset{mstate/.style={circle, draw, blue, text=black, minimum width=0.7cm}}

\tikzset{commutative diagrams/.cd,cdmap/.style={/tikz/column 1/.append style={anchor=base east},/tikz/column 2/.append style={anchor=base west},row sep=tiny}}

\definecolor{mblue}{rgb}{0.2, 0.3, 0.8}
\definecolor{morange}{rgb}{1, 0.5, 0}
\definecolor{mgreen}{rgb}{0.1, 0.4, 0.2}
\definecolor{mred}{rgb}{0.5, 0, 0}

\def\drawcirculararc(#1,#2)(#3,#4)(#5,#6){%
    \pgfmathsetmacro\cA{(#1*#1+#2*#2-#3*#3-#4*#4)/2}%
    \pgfmathsetmacro\cB{(#1*#1+#2*#2-#5*#5-#6*#6)/2}%
    \pgfmathsetmacro\cy{(\cB*(#1-#3)-\cA*(#1-#5))/%
                        ((#2-#6)*(#1-#3)-(#2-#4)*(#1-#5))}%
    \pgfmathsetmacro\cx{(\cA-\cy*(#2-#4))/(#1-#3)}%
    \pgfmathsetmacro\cr{sqrt((#1-\cx)*(#1-\cx)+(#2-\cy)*(#2-\cy))}%
    \pgfmathsetmacro\cA{atan2(#2-\cy,#1-\cx)}%
    \pgfmathsetmacro\cB{atan2(#6-\cy,#5-\cx)}%
    \pgfmathparse{\cB<\cA}%
    \ifnum\pgfmathresult=1
        \pgfmathsetmacro\cB{\cB+360}%
    \fi
    \draw (#1,#2) arc (\cA:\cB:\cr);%
}
\newcommand\getCoord[3]{\newdimen{#1}\newdimen{#2}\pgfextractx{#1}{\pgfpointanchor{#3}{center}}\pgfextracty{#2}{\pgfpointanchor{#3}{center}}}

\def\Xint#1{\mathchoice
   {\XXint\displaystyle\textstyle{#1}}%
   {\XXint\textstyle\scriptstyle{#1}}%
   {\XXint\scriptstyle\scriptscriptstyle{#1}}%
   {\XXint\scriptscriptstyle\scriptscriptstyle{#1}}%
   \!\int}
\def\XXint#1#2#3{{\setbox0=\hbox{$#1{#2#3}{\int}$}
     \vcenter{\hbox{$#2#3$}}\kern-.5\wd0}}
\def\ddashint{\Xint=}
\def\dashint{\Xint-}

\newcommand\separator{{\centering\rule{2cm}{0.2pt}\vspace{2pt}\par}}

\newenvironment{own}{\color{gray!70!black}}{}

\newcommand\makecenter[1]{\raisebox{-0.5\height}{#1}}

\begin{document}
\maketitle

\setcounter{section}{-1}
\section{Introduction}



\section{Vector Spaces}

\begin{defi}
	An $ \F $-Vector space (a vector space on $ \F $) is an abelian group $ (V, +) $ equipped with a function\footnote{scalar multiplication} $ F \times V \to V $, $ (\lambda,v) \mapsto \lambda V $
	
	\[ \lambda ( v_{1} + v_{2}) = \lambda v_{1} + \lambda v_{2} \]
	
	\[ (\lambda_{1} + \lambda_{2}) v = \lambda_{1} v + \lambda_{2} v \]
	
	\[ \lambda (\mu v) = \lambda \mu v \]
	
	\[ 1v = v \]
	
	\[ v + \mathbf{0} = v \]
	
	for all $ \lambda_{i}, \mu \in F $, $ v_{i} \in V $
	
\end{defi}

Note that we will not be underlining our vectors, as this is cumbersome here. We will however be using $ \mathbf{0} $ to denote the zero vector. 

\begin{eg}
	For all $ n \in \N  $, $ \F^{n} = $ space of column vectors of length $ n $, entries in $ \F $. We understand the definition as entry-wise addition, entry-wise scalar multiplication
	  
\end{eg}


\begin{eg}
	$ M_{m,m}(\F) $, the set of $ m \times m $ matrices with entries in $ \F $
	
	\[ \begin{pmatrix}
	a & b \\
	c & d
	\end{pmatrix} + \begin{pmatrix}
	e & f \\
	g & h
	\end{pmatrix} = \begin{pmatrix}
	a + e & b + f\\
	c + g & d + h
	\end{pmatrix} \]
	again all operations defined entry-wise
\end{eg}

\begin{eg}
	For any set $ X $, $ \R^{X} = \{ f : X \to \R \}$
	Addition and scalar multiplication defined pointwise $ = f_{1}(x) + f_{2} (x) $.
\end{eg}


\begin{ex}
	Show that the above examples satisfy the axioms
\end{ex}

\begin{prop} 
	$ 0 v = \mathbf{0} $ for all $ v \in V $.
\end{prop}

\begin{proof}
	( $ (0 + 0)v = 0v \iff 0 v + 0v = 0v \iff 0v = \mathbf{0} $)
\end{proof}

\begin{ex}
	Show\footnote{Hint: Use the previous proposition} that $ (-1)v = -v $
\end{ex}

\begin{defi}
	Let V be an $ \F $-vector space. A subset $ U $ of $ V $ is a subspace ( $ U \leq V $) if: 
	\begin{enumerate}
		\item $ \mathbf{0} \in U $
		\item $ u_{1}, u_{2} \in U  \Rightarrow u_{1} + u_{2} \in U $ ``$ U $ is closed under addition...''
		\item $ u \in U $, any $ \lambda \in \F \Rightarrow \lambda u \in U$ ``...and scalar multiplication''
	\end{enumerate}
\end{defi}


\begin{ex}
	If $ U $ is a subspace of $ V $, then $ U $ is also an $ \F $-vector space.
\end{ex}

\begin{eg}
	Let $ V = \R^{\R} $, then $ f : R \to R $. The set of all continuous functions $ C(\R) $ are a subspace. An even smaller subspace is the set of all polynomials.
\end{eg}


\begin{ex} Define $ U \subseteq R^{3} $ as: 
	\[ \begin{pmatrix}
a_{1} \\
a_{2} \\
a_{3}
\end{pmatrix} \qquad a_{1} + a_{2} + a_{3} = t \]
for some constant  $ t $. 
Check that this is a subspace of $ \R^{3} $ if and only if $ t = 0 $.
\end{ex}

\begin{prop} 
	Let $ V $ be an $ F $-vector space, $ U,W \leq V $. Then $ U \cap W \leq V $.	
\end{prop}

\begin{proof}
	\begin{enumerate}
		\item 	$ 0 \in U $, $ 0 \in W  \Rightarrow 0 \in U \cap W $ 
		\item Suppose $ u,v \in U \cap W $, $ \lambda, \mu \in F $.
		$ U $ is a subspace $ \Rightarrow  \lambda u + \mu v \in U $. Similarly $ \lambda u + \mu v \in U \in W $, so it is in the intersection. 
	\end{enumerate}
\end{proof}

\begin{eg}
	$ V = \R^{3} $, $ U = \left\{ \begin{pmatrix}
	x\\
	y\\
	z
	\end{pmatrix} \; | \; x = 0 \right\} $, $ V = \left\{ \begin{pmatrix}
	x\\
	y\\
	z
	\end{pmatrix} \; | \; y = 0 \right\}  $ then $ U \cap W = U = \left\{ \begin{pmatrix}
	x\\
	y\\
	z
	\end{pmatrix} \; | \; x = 0, y= 0 \right\}  $ (intersect along the $ z $-axis)
\end{eg}

Note: union of family of subspaces is almost never a subspace itself.


\begin{defi}
	Let $ V $ be an $ F $-vector space, $ U,W \leq V $. The \emph{sum} of $ U $ and $ W $ is the set:
	
	\[ U + W = \left\{  u + w \; | u \in U, w \in W    \right\}  \]
\end{defi}

\begin{prop} 
	$ U + W \leq V $
\end{prop}

\begin{proof}
	$ \mathbf{0} \in U,W \Rightarrow \mathbf{0} + \mathbf{0} = \mathbf{0} \in U + W $
	
	$ u_{1},u_{2} \in U $, $ w_{1},w_{2} \in W $, 
	
	\[ (u_{1} + w_{1}) + (u_{2} + w_{2}) = \underbrace{(u_{1} + u_{2})}_{\in U}  + \underbrace{(w_{1} + w_{2})}_{\in W} \]
	
	Similarly for scalar multiplication (ex.)
\end{proof}

Note: $ U + W $ is the smallest subspace containing both $ U $ and $ W $. (This is becaues all elements of the form $ u + w $ ae forced to be in such a subspace by the ``closed under addition'' axiom)


\begin{defi}
	$ V $ is an $ \F $-vector space, $ U \leq V $. The quotient space\footnote{think of this as the collection of cosets of $ U $ in $ V $ } $ V / U $ is the abelian group $ V / U $ equipped with scalar multiplication;
	
	\[ F \times V / U \to V / U \]
	
	\[ (\lambda, v + U) \mapsto \lambda v + U \]
\end{defi}

\begin{prop} 
	This is well-defined, and $ V/U $ is an $ F $-vector space.
\end{prop}

\begin{proof}
	Well-defined: Suppose $ v_{1} + U = v_{2} + U \in V / U $. $ \Rightarrow (v_{2} - v_{1}) \in U \Rightarrow ( \lambda v_{2} - \lambda v_{1}) \in U \Rightarrow \lambda v_{2} + U = \lambda v_{1} + U \in V / U $

To show that it is an $ \F $-vector space, we must show that the axioms hold. These follow from the axioms of $ V $.
$ \lambda ( \mu (v + U)) = \lambda ( \mu v + U) = \lambda(u v) + U = (\lambda u) v + U  = \lambda u (v \in U) $ (scalar multiplication on $ V / U $). 

Ex. Other axioms follow similarly from using vecton space axioms

\end{proof}

\begin{defi}
	$ V $ is an $ \F $-vector space, $ S \subset V $. The \emph{span} of $ S $ is denoted by 
	
	\[ <S> = \left\{ \sum_{s \in S}  \lambda_{s} s \; | \; \lambda_{s} \in \F \right\}  \]
	
	ie. the set of all finite linear combinations, all but finitely many of the $ \lambda_{s} $ are zero.
\end{defi}

Remark: $ <S> $ is the smallest subspace of $ V $ which contains\footnote{This is essentially a tautology} all of the elements of $ S $

Convention: $  < \emptyset > = \{ \mathbf{0} \} $.

\begin{eg}
	$ V = \R^{3} $, 
	
	\[ S = \left\{ \begin{pmatrix}
	1 \\
	0 \\
	0
	
	\end{pmatrix}, \begin{pmatrix}
	0\\
	1\\
	2
	
	\end{pmatrix}, \begin{pmatrix}
	3\\
	-2\\
	-4
	\end{pmatrix} \right\}  \]
	5
	\[ <S> = \{ \begin{pmatrix}
	a\\
	b\\
	2b
	\end{pmatrix} \} \; | \; a,b \in \R \]
	
	ie. we have took linear combinations of the first two. We don't need the third one.
	
\end{eg}



\begin{eg}
	For $ X $ a set, 
	
	
	\[ \delta_{x}(y) = \begin{cases} 1  & \text{ if } x = y \\ 0  & \text{ if } x \neq y \end{cases}  \]
	
	\[ < \delta_{x} \; | \; x \in X > = \{  f \in \R^{X} \; | \; f \text{ has finite support} \}  \]
	
	\[ = <  x \in X \; | \; f(x) \neq 0 > \]
\end{eg}

\begin{defi}
	$ S $ \emph{spans} $ V $ if $ <S> = V $
\end{defi}

\begin{defi}
	$ V $ is \emph{finite dimensiona}l over $ \F $ if it is spanned by a set that is finite.
\end{defi}

\begin{defi}
	The vectors $ v_{1},\cdots,v_{n} $ are \emph{linearly indepedent} over $ \F $ if 
	
	\[ \sum_{i = 1}^{n} \lambda_{i} v_{i} = 0 \Rightarrow \lambda_{i} \text{ for all } i \]
	
	some coefficients $ \lambda_{i} \in \F  $. $ S \subset V $ is linearly independent if every finite subset of it is. 
\end{defi}

\begin{eg}
	The fist example, $ u,v,w $ are not linearly independent.
	
\end{eg}

\begin{eg}
	The set $ \{  \delta_{X} \; | \; x \in X \} $ is linearly indepndent.
\end{eg}


\begin{defi}
	If \emph{not} linearly indepnedent, say a set is linearly dependent.
\end{defi}

\begin{defi}
	$ S $ is a \emph{basis} of $ V $ if it is linearly indepnedent and spans $ V $
\end{defi}


\begin{eg}
	$ \F^{n}  $  standard basis: $ e_{1},e_{2},\cdots,e_{n} $.
\end{eg}

\begin{eg}
	$ V = \C $ over $ \C $ has natural basis $ \{ 1\} $, over $ \R $ has natural basis $ \{ 1,i \} $
\end{eg}


\begin{eg}
	$ V = \mathcal{P}(\R) $ space of all polynomials, has natural basis 
	
	\[ \{ 1,x,x^{2},x^{3},\cdots \} \]
\end{eg}

\begin{ex}
	Check this carefully 
\end{ex}


\begin{lemma} 
	$ V $ is an $ \F $-vector space. The vectors $ v_{1}, \cdots, v_{n} $ form a basis of $ V $ iff each vector $ v \in V $ has a unique expression
	
	\[ v = \sum_{i=1}^{n} \lambda_{i}v_{i}, \text{ with } \lambda_{i} \in \F \] 
	
\end{lemma}

\begin{proof}
	$ (\Rightarrow) $ Fix $ v \in V$. The $ v_{i} $ span, so 
	
	\[ \exists \lambda_{i} \in \F \text{ s.t. } v = \sum \lambda_{i} v_{i} \]
	
	Suppose also $ v = \sum \mu_{i} v_{i} $ for some $ \mu_{i} \in \F $. $ \sum \left( \mu_{i} - \lambda_{i} \right)  v_{i} = \mathbf{0} $.
	
	The $ v_{i} $ are linearly indepnedent so $ \mu_{i} - \lambda_{i} = 0 $ for all $ i $, $ \lambda_{i} = \mu_{i} $
	
	
	$ (\Leftarrow) $ The $ v_{i} $ span $ V $, since any $ v \in V $ is a linear combination of them.
	IF $ \sum_{i=1}^{n} \lambda_{i} v_{i} = \mathbf{0} $. Note that $ \mathbf{0} = \sum_{i=1}^{n} 0 v_{i} $. By uniqueness (applied to $ \mathbf{0} $), $ \lambda_{i}  = 0 $ for all $ i $. 
\end{proof} 


\begin{lemma} 
	If $ v_{1},\cdots,v_{n} $ span $ V $ (over $ \F $), then some subset of $ v_{1},\cdots,v_{n} $ is a basis for $ V $ (over $ \F $).	
\end{lemma}

\begin{proof}
		If $ v_{1},\cdots,v_{n} $ ilnearly indepnedent, done.
		Otherwise for some $ l $, there exist $ \alpha_{1},\cdots,\alpha_{l-1} \in \F $ such that
		
		\[ v_{l} = \alpha_{1} v_{1} + \cdots + \alpha_{l-1}v_{l-1} \]
		
		
		( If $ \sum \lambda_{i} v_{i} = \mathbf{0}$, not all $ \lambda_{i} = 0 $. Take $ l $ maximaml with $ \lambda_{i} \neq  0$, just $ \alpha_{i} = - \lambda_{i} / \lambda_{l} $ ).
		
		Now $ v_{2}, \cdots, v_{l-1},v_{l+1},\cdots,v_{n} $ still span $ V $. Continue interatively until get linear independence. 
		
\end{proof}

\begin{thm} (Steinitz exchange lemma)
	Let $ V $ be a finite dimensional vector space over $ \F $. Take $ v_{1},\cdots,v_{m} $ to be linearly independent $ w_{1},\cdots,w_{n} $ to span $ V $. 
	
	Then $ m \leq n $, and reordering the spanning set if needed,
	
	\[ v_{1},\cdots, v_{m}, w_{m+1},\cdots,w_{n} \] span $ V $.
	
\end{thm}

\begin{proof} (Induction)
	Suppose that we've replaced $ l (\geq 0)$ of the $ w_{i} $. Reordering the $ w_{i} $ if needed, $ v_{1},\cdots,v_{l},w_{l+1},\cdots,w_{n} $ span $ V $. 
	
	If $ l = m $, done.
	
	If $ l < m $, then
	
	\[ v_{l+1} = \sum_{i=1}^{l} \alpha_{i} v_{i}  + \sum_{i > l} \beta_{i} w_{i} \]
	
	$ \alpha_{i}, \beta_{i} \in \F $. As the $ v_{i} $ are lin. indep, $ \beta_{i} \neq 0 $ for some $ i $. (After reordering, $ \beta_{l+1} \neq 0 $).
	
	\[ w_{l+1} = \frac{1}{\beta_{l+1}} \left( v_{l+1} - \sum_{i \leq l} \alpha_{i} v_{i}  - \sum_{i > l+1} \beta_{i} w_{i} \right)  \]
	
	This $ v_{1},\cdots,v_{l+1},w_{l+2},\cdots,w_{n} $ also spans $ V $. After $ m $ steps, $ w_{i} $ will have replaced $ m $ of the $ w_{i} $ by $ v_{i} $. Thus $ m \leq n $.
\end{proof}


\begin{thm} 
	If $ V $ is a finite dimensional vector space over $ \F $, then any two bases for $ V $ have the same number of elements. This is what we call the \emph{dimension} of $ V $, denoted $ \dim_{\F} V $.
\end{thm}

\begin{proof}
	If $ v_{\{1},\cdots,v_{n} $ is a basis and $ w_{1},\cdots,w_{m} $ is another basis, the $ \{ v_{i} \} $ span and $ \{ w_{i} \} $ is linearly indepnedent' so by Steinitz $ m \leq n $. Likewise, $ n \leq m $.
\end{proof}


\begin{eg}
	$ \dim_{\C} \C = 1 $, $ \dim_{\R} \C = 2 $
\end{eg}





\subsection{}




\end{document}