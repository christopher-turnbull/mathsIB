\documentclass[a4paper]{article}

\def\npart {IB}
\def\nterm {Michaelmas}
\def\nyear {2017}
\def\nlecturer {J.M.Evans (j.m.evans@damtp.cam.ac.uk)}
\def\ncourse {Quantum Mechanics}

% Imports
\ifx \nauthor\undefined
  \def\nauthor{Christopher Turnbull}
\else
\fi

\author{Supervised by \nlecturer \\\small Solutions presented by \nauthor}
\date{\nterm\ \nyear}

\usepackage{alltt}
\usepackage{amsfonts}
\usepackage{amsmath}
\usepackage{amssymb}
\usepackage{amsthm}
\usepackage{booktabs}
\usepackage{caption}
\usepackage{enumitem}
\usepackage{fancyhdr}
\usepackage{graphicx}
\usepackage{mathdots}
\usepackage{mathtools}
\usepackage{microtype}
\usepackage{multirow}
\usepackage{pdflscape}
\usepackage{pgfplots}
\usepackage{siunitx}
\usepackage{slashed}
\usepackage{tabularx}
\usepackage{tikz}
\usepackage{tkz-euclide}
\usepackage[normalem]{ulem}
\usepackage[all]{xy}
\usepackage{imakeidx}

\makeindex[intoc, title=Index]
\indexsetup{othercode={\lhead{\emph{Index}}}}

\ifx \nextra \undefined
  \usepackage[pdftex,
    hidelinks,
    pdfauthor={Christopher Turnbull},
    pdfsubject={Cambridge Maths Notes: Part \npart\ - \ncourse},
    pdftitle={Part \npart\ - \ncourse},
  pdfkeywords={Cambridge Mathematics Maths Math \npart\ \nterm\ \nyear\ \ncourse}]{hyperref}
  \title{Part \npart\ --- \ncourse}
\else
  \usepackage[pdftex,
    hidelinks,
    pdfauthor={Christopher Turnbull},
    pdfsubject={Cambridge Maths Notes: Part \npart\ - \ncourse\ (\nextra)},
    pdftitle={Part \npart\ - \ncourse\ (\nextra)},
  pdfkeywords={Cambridge Mathematics Maths Math \npart\ \nterm\ \nyear\ \ncourse\ \nextra}]{hyperref}

  \title{Part \npart\ --- \ncourse \\ {\Large \nextra}}
  \renewcommand\printindex{}
\fi

\pgfplotsset{compat=1.12}

\pagestyle{fancyplain}
\lhead{\emph{\nouppercase{\leftmark}}}
\ifx \nextra \undefined
  \rhead{
    \ifnum\thepage=1
    \else
      \npart\ \ncourse
    \fi}
\else
  \rhead{
    \ifnum\thepage=1
    \else
      \npart\ \ncourse\ (\nextra)
    \fi}
\fi
\usetikzlibrary{arrows.meta}
\usetikzlibrary{decorations.markings}
\usetikzlibrary{decorations.pathmorphing}
\usetikzlibrary{positioning}
\usetikzlibrary{fadings}
\usetikzlibrary{intersections}
\usetikzlibrary{cd}

\newcommand*{\Cdot}{{\raisebox{-0.25ex}{\scalebox{1.5}{$\cdot$}}}}
\newcommand {\pd}[2][ ]{
  \ifx #1 { }
    \frac{\partial}{\partial #2}
  \else
    \frac{\partial^{#1}}{\partial #2^{#1}}
  \fi
}
\ifx \nhtml \undefined
\else
  \renewcommand\printindex{}
  \makeatletter
  \DisableLigatures[f]{family = *}
  \let\Contentsline\contentsline
  \renewcommand\contentsline[3]{\Contentsline{#1}{#2}{}}
  \renewcommand{\@dotsep}{10000}
  \newlength\currentparindent
  \setlength\currentparindent\parindent

  \newcommand\@minipagerestore{\setlength{\parindent}{\currentparindent}}
  \usepackage[active,tightpage,pdftex]{preview}
  \renewcommand{\PreviewBorder}{0.1cm}

  \newenvironment{stretchpage}%
  {\begin{preview}\begin{minipage}{\hsize}}%
    {\end{minipage}\end{preview}}
  \AtBeginDocument{\begin{stretchpage}}
  \AtEndDocument{\end{stretchpage}}

  \newcommand{\@@newpage}{\end{stretchpage}\begin{stretchpage}}

  \let\@real@section\section
  \renewcommand{\section}{\@@newpage\@real@section}
  \let\@real@subsection\subsection
  \renewcommand{\subsection}{\@@newpage\@real@subsection}
  \makeatother
\fi

% Theorems
\theoremstyle{definition}
\newtheorem*{aim}{Aim}
\newtheorem*{axiom}{Axiom}
\newtheorem*{claim}{Claim}
\newtheorem*{cor}{Corollary}
\newtheorem*{conjecture}{Conjecture}
\newtheorem*{defi}{Definition}
\newtheorem*{eg}{Example}
\newtheorem*{ex}{Exercise}
\newtheorem*{fact}{Fact}
\newtheorem*{law}{Law}
\newtheorem*{lemma}{Lemma}
\newtheorem*{notation}{Notation}
\newtheorem*{prop}{Proposition}
\newtheorem*{soln}{Solution}
\newtheorem*{thm}{Theorem}

\newtheorem*{remark}{Remark}
\newtheorem*{warning}{Warning}
\newtheorem*{exercise}{Exercise}

\newtheorem{nthm}{Theorem}[section]
\newtheorem{nlemma}[nthm]{Lemma}
\newtheorem{nprop}[nthm]{Proposition}
\newtheorem{ncor}[nthm]{Corollary}


\renewcommand{\labelitemi}{--}
\renewcommand{\labelitemii}{$\circ$}
\renewcommand{\labelenumi}{(\roman{*})}

\let\stdsection\section
\renewcommand\section{\newpage\stdsection}

% Strike through
\def\st{\bgroup \ULdepth=-.55ex \ULset}

% Maths symbols
\newcommand{\abs}[1]{\left\lvert #1\right\rvert}
\newcommand\ad{\mathrm{ad}}
\newcommand\AND{\mathsf{AND}}
\newcommand\Art{\mathrm{Art}}
\newcommand{\Bilin}{\mathrm{Bilin}}
\newcommand{\bket}[1]{\left\lvert #1\right\rangle}
\newcommand{\B}{\mathcal{B}}
\newcommand{\bolds}[1]{{\bfseries #1}}
\newcommand{\brak}[1]{\left\langle #1 \right\rvert}
\newcommand{\braket}[2]{\left\langle #1\middle\vert #2 \right\rangle}
\newcommand{\bra}{\langle}
\newcommand{\cat}[1]{\mathsf{#1}}
\newcommand{\C}{\mathbb{C}}
\newcommand{\CP}{\mathbb{CP}}
\newcommand{\cU}{\mathcal{U}}
\newcommand{\Der}{\mathrm{Der}}
\newcommand{\D}{\mathrm{D}}
\newcommand{\dR}{\mathrm{dR}}
\newcommand{\E}{\mathbb{E}}
\newcommand{\F}{\mathbb{F}}
\newcommand{\Frob}{\mathrm{Frob}}
\newcommand{\GG}{\mathbb{G}}
\newcommand{\gl}{\mathfrak{gl}}
\newcommand{\GL}{\mathrm{GL}}
\newcommand{\G}{\mathcal{G}}
\newcommand{\Gr}{\mathrm{Gr}}
\newcommand{\haut}{\mathrm{ht}}
\newcommand{\Id}{\mathrm{Id}}
\newcommand{\ket}{\rangle}
\newcommand{\lie}[1]{\mathfrak{#1}}
\newcommand{\Mat}{\mathrm{Mat}}
\newcommand{\N}{\mathbb{N}}
\newcommand{\norm}[1]{\left\lVert #1\right\rVert}
\newcommand{\normalorder}[1]{\mathop{:}\nolimits\!#1\!\mathop{:}\nolimits}
\newcommand\NOT{\mathsf{NOT}}
\newcommand{\Oc}{\mathcal{O}}
\newcommand{\Or}{\mathrm{O}}
\newcommand\OR{\mathsf{OR}}
\newcommand{\ort}{\mathfrak{o}}
\newcommand{\PGL}{\mathrm{PGL}}
\newcommand{\ph}{\,\cdot\,}
\newcommand{\pr}{\mathrm{pr}}
\newcommand{\Prob}{\mathbb{P}}
\newcommand{\PSL}{\mathrm{PSL}}
\newcommand{\Ps}{\mathcal{P}}
\newcommand{\PSU}{\mathrm{PSU}}
\newcommand{\pt}{\mathrm{pt}}
\newcommand{\qeq}{\mathrel{``{=}"}}
\newcommand{\Q}{\mathbb{Q}}
\newcommand{\R}{\mathbb{R}}
\newcommand{\RP}{\mathbb{RP}}
\newcommand{\Rs}{\mathcal{R}}
\newcommand{\SL}{\mathrm{SL}}
\newcommand{\so}{\mathfrak{so}}
\newcommand{\SO}{\mathrm{SO}}
\newcommand{\Spin}{\mathrm{Spin}}
\newcommand{\Sp}{\mathrm{Sp}}
\newcommand{\su}{\mathfrak{su}}
\newcommand{\SU}{\mathrm{SU}}
\newcommand{\term}[1]{\emph{#1}\index{#1}}
\newcommand{\T}{\mathbb{T}}
\newcommand{\tv}[1]{|#1|}
\newcommand{\U}{\mathrm{U}}
\newcommand{\uu}{\mathfrak{u}}
\newcommand{\Vect}{\mathrm{Vect}}
\newcommand{\wsto}{\stackrel{\mathrm{w}^*}{\to}}
\newcommand{\wt}{\mathrm{wt}}
\newcommand{\wto}{\stackrel{\mathrm{w}}{\to}}
\newcommand{\Z}{\mathbb{Z}}
\renewcommand{\d}{\mathrm{d}}
\renewcommand{\H}{\mathbb{H}}
\renewcommand{\P}{\mathbb{P}}
\renewcommand{\sl}{\mathfrak{sl}}
\renewcommand{\vec}[1]{\boldsymbol{\mathbf{#1}}}
%\renewcommand{\F}{\mathcal{F}}

\let\Im\relax
\let\Re\relax

\DeclareMathOperator{\adj}{adj}
\DeclareMathOperator{\Ann}{Ann}
\DeclareMathOperator{\area}{area}
\DeclareMathOperator{\Aut}{Aut}
\DeclareMathOperator{\Bernoulli}{Bernoulli}
\DeclareMathOperator{\betaD}{beta}
\DeclareMathOperator{\bias}{bias}
\DeclareMathOperator{\binomial}{binomial}
\DeclareMathOperator{\card}{card}
\DeclareMathOperator{\ccl}{ccl}
\DeclareMathOperator{\Char}{char}
\DeclareMathOperator{\ch}{ch}
\DeclareMathOperator{\cl}{cl}
\DeclareMathOperator{\cls}{\overline{\mathrm{span}}}
\DeclareMathOperator{\conv}{conv}
\DeclareMathOperator{\corr}{corr}
\DeclareMathOperator{\cosec}{cosec}
\DeclareMathOperator{\cosech}{cosech}
\DeclareMathOperator{\cov}{cov}
\DeclareMathOperator{\covol}{covol}
\DeclareMathOperator{\diag}{diag}
\DeclareMathOperator{\diam}{diam}
\DeclareMathOperator{\Diff}{Diff}
\DeclareMathOperator{\disc}{disc}
\DeclareMathOperator{\dom}{dom}
\DeclareMathOperator{\End}{End}
\DeclareMathOperator{\energy}{energy}
\DeclareMathOperator{\erfc}{erfc}
\DeclareMathOperator{\erf}{erf}
\DeclareMathOperator*{\esssup}{ess\,sup}
\DeclareMathOperator{\ev}{ev}
\DeclareMathOperator{\Ext}{Ext}
\DeclareMathOperator{\Fit}{Fit}
\DeclareMathOperator{\fix}{fix}
\DeclareMathOperator{\Frac}{Frac}
\DeclareMathOperator{\Gal}{Gal}
\DeclareMathOperator{\gammaD}{gamma}
\DeclareMathOperator{\gr}{gr}
\DeclareMathOperator{\hcf}{hcf}
\DeclareMathOperator{\Hom}{Hom}
\DeclareMathOperator{\id}{id}
\DeclareMathOperator{\image}{image}
\DeclareMathOperator{\im}{im}
\DeclareMathOperator{\Im}{Im}
\DeclareMathOperator{\Ind}{Ind}
\DeclareMathOperator{\Int}{Int}
\DeclareMathOperator{\Isom}{Isom}
\DeclareMathOperator{\lcm}{lcm}
\DeclareMathOperator{\length}{length}
\DeclareMathOperator{\Lie}{Lie}
\DeclareMathOperator{\like}{like}
\DeclareMathOperator{\Lk}{Lk}
\DeclareMathOperator{\mse}{mse}
\DeclareMathOperator{\multinomial}{multinomial}
\DeclareMathOperator{\orb}{orb}
\DeclareMathOperator{\ord}{ord}
\DeclareMathOperator{\otp}{otp}
\DeclareMathOperator{\Poisson}{Poisson}
\DeclareMathOperator{\poly}{poly}
\DeclareMathOperator{\rank}{rank}
\DeclareMathOperator{\rel}{rel}
\DeclareMathOperator{\Re}{Re}
\DeclareMathOperator*{\res}{res}
\DeclareMathOperator{\Res}{Res}
\DeclareMathOperator{\rk}{rk}
\DeclareMathOperator{\Root}{Root}
\DeclareMathOperator{\sech}{sech}
\DeclareMathOperator{\sgn}{sgn}
\DeclareMathOperator{\spn}{span}
\DeclareMathOperator{\stab}{stab}
\DeclareMathOperator{\St}{St}
\DeclareMathOperator{\supp}{supp}
\DeclareMathOperator{\Syl}{Syl}
\DeclareMathOperator{\Sym}{Sym}
\DeclareMathOperator{\tr}{tr}
\DeclareMathOperator{\Tr}{Tr}
\DeclareMathOperator{\var}{var}
\DeclareMathOperator{\vol}{vol}

\pgfarrowsdeclarecombine{twolatex'}{twolatex'}{latex'}{latex'}{latex'}{latex'}
\tikzset{->/.style = {decoration={markings,
                                  mark=at position 1 with {\arrow[scale=2]{latex'}}},
                      postaction={decorate}}}
\tikzset{<-/.style = {decoration={markings,
                                  mark=at position 0 with {\arrowreversed[scale=2]{latex'}}},
                      postaction={decorate}}}
\tikzset{<->/.style = {decoration={markings,
                                   mark=at position 0 with {\arrowreversed[scale=2]{latex'}},
                                   mark=at position 1 with {\arrow[scale=2]{latex'}}},
                       postaction={decorate}}}
\tikzset{->-/.style = {decoration={markings,
                                   mark=at position #1 with {\arrow[scale=2]{latex'}}},
                       postaction={decorate}}}
\tikzset{-<-/.style = {decoration={markings,
                                   mark=at position #1 with {\arrowreversed[scale=2]{latex'}}},
                       postaction={decorate}}}
\tikzset{->>/.style = {decoration={markings,
                                  mark=at position 1 with {\arrow[scale=2]{latex'}}},
                      postaction={decorate}}}
\tikzset{<<-/.style = {decoration={markings,
                                  mark=at position 0 with {\arrowreversed[scale=2]{twolatex'}}},
                      postaction={decorate}}}
\tikzset{<<->>/.style = {decoration={markings,
                                   mark=at position 0 with {\arrowreversed[scale=2]{twolatex'}},
                                   mark=at position 1 with {\arrow[scale=2]{twolatex'}}},
                       postaction={decorate}}}
\tikzset{->>-/.style = {decoration={markings,
                                   mark=at position #1 with {\arrow[scale=2]{twolatex'}}},
                       postaction={decorate}}}
\tikzset{-<<-/.style = {decoration={markings,
                                   mark=at position #1 with {\arrowreversed[scale=2]{twolatex'}}},
                       postaction={decorate}}}

\tikzset{circ/.style = {fill, circle, inner sep = 0, minimum size = 3}}
\tikzset{mstate/.style={circle, draw, blue, text=black, minimum width=0.7cm}}

\tikzset{commutative diagrams/.cd,cdmap/.style={/tikz/column 1/.append style={anchor=base east},/tikz/column 2/.append style={anchor=base west},row sep=tiny}}

\definecolor{mblue}{rgb}{0.2, 0.3, 0.8}
\definecolor{morange}{rgb}{1, 0.5, 0}
\definecolor{mgreen}{rgb}{0.1, 0.4, 0.2}
\definecolor{mred}{rgb}{0.5, 0, 0}

\def\drawcirculararc(#1,#2)(#3,#4)(#5,#6){%
    \pgfmathsetmacro\cA{(#1*#1+#2*#2-#3*#3-#4*#4)/2}%
    \pgfmathsetmacro\cB{(#1*#1+#2*#2-#5*#5-#6*#6)/2}%
    \pgfmathsetmacro\cy{(\cB*(#1-#3)-\cA*(#1-#5))/%
                        ((#2-#6)*(#1-#3)-(#2-#4)*(#1-#5))}%
    \pgfmathsetmacro\cx{(\cA-\cy*(#2-#4))/(#1-#3)}%
    \pgfmathsetmacro\cr{sqrt((#1-\cx)*(#1-\cx)+(#2-\cy)*(#2-\cy))}%
    \pgfmathsetmacro\cA{atan2(#2-\cy,#1-\cx)}%
    \pgfmathsetmacro\cB{atan2(#6-\cy,#5-\cx)}%
    \pgfmathparse{\cB<\cA}%
    \ifnum\pgfmathresult=1
        \pgfmathsetmacro\cB{\cB+360}%
    \fi
    \draw (#1,#2) arc (\cA:\cB:\cr);%
}
\newcommand\getCoord[3]{\newdimen{#1}\newdimen{#2}\pgfextractx{#1}{\pgfpointanchor{#3}{center}}\pgfextracty{#2}{\pgfpointanchor{#3}{center}}}

\def\Xint#1{\mathchoice
   {\XXint\displaystyle\textstyle{#1}}%
   {\XXint\textstyle\scriptstyle{#1}}%
   {\XXint\scriptstyle\scriptscriptstyle{#1}}%
   {\XXint\scriptscriptstyle\scriptscriptstyle{#1}}%
   \!\int}
\def\XXint#1#2#3{{\setbox0=\hbox{$#1{#2#3}{\int}$}
     \vcenter{\hbox{$#2#3$}}\kern-.5\wd0}}
\def\ddashint{\Xint=}
\def\dashint{\Xint-}

\newcommand\separator{{\centering\rule{2cm}{0.2pt}\vspace{2pt}\par}}

\newenvironment{own}{\color{gray!70!black}}{}

\newcommand\makecenter[1]{\raisebox{-0.5\height}{#1}}

\begin{document}
\maketitle

\setcounter{section}{-1}
\section{Introduction}




Quantum Mechanics (QM) is a radical generalization of classical physics involving a new fundamental constant, \emph{Planck's constant}:

\[ \hbar = h / 2\pi \approx 1.05 \times 10^{-34} \text{Js},
 \]
 

 
 with dimensions 
 
 \begin{align*}
 [\hbar] = ML^{2} T^{-1} & =  \text{[position]} \times \text{[momentum]} \\
 & =  \text{[energy]} \times \text{[time]}
 \end{align*}
 
 Profound new features of QM include:
 
 \begin{itemize}
 	\item \emph{ Quantisation.} Physical quantities such as energy may be restricted to discrete sets of values, or may appear only in specific amounts, called \emph{quanta}.
 	
 	\item \emph{Wave-particle duality}. Classical concepts of a particle and a wave are merged; they become different aspects of a single entity that shows either particle-like or wave-like behaviour, depending on the circumstance.
 	
 	\item \emph{Probability and uncertainty}. Predictions in QM involve probability in a fundamental way and there are limits to what can be asked about a physical system, even in principle\footnote{If we were able to keep track of every single particle we would know exactly what the system is doing. But in QM, we \emph{still} can't know what the system is doing precisely. }. A famous example is the \emph{Heisenberg uncertainty principle relation} for position and momentum. 
\end{itemize}

Despite these radical changes, classical physics must be recovered in the limit $ \hbar \to 0 $ (which may require careful interpretation).

The following sections provide some physical background and summarise key experimental evidence for these novel features of QM.
 
\subsection{Light Quanta}

An electromagnetic (EM) wave, eg. light, consists of quanta called \emph{photons}. Photons can be regarded as particles with energy, $ E $, and momentum, $ p $, related to frequency\footnote{$ v,\omega $  both called frequency, and differ by a factor of $ 2 \pi $.} $ \nu $ or $ \omega $, and wavelength $ \lambda $, or wavenumber $ k $, according to

\[ E = h v = \hbar \omega \]
\[ p = h / \lambda = \hbar k \]

  

(Think of periodic functions, $ e^{\pm i \omega t}, e^{\pm i k x} $

From the wave equation (satisfied by each EM field component) 

\[ c = \omega / k  = v k \quad \text{ or } \quad E = cp \text{ massless particle}\]

so the relations are consistent with photons being particles of rest mass zero, moving with the speed of light, $ c $.

Compelling evidence for the existence of photons is provided by the \emph{photoelectric effect}. Consider (Fig. 1) light of EM radiation $ (\gamma) $ of frequency $ \omega $ incident on a metal surface. For certain metals and suitable frequencies this results in the emission of electrons $ (e^{-}) $ and their maximum kinetic energy $ K $ can be measured. 

[fig 1] 

Experiments find that (i) the rate at which electrons are emitted is proportional to the intensity of the radiation (the `brightness' of the source); (ii) $ K $ depends linearly on $ \omega $ but \emph{not} on the intensity; (iii) for $ \omega < \omega_{0} $, some critical value, \emph{no} electrons are emitted, irrespective of the intensity. 

The results are extremely hard to understand in terms of classical EM waves. However, they follow naturally from the assumption that the wave consists of photons, each with energy $ E = \hbar \omega $, and with the intensity of the radiation proportional to the number of photons incident per unit time. Suppose that an electron is emitted as a result of absorbing a single photon with sufficiently high energy. If $ W $ is the minimum energy needed to liberate an electron from the metal then

\[ K = \hbar \omega - W \]

is the maximum kinetic energy of an emitted electron if $ \omega > \omega_{0} $, where $ \omega_{0}  = W / \hbar $, and no emission is possible if $ \omega < \omega_{0} $ (Fig 2). Furthermore, the rate at which electrons are emitted will be proportional to the rate at which incident photons arrive, and hence the intensity.

Figure 2:




The energy-frequency relation for photons was introduced by Planck and used to derive the \emph{black body spectrum}. This is the distribution of energy with frequency for EM radiation in thermal equilibrium, a fundamental result in thermodynamics of far-reaching importance (understanding the \emph{cosmic microwave background}, for example). Einstein then applied the energy-frequency relation to explain the photoelectric effect. Further conclusive evidence for photons as particles, including the momentum-wavelength relation, came from subsequent experiments involving \emph{Compton scattering. }

Consider a photon of wavelength $ \lambda $ colliding with an electron that is stationary in the laboratory frame. Let $ \lambda' $ be the wavelength of the photon after the collision and $ \theta $ the angle through which it is deflected. Treating the photon as a massless relativistic particle, conservation of four-momentum implies

\[ \lambda' - \lambda = \frac{h}{m_{e}c} (1 - cos \theta) \] 

Figrue 3

This dependence of the change in wavelength (or decrease in energy) can be verified experimentally (for $ X $-rays, or $ \gamma $-rays, for instance). 

\subsection{Bohr Model of the Atom}

The \emph{Rutherford model} of the atom was proposed to explain the results of scattering experiments (eg. alpha particles scattered by gold foil). The key assumption is that most of the mass of the atom is concentrated in a compact, positively-charged \emph{nucleus} (subsequently understood to consist of protons and neutrons), with light, negatively charged electrons orbiting around it. The simplest case is the Hydrogen atom, in which a single electron with charge $ - e $ and mass $ m_{e} $ orbits a nucleus consisting of a single proton with charge $ +e $ and mass $ m_{p} $. Since $ m_{p} \geq\geq m_{e} $ it is a good approximation to assume the proton is stationary, at the origin, say. The electron and proton interact via Coulomb's Law: the potential energy of the electron and the force it experiences are:
\[ V(r) = - \frac{e^{2}}{4\pi \varepsilon_{0}} \frac{1}{r} \qquad \mathbf{F}(\mathbf{r}) =  - \nabla V = - \frac{e^{2}}{4\pi \varepsilon_{0}} \hat{\mathbf{r}}\]


EM radiaton of accerelated charge $ \Rightarrow $ this would be unstable. Also - experimental evidence for complex structure in atoms. Line spectra:

Bohr quantisation condition: angular momentum takes discrete set of possile values: $ L = n \hbar, n = 1,2,\cdots $

So if you don't have a continuous set of orbits, it can't lose a bit of energy, seems dubious, but there's no denying that this assmuption is a simple way of reproducting the line specra shit.

This implies discrete set of allwoed orbits with energies 

\[ E_{n} = - \frac{1}{2} m_{e} \left( \frac{e^{2}}{4\pi \varepsilon_{0} \hbar} \right)^{2} \frac{1}{n^{2}}  \]

Note that the allowed energy levels are now discrete. 

Suppose that an electron makes a transition between levels $ n $ and $ n' $ (with $ n > n' $ say ) accompanied by emission or absorption of a photon of frequecy $ \omega $ (Fig 5). Then

For emission or absorption of radiation with atom making transition between levels $ n $ \& $ n' $ with $ n'  > n$


\subsection{Matter Waves}

Use same relations:

$ E = \hbar \omega $, $ p = \hbar k $

to say particles behave like waves.

Note Bohr quantisation conditios becomes:

\[ L = pr = n \hbar \iff n \lambda = 2 \pi r \]

$ n = 3 $ - Fig 6.


To confirm wave-behaviour of electrons, two slit experiment.

But Fig 8.... Interference pattern

Fig 9. I

In diffraction experimets with electorns, we cannot predict what will happen to any \emph{single} particle; the most that can be said is that it will be detected at a given postiion with a certain \emph{probability}
 



\section{Wave functions and Operators}

In first few chapters we consider a quantum particle in one dimension, introduce some key ideas and three postulates for how to extract physical information from mathematical framework. Later (Chapter 6) we will state general axioms from which these follow. 

\subsection{Wave functions and States  }

A classical point particle in one-dimension has a position $ x $ at each time. In QM a particle has a \emph{state} at each time given by a complex-valued wavefunction 

\[ \psi(x) \]

Postulate(P1): A measurement of position gives a result with probability density $ | \psi(x) |^{2} $. ie. $ | \psi(x) |^{2} \delta x $ prob particle is found between $ x $ and $ x + \delta x $, or 

\[ \int_{a}^{b} | \psi(x) |^{2} \; \d x \]

is the probability the particle is found in the interval $ a \leq x \leq b $.
This requires $ \psi(x) $ is \emph{normalised}.

\[ \int_{-\infty}^{\infty} | \psi(x) |^{2} \; \d x = 1 \qquad \text{ single particle - total probability 1 } \]

\begin{eg}(Gaussian wavefunction)
	
	\[ \psi(x) = C e^{- (x - x_{0})^{2} / 2 \alpha } \qquad \text{ real } \alpha > 0 \]
	
	
	 \begin{center}
	 	%label axis x,y as x, |psi(x)|^{2}
		\begin{tikzpicture}[yscale=1.5]
		\draw [->] (-3, 0) -- (3, 0) node [right] {$x$};
		\draw [->](0, 0) node [below] {$x_{0}$}  -- (0, 1.3) node [above] {$| \psi(x) |^{2}$};
		\draw [domain=-3:3,samples=50, mblue] plot (\x, {exp(-\x * \x)});
		\end{tikzpicture}
	\end{center}



\begin{align*}
	\int_{-\infty}^{\infty} | \psi(x) |^{2} \; \d x & = | C |^{2} \int_{- \infty}^{\infty} e ^{ - (x - x_{0})^{2} / \alpha} \; \d x \\
	& = | C |^{2} (\alpha \pi)^{\frac{1}{2}} = 1
\end{align*}

So \footnote{Recall $ \int_{-\infty}^{\infty}  e^{-\frac{(x-\mu)^{2}}{2 \sigma^{2}}} \; \d x = \sqrt{2\pi \sigma^{2}}  $}   $ \psi $ normalised if $ C = \left( \frac{1}{\alpha \pi} \right)^{\frac{1}{4}}  $.
$ \alpha $ small $ \Rightarrow $ sharp peak around $ x = x_{0} $, $ \Rightarrow $ ``particle-like".
$ \alpha $ large $ \Rightarrow $ more spread out (diffuse)
 
\end{eg}

It is convenient to deal more generally with \emph{normalisable} wavefunctios

\[ \int_{-\infty}^{\infty} | \psi(x) |^{2} \; \d x < \infty \qquad \text{ finite/convergent} \]
   
Then $ \psi(x) $ and   $  \phi(x) = \lambda \psi(x) $ are physically equivalent and represent same state for any $ \lambda \neq 0 $. Provided $ \psi $ normalisable, we can choose $ \lambda $ so that $ \phi $ normalised. But if $ \psi $ normalised already, then $ \phi(x) = e^{i \alpha} \psi(x) $ (for any real $ \alpha $) gives some probability distribution. 

\[ | \phi(x) |^{2} = | \psi(x) |^{2} \]

So quantum state is strictly an equivalence class of non-zero wave functions but in practice we often refer to $ \psi(x) $ as ``the state''.

Any non-zero normalisable wave function $ \phi(x) $ represents a physical state. For our purposes we can assume, \emph{unless} we say otherwise, $ \psi(x) $ is smooth (can be differentiated any number of times), and $ \psi(x) \to 0 $ as $ | x | \to \infty $

If $ \psi_{1}(x) $ and $ \psi_{2}(x) $ are normalisable then so is 

\[ \psi = \lambda_{1} \psi_{1} + \lambda_{2} \psi_{2} \] 

for any complex $ \lambda_{1},\lambda_{2} $.

Physically: principle of super position
Mathematically: structure of complex vector space.

\begin{eg} (Superposition of Gaussians)
	\[ \psi(x) = B( e^{\frac{-x^{2}}{2\alpha}}  + e^{-(x-x_{0})^{2}/2\beta}) \]
	
	Can choose $ B $ so that $ \psi $ is normalised
	
	
	 \begin{center}
	 	% \draw [ ->](0, 0) -- (0,1.3) node[above] {$| \psi(x) |^{2}$
	 	%dashed vertical line going through the second wave, label with x_{0}
		\begin{tikzpicture}[xscale=0.75, yscale=1.5]
		\draw [ ->](-3, 0) -- (9, 0) node[right] {$x$};
		\draw [ ->](0, 0) -- (0,1.3) node[above] {$| \psi(x) |^{2}$};
		\draw [dashed] (6, -0.5) node[below] {$ x_{0} $} -- (6,1.3) ;
		\draw [domain=-3:9,samples=80, mblue] plot (\x, {exp(-\x * \x) + exp(-(\x - 6)^2/4)});
		\end{tikzpicture}
	\end{center}
\end{eg}


\subsection{Operators and Observables}

A quantum state contains information about other physical quantities or \emph{observables} (momentum, energy) not just position. In QM each observable is represented by an \emph{operator} (denoted by a hat when necessary) acting on wave functions

\begin{center}
	\begin{tabular}{rll}
		position & $\hat{x} = x$ & $\hat{x} \psi = x\psi(x)$\\
		momentum & $\hat{p} = -i\hbar \frac{\partial}{\partial x}$ & $\hat{p}\psi = -i\hbar \psi'(x)$\\
		energy & $H = \frac{\hat{p}^2}{2m} + V(\hat{x})$ & $H\psi = -\hbar^2 \frac{\partial^2}{\partial x^2}\psi + V(x)\psi(x)$
	\end{tabular}
\end{center}

for a particle of mass $ m $ in a potential $ V(x) $.

If we measure one of these quantities, what answers can we get and what are the probabilities? Partial answers provided by (P2) and (P3)

\subsubsection{Expectation Values}

For any (normalisable) $ \psi(x) $ and $ \phi(x) $, define 

\[ (\psi,\phi) : = \int_{-\infty}^{\infty} \psi(x)^{*}\phi(x) \; \d x \]

Note that this is the complex inner product on vector space,

For $ \psi(x) $ normalised, define the \emph{expectation value} of an observable $ Q $ in this state, to be

\begin{align*}
	<Q>_{\psi}& : = (\psi, Q \psi) \\
	& = \int_{-\infty}^{\infty} \psi^{*} (Q \psi) \; \d x
\end{align*} 

Note

\begin{align*}
	<\hat{x}>  & =  (\psi, \hat{x} \psi) \\
	& =  \int_{-\infty}^{\infty} x | \psi(x) |^{2} \; \d x
\end{align*}

standard expression for mean, or expected value of $ x $, given (P1)



Postulate (P2): For any observable, $ <Q>_{\psi} $ is the mean result (expected value) if $ Q $ is measured many times ($ N $ times and then $ N \to \infty $) with the particle in state $ \psi $ before each measurement.

Consider wave functions:

\[ \phi(x) = \psi(x) e^{ikx} \qquad \text{with } k \text{ real constant} \]

Clearly

\[ | \phi(x) |^{2} = | \psi(x) |^{2} \]

and so 

\[ <\hat{x}>_{\phi} = <\hat{x}>_{\psi} \]
But

\begin{align*}
	<\hat{p}>_{\phi} & = \int_{-\infty}^{\infty} \phi^{*} (-i \hbar \phi') \; \d x \\
	& = \int_{-\infty}^{\infty} \psi^{*} (-i \hbar \psi^{'}) \; \d x + \hbar k \int_{-\infty}^{\infty} \psi^{*} \psi \; \d x \\
	& = <\hat{p}>_{\psi} + \hbar k
\end{align*}

where $ \hat{p} = - i \hbar \frac{\d }{\d x} $


\begin{eg}
\[ \psi(x) = C e^{- x^{2} / 2 \alpha} \]

as in 1.1 with $ x_{0} = 0 $

\[ \Rightarrow \;  <\hat{p}>_{\psi} = 0 \]

and 

\[ \phi(x) = C e^{-x^{2} / 2 \alpha} e^{i k x} \]

\[ \Rightarrow \; <\hat{p}>_{\psi} = \hbar k  \]
\end{eg}
  \end{document}